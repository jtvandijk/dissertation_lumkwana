\chapter{Conclusion}
\section{Summary of results}
This study aimed assessing models systems in order to gain a better understanding of the effect of a low and high concentration of spermidine and rilmenidine.
% rewrite the top paragraph
Our findings show indeed that  there is a concentration dependent effect of spermidine and rilmenidine on autophagy which can only be revealed by an accurate assessment of autophagy using different tools, as not all tools that are widely used to monitor autophagy, are always sensitive enough as revealed in this study. Our results suggest that drug screening should be performed using multiple tools, and most important screening of a concentration range and subsequent effects on autophagic flux which can be aligned with the given disease. This suggests, that it must be taken into account which drug concentration is of importance, in order to better align flux control with in context of autophagy dynamicsis most favorable with which offset, or with which decrease or dysfunction of autophagy as not all autophagy modulators are beneficial, as we have shown in the case for rilmenidine, suggesting drug specificity. 

Indeed, our finding using the \textit{in vitro} model of PQ-induced toxicity showed that spermidine at a low concentration and not a high concentration protected against cell toxicity, ROS damage and cell death as well as microtubule destabilization in a manner that was dependent on autophagy. These results further suggest that only low concentration of spermidine was beneficial in the context of PQ induced neurodegeneration textit{in vitro}. Moreover, rilmenidine at both concentrations failed to protect in the same context, suggesting that neuronal protection against PQ induced neuronal injury is drug specific. This data supports the notion of  aligning a specific drug and specific concentration concentration with the disease, cell type specific based on its impact on autophagic flux. In line with this notion, our results in the \textit{in vitro} model of neuronal toxicity induced by APP over-expression showed for the first time that both spermidine and rilmenidine protected against cytotoxicity effects induced by APP over-expression. Hence, the mode of injury has major impact on whether autophagy enhancement  confers protection. 

In addition, our findings reveal that spermidine at low concentration effectively cleared APP clusters and reduced their size, especially after 48 h of APP over-expression as discussed in chapter 5. More so, when using rilmenidine, interestingly, a concentration dependent effect on APP protein clearance in terms of  cluster number and size, particularly, after 48 h APP over-expression (chapter 5). Lastly, we demonstrated that PQ-induced toxicity impacts the brain regions differentially, with the hippocampus being high susceptible to PQ injury followed by the cortex (chapter 7). These results may point towards region-specific autophagy activity and are in line with the findings that the pathological changes have been shown to manifest first in the hippocampus and then later in the cortex as the disease pathology progresses \citep{Braak2004,Braak1998,Braak1991,Braak2012}. Moreover, we show that both concentrations of spermidine confer protection against oxidative stress, neuronal damage, microtubule destabilization in a similar manner in the hippocampus and cortex by upregulating autophagy. Autophagy was upregulated in a manner dependent on the concentration with the highest concentration spermidine increasing LC3 II in the hippocampus, while a lower concentration increased LC3 II in the cortex, while both concentration affected LAMP2A levels in a similar manner in both brain regions. 
%cross ref chapter 5
\section{Conclusions and future outlook}
In overall, our results showed the importance of using different tools to assess a concentration dependent autophagic flux modulation under normal homeostasis and to assess their protective effects under pathological conditions in different cell types and brain regions. We also provided evidence of the protective roles of rilmenidine and spemidine in an in vitro model of AD as well as protective roles of spermidine \textit{in vitro} and \textit{in vivo} models of PQ induced toxicity. These results suggest that the type of the cell, brain region and concentration should be taken into account when performing drug screening so that a specific drug and a particular concentration can be aligned with a particular disease with the aim of offsetting the deviation. It is widely known that neurons depend vastly on a high degree of efficient autophagic flux in order to survive due to their post-mitotic nature, high protein synthesis rate and energy demands. However, protein degradation and proteolysis is regional and cell-specific, requiring sensitive characterization of autophagic flux, and autophagosomal and lysosomal pool sizes in order to better understand the underlying mechanisms that govern a particular autophagic activity. While extensive data from preclinical studies in animals provide strong evidence that autophagy up-regulation does indeed have therapeutic potential for the treatment of AD, there are numerous unanswered questions regarding the efficiency of autophagy induction using such interventions. Therefore, a better understanding is required regarding the brain-specific spatio-temporal autophagic activity in physiology and in neuronal pathology, so as to better regulate autophagy and thereby to control neuronal fate. In addition, autophagy modulation requires careful alignment with aggregate-prone protein cargo levels as well as markers of disease, in order to precision target the autophagic machinery according to defect localization. Future work should focus on enhancing methodologies in order to advance research. For example, we have performed volume analysis in 3D, however, this is labour intensive. Generating 3D EM analysis in a high throughput system would be beneficial, suggesting the need for developing and investing on technologies that focus into highly specific autophagy assessment. Applications with this approach within the \textit{in vitro} and \textit{in vivo} environment will allow improved and progressive translation in preclinical and clinical settings, and may make autophagy modulation a prime candidate tool in the treatment of neurodegeneration. 

\section{Limitations}