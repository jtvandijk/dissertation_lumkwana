\chapter{Conclusion}
\section{Summary}
This study aimed to unravel the role of autophagy modulation, using a low and a high concentration of autophagy enhancing drugs, spermidine and rilmenidine, on autophagic flux, neuronal toxicity and protein clearance using distinct model systems. Our findings reveal four critical aspects with regards to autophagy and cell death control. Firstly, indeed, a concentration dependent effect of autophagy-inducing drugs on autophagic flux exists, particularly highlighted by concentration dependency (\Cref{sec:chapter3} \& \Cref{sec:chapter6}), cell type specificity (\Cref{sec:chapter4} \& \Cref{sec:chapter5}), injury specificity (\Cref{sec:chapter4}, \Cref{sec:chapter5} \& \Cref{sec:chapter7}), \textit{in vivo} region specificity (\Cref{sec:chapter7}) and method detection sensitivity (WB, FM, TEM, and CLEM) (\Cref{sec:chapter3} \& \Cref{sec:chapter6}) (\Cref{fig:80_spermidine} \& \Cref{fig:80_rilmenidine}). Our results suggest that drug screening should indeed be performed using multiple tools, and most importantly screening of a concentration range and subsequent effects on autophagic flux be included, so as to be aligned with the given disease. This suggests, that it must be taken into account which drug concentration to use, in order to better align flux control with the most favourable autophagy activity in context of autophagy dysfunction. We have shown in a PQ-induced neuronal toxicity model that rilmenidine at both concentrations failed to upregulate autophagy and failed to protect against ROS damage and subsequent cell death (\Cref{sec:chapter4}). Spermidine at a low concentration upregulated autophagy but not significantly so, while spermidine at a high concentration significantly upregulated autophagy. However, only the low concentration protected against ROS damage and cell death, and was therefore beneficial in the context of PQ-induced neurodegeneration in \textit{in vitro}. These results together with the results obtained when using rilmenidine suggest a drug specificity and that higher flux does not necessarily translate to a higher protection. In an \textit{in vitro} model of neuronal toxicity induced by APP over-expression, we showed for the first time that both spermidine and rilmenidine protected against cytotoxicity effects induced by APP over-expression (\Cref{sec:chapter5}), suggesting cell specificity. In addition, these results suggest that the mode of injury has major impact on whether autophagy enhancement  confers protection. In addition, our findings reveal that spermidine at low concentration effectively cleared APP clusters and reduced their size, especially after 48 h of APP over-expression (\Cref{sec:chapter5}). More so, when using rilmenidine, a concentration-dependent effect on APP protein clearance was observed in terms of  cluster number and size, particularly, after 48 h APP over-expression (\Cref{sec:chapter5}). Taken together, this data supports the notion of aligning a specific drug and specific concentration with the pathology, here proteotoxicity, based on its impact on autophagic flux. Moreover, we have successfully implemented a 3D CLEM protocol and revealed that a high concentration of spermidine, and not a low concentration, decreases autophagosome volume (\Cref{sec:chapter6}). Lastly, we demonstrated that PQ-induced toxicity impacts the brain regions differentially, with the hippocampus being highly susceptible to PQ-induced injury followed by the cortex (\Cref{sec:chapter7}). These results may point towards region-specific autophagy activity and are in line with the findings that the pathological changes of protein aggregation and neuronal loss are manifesting first in the hippocampus and later in the cortex, as the disease pathology progresses \citep{Braak2004,Braak1998,Braak1991,Braak2012}. Moreover, we showed that \textit{in vivo}, in contrast to the \textit{in vitro} model (\Cref{sec:chapter4}), both dosages of spermidine confer protection against oxidative stress, neuronal damage, microtubule destabilization, and upregulated autophagy, similarly in the hippocampus and cortex. However, the low dose of spermidine conferred enhanced protection, although, autophagy was here upregulated in a manner dependent on the dose utilized, with the high dose of spermidine increasing LC3-II in the hippocampus, while the low dose increased LC3-II in the cortex. In addition, both dosages of spermidine increased LAMP2A levels in a similar manner in both brain regions. 
 
\section{Conclusions and future outlook}
Taken together, our results showed firstly the importance of using multiple tools to assess autophagy and that a concentration-dependent effect on autophagic flux exists, however, not all techniques are sensititive enough to detect the differences in flux. These effects manifest differently under normal homeostasic and pathological conditions, depending on the injury model, cell type, and model system used. We also provided evidence of the distinct, context dependent protective role of spermidine and rilmenidine in an \textit{in vitro} model of APP over-expression as well as of the protective role of spermidine in \textit{in vitro} and \textit{in vivo} models of PQ-induced neuronal toxicity. These results suggest that the cell type, brain region and the concentration should be critically taken into account when performing drug screening so that a specific drug and a particular concentration can be most favourably aligned with a particular disease with the aim to offset autophagy dysfunction. It is widely known that neurons depend vastly on a high degree of efficient autophagic flux in order to survive due to their post-mitotic nature, high protein synthesis rate and energy demands. However, protein degradation and proteolysis is regional and cell-specific, requiring sensitive characterization of autophagic flux, autophagosomal and lysosomal pool sizes and transition time  in order to better understand the underlying mechanisms that govern a particular autophagic activity. While extensive data from preclinical studies in animal models provide strong evidence that autophagy upregulation does indeed have therapeutic potential for the treatment of AD, there are numerous unanswered questions regarding the efficiency of autophagy induction using such interventions. Therefore, a better understanding is required regarding the context of neuronal protection, with a quantitative approach to assess autophagy. This is critical in the context of brain-specific spatio-temporal autophagic activity in physiology and in neuronal pathology, so as to better regulate autophagy and thereby to control neuronal fate. In addition, autophagy modulation requires careful alignment with aggregate-prone protein cargo levels as well as markers of disease, in order to precision-target the autophagic machinery according to defect localization. Here, we have utilized STORM to assess APP clusters and 3D CLEM to assess autophagosome volume as a point of departure in this regard. Future work may, therefore, focus on enhancing methodologies, in particular high throughput-based approaches in order to advance translation. For example, we have performed morphometric analysis of autophagosomes in 3D, however, this is currently highly labour intensive. Efforts are on the way of generating 3D EM data in a high throughput manner, which would be highly beneficial in the context of autophagy assessment. Our results suggest that administration of spermidine may represent a favourable therapeutic strategy for the treatment of AD. Further studies, using transgenic mice models of APPSwe, PS1 and 3xTg AD are warranted to further verify the protective effects of spermidine. In such a models, behavioural studies using y-maze or Morris water maze test  to assess cognitive functions (learning and memory) will provide invaluable information about the roles of spermidine in learning and memory function, i.e. does spermidine intervention rescues spatial learning and memory deficits in these mice. Applications with this approach within the \textit{in vitro} and \textit{in vivo} environment will undoubtedly allow improved and progressive translation of autophagy control in preclinical and clinical settings, and may advance autophagy modulation as a prime candidate for the treatment of neurodegeneration.

\begin{landscape}
\begin{figure}[!htbp]
\vspace{2.5cm}
\center
 \includegraphics[width=\linewidth]{figures/chapter80/80_spermidine}
 \caption[Summary of findings I: Protective roles of spermidine against PQ-induced and APP induced neurodegeneration]{\textbf{Summary of findings I: Protective roles of spermidine against PQ-induced and APP induced neurodegeneration.}  In the PQ-induced model, PQ increased ROS, cellular toxicity, oxidative stress, neuronal injury, microtubule destabilization and cell death, while APP over-expression increased APP clusters and cellular toxicity, both leading to molecular hallmarks of neurodegeneration. Spermidine ameliorated these defects in a concentration-dependent manner and conferred neuroprotection.}
 \label{fig:80_spermidine}
\end{figure} 
\end{landscape} 
 
 \begin{landscape}
\begin{figure}[!htbp]
\vspace{3cm}
\center
 \includegraphics[width=\linewidth]{figures/chapter80/80_rilmenidine}
 \caption[Summary of findings II: Protective roles of rilmenidine against PQ-induced and APP-induced neurodegeneration]{\textbf{Summary of findings II: Protective roles of rilmenidine against PQ-induced and APP-induced neurodegeneration.}  In the PQ-induced model, PQ increased ROS, cellular toxicity and cell death, while APP over-expression increased APP clusters and cellular toxicity, both leading to molecular hallmarks of neurodegeneration. Rilmenidine ameliorated APP-related defects and conferred neuroprotection.}
 \label{fig:80_rilmenidine}
\end{figure} 
\end{landscape} 






















