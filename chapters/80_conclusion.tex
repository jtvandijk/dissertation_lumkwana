\chapter{Conclusion}
\section{Summary of results}
This study aimed to unravel the role of autophagy modulation, using high and low concentrations of autophagy enhancing drugs, spermidine and rilmenidine, on autophagic flux, neuronal toxicity and protein clearance using distinct model systems. Our findings reveal four critical aspects with regards to autophagy and cell death control. Firstly, indeed, a concentration dependent effect of autophagy drugs exists, particularly shown by concentration dependency (\Cref{sec:chapter3} \& \Cref{sec:chapter6}), cell type specificity (\Cref{sec:chapter4} \& \Cref{sec:chapter5}), injury specificity (\Cref{sec:chapter4}, \Cref{sec:chapter5} \& \Cref{sec:chapter7}), \textit{in vivo} region specificity (\Cref{sec:chapter7}) and method specificity (WB, FM, TEM, and CLEM). Our results suggest that drug screening should be performed using multiple tools, and most important screening of a concentration range and subsequent effects on autophagic flux which can be aligned with the given disease. This suggests, that it must be taken into account which drug concentration to use, in order to better align flux control with the most favourable offset in context of autophagy dysfunction. For example, we have shown in the case of rilmenidine that both concentrations failed to upregulate autophagy (decreased the LC3-II turnover and autophagosome flux and increased p62 protein expression) and failed to protect against ROS damage and subsequent cell death (\Cref{sec:chapter3} \& \Cref{sec:chapter4}), while spermidine upregulated autophagy and protected against ROS damage and cell death, suggesting drug specificity. Indeed, our finding using the \textit{in vitro} model of PQ-induced toxicity showed that spermidine at a low concentration and not a high concentration protected against cell toxicity, ROS damage and cell death as well as microtubule destabilization in a manner that was dependent on autophagy. These results further suggest that only a low concentration of spermidine was beneficial in the context of PQ induced neurodegeneration textit{in vitro}. Moreover, rilmenidine at both concentrations failed to protect in the same context, suggesting that neuronal protection against PQ induced neuronal injury is drug-specific and cell type-specific. This data supports the notion of  aligning a specific drug and specific concentration with the disease, based on its impact on autophagic flux. In line with this notion, our results in the \textit{in vitro} model of neuronal toxicity induced by APP over-expression showed for the first time that both spermidine and rilmenidine protected against cytotoxicity effects induced by APP over-expression (\Cref{sec:chapter5}). Hence, the mode of injury has a major impact on whether autophagy enhancement  confers protection. In addition, our findings reveal that spermidine at low concentration effectively cleared APP clusters and reduced their size, especially after 48 h of APP over-expression (\Cref{sec:chapter5}). More so, when using rilmenidine, interestingly, a concentration-dependent effect on APP protein clearance was observed in terms of  cluster number and size, particularly, after 48 h APP over-expression (\Cref{sec:chapter5}). Lastly, we demonstrated that PQ-induced toxicity impacts the brain regions differentially, with the hippocampus being highly susceptible to PQ injury followed by the cortex (\Cref{sec:chapter7}). These results may point towards region-specific autophagy activity and are in line with the findings that the pathological changes of protein aggregation and neuronal loss manifesting first in the hippocampus and later in the cortex as the disease pathology progresses \citep{Braak2004,Braak1998,Braak1991,Braak2012}. Moreover, we showed that in \textit{in vivo}, in contrast to the \textit{in vitro} model (\Cref{sec:chapter4}), both dosages of spermidine confer protection against oxidative stress, neuronal damage, microtubule destabilization, and upregulated autophagy, similarly in the hippocampus and cortex. However, the lower dose of spermidine gave a stronger protection, although, autophagy was here upregulated in a manner dependent on the dose utilized, with the higher dose of spermidine increasing LC3-II in the hippocampus, while a lower dose increased LC3-II in the cortex. In addition, both dosages of spermidine increased LAMP2A levels in a similar manner in both brain regions. 

\section{Conclusions and future outlook}
Taken together, our results showed firstly the importance of using multiple tools to assess autophagy and that a concentration-dependent effect on autophagic flux modulation exists. These effects manifest differently under normal homeostasis and pathological conditions, depending on the injury model, cell type, and model system used. We also provided evidence of the distinct, context dependent protective roles of spemidine and rilmefnidine in an \textit{in vitro} model of APP over-expression as well as the protective roles of spermidine \textit{in vitro} and \textit{in vivo} models of PQ-induced neuronal toxicity. These results suggest that the cell type, brain region and the concentration should be critically taken into account when performing drug screening so that a specific drug and a particular concentration can be most favourable aligned with a particular disease with the aim of offsetting autophagy dysfunction. It is widely known that neurons depend vastly on a high degree of efficient autophagic flux in order to survive due to their post-mitotic nature, high protein synthesis rate and energy demands. However, protein degradation and proteolysis is regional and cell-specific, requiring sensitive characterization of autophagic flux, autophagosomal and lysosomal pool sizes in order to better understand the underlying mechanisms that govern a particular autophagic activity. While extensive data from preclinical studies in animals models provide strong evidence that autophagy upregulation does indeed have therapeutic potential for the treatment of AD, there are numerous unanswered questions regarding the efficiency of autophagy induction using such interventions. Therefore, a better understanding is required regarding the context of neuronal protection, with a quantitative approach to assess autophagy. Moreover, this is critical in the context of brain-specific spatio-temporal autophagic activity in physiology and in neuronal pathology, so as to better regulate autophagy and thereby to control neuronal fate. In addition, autophagy modulation requires careful alignment with aggregate-prone protein cargo levels as well as markers of disease, in order to precision target the autophagic machinery according to defect localization. Here, we have utilized STORM to assess APP clusters as a point of departure in this regard. Future work may, therefore, focus on enhancing methodologies, in particular high throughput-based in order to advance translation. For example, we have performed morphometric analysis of autophagosomes in 3D, however, this is currently labour intensive. Currently, efforts are on the way of generating 3D EM data in a high throughput manner, which, would be highly beneficial in the context of autophagy assessment. Applications with this approach within the \textit{in vitro} and \textit{in vivo} environment will undoubtedly allow improved and progressive translation of autophagy control in preclinical and clinical settings, and may advance autophagy modulation as a prime candidate for the treatment of neurodegeneration. 