%-------------------------------------------------------------------------------
% Titlepage
%-------------------------------------------------------------------------------
\WaterMark[0.15\paperwidth]{UScrest-WM} 
\TitlePage

%-------------------------------------------------------------------------------
% Declaration page
%-------------------------------------------------------------------------------
\chapter{Declaration}
By submitting this dissertation electronically, I declare that the entirety of the work contained therein is my own, original work, that I am the sole author thereof (save to the extent explicitly otherwise stated), that reproduction and publication thereof by \mbox{Stellenbosch University} will not infringe any third party rights and that I have not \mbox{previously} in its entirety or in part submitted it for obtaining any qualification. \\ \\

\noindent
\begin{tabular}{@{}ll}
\textbf{Date:} & November 4, 2019 \\ 
\textbf{Name:} & Dumisile Lumkwana \\ 
\end{tabular}

\vspace*{\fill}
\begin{center}
Copyright $\textcopyright$ Stellenbosch University \\
All rights reserved.
\end{center}

%-------------------------------------------------------------------------------
% Abstract
%-------------------------------------------------------------------------------
\chapter{Abstract}

\textit{Introduction} - Alzheimer’s disease is a neurodegenerative disease characterized by progressive cognitive impairment, particularly in the brain regions important for learning and memory. The neuropathology of AD is characterised by two molecular hallmarks apparent in the brain tissue; the intracellular protein aggregates known as neurofibrillary tangles (NFTs), composed of hyper-phosphorylated Tau and extracellular amyloid beta (A$\beta$) plaques, composed of A$\beta$ peptides derived from the amyloid precursor protein (APP), both of which occur as a result of an imbalance in proteostasis. Although, we have advanced our understanding of the molecular machinery that regulates the rate of protein degradation through autophagy at basal levels and the many aspects of its dysfunction in AD, the deviation of autophagic activity from basal levels and its change during disease pathogenesis in neuronal tissue remains largely unclear. Over the recent years, we have made substantial progress in modulating autophagy using pharmacological agents \textit{in vitro} and \textit{in vivo} and mounting evidence points towards autophagy modulation using pharmacological agents as one of the major therapeutic strategies for neurodegenerative diseases. Although, spermidine and rilmenidine both enhance autophagy, the exact relationship between autophagy activity, the extent of protein clearance and death onset remains unclear. Moreover, the impact of concentration differences on autophagic flux and on subsequent protein clearance and neuronal toxicity is unclear. Therefore, this study aimed to unravel the role of autophagy modulation, using high and low concentrations of spermidine and rilmenidine, on autophagic flux, neuronal toxicity and protein clearance using distinct model systems. 

\textit{Methods} - For the \textit{in vitro} model, murine hypothalamus-derived GT1-7 neuronal cell and the mouse neuroblastoma (N2a) cell line stably expressing Swedish  double mutation APP695 (Swe) associated with AD pathology were used. GT 1-7 cells and GT 1-7 transfected with mRFP-GFP-LC3 and GFP-LC3-RFP-LC3$\Delta$G were treated with low and high concentrations of spermidine and rilmenidine in the presence of saturating concentrations of bafilomycin after which autophagic profile of these two drugs was characterized by assessing for cellular viability, autophagic flux, autophagosomes pool, autolysosome pool, autophagosome flux, transition time, p62 puncta and autophagic vacuoles using cellular viability assay, western blotting, fluorescence microscopy, transmission electron microscopy and correlative light and electron microscopy linked quantitative morphometric analysis. In addition, potential protective effects of low and high concentration of  spermidine and rilmenidine were assessed in a PQ-induced neuronal toxicity model and in an APP over-expression model. Cellular viability assays, ROS damage, cell death onset and autophagic activity were performed in the PQ-induced toxicity model, while cellular viability assays, western blotting and protein clusters were assessed and quantified using single molecule imaging, i.e, d-STORM. Moreover, potential protective effects of spermidine at two distinct concentrations were assessed using transgenic mice expressing GFP-LC3 that were treated with PQ to induce neuronal toxicity associated with neurodegeneration. Lastly, a CLEM method was developed and used to assess the effects of low and high concentrations of spermidine on the localization of autophagosomes in 2 and 3 dimensions. 

\textit{Results} - Our results indicate a concentration-dependent effect of spermidine and rilmenidine on autophagic flux and that the detected change in the flux depends heavily on the specificity on the method used.  In addition, in an \textit{in vitro} model of PQ-induced toxicity, our results revealed that spermidine at a low concentration and not a high concentration protected against cell toxicity, ROS damage, cell death as well as microtubule destabilization in a manner that was dependent on autophagy. To our surprise, both concentrations of rilmenidine failed to protect against ROS damage and cell death and also failed to up-regulate autophagy in the same model.  Moreover, in an \textit{in vitro} model of neuronal toxicity induced by APP over-expression, our results showed that both concentrations of spermidine and rilmenidine protected against cytotoxicity. Spermidine at low concentration effectively cleared APP clusters and reduced their size, especially after 48 h of APP over-expression, while rilmenidine, reduced the number and size of APP clusters in a concentration-dependent manner at the same time point.  
Taken together, the \textit{in vivo} results reveal a concentration dependency effect, drug dependency effect, method specificity, a cell type specificity and injury specificity on autophagy and cell death control. \textit{In vivo}, our results revealed that PQ-induced toxicity impacts the brain regions differentially, with the hippocampus being highly susceptible to PQ injury followed by the cortex. Moreover, our results showed that both dosages of spermidine protected against oxidative stress, neuronal damage, microtubule destabilization, and upregulated autophagy, in the hippocampus and cortex. However, the lower dose of spermidine gave a stronger protection, although, autophagy was here upregulated in a manner dependent on the dose utilized, with the higher dose of spermidine increasing LC3-II in the hippocampus, while a lower dose increased LC3-II in the cortex. Lastly, we have successfully established a CLEM protocol and showed with this method that spermidine in combination with BafA1 decreases the volume while increasing the surface area of autophagosomes in a concentration-dependent manner. 

\textit{Conclusion} - Our results indicate the importance of using multiple tools to assess autophagy and show that a concentration-dependent effect on autophagic flux modulation exists. In addition, we also provided evidence of the distinct, context-dependent protective roles of spermidine and rilmenidine in an \textit{in vitro} model of APP over-expression as well as the protective roles of spermidine \textit{in vitro} and \textit{in vivo} models of PQ-induced neuronal toxicity. These results suggest that administration of spermidine may represent a favourable therapeutic strategy for the treatment of AD. Further studies, using an \textit{in vivo} model over-expressing APP are warranted to further verify the protective effects of spermidine. 

\chapter{Opsomming}
\newpage

%-------------------------------------------------------------------------------
% Acknowledgements
%-------------------------------------------------------------------------------
\chapter{Acknowledgements}

\noindent
I would like to express my sincere gratitude to the following people for their help and support without whom this dissertation would not
have been possible. \\

\noindent
My supervisor, Prof Ben Loos, thank you so much for always believing in me even at times when I doubted myself. I will be forever grateful to you for all of the opportunities you have provided me with and for guiding me to become a better researcher. I cannot imagine having gone through my postgraduate degrees without you as my supervisor and mentor. It has been an immense honour to be one of your students. \\

\noindent
I would also like to thank the staff and the many students in Department of Physiological Sciences. A special thank you goes to the NRG group members for your friendship, critical discussions and your willingness to lend a helping hand. To Andr{\'e} du Toit, thank you for everything you have assisted me with as well as the much needed beers. Words cannot express my gratitude. To Jurgen, thank you for all your help. To Leah and Nicola, thank you for trusting me to co-supervise your Honours project; it definitely was a valuable experience during my PhD journey. To Bali and Theo, thank you for the help as well as the reoccuring Friday relaxation and chats at the Happy Oak. Lastly, also thank you to all the other staff members in the Department of Physiological Sciences, including Judy, Grazelda, Jonnifer, Annadie and Denzel. \\

\noindent
Thank you to Stellenbosch University’s Central Analytical Facility (CAF) staff, Lize Engelbrecht and Rozanne Adams for your assistance whenever needed.  Thank you to Reggie Williams for assisting with the brain tissue sections, and Noel Markgraff for the invaluable assistance with the GFP-LC3 mice. \\

\noindent
Thank you to Dr Lucy Collinson for giving me an opportunity to visit you at The Francis Crick in London. To Matt Russell, Christopher Peddie and Marie-Charlotte Domart: thank you for sharing your knowledge on Correlative Microscopy with me. \\

\noindent
To my family, my dad and my mom, I can honestly not express how thankful I am for all of your support. Thank you for your endless love, support and most importantly your sacrifices. None of this would have been possible for me to achieve without you. I am truly blessed to have both of you as parents.\\

\noindent
To my sisters, Zodumo, Tulani, Zandile, Nwabisa and Celi, thank you for being there when I needed a shoulder to cry on. Thank for your endless support.\\

\noindent
And lastly, to my husband, Dr Justin van Dijk, who keeps insisting to be called Dr, thank you for sharing my ambitions. Thank you for believing in me. Thank you for pushing me to be the best that I can be. And most importantly, thank you for your love and support.\\

%-------------------------------------------------------------------------------
% Output
%-------------------------------------------------------------------------------
\chapter{Academic output}

\subsection*{Refereed publications: International journals}
\begin{enumerate}
\item Mbizana, S., Hlalele, L., Pfukwa, R., Du Toit, A., \textbf{Lumkwana, D.}, Loos, B. \& Klumperman, B., 2018. Synthesis and Cell Interaction of Statistical l -Arginine-Glycine- l -Aspartic Acid Terpolypeptides. \textit{Biomacromolecules} 19: 3058–3066.
\item \textbf{Lumkwana, D.}, Du Toit, A., Kinnear, C. \& Loos, B., 2017. Autophagic flux control in neurodegeneration: Progress and precision targeting — Where do we stand? \textit{Progress in Neurobiology} 153: 65-85.
\item \textbf{Lumkwana, D.}, Botha, A., Samodien, E., Hanser, S., Lopes, J., 2017. Laminin, laminin-entactin and extracellular matrix are equally appropriate adhesive substrates for isolated adult rat cardiomyocyte culture and experimentation. \textit{Cell Adhesion \& Migration} 12: 503-511.
\item Huisamen, B., Hafver, T.L., \textbf{Lumkwana, D.} \& Lochner, A., 2016. The impact of chronic glycogen synthase kinase-3 inhibition on remodeling of normal and pre-diabetic rat hearts. \textit{Cardiovascular Drugs and Therapy} 30: 237–246.
\end{enumerate}

\subsection*{Refereed publications: Book chapters}
\begin{enumerate}
\item Ntsapi, C., \textbf{Lumkwana, D.}, Swart, C., Du Toit, A. \& Loos, B., 2017. New insights into autophagy dysfunction related to amyloid beta toxicity and neuropathology in Alzheimer’s Disease. In: International Review of Cell and Molecular Biology, Volume 336. London: Elsevier, pp. 321-361.
\item Ntsapi, C., Swart, C., \textbf{Lumkwana, D.} \& Loos, B., 2016. Autophagic flux failure in neurodegeneration: Identifying the defect and compensating flux offset. In: Gorbunov N.V. \& Schneider, M. (eds.). Autophagy in current trends in cellular physiology and pathology. London: InTech Open, pp. 157-176.
\end{enumerate}

\subsection*{Presentations at international conferences}
\begin{enumerate}
\item Poster presented at the Focus On Microscopy (FOM) conference. \textit{April 2019}. London, United Kingdom.
\item Paper presented at the Microscopy Society of Southern Africa (MMSA). \textit{December 2016}. Port Elizabeth, South Africa.
\item Poster presented at Physiology Society of Southern Africa (PSSA). \textit{September 2016}. Cape Town, South Africa.
\item Paper presented at the Microscopy Society of Southern Africa (MMSA). \textit{December 2014}. Stellenbosch, South Africa.
\item Paper presented at the Annual Academic Year Day. \textit{August 2013}. Stellenbosch, South Africa.
\end{enumerate}

\subsection*{Laboratory visits}
\begin{enumerate}
\item Electron Microscopy laboratory, The Francis Crick Institute. \textit{January 2018}. London, United Kingdom.
\item Department of Molecular Biology (Autophagy lab), University of T{\"u}bingen. \textit{March - August 2018}. T{\"u}bingen, Germany.
\end{enumerate}
\newpage

%-------------------------------------------------------------------------------
% Table and Lists of Content
%-------------------------------------------------------------------------------
\setcounter{secnumdepth}{4}
\setcounter{tocdepth}{4}

\tableofcontents
\newpage

\listoffigures
\newpage

\listoftables
\newpage

\chapter*{Nomenclature}
\addcontentsline{toc}{chapter}{Nomenclature}
\subsection*{Abbreviations}
\begin{itemize}[leftmargin=*,wide=0pt,]
\DTLforeach*{acronyms}{\thisAcronym=Acronym,\thisDesc=Description}
   {\item[] \textbf{\thisAcronym} \thisDesc}
\end{itemize}

\subsection*{Units of Measurements}
\begin{itemize}[leftmargin=*,wide=0pt,]
\DTLforeach*{units}{\thisUnit=Unit,\thisDesc=Description}
   {\item[] \textbf{\thisUnit} \thisDesc}
\end{itemize}
