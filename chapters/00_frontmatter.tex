%-------------------------------------------------------------------------------
% Titlepage
%-------------------------------------------------------------------------------
\WaterMark[0.15\paperwidth]{UScrest-WM} 
\TitlePage

%-------------------------------------------------------------------------------
% Declaration page
%-------------------------------------------------------------------------------
\chapter{Declaration}
By submitting this dissertation electronically, I declare that the entirety of the work contained therein is my own, original work, that I am the sole author thereof (save to the extent explicitly otherwise stated), that reproduction and publication thereof by \mbox{Stellenbosch University} will not infringe any third party rights and that I have not \mbox{previously} in its entirety or in part submitted it for obtaining any qualification. \\ \\

\noindent
\begin{tabular}{@{}ll}
\textbf{Date:} & November 7, 2019 \\ 
\textbf{Name:} & Dumisile Lumkwana \\ 
\end{tabular}

\vspace*{\fill}
\begin{center}
Copyright $\textcopyright$ Stellenbosch University \\
All rights reserved.
\end{center}

%-------------------------------------------------------------------------------
% Abstract
%-------------------------------------------------------------------------------
\chapter{Abstract}

\textit{Introduction} - Alzheimer’s disease is a progressive neurodegenerative disorder characterized by multiple cognitive deficits. The neuropathology of AD is underpinned by two molecular hallmarks; intracellular protein aggregates known as neurofibrillary tangles (NFTs), composed of hyper-phosphorylated Tau and extracellular amyloid beta (A$\beta$) plaques, composed of A$\beta$ peptides derived from the amyloid precursor protein (APP). Both of these occur as a result of an imbalance in proteostasis, leading to neuronal toxicity. Although, we have advanced our understanding of the molecular machinery that regulates the rate of protein degradation through autophagy at basal levels and increasingly so the many aspects of its dysfunction in AD, the deviation of autophagic activity from basal levels and its change during disease pathogenesis in neuronal tissue remains largely unclear. Over the recent years, substantial progress has been made in modulating autophagy using pharmacological agents \textit{in vitro} and \textit{in vivo} and mounting evidence points towards autophagy modulation using pharmacological agents as one of the major therapeutic strategies for neurodegenerative diseases. Although spermidine and rilmenidine both enhance autophagy, the relationship between autophagy activity, the extent of protein clearance and cell death onset remains unclear. Moreover, the impact of their concentration on autophagic flux and subsequent protein clearance as well as neuronal toxicity is unclear. Therefore, this study aimed to unravel the impact of using a high and low concentration of spermidine and rilmenidine on autophagic flux, neuronal toxicity and protein clearance using distinct neuronal injury model systems. 

\textit{Methods} - Murine hypothalamus-derived GT1-7 neuronal cells and the mouse neuroblastoma (N2a) cell line stably expressing Swedish  double mutation APP695 (Swe) associated with AD pathology were used. GT 1-7 cells transfected with mRFP-GFP-LC3 and GFP-LC3-RFP-LC3$\Delta$G were treated with a low and high concentration of spermidine and rilmenidine in the absence and presence of saturating concentrations of bafilomycin, after which the autophagic flux profile was characterized, assessing cellular viability, autophagosome pool, autolysosome pool, autophagosome flux, transition time, p62 puncta and autophagic vacuoles. Cellular viability assays, western blotting, fluorescence microscopy, transmission electron microscopy and correlative light and electron microscopy linked to quantitative morphometric analysis was performed. In addition, potential protective effects of a low and high concentration of spermidine and rilmenidine were assessed in a paraquat (PQ)-induced neuronal toxicity model and in an APP over-expression model. Cellular viability, ROS damage, cell death onset and autophagic activity were assessed in the PQ-induced toxicity model, while in the APP model, cellular viability and protein clusters were assessed and quantified using cellular viability assays and single molecule imaging, i.e. d-STORM. Moreover, CLEM method was developed and implemented to morphometrically assess the effects of a low and high concentration of spermidine on the localization of autophagosomes in 3 dimensions. Finally, potential protective effects of spermidine at two distinct concentrations were assessed using transgenic mice expressing GFP-LC3, treated with PQ to induce neuronal toxicity associated with neurodegeneration.

\textit{Results} - Our results indicate a concentration-dependent effect of spermidine and rilmenidine on autophagic flux with the detected change in flux depending on the specificity and sensitivity of the method employed.  In addition, in the \textit{in vitro} model of PQ-induced toxicity, our results revealed that spermidine at a low concentration and not a high concentration protected against cell toxicity, ROS damage, cell death as well as microtubule destabilization. To our surprise, both concentrations of rilmenidine failed to protect against ROS damage and cell death and also failed to robustly upregulate autophagy in the same model.  Moreover, in the \textit{in vitro} model of neuronal toxicity induced by APP over-expression, our results showed that both concentrations of spermidine and rilmenidine protected against cytotoxicity. Here, spermidine at a low concentration effectively cleared APP clusters and reduced their size, especially after 48 h of APP over-expression, while rilmenidine, reduced the number and size of APP clusters in a concentration-dependent manner at the same time point.  
Taken together, the \textit{in vitro} results reveal a concentration-dependent effect, that is cell type and injury specific, impacting autophagy and cell death control. Moreover, we have successfully implemented a 3D CLEM protocol and revealed that spermidine, in combination with BafA1 decreases autophagosome volume while increasing their surface area in a concentration-dependent manner. Lastly, \textit{in vivo}, our results reveal that PQ-induced toxicity impacts the brain regions differentially, with the hippocampus being highly susceptible to PQ-induced injury followed by the cortex. Moreover, our results show that both dosages of spermidine robustly protected against oxidative stress, neuronal damage, microtubule destabilization, and upregulated autophagy, in the hippocampus and cortex. However, the low dose of spermidine resulted in more enhanced protection, although, autophagy was here region-specifically upregulated in a manner dependent on the dose utilized, with the higher dose of spermidine increasing LC3-II in the hippocampus, and the lower dose increasing LC3-II in the cortex. 

\textit{Conclusion} - Our results indicate the critical importance of using multiple tools to assess autophagy and show that a concentration-dependent effect of the two selected drugs on autophagic flux exists. In addition, we provide evidence of the distinct, context-dependent protective roles of spermidine and rilmenidine in an \textit{in vitro} model of APP over-expression as well as the protective roles of spermidine using \textit{in vitro} and \textit{in vivo} models of PQ-induced neuronal toxicity. These results suggest that firstly administration of spermidine may represent a favourable therapeutic strategy for the treatment of AD and secondly, concentration / flux screening may be more critical for optimal autophagy control than previously thought. Future studies, using an \textit{in vivo} model over-expressing APP are warranted to further verify the protective effects of spermidine, to foster clinical translation and therapeutic intervention in neurodegenerative disorders. 

\chapter{Opsomming}
\newpage

%-------------------------------------------------------------------------------
% Acknowledgements
%-------------------------------------------------------------------------------
\chapter{Acknowledgements}

\noindent
I would like to express my sincere gratitude to the following people for their help and support without whom this dissertation would not
have been possible. \\

\noindent
My supervisor, Prof Ben Loos, thank you so much for always believing in me even at times when I doubted myself. I will be forever grateful to you for all of the opportunities you have provided me with and for guiding me to become a better researcher. I cannot imagine having gone through my PhD studies without you as my supervisor and mentor. It has been an immense honour to be one of your students. \\

\noindent
My co-supervisor, Prof Craig Kinnear, thank you so much for your assistance with all plasmid expansion related work. Without your help, the work would not have been possible.\\

\noindent
I would also like to thank the staff and students in Department of Physiological Sciences. A special thank you goes to the NRG group members for your friendship, critical discussions and your willingness to lend a helping hand. To Andr{\'e} du Toit, thank you for everything you have assisted me with including the relaxing chats at the Oak. Words cannot express my gratitude. To Jurgen, thank you for all your help, you are the best. To Leah and Nicola, thank you for trusting me to co-supervise your Honours project; it definitely was a valuable experience during my PhD journey. To Bali, Theo and Danzil, thank you for the help as well as the reoccuring Friday relaxation and chats at the Happy Oak. Lastly, also thank you to all the other staff members in the Department of Physiological Sciences, including Judy, Grazelda, Jonnifer, Annadie. \\

\noindent
Thank you to Lize Engelbrecht, Lydia Joubert and Madelaine Frazenburg at the Central Analytical Facility (CAF) for all your assistance, particularly with the sectioning and imaging of the correlative work. We made a great team. Thank you to Reggie Williams for assisting with the brain tissue sections, and Noel Markgraff for the assistance with the GFP-LC3 mice. \\

\noindent
Thank you to Dr Lucy Collinson for giving me an opportunity to visit her at The Francis Crick in London. To Matt Russell, Christopher Peddie and Marie-Charlotte Domart: thank you for sharing your knowledge on Correlative Light and Electron Microscopy with me. More importantly, thank you to all of you for the beautiful FIB-SEM images and for being thee best collaborators. The CLEM work would not have been possible without your guidance and your willingness to help. \\

\noindent
Thank you to the post graduate international office at Stellenbosch University and University of T{\"u}bingen for granting me the opportunity to go for a lab visit in Germany. To Prof Tassula Proikas-Cezanne; thank you for opening your lab and for welcoming me as one of your group members.\\

\noindent
To Madelaine, thank you for being more than a colleague to me. Thank you for all your support and thank you for pushing me to finish. Most importantly, thank you for your friendship and guidance. You simply are the best.\\

\noindent
To Rozanne, I honestly cannot express how thankful I am for having you as a friend, a colleague, and partner throughout my PhD journey. Thank you for allowing me to drag you into my night shifts in the lab running western blots. You have made this work-PhD life balance a whole lot easier to handle and I could not have wished for anyone else to walk through this journey with me.\\

\noindent
To my sisters from another mother, Zandile, Nwabisa and Celi, thank you for being there when I needed a shoulder to cry on. Thank for your endless support. Thank you for going through the PhD journey before I did, your advices were of tremendous help and I could not have done it without you. And most importantly, thank you for your friendship.\\

\noindent
To my sisters Zodumo and Tulani, my brother Sango and my in-laws, Annemieke, Rik, Sean, Sarah and Lenore, thank you for all your endless support.\\

\noindent
To my kids Sihle, Amzolele, Amvelo and Yolatha, thank you for being the bundle of  joy in my life. I love you.\\

\noindent
Ku Tata uMvulane no mama uMamncotshe, enkosi ngenxaso enindinikeze yona, amagama akanokuze acacise umbulelo endinawo kuni. Enkosi ngondithanda. Enkosi ngenxaso yenu. Ekubalulekileyo ngaphezu kwazo zonke, enkosi ngokuncama yonke into kuba ninomqweno wokufezekisa amaphupha am. Bendingenoze ndiwafezekise amaphupha am ngaphandle kwenxaso yenu. This one is for you guys.\\

\noindent
And lastly, to my husband, Dr Justin van Dijk, who keeps insisting that I call him Dr, thank you for sharing my ambitions. Thank you for believing in me. Thank you for pushing me to be the best that I can be. Thank you for handling all the cooking. And most importantly, thank you for your love and support. Ik houd van je.

%-------------------------------------------------------------------------------
% Output
%-------------------------------------------------------------------------------
\chapter{Publications and conference presentations}

\subsection*{Refereed publications}
\begin{enumerate}
\item \textbf{Lumkwana, D.}, Du Toit, A., Kinnear, C. \& Loos, B., 2017. Autophagic flux control in neurodegeneration: Progress and precision targeting — Where do we stand? \textit{Progress in Neurobiology} 153: 65-85.
\item Ntsapi, C., \textbf{Lumkwana, D.}, Swart, C., Du Toit, A. \& Loos, B., 2017. New insights into autophagy dysfunction related to amyloid beta toxicity and neuropathology in Alzheimer’s Disease. In: International Review of Cell and Molecular Biology, Volume 336. London: Elsevier, pp. 321-361.
\item Ntsapi, C., Swart, C., \textbf{Lumkwana, D.} \& Loos, B., 2016. Autophagic flux failure in neurodegeneration: Identifying the defect and compensating flux offset. In: Gorbunov N.V. \& Schneider, M. (eds.). Autophagy in current trends in cellular physiology and pathology. London: InTech Open, pp. 157-176.
\end{enumerate}

\subsection*{Presentations at international conferences}
\begin{enumerate}
\item Poster presented at the Focus On Microscopy (FOM) conference. \textit{Assessment of amyloid precursor protein (APP) and tubulin network using stochastic optical reconstruction microscopy (STORM) in an in vitro model of Alzheimer's disease.} April 2019. London, United Kingdom.
\item Paper presented at the Microscopy Society of Southern Africa (MMSA). \textit{Investigating the role of Rilmenidine and Spermidine in an in vitro model of Alzheimer's disease.} December 2016. Port Elizabeth, South Africa.
\item Poster presented at Physiology Society of Southern Africa (PSSA). \textit{The role of Rilmenidine and Spermidine in autophagic flux and cell death in a model of paraquat-induced neuronal toxicity.} September 2016. Cape Town, South Africa.
\end{enumerate}

\subsection*{Laboratory visits}
\begin{enumerate}
\item Electron Microscopy laboratory, The Francis Crick Institute. \textit{January 2018}. London, United Kingdom.
\item Department of Molecular Biology (Autophagy lab), University of T{\"u}bingen. \textit{March - August 2018}. T{\"u}bingen, Germany.
\end{enumerate}
\newpage

%-------------------------------------------------------------------------------
% Table and Lists of Content
%-------------------------------------------------------------------------------
\setcounter{secnumdepth}{4}
\setcounter{tocdepth}{4}

\tableofcontents
\newpage

\listoffigures
\newpage

\listoftables
\newpage

\chapter*{Nomenclature}
\addcontentsline{toc}{chapter}{Nomenclature}
\subsection*{Abbreviations}
\begin{itemize}[leftmargin=*,wide=0pt,]
\DTLforeach*{acronyms}{\thisAcronym=Acronym,\thisDesc=Description}
   {\item[] \textbf{\thisAcronym} \thisDesc}
\end{itemize}

\subsection*{Units of Measurements}
\begin{itemize}[leftmargin=*,wide=0pt,]
\DTLforeach*{units}{\thisUnit=Unit,\thisDesc=Description}
   {\item[] \textbf{\thisUnit} \thisDesc}
\end{itemize}
