\chapter{Literature review}
\section{Introduction}

Alzheimer’s disease (AD) is the most common neurodegenerative disease and a leading cause of dementia in the elderly \citep{andrieu2015}. This disease is characterized by a progressive loss of synapses and neurons in certain brain regions (e.g. cerebral cortex and hippocampus), leading to impaired memory and deterioration of cognitive functions \citep{dekosky1990,scheff2006,zare-shahabadi2015}, thus necessitating full-time medical care \citep{prince2013}. Currently the disease is incurable. Although a lot of efforts have been directed towards developing AD disease modifying therapies, current treatment strategies are aimed at ameliorating disease symptoms alone \citep{anand2014,disanto2013}. Age is the most prominent risk factor for AD and about 44 million people are currently affected globally. While only 5\% of individuals over the age of 65 years are affected by AD, the prevalence doubles with every 5 years of increasing age \citep{pimenova2018,qiu2009}. Given the rapidly aging population in first-and third- world countries, the prevalence of AD is predicted to rise to 81 million by 2025 \citep{AlzheimersAssociation2014,Ferri2005}. Moreover, it has been estimated that by 2050, 22\% of the global population will be over the age of 60, with the majority residing in the developing nations \citep{Annear2015,Paddick2013}. In South Africa, there is limited statistics on the prevalence of AD. According to the census conducted in 2011, there are approximately 2.2 million people with some form of dementia. Very little is known about the prevalence of dementia and how it impacts on older adults residing in low and middle classes, as well as in the rural areas, where most of the adult population reside. Despite of this, it was reported that about 79\% of patients were being cared for by family members \citep{Kalula2010}. In the continued absence of effective therapeutic strategies to either delay or slow down the disease progression, AD will pose a tremendously burden on the healthcare and social services. 

AD is a multifactorial disease that is highly complex, with genetic as well as and environmental causes \citep{Dorszewska2016}.This disease is classified into two subsets. The first being the early-onset Familial Alzheimer’s disease (FAD, onset $<$ 65), which contributes to the least cases of AD (1-5\%) with strong genetic association \citep{Musiek2015,Reitz2014,Swerdlow2007}. The second being sporadic late-onset Alzheimer’s disease (LOAD, onset $\geq$ 65) which contributes to the majority of all AD cases ($>$ 95\%), \citep{Musiek2015,Reitz2014,Swerdlow2007}, with an unclear cause \citep{Dorszewska2016,pimenova2018}. Certain genes such as amyloid precursor protein (APP), Presenilin 1 (PSEN1) and Presenilin 2  (PSEN2) are responsible for occurrence of FAD, while APOE gene is responsible for LOAD \citep{Dorszewska2016}.

AD pathology occurs in 5 stages; the pre-symptomatic, Mild Cognitive Impairment (MCI), mild AD, moderate AD and severe AD \citep{Caldwell2015}. The first two stages encompasses a prodromal stage, which, in the majority of cases, precedes symptom onset by several years (at least 20 - 30 years) \citep{Caldwell2015,Caselli2013,Penn1993}. During this time, pathological molecular changes occur inside the brain and are present as two molecular hallmarks which occur as a result of perturbations in cellular proteostasis. The senile plaques, which are extracellular deposits of amyloid beta ($A\beta$)
peptides, and intraneuronal neurofibrillary tangles (NFTs), which are somatic inclusions of hyperphosphorylated, microtubule-associated protein tau, respectively \citep{Mattson2008} Figure \ref{fig:10_ad_model}. To better understand the multifactorial pathophysiology of AD, several hypothesis have been put forward, including the amyloid cascade hypothesis (ACH), the cholinergic and tau hypothesis, as well as inflammation \citep{Kurz2011}; however, many molecular aspects and their dynamic changes during disease progression remain unclear. The ACH, the most widely accepted mechanistic hypothesis for AD, posits that an imbalance between the production and clearance of ($A\beta$) peptides is a very early, often initiating factor in disease onset \citep{Hardy2009,Hardy1992}.

\begin{figure}[h!]
  \includegraphics[width=\linewidth]{figures/10_ad_model}
  \caption{Hypothetical model of AD clinical disease stages in relation to biomarkers.  $A\beta$ levels are deposited first and are the initial trigger AD while tau mediated neuronal injury manifests only late in the disease progression, ultimately leading neuronal death \citep{Petrella2013} .}
  \label{fig:10_ad_model}
\end{figure}

Proteolytic systems; the ubiquitin-proteasome system (UPS) and the lysyosomal systems [the autophagy-lysosomal pathway (ALP) and the endocytic- lysosomal pathway (ELP)] are responsible for the degradation of mis-folded or aggregated proteins in order to maintain cellular homeostasis. For example, the UPS targets and degrades short-lived proteins in the cytoplasm and nucleus, while the lysosomal system removes primarily long lived cytoplasmic proteins and damaged organelles \citep{Ravikumar2003,Rubinsztein2005}. A large body of evidence implicates these proteolytic systems in the AD pathophysiology. Dysfunction of these system has been well documented in AD pathophysiology and although extracellular $A\beta$ plaques and intraneuronal NFTs are defining hallmarks of AD neuropathology, a growing body of literature suggests that deficits in the autophagy–lysosomal pathway are likely to precede the formation of these pathological hallmarks \citep{Cataldo2000,Nixon2011,Perez2015,zare-shahabadi2015}, suggesting a potential causality. More recently, a systems biology study highlighted the pivotal role of dysregulated autophagy in neurodegenerative diseases, where toxic protein aggregates and damaged organelles accumulate within specific types of neurons and lead to neuronal dysfunction and ultimately, demise \citep{Caberlotto2014}.

Although we have advanced our understanding of the molecular machinery that regulates the rate of protein degradation through autophagy at basal levels and the many aspects of its dysfunction in AD, the deviation of autophagic activity from basal levels and its change during disease pathogenesis in neuronal tissue remains largely unclear. Over the recent years, we have made substantial progress in modulating autophagy using pharmacological agents \citep{Berger2006,Hebron2013,Ravikumar2002,Ravikumar2004,Rose2010} or lifestyle interventions \citep{Alirezaei2010,Kuma2004,Mizushima2004a,Scott2004} in vitro and in vivo; yet many questions regarding an effective implementation of autophagy control remain unanswered. Therefore understanding the deviation of autophagic activity from basal levels and its change during disease pathogenesis as well as targeting or modulating autophagy activity precisely with the aim to restore autophagic flux may drive the development of better targeted therapeutic interventions that will slow down the disease progression and ultimately curing the disease instead of treating the symptoms. In this review, we start by describing neuronal metabolism and their metabolic profile and move to the involvement of reactive oxygen species in neurodegeneration, focusing on the role of oxidative stress and mitochondrial dysfunction in the pathogenesis of Alzheimer’s disease. We then turn our focus to amyloid beta biogenesis and its pathology in AD.  We highlight the recent advances in the use super resolution techniques in molecular biology. We introduce autophagy and its molecular machinery and discuss its efficiency and essential function in neuronal cells, followed by its role in neurodegenerative diseases. We discuss how autophagic flux may differ in brain regions, and how the deviation of flux may relate to the state of protein aggregation in pathology. Furthermore, we review studies that localize the defect in autophagy in AD and assess the relationship between autophagic flux and neuronal cell death. Finally, we indicate the importance of accurately measuring and targeting autophagy and provide an overview of how autophagy may be modulated for therapeutic purposes in various model systems aimed at restoring autophagic flux. 

\section{Neuronal metabolism}
A brain is made up of two cells types, neurons and astrocytes, which are highly interconnected and form functional networks through their spatial organization. The co-dependency of these cells types is reflected by their metabolic profiles where different yet complimentary pathways are used \citep{Belanger2011,Schonfeld2013} The brain has a high energy demand. This is evident by the fact that 20\% of oxygen and 25\% of the glucose utilized by the human body is dedicated to only cerebral function, yet the brain encompasses only 2\% of the total body mass \citep{Belanger2011}. This means, uninterrupted supply of energy substrates from the circulation is required to meet the energy demands. Glucose is the main substrate of energy in the brain 
\citep{Dienel2012,Pellerin2012} , however, other energy substrates such as lactate, pyruvate, glutamate, and glutamine can be utilized \citep{Zielke2009}.

Glucose is delivered into a cell via specific glucose transporters (GLUTs) and once inside, it is phosphorylated by an enzyme called hexokinase (HK) to generate glucose-6-phosphate (glucose-6P) \citep{Belanger2011,Herrero-Mendez2009}. The latter can enter three main metabolic pathways; glycolysis, pentose phosphate pathway (PPP), and glycogenesis in order to produce adenosine triphosphate (ATP). In glycolysis, glucose-6P is metabolized to produce pyruvate, 2 ATP molecules and NADH \citep{Belanger2011}. Pyruvate can be metabolised further in the tricarboxylic acid (TCA) cycle and oxidative phosphorylation (OxPhos) in the mitochondria using oxygen to generate 30-34 ATP molecules and CO2 or it can be reduced to lactate by lactate dehydrogenase (LDH) in the cytosol \citep{Belanger2011}. In PPP, glucose-6P is metabolized to generate NADPH, while it is stored as glycogen in glycogenesis, with the latter occurring in astrocytes \citep{Belanger2011} (Figure \ref{fig:10_glucose_metabolism}). More importantly, astrocytes and neurons have the ability to oxidized glucose and/or lactate \citep{Zielke2009}.

\begin{figure}[h!]
  \includegraphics[width=\linewidth]{figures/10_glucose_metabolism}
  \caption{Glucose metabolism. A schematic diagram showing three main pathways; glycolysis, pentose phosphate pathway, and glycogenesis where glucose can be metabolized \citep{Belanger2011} .}
  \label{fig:10_glucose_metabolism}
\end{figure}

\section{Metabolic profile of neurons}
Neurons are post-mitotic, highly differentiated cells that are characterized by high energy demands. This is because of their high levels of protein synthesis, which consumes high amount of ATP within the mammalian cells \citep{Buttgereit1995}. Neurons depend almost exclusively on the energy produced through OxPhos (30 - 34 ATP molecules) compared to glycolysis (2 ATP) in order to meet their high energy demand needed to perform cellular functions, such as synaptic plasticity and neurotransmitter synthesis \citep{Cenini2019,Mattson2008,Schonfeld2013}. Mounting evidence demonstrate that neurons are capable of using lactate as an energy substrate \citep{Boumezbeur2010,Bouzier2000,Serres2005} and prefer lactate over glucose when both substrates are available \citep{Bouzier-Sore2006,Itoh2003}. Thus the specific characteristics of neurons are probably underly their distinct metabolic profile. For example, glycolytic enzyme 6-phosphofructose-2-kinase/fructose-2, 6-bisphosphatase-3 (PFKFB3) is highly expressed in astrocytes, but virtually absent in neurons because of a constant proteasomal degradation \citep{Almeida2004,Herrero-Mendez2009}. Because of this, neurons unlike astrocytes display a lower glycolytic rate, and thus cannot upregulate this pathway in response to cellular stress \citep{Almeida2004,Herrero-Mendez2009}. Indeed, a previous study showed that upregulation of glycolysis via PFKFB3 in neurons is in fact detrimental, resulting to oxidative stress and apoptosis \citep{Herrero-Mendez2009}. In this study, it was though that the upregulation of glycolysis occurs at a cost of PPP metabolism that is needed to produce NADPH, an antioxidant vital for maintaining cellular redox state \citep{Herrero-Mendez2009}. Evidently, it has been shown that neurons have low NADPH compared to astrocytes \citep{Ben-Yoseph1996,Garcia-Nogales2003}, and since antioxidant system and the related enzymes are important for maintaining neuronal integrity and survival by keeping the levels of reactive oxygen species (ROS) relatively low \citep{Cenini2019}, it is not surprising that neurons are vulnerable to oxidative damage, implicated in neurodegeneration

\section{ROS and its role in neurodegeneration}
\section{Neuronal metabolism}
A brain is made up of two cells types, neurons and astrocytes, which are highly interconnected and form functional networks through their spatial organization. The co-dependency of these cells types is reflected by their metabolic profiles where different yet complimentary pathways are used \citep{Belanger2011,Schonfeld2013} The brain has a high energy demand. This is evident by the fact that 20\% of oxygen and 25\% of the glucose utilized by the human body is dedicated to only cerebral function, yet the brain encompasses only 2\% of the total body mass \citep{Belanger2011}. This means, uninterrupted supply of energy substrates from the circulation is required to meet the energy demands. Glucose is the main substrate of energy in the brain 
\citep{Dienel2012,Pellerin2012} , however, other energy substrates such as lactate, pyruvate, glutamate, and glutamine can be utilized \citep{Zielke2009}.

Glucose is delivered into a cell via specific glucose transporters (GLUTs) and once inside, it is phosphorylated by an enzyme called hexokinase (HK) to generate glucose-6-phosphate (glucose-6P) \citep{Belanger2011,Herrero-Mendez2009}. The latter can enter three main metabolic pathways; glycolysis, pentose phosphate pathway (PPP), and glycogenesis in order to produce adenosine triphosphate (ATP). In glycolysis, glucose-6P is metabolized to produce pyruvate, 2 ATP molecules and NADH \citep{Belanger2011}. Pyruvate can be metabolised further in the tricarboxylic acid (TCA) cycle and oxidative phosphorylation (OxPhos) in the mitochondria using oxygen to generate 30-34 ATP molecules and CO2 or it can be reduced to lactate by lactate dehydrogenase (LDH) in the cytosol \citep{Belanger2011}. In PPP, glucose-6P is metabolized to generate NADPH, while it is stored as glycogen in glycogenesis, with the latter occurring in astrocytes \citep{Belanger2011} (Figure \ref{fig:10_glucose_metabolism}). More importantly, astrocytes and neurons have the ability to oxidized glucose and/or lactate \citep{Zielke2009}.

\begin{figure}[h!]
  \includegraphics[width=\linewidth]{figures/10_glucose_metabolism}
  \caption{Glucose metabolism. A schematic diagram showing three main pathways; glycolysis, pentose phosphate pathway, and glycogenesis where glucose can be metabolized \citep{Belanger2011} .}
  \label{fig:10_glucose_metabolism}
\end{figure}

\section{Metabolic profile of neurons}
Neurons are post-mitotic, highly differentiated cells that are characterized by high energy demands. This is because of their high levels of protein synthesis, which consumes high amount of ATP within the mammalian cells \citep{Buttgereit1995}. Neurons depend almost exclusively on the energy produced through OxPhos (30 - 34 ATP molecules) compared to glycolysis (2 ATP) in order to meet their high energy demand needed to perform cellular functions, such as synaptic plasticity and neurotransmitter synthesis \citep{Cenini2019,Mattson2008,Schonfeld2013}. Mounting evidence demonstrate that neurons are capable of using lactate as an energy substrate \citep{Boumezbeur2010,Bouzier2000,Serres2005} and prefer lactate over glucose when both substrates are available \citep{Bouzier-Sore2006,Itoh2003}. Thus the specific characteristics of neurons are probably underly their distinct metabolic profile. For example, glycolytic enzyme 6-phosphofructose-2-kinase/fructose-2, 6-bisphosphatase-3 (PFKFB3) is highly expressed in astrocytes, but virtually absent in neurons because of a constant proteasomal degradation \citep{Almeida2004,Herrero-Mendez2009}. Because of this, neurons unlike astrocytes display a lower glycolytic rate, and thus cannot upregulate this pathway in response to cellular stress \citep{Almeida2004,Herrero-Mendez2009}. Indeed, a previous study showed that upregulation of glycolysis via PFKFB3 in neurons is in fact detrimental, resulting to oxidative stress and apoptosis \citep{Herrero-Mendez2009}. In this study, it was though that the upregulation of glycolysis occurs at a cost of PPP metabolism that is needed to produce NADPH, an antioxidant vital for maintaining cellular redox state \citep{Herrero-Mendez2009}. Evidently, it has been shown that neurons have low NADPH compared to astrocytes \citep{Ben-Yoseph1996,Garcia-Nogales2003}, and since antioxidant system and the related enzymes are important for maintaining neuronal integrity and survival by keeping the levels of reactive oxygen species (ROS) relatively low \citep{Cenini2019}, it is not surprising that neurons are vulnerable to oxidative damage, implicated in neurodegeneration

\section{ROS and its role in neurodegeneration}
ROS are a group of reactive molecules that produced naturally in biological systems as part of normal cellular metabolism and are important in maintaining cellular homeostasis \citep{Cenini2019}. These include superoxide (O\textsubscript{2}\textsuperscript{-}), hydroxyl radical ($\cdot$OH), hydroxyl ion (OH\textsuperscript{-}) and hydrogen peroxide (H\textsubscript{2}O\textsubscript{2}), all of which are generated from oxygen. O\textsubscript{2}\textsuperscript{-} is generated from O2 in the mitochondria as a result of the respiratory chain complex or NADPH oxidase and can be converted by superoxide dismutase (SOD) enzyme to produce H\textsubscript{2}0\textsubscript{2}. The later  in turn can be converted to other types of ROS, for example OH and OH\textsuperscript{-} \citep{Kim2015a}, with $\cdot$OH being the most reactive ROS responsible for cytotoxicity \citep{Bolisetty2013}.

Under physiological conditions, ROS are maintained at relatively low levels by antioxidant system \citep{Dasuri2013,Gandhi2012}, and are involved cellular processes such as inflammation, immune response, cell survival, synaptic plasticity, learning, and memory \citep{Cenini2019,Kishida2007,Liu2017}. However, increased ROS production can be harmful because of its ability to oxidise nucleic acids, protein and lipids \citep{Wang2014}. Increased ROS accumulation has been implicated in oxidative stress, mitochondrial dysfunction and in gliosis. However, role of ROS in gliosis is poorly understood and only one study provides evidence \citep{Kishida2007}.

Excessive accumulation of ROS due failure of antioxidant system or increased ROS production can result in oxidative stress, an imbalance between rate of ROS production and clearance \citep{Wang2014}. High levels of oxidative stress have been implicated in aging and in the pathogenesis of various neurodegenerative diseases \citep{Bonda2010,Cenini2019,Liu2017,Shibata2008}. Neurons are mostly susceptible to oxidative stress and damage due to its high oxygen consumption, high energy demand, low antioxidant defenses as well as high abundance of polyunsaturated fatty acid which are susceptible to lipid peroxidation \citep{Cobley2018}. Thus, it is not surprising that ROS induced oxidative damage is widely reported in AD. In addition, since mitochondria are the major source of ROS production and the main target of oxidative stress, progressive mitochondrial dysfunction has also been implicated in the pathogenesis of AD \citep{Swerdlow2007}. The involvement of oxidative stress and mitochondrial damage in AD has been demonstrated in different models and are described below. 

\subsection{Evidence of oxidative stress in AD}
Oxidative damage is one of the earliest events in AD \citep{Nunomura2001}. This is supported by several studies that demonstrated elevated levels of oxidative stress in patients with mild cognitive impairment (MCI) \citep{Ansari2010,Pratico2004,Williams2006}. In addition, antioxidants including uric acid, vitamin C and E as well as antioxidant enzymes such as superoxide dismutase (SOD) were found to be decreased in MCI patients \citep{Rinaldi2003,Torres2011}. Increased oxidative stress has also been implicated in AD. Excessive production of ROS is thought to play an essential role in the accumulation and deposition of $A\beta$ peptides \citep{Bonda2010}. 

\citet{Ferreiro2008) reported that $A\beta$ plaques depleted Ca\textsuperscript{2+} ions storage in the endoplasmic reticulum (ER), leading to in cytosolic Ca\textsuperscript{2+} overload, which resulted in the reduction of endogenous GSH levels and ROS accumulation of ROS. 

In addition, increased H\textsubscript{2}O\textsubscript{2} levels and increased peroxidation of proteins and lipids were shown in transgenic mouse expressing APP/PS-1, suggesting that $A\beta$ may exacerbate oxidative stress in AD \citep{Matsuoka2001,Zhao2013}. In addition, products of lipid peroxidation such as 4-hydroxynonal (4HNE), malondialdehyde (MDA), and 2-propenal (acrolein) have been found to be elevated in multiple studies performed in patient with AD \citep{Wang2014,Zhao2013}. For example, significantly increased levels of 4HNE were reported in the hippocampus \citep{Lovell1995,Markesbery1998,Montine1998}, parahippocampal gyrus \citep{Markesberry1998}, entorhinal and temporal cortex \citep{Montine1998}, amygdala \citep{Lovell1995,Markesberry1998}, ventricular fluid \citep{Lovell1997}, and plasma \citep{McGrath2001} in AD patients versus control subjects of the same age. Similar findings were observed with MDA and acrolein in AD patients. MDA was found to be increased in the hippocampus \citep{Lovell1995}, pyriform cortex \citep{Lovell1995}, temporal cortex \citep{Marcus1998,Palmer1994} and occipital cortices \citep{Miranda2000}, while elevated levels of acrolein were reported in the hippocampus/parahippocampal gyrus \citep{Bradley2010,Calingasan1999,Lovell2001,Williams2006}, amygdala \citep{Lovell2001},superior and middle temporal gyri 
\citep{Bradley2010,Williams2006}, and cerebellum \citep{Bradley2010,Williams2006}. Altogether, these studies demonstrate the involvement of oxidative stress in AD
