\chapter{Materials and methods}
\section{Reagents and consumables}
\subsection{Cell culture reagents}
Cell culture flasks; T175 (175 cm\textsuperscript{2}) (\#431080), T75 (75 cm\textsuperscript{2}) (\#708003) and T25 (25 cm\textsuperscript{2}) (\#707003), cell culture plates; 6 well (703001), 48 well (\#748001) and serological pipettes; 5 ml (\#326001), 10 ml (\#327001), 25 ml (\#328001) were purchased from Whitehead Scientific. Cell culture medium, Dulbecco’s Modified Eagles Medium (DMEM) (\#41965-039), Opti-MEM Reduced Serum media (\#319850470), Penicillin-Streptomycin (PenStrep) (\#151140-122) and 0.25\% Trypsin-EDTA (\#25200-072) were obtained from Thermo Fischer Scientific, Foetal Bovine Serum (FBS) (\#FBS-GI-12A) from Biocom-Biotech. Tissue culture falcon tubes, 15 ml (\#50015) ml and 50 ml (\#50050), pipette tips and Eppendorf tubes were purchased from B\&M Scientific. Glass bottom Gridded dishes (\#P35G-1.5-14-CGRD) were obtained from MatTek Corporation, USA.

\subsection{Experimental and treatment reagents}
Spermidine (\#S0266), Paraquat dichloride x-hydrate pestnal (\#36541), 2-Cysteamine (\#M9768), WST-1 Cell Proliferative Reagent (\#11644807001) and butyric acid (\#B103500) were purchased from Sigma Aldrich. Rilmenidine hemifumarate (\#0790) was obtained from Whitehead Scientific and Bafilomycin A1 (\#B0025) from Lkt$^{\tiny{\textregistered}}$ Labs. 

\section{Antibodies, Plasmids Constructs, Transfection kits and Fluorescence probes}
\subsubsection{Antibodies and fluorescence probes}
Primary antibodies and secondary antibodies [Horse Radish Peroxidase (HRP) and Fluorophore conjugated secondary antibodies] used for western blotting (WB) and immunofluorescence (IF), the company of purchase and the dilution used are listed in \Cref{tab:20_prim} and \Cref{tab:20_sec} respectively. Fluorescence probes such as LysoTracker${\tiny{\texttrademark}}$ Red DND-99 (\#L7528) and MitoSOX TM Red (\#M36008) were purchased from Thermo Fischer Scientific, 2', 7'-Dichlorofluorescin diacetate (DCF) (\#D6883) and Hoechst 33342 (\#H6024) were obtained from Sigma Aldrich.

\subsubsection{DNA plasmids and Transfection reagents}
The DNA plasmids mRFP-GFP-LC3 and GFP-LC3-RFP$\delta$LC3 (84572) were purchased from AddGene, Lipofectamine$^{\tiny{\textregistered}}$ 3000 Transfection kit (\#L3000008) and Neon$^{\tiny{\textregistered}}$ 10 $\mu$L Transfection Kit (\#MPK1025) were from Thermo Fischer Scientific.

\section{Mammalian cell culture protocol}
Murine hypothalamus-derived GT1-7neuronal cell and the murine neuroblastoma (Neuro 2a) cell line were used. GT1-7neuronal cells were received as a gift from Prof Pamela Mellon (University of California, San Diego, USA) \citep{Mellon1990}. Cells were maintained in DMEM supplemented with 10\% FBS and 1\% PenStrep. Stably transfected mouse neuroblastoma cell lines, the N2a Swedish mutant form (N2aSwe) of the APP695 plasmid were kindly provided by Professor Sangram Sisodia (University of Chicago, USA) \citep{Sisodia1990}. The generation of the N2aSwe cell line has been extensively described elsewhere \citep{Lo1994}. These cells were cultured in a 1:1 mixture of DMEM and Opti-MEM Reduced Serum supplemented with 5\% FBS and 1\% PenStrep. Both cell types were maintained in a humidified incubator (C01901,Snijders Scientific) at 37 degrees Celsius and 5\% CO\textsubscript{2} atmosphere.

\begin{landscape}
\begin{table}[p]
\centering
\caption[Primary antibodies]{Primary antibodies}
\label{tab:20_prim}
\begin{tabular}{lllccc}
\toprule
Antibody & Company & Source & Expected size (kDa) & Dilution WB & Dilution IF \\
\midrule
Anti-LC3B & Cell Signalling (\#2775) & Rabbit Polyclonal & 16 \& 18 & 1:1000 & 1:200 \\
Anti-SQSTM1/p62 & Abcam (\#ab109012) & Rabbit Ployclonal & 62 & 1:50000 & 1:200 \\
Anti-SQSTM1/p62 & Abcam (\#ab56416) & Mouse Ployclonal & 62 & 1:5000 & 1:200 \\
Anti- acetylated-$\alpha$- tubulin (6-11B) & Santa Cruz (\#23950) & Mouse Monoclonal & 55 & 1:1000 & 1:200 \\
Anti-alpha/beta tubulin & Cell Signalling (\#2148) & Rabbit Polyclonal & - & N/A & 1:100 \\
Anti-APP(NAB228) & Cell Signalling (2450) & Mouse Monoclonal & 140 & 1:1000 & 1:200 \\
Anti-APP & Cell Signalling (2452) & Rabbit Polyclonal & 140 & 1:1000 & 1:200 \\
Anti-Lamp2a & Abcam (ab18528) & Rabbit Polyclonal & 120 & 1:1000 & N/A \\
Anti-cleaved PARP & Cell Signalling (9541) & Rabbit Polyclonal & 89 & 1:1000 & N/A \\
Anti-4-Hydroxy-2-nonenal & Abcam (ab46545) & Rabbit Polyclonal & 35-76 & 1:1000 & 1:200 \\
\bottomrule
\end{tabular}
\end{table}
\begin{table}[p]
\centering
\caption[Secondary antibodies and probes]{Secondary antibodies and probes}
\label{tab:20_sec}
\begin{tabular}{llcc}
\toprule
Antibody/Probe & Company & Dilution WB & Dilution IF \\
\midrule
Anti-Rabbit IgG,HRP-linked antibody & Cell Signalling (\#CST7074S) & 1:5000 & N/A \\
Anti-Mouse IgG,HRP-linked antibody & Cell Signalling (\#CST7076S) & 1:5000 & N/A \\
Alexa Fluor 488 donkey anti-mouse & ThermoFischerScientific (\#1226927) & N/A & 1:200 \\
Alexa Fluor 488 donkey anti-rabbit & ThermoFischer Scientific (\#1754421) & N/A & 1:200 \\
Alexa Fluor 568 donkey anti-mouse & ThermoFischer Scientific (\#1696197) & N/A & 1:200 \\
Alexa Fluor 568 donkey anti-rabbit & ThermoFischer Scientific (\#1235798) & N/A & 1:200 \\
Alexa Fluor 647 Phalloidin & ThermoFischer Scientific (\#1466629) & N/A & 1:200 \\
\bottomrule
\end{tabular}
\end{table}
\end{landscape}

\section{Thawing and culturing of GT1-7 and N2a-Swe cells}
Cryovials containing approximately 1x106 cells in 1 mL FBS, with 10\% DMSO (Dimethyl sulfoxide) (\#D2650, Whitehead Scientific), were thawed at 37 degrees Celsius warm water. The suspension was added in a T25 culture flask (25 cm\textsuperscript{2}) containing 4 ml of pre-warmed growth media and incubated at 37 degrees Celsius for 1 - 2 h to allow cells to attach. After cell attachment, media with DMSO was removed and 5 mL of fresh media was added. The media was changed every 2-3 days until the cells reached 70 - 80 \% confluency. At this point, the cells were sub-cultured and seeded for experimental conditions as indicated below.

\subsection{Passaging and sub-culturing of GT1-7 and N2a-Swe cells for experiments}
For passaging and sub-culturing of cells, growth medium was discarded and the cell monolayer was rinsed briefly with 1 mL of pre-warmed 1x phosphate buffer saline (PBS) (ADDENDUM A). 2 mL of fresh trypsin was added to the cell monolayer and incubated until the cells detached (3 - 5 min). Growth medium (twice the volume of trypsin) was added to the cell suspension to neutralize trypsin and the cell suspension was transferred into a 15 mL falcon tube and centrifuged for 3 min at 1500 RPM. The medium was removed, the pellet was re-suspended in fresh medium and cells were counted in a Neubauer Improved haemocytometer (Marenfield) and seeded at desired density into flasks and/or experimental plates for various experiments.

\section{Treatment protocol}
Prior to treatment, growth medium was removed and the cell monolayer was washed twice with warm PBS. GT1-7 cells were left untreated or treated with varying concentrations, 0.1 $\mu$M, 1 $\mu$M and 10 $\mu$M of Spd and Ril for 8 hr. After 8 h, cells were either incubated with WST-1 reagent for viability analysis or treated for another 4 h with saturating concentrations (400 nM) of the lysosomal inhibitor bafilomycin A1 (BafA1) \citep{DuToit2018b,loos2014}.

\subsection{Effect of low and high concentration of Spd and Ril on autophagic flux}
In brief, cells were treated with 1 $\mu$M and 10 $\mu$M Spd and Ril for 8 h. After 8 h, the media was aspirated and cells were treated for another 4 h with Baf A1 (400 nM).

\subsection{Paraquat (PQ) concentration test}
Prior to treatment, growth medium was removed and the cell monolayer was washed twice with warm PBS. GT1-7 cells were treated with the following concentrations of PQ: 500 $\mu$M,1,2,3,5 and 10 mM. 

\subsection{Effect of low and high concentration spermidine and rilmenidine on PQ induced toxicity}
After removal of growth medium and rinsing of the cell monolayer with warm PBS, cells were treated with 1 $\mu$M and 10 $\mu$M Spd and Ril for 8 h. After 8 h, the treatment was aspirated and cells were treated for another 6 h with 3 mM PQ. 

\subsection{Effect of low and high concentration spermidine and rilmenidine on APP overexpression}
Prior to treatment, growth medium was aspirated and the cell monolayer was washed twice with warm PBS. N2aSwe cells were treated with 5 mM butyric acid (BA) for 24 and 48 h to induce butyrate-inducible CMV promoter-driven APP transgene expression. 8 h prior to the end of the treatment, cells were incubated for further 8 h in combination medium containing 5 mM BA and 1 $\mu$M and 10 $\mu$M Spd and Ril.

\section{Analysis}
\subsection{Reductive Capacity Assay}
The colorimetric assay WST-1 was used to measure cell viabilty. This assay is frequently used to measure cell viability or cell proliferation based on the reduction of the tetrazolium salts into formazan crystals by mitochondrial dehydrogenase enzyme. Therefore, media with solubilized formazan crystals can be measured spectrophotometrically as an indicator of cellular viability. An increase in the absorbance signal is directly proportional to an increase in metabolically healthy cells, while its decrease indicates a decrease in viable cells due to cytotoxic effects of treatment compounds. 

Briefly, GT 1-7 cells were seeded at a density of 75 000 cells per well, while N2aSwe were seeded at 20 000 cells per well in 48 well dishes and cultured overnight in 200 $\mu$L media. The following day, media was aspirated and cells were treated as desired over the required period of time. Thereafter, 10 $\mu$L of WST-1 reagent was added to each well and the culture plates were incubated for 2 h protected from light. Subsequently, culture plates were placed in shaking incubator (37 degrees Celsius, 200 RPM) and gently shaken for 2 min to dissolve the formazan crystals. Plates were analysed spectrophotometrically at a wavelength of 595 nm with the KC Junior software using a universal micro plate reader (EL800, Bio-Tek Instruments Inc.). Absorbance values of the treated and untreated control cells were subtracted from the background control (medium containing WST-1). All results were expressed as a percentage of the untreated control.

\subsection{Western blot analysis}
For the Western blot analysis, GT1-7 were seeded at 3 000 000 and N2a-Swe at 100 000 into T75 flasks with 8 ml of growth media per flask. GT 1-7 cells were allowed to attach and proliferate for 48 h while N2aSwe were incubated for 24 h, both at 37 degrees Celsius, 5\% CO2 environment before treatment.

