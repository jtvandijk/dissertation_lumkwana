\chapter{Materials and methods}
\section{Reagents and consumables}
\subsection{Cell culture reagents}
Cell culture flasks; T175 (175 cm\textsuperscript{2}) (\#431080), T75 (75 cm\textsuperscript{2}) (\#708003) and T25 (25 cm\textsuperscript{2}) (\#707003), cell culture plates; 6 well (703001), 48 well (\#748001) and serological pipettes; 5 ml (\#326001), 10 ml (\#327001), 25 ml (\#328001) were purchased from Whitehead Scientific. Cell culture medium, Dulbecco’s Modified Eagles Medium (DMEM) (\#41965-039), Opti-MEM Reduced Serum media (\#319850470), Penicillin-Streptomycin (PenStrep) (\#151140-122) and 0.25\% Trypsin-EDTA (\#25200-072) were obtained from Thermo Fischer Scientific, Foetal Bovine Serum (FBS) (\#FBS-GI-12A) from Biocom-Biotech. Tissue culture falcon tubes, 15 ml (\#50015) ml and 50 ml (\#50050), pipette tips and Eppendorf tubes were purchased from B\&M Scientific. Glass bottom Gridded dishes (\#P35G-1.5-14-CGRD) were obtained from MatTek Corporation, USA.

\subsection{Experimental and treatment reagents}
Spermidine (\#S0266), Paraquat dichloride x-hydrate pestnal (\#36541), 2-Cysteamine (\#M9768), WST-1 Cell Proliferative Reagent (\#11644807001) and butyric acid (\#B103500) were purchased from Sigma Aldrich. Rilmenidine hemifumarate (\#0790) was obtained from Whitehead Scientific and Bafilomycin A1 (\#B0025) from Lkt${\textsuperscript{\tiny{\textregistered}}}$ Labs. 

\subsection{Antibodies, Plasmids Constructs, Transfection kits and Fluorescence probes}
\subsubsection{Antibodies and fluorescence probes}
Primary antibodies and secondary antibodies [Horse Radish Peroxidase (HRP) and Fluorophore conjugated secondary antibodies] used for western blotting (WB) and immunofluorescence (IF), the company of purchase and the dilution used are listed in \Cref{tab:20_prim} and \Cref{tab:20_sec} respectively. Fluorescence probes such as LysoTracker${\textsuperscript{\tiny{\texttrademark}}}$ Red DND-99 (\#L7528) and MitoSOX TM Red (\#M36008) were purchased from Thermo Fischer Scientific, 2', 7'-Dichlorofluorescin diacetate (DCF) (\#D6883) and Hoechst 33342 (\#H6024) were obtained from Sigma Aldrich.

\subsubsection{DNA plasmids and Transfection reagents}
The DNA plasmids mRFP-GFP-LC3 and GFP-LC3-RFP$\delta$LC3 (84572) were purchased from AddGene, Lipofectamine${\textsuperscript{\tiny{\textregistered}}}$ 3000 Transfection kit (\#L3000008) and Neon${\textsuperscript{\tiny{\textregistered}}}$ 10 $\mu$L Transfection Kit (\#MPK1025) were from Thermo Fischer Scientific.

\section{Mammalian cell culture protocol}
Murine hypothalamus-derived GT1-7neuronal cell and the murine neuroblastoma (Neuro 2a) cell line were used. GT1-7neuronal cells were received as a gift from Prof Pamela Mellon (University of California, San Diego, USA) \citep{Mellon1990}. Cells were maintained in DMEM supplemented with 10\% FBS and 1\% PenStrep. Stably transfected mouse neuroblastoma cell lines, the N2a Swedish mutant form (N2aSwe) of the APP695 plasmid were kindly provided by Professor Sangram Sisodia (University of Chicago, USA) \citep{Sisodia1990}. The generation of the N2aSwe cell line has been extensively described elsewhere \citep{Lo1994}. These cells were cultured in a 1:1 mixture of DMEM and Opti-MEM Reduced Serum supplemented with 5\% FBS and 1\% PenStrep. Both cell types were maintained in a humidified incubator (C01901,Snijders Scientific) at 37 degrees Celsius and 5\% CO\textsubscript{2} atmosphere.

\subsection{Thawing and culturing of GT1-7 and N2a-Swe cells}
Cryovials containing approximately 1x106 cells in 1 mL FBS, with 10\% DMSO (Dimethyl sulfoxide) (\#D2650, Whitehead Scientific), were thawed at 37 degrees Celsius warm water. The suspension was added in a T25 culture flask (25 cm\textsuperscript{2}) containing 4 ml of pre-warmed growth media and incubated at 37 degrees Celsius for 1 - 2 h to allow cells to attach. After cell attachment, media with DMSO was removed and 5 mL of fresh media was added. The media was changed every 2-3 days until the cells reached 70 - 80 \% confluency. At this point, the cells were sub-cultured and seeded for experimental conditions as indicated below.

\begin{landscape}
\begin{table}[p]
\centering
\caption[Primary antibodies]{Primary antibodies}
\label{tab:20_prim}
\begin{tabular}{lllccc}
\toprule
Antibody & Company & Source & Expected size (kDa) & Dilution WB & Dilution IF \\
\midrule
Anti-LC3B & Cell Signalling (\#2775) & Rabbit Polyclonal & 16 \& 18 & 1:1000 & 1:200 \\
Anti-SQSTM1/p62 & Abcam (\#ab109012) & Rabbit Ployclonal & 62 & 1:50000 & 1:200 \\
Anti-SQSTM1/p62 & Abcam (\#ab56416) & Mouse Ployclonal & 62 & 1:5000 & 1:200 \\
Anti- acetylated-$\alpha$- tubulin (6-11B) & Santa Cruz (\#23950) & Mouse Monoclonal & 55 & 1:1000 & 1:200 \\
Anti-alpha/beta tubulin & Cell Signalling (\#2148) & Rabbit Polyclonal & - & N/A & 1:100 \\
Anti-APP(NAB228) & Cell Signalling (2450) & Mouse Monoclonal & 140 & 1:1000 & 1:200 \\
Anti-APP & Cell Signalling (2452) & Rabbit Polyclonal & 140 & 1:1000 & 1:200 \\
Anti-Lamp2a & Abcam (ab18528) & Rabbit Polyclonal & 120 & 1:1000 & N/A \\
Anti-cleaved PARP & Cell Signalling (9541) & Rabbit Polyclonal & 89 & 1:1000 & N/A \\
Anti-4-Hydroxy-2-nonenal & Abcam (ab46545) & Rabbit Polyclonal & 35-76 & 1:1000 & 1:200 \\
\bottomrule
\end{tabular}
\end{table}
\begin{table}[p]
\centering
\caption[Secondary antibodies and probes]{Secondary antibodies and probes}
\label{tab:20_sec}
\begin{tabular}{llcc}
\toprule
Antibody/Probe & Company & Dilution WB & Dilution IF \\
\midrule
Anti-Rabbit IgG,HRP-linked antibody & Cell Signalling (\#CST7074S) & 1:5000 & N/A \\
Anti-Mouse IgG,HRP-linked antibody & Cell Signalling (\#CST7076S) & 1:5000 & N/A \\
Alexa Fluor 488 donkey anti-mouse & ThermoFischerScientific (\#1226927) & N/A & 1:200 \\
Alexa Fluor 488 donkey anti-rabbit & ThermoFischer Scientific (\#1754421) & N/A & 1:200 \\
Alexa Fluor 568 donkey anti-mouse & ThermoFischer Scientific (\#1696197) & N/A & 1:200 \\
Alexa Fluor 568 donkey anti-rabbit & ThermoFischer Scientific (\#1235798) & N/A & 1:200 \\
Alexa Fluor 647 Phalloidin & ThermoFischer Scientific (\#1466629) & N/A & 1:200 \\
\bottomrule
\end{tabular}
\end{table}
\end{landscape}

\subsection{Passaging and sub-culturing of GT1-7 and N2a-Swe cells for experiments}
For passaging and sub-culturing of cells, growth medium was discarded and the cell monolayer was rinsed briefly with 1 mL of pre-warmed 1x phosphate buffer saline (PBS) (ADDENDUM A). 2 mL of fresh trypsin was added to the cell monolayer and incubated until the cells detached (3 - 5 min). Growth medium (twice the volume of trypsin) was added to the cell suspension to neutralize trypsin and the cell suspension was transferred into a 15 mL falcon tube and centrifuged for 3 min at 1500 RPM. The medium was removed, the pellet was re-suspended in fresh medium and cells were counted in a Neubauer Improved haemocytometer (Marenfield) and seeded at desired density into flasks and/or experimental plates for various experiments.

\section{Treatment protocol}
Prior to treatment, growth medium was removed and the cell monolayer was washed twice with warm PBS. GT1-7 cells were left untreated or treated with varying concentrations, 0.1 $\mu$M, 1 $\mu$M and 10 $\mu$M of Spd and Ril for 8 hr. After 8 h, cells were either incubated with WST-1 reagent for viability analysis or treated for another 4 h with saturating concentrations (400 nM) of the lysosomal inhibitor bafilomycin A1 (BafA1) \citep{DuToit2018b,loos2014}.

\subsection{Effect of low and high concentration of Spd and Ril on autophagic flux}
In brief, cells were treated with 1 $\mu$M and 10 $\mu$M Spd and Ril for 8 h. After 8 h, the media was aspirated and cells were treated for another 4 h with Baf A1 (400 nM).

\subsection{Paraquat (PQ) concentration test}
Prior to treatment, growth medium was removed and the cell monolayer was washed twice with warm PBS. GT1-7 cells were treated with the following concentrations of PQ: 500 $\mu$M,1,2,3,5 and 10 mM. 

\subsection{Effect of low and high concentration spermidine and rilmenidine on PQ induced toxicity}
After removal of growth medium and rinsing of the cell monolayer with warm PBS, cells were treated with 1 $\mu$M and 10 $\mu$M Spd and Ril for 8 h. After 8 h, the treatment was aspirated and cells were treated for another 6 h with 3 mM PQ. 

\subsection{Effect of low and high concentration spermidine and rilmenidine on APP overexpression}
Prior to treatment, growth medium was aspirated and the cell monolayer was washed twice with warm PBS. N2aSwe cells were treated with 5 mM butyric acid (BA) for 24 and 48 h to induce butyrate-inducible CMV promoter-driven APP transgene expression. 8 h prior to the end of the treatment, cells were incubated for further 8 h in combination medium containing 5 mM BA and 1 $\mu$M and 10 $\mu$M Spd and Ril.

\section{Reductive Capacity Assay}
The colorimetric assay WST-1 was used to measure cell viabilty. This assay is frequently used to measure cell viability or cell proliferation based on the reduction of the tetrazolium salts into formazan crystals by mitochondrial dehydrogenase enzyme. Therefore, media with solubilized formazan crystals can be measured spectrophotometrically as an indicator of cellular viability. An increase in the absorbance signal is directly proportional to an increase in metabolically healthy cells, while its decrease indicates a decrease in viable cells due to cytotoxic effects of treatment compounds. 

Briefly, GT 1-7 cells were seeded at a density of 75 000 cells per well, while N2aSwe were seeded at 20 000 cells per well in 48 well dishes and cultured overnight in 200 $\mu$L media. The following day, media was aspirated and cells were treated as desired over the required period of time. Thereafter, 10 $\mu$L of WST-1 reagent was added to each well and the culture plates were incubated for 2 h protected from light. Subsequently, culture plates were placed in shaking incubator (37 degrees Celsius, 200 RPM) and gently shaken for 2 min to dissolve the formazan crystals. Plates were analysed spectrophotometrically at a wavelength of 595 nm with the KC Junior software using a universal micro plate reader (EL800, Bio-Tek Instruments Inc.). Absorbance values of the treated and untreated control cells were subtracted from the background control (medium containing WST-1). All results were expressed as a percentage of the untreated control.

\section{Western blot analysis}
For the western blot analysis, GT1-7 were seeded at 3 000 000 and N2a-Swe at 100 000 into T75 flasks with 8 ml of growth media per flask. GT 1-7 cells were allowed to attach and proliferate for 48 h while N2aSwe were incubated for 24 h, both at 37 degrees Celsius, 5\% CO2 environment before treatment.

\subsection{Protein extraction}
After appropriate treatments were completed, growth medium was aspirated and cells were immediately placed on ice and washed twice with ice cold 1x PBS. In order to extract total protein, cells were incubated for 5 mins with 100 uL of radioprecipitation (RIPA) lysis buffer (\textbf{ADDENDUM A}), which was supplemented with 1x complete Protease Inhibitor Cocktail (as per manufacturers protocol) and phosphatase inhibitors such as 1 mM phenylmethylsulfonyl fluoride (PMSF) (\#93482, Sigma Alrich, ), 1 mM Sodium Fluoride (NaF) (\#193270, Merck,), and 1 mM Sodium Orthovanadate (Na3VO4) (\#S6508, Sigma Alrich). Cells were detached by a scrapping method. Cell lysates were sonicated on ice at 4 Hz for 8 sec and were left for 30 min on ice until the foam subsided and then centrifuged at 8000 RPM for 10 min. Subsequently, supernatant was added into pre-chilled Eppendorf tubes and stored at -80 degrees Celsius until further analysis.

\subsection{Protein determination and sample preparation}
Protein content of cell lysates was determined using a Direct Detect${\textsuperscript{\tiny{\textregistered}}}$ infrared spectrometer (DDHW00010-WW, Merck). Aliquots of 50$\mu$g/ $\mu$L protein diluted in 2:1 ratio in Laemli’s sample buffer (\textbf{ADDENDUM A}), were prepared and stored at -80 degrees Celsius. Prior to loading onto the gels, samples were thawed on ice and boiled at 95 degrees Celsius for 5 min to denature proteins.

\subsection{Sodium-dodecyl-sulfate-polycrylamide gel electrophoresis (SDS-PAGE) and western blot analysis}
Samples (50 $\mu$g/ $\mu$L protein) were loaded into the 4-20 \% polyacrylamide precast gels (CriterionTM TGX TM Midi protein gel, Biorad, 5671094). 5 $\mu$l of a prestained protein ladder (PM007-0500, Biocom Biotech) was loaded into the first well of each gel in order to orientate the gel and determine molecular weights of separated proteins. Proteins were separated at 100 V and 400 mA using Power Pac 300 (BioRad). Once completed, proteins were transferred to a 0.2 $\mu$m polyvinylidine fluoride (PVDF) membrane using Midi Trans-Blot Turbo Transfer Packs (\#170-4157, Biorad) and the BioRad Trans-Blot Turbo electrotransfer system (170-4155, BioRad,), which was programmed at 120 V and 400 mA for 7 min. To confirm that proteins were transferred correctly, membranes were visualized using the stain free blot protocol provided on the Chemi-DocTM MP system from BioRad. Subsequently, membranes were incubated in 5\% fat-free milk made up in Tris- buffered saline/0.1\% Tween 20 (TBS-T) (\textbf{ADDENDUM A}), for 1 h using a bellydancer. Following blocking, membranes were washed 3x for 5 min with TBS-T and incubated with specific primary antibodies diluted in TBS-T (1:1000 or 1:5000) overnight at 4 degrees Celsius. The following day, membranes were washed 3x for 5 min with TBS-T and incubated in corresponding secondary antibodies, Goat anti-rabbit IgG or anti-mouse horseradish peroxidase conjugated secondary antibody (1:5000), gently shaking for 1 h at room temperature (RT). Following incubation, membranes were washed 3x for 5 min with TBS-T and exposed to Clarity${\textsuperscript{\tiny{\textregistered}}}$ ECL (Enhanced chemiluminescence) peroxide reagent kit (\#1705061, Biorad) for 2 min and protein bands were detected and captured using a Chemi-DocTM MP imaging system (BioRad). Detected band intensities were quantified and normalized against total protein using Bio-Rad Image Lab software.

\subsection{Membrane stripping and re-probing}
Membranes were washed 2x 10 min with TBS-T and incubated at RT in the stripping buffer (\textbf{ADDENDUM A}) for 2x 15 min. Subsequently, membranes were rinsed twice in TBS-T, washed for 15 min in TBS-T and again washed for 3x for 5 min in 1x TBS-T. Once completed, the membrane was blocked and re-probed.

\section{Immunofluorescence}
\subsection{Sample preparation for confocal microscopy and SR-SIM}
GT1-7 cells were seeded at 150 000 cells per well on coverslips in 6 well dishes and cultured overnight in 2 ml of growth medium at 37 degrees Celsius, in a 5\% CO\textsubscript{2} environment. The following day, cells were subjected to the desired treatment protocol. When required, LysoTracker${\textsuperscript{\tiny{\texttrademark}}}$ Red DND-99 at 75 nM was added to the treatment medium 2 h prior ro the completion of the period. Next, growth medium was aspirated and cells were washed 3x 5 min with pre-warmed in 1x PBS-A (\textbf{ADDENDUM A}), fixed in pre-warmed double strength fixative at a 1:1 ratio of 8\% v/v formaldehyde (FA): growth medium for 15 min at 37 degrees Celsius, in a 5\% CO\textsubscript{2} environment. Following fixation, cells were washed 3x for 5 min and permeabilized with 0.2\%Triton X in 1x PBS-A for 2 min. Subsequently, cells were washed 3x for 5 min and blocked for 30 min in 5\% v/v donkey serum in 1x PBS-A to prevent non-specific binding of antibodies. Blocking buffer was removed and cells were probed with specific primary antibodies (\Cref{tab:20_prim}); anti-p62 mouse, anti-LC3 rabbit, anti-Lamp2A mouse, anti-alpha ($\alpha$)/beta ($\beta$)-tubulin rabbit, anti-acetylated-$\alpha$ tubulin mouse, diluted in 3\% BSA (\#10735078001, Sigma Aldrich) and incubated overnight at 4 degrees Celsius in a humidified chamber. Next, cells were washed 3x for 5 min and incubated with corresponding secondary antibodies diluted in 3\% BSA for 50 min in a humidified chamber at RT. In this case, Alexa Fluor 488 donkey anti-mouse, Alexa Fluor 488 donkey anti-rabbit, Alexa Fluor 568 donkey anti-mouse, Alexa Fluor 568 donkey anti-rabbit, and Alexa Fluor 647 phalloidin secondary antibodies were used. Cells were counterstained with Hoechst 33342 for further 10 min and then washed 3x for 5 min. Coverslips were inverted and mounted onto glass slide using Dako${\textsuperscript{\tiny{\textregistered}}}$ fluorescent mounting media (\#S302380, Diagnostech), sealed with clear nail polish and stored at 4 degrees Celsius in a dark slide container until image acquisition.

\subsection{dSTORM sample preparation}
In brief, GT1-7 and N2a-swe cells were seeded at a density of 150,000 cells per well using matTek culture dishes with glass bottom (MatTek Corporation) and cultured overnight. Thereafter, cells were treated as desired over the required period of time. Following completion of treatment, growth medium was aspirated and cells were washed, fixed and prepared for immunofluorescence. Specifically, anti-acetylated-$\alpha$ tubulin mouse, anti-APP rabbit, Alexa Fluor 568 donkey anti-mouse and Alexa Fluor 568 donkey anti-rabbit were used. After secondary antibody incubation was complete cells were fixed in 4\% v/v FA for 10 min in order to stabilize fluorophore labelling. Subsequently, fixative was aspirated and cells were incubated in PBS-G (\textbf{ADDENDUM A}) for 2 min to terminate fixation, washed 3x for 5 min and stored in PBS-A until image acquisition. 

\section{Transfection} 
The electroporation (Neon, Thermofischer) or chemical transfection [Lipofectamine 3000 transfection reagent (Invitrogen, \#L3000008)] was used to transfect cells using GFP-LC3-RFP-LC3$\delta$G or mRFP-GFP-LC3 DNA plasmid into GT 1-7 cell. Manufactures instructions for electroporation and Lipofectamine 3000 transfection reagent were used as a point of departure for transfection optimisation. Different conditions such as cell density, adherent or cell suspension conditions, temperature of transfection buffers and amount of plasmid DNA in using a 6 well plate and T25 culture flask were tested. The Olympus${\textsuperscript{\tiny{\textregistered}}}$ IX81 inverted widefield fluorescence microscope (Olympus${\textsuperscript{\tiny{\textregistered}}}$ Biosystems, GMBH Japan) at 10x objective was used to determine the transfection efficiency.

Once optimised, the following protocols were implemented for the T25 ml flask, 100,000 cells exposed to a total of 10 $\mu$g of plasmid DNA were used while for a 6 well plate, 200,000 cells with 2 $\mu$g was were employed. Briefly, when following the electroporation protocol, appropriate number of cells were aliquoted out, washed in PBS and spun down. The supernatant was discarded, and the pallet of cells was re-suspended in 8.34 or 41.7 $\mu$L of sterile Neon${\textsuperscript{\tiny{\textregistered}}}$ resuspension buffer containing 1.66 or 8.3 $\mu$L of GFP-LC3-RFP-LC3$\delta$G or mRFP-GFP-LC3 DNA plasmid, a total of 2.0 or 10 $\mu$g DNA plasmid. Subsequently, the cell suspension was pipetted into a gold plated Neon${\textsuperscript{\tiny{\textregistered}}}$ Tip using a Neon${\textsuperscript{\tiny{\textregistered}}}$ Pipette, with the tip inserted into Neon${\textsuperscript{\tiny{\textregistered}}}$ Electrolytic buffer inside the Neon${\textsuperscript{\tiny{\textregistered}}}$ Pipette Station. Following this process, cells were electroporated at 1350 V for 1 pulse lasting for 30 ms and plated into T25 flasks or 6 well dish containing fresh media (1:1 DMEM/OptiMEM with 10\% FBS) without antibiotics and incubated 48 hrs to allow transfection to take place.

Following the chemical transfection protocol and after removing the supernatant, the pallet of cells was re-suspended and cells seeded at a density of 200,000 per well in 6 well dishes in media (1:1 DMEM/OptiMEM) containing 10\% FBS without antibiotics and transfected immediately in suspension. Transfections were carried out in OptiMEM using a total of 2$\mu$g plasmid DNA per well. Briefly, two mixtures were prepared, one containing optiMEM with 2 $\mu$g of DNA and 4 $\mu$l P3000 reagent to a total of 125 $\mu$l and and another mixture containing 121.5 $\mu$l optiMEM with 3.75 $\mu$l of Lipofectamine 3000 reagent.  These solutions were then mixed together and incubated 15 mins at RT prior to adding to wells in a drop-wise manner. Cells were incubated for 48 hours. 

\section{Fluorescence Microscopy}
All images were acquired at the Central Analytical Facility (Stellenbosch University) using a Carl Zeiss laser scanning confocal microscope (LSM) 780 equipped with the ELYRA PS1 super-resolution and PALM/STORM platform (Carl Zeiss, Germany) in order to harness the distinct resolving power of each component, i.e. achieving 80 nm (SR-SIM) or 20 nm (PALM/STORM) for a given application.

\subsection{Confocal Laser Scanning Microscopy (LSM)}
Fluorescently labelled and transfected cells were acquired and processed using Zen 2011 imaging software (Carl Zeiss, Germany). Micrographs were collected in a z-stacking mode, acquiring 8-10 image frames with increments of 0.60 $\mu$m step width were routinely acquired. Depending on the type of analysis, a LCI Plan-Apochromat 63x/1.4 Oil DIC M27 or Alpha Plan-Apochromat 100x/1.46 Oil DIC M27 objective lens was used. Hoechst, Alexa 488, Alexa 568 and Alexa 647 and LysoTracker were excited with a suitable range of lasers, (405 nm, 488 nm, 561 nm and 633 nm) and signal detected with a GaAsP detector (32$\pm$2 PMT). Beam splitters (MBS 458, MBS 488, MBS 458/514, MBS 488/561, MBS 488/561/633) were selected accordingly depending on the laser line used. Z-stacks were projected as maximum intensity using the Zeiss Zen Black Software (2012) and used for subsequent analysis using FIJI.

\subsection{Super Resolution-Structured Illumination Microscopy (SR-SIM)}
Cells labelled for acetylated-$\alpha$ tubulin and -$\alpha$/$\gamma$ tubulin were imaged using SR-SIM on the Elyra PS1 platform (Carl Zeiss, Germany). Beads were acquired in a z-stacking mode at 5 phases and 3 rotations  of the illumination grid using Alpha Plan-Apochromat 100x/1.46 Oil DIC M27 objective lenses and 405,488 and 561 laser lines in order to correct for chromatic aberration using the channel alignment algorithm. Subsequently, images were processed and aligned using the ZEN Black Elyra edition sofware (Carl Zeiss Microscopy). Samples were acquired using the above mentioned settings and processed for SIM, corrected for channel alignment and projected to maximum intensity. 

\subsection{d-STORM}
Single molecule microscopy of cells labelled for APP and acetylated-$\alpha$ tubulin was performed. Molecules were activated using super resolution Abbelight buffer (a kind gift from Pierre Bauër, Abbelight) or 1M MEA buffer (M9768, Sigma Aldrich) (\textbf{ADDENDUM A}). Blinking events were acquired using an Alpha Plan-Apochromat 100x/1.46 Oil DIC M27 objective lens using a 561 nm laser and fluorescence was detected with the EMCCD camera. Drifting was minimised using nitrogen gas and definite focus. Imaging was performed in TIRF-uHP mode, 100\% laser power, camera integration of 33ms and EM gain of 150 initially in order to bleach the sample and force fluorephores to enter the dark state. A recording of 50,000 events were collected per cell. After fluorescence was bleached, the laser power was reduced to 2\% while recording the events. Raw data images were processed using a ZEN Black software where data was corrected for outliers, drifting, grouping and rendering. For APP analysis, images were further processed using FIJI analysis software for cluster analysis. 

\section{Transmission Electron Microscopy (TEM)}





 

