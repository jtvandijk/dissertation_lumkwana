\chapter{Materials and methods}
\section{Reagents and consumables}
\subsection{Cell culture reagents}
Cell culture flasks; T175 (175 cm\textsuperscript{2}) (\#431080), T75 (75 cm\textsuperscript{2}) (\#708003) and T25 (25 cm\textsuperscript{2}) (\#707003), cell culture plates; 6 well (703001), 48 well (\#748001) and serological pipettes; 5 ml (\#326001), 10 ml (\#327001), 25 ml (\#328001) and Dimethyl sulfoxide (DMSO) (\#D2650), were purchased from Whitehead Scientific. Cell culture medium, Dulbecco’s Modified Eagles Medium (DMEM) (\#41965-039), Opti-MEM Reduced Serum media (\#319850470), Penicillin-Streptomycin (PenStrep) (\#151140-122) and 0.25 \% Trypsin-EDTA (\#25200-072) were obtained from Thermo Fischer Scientific, Foetal Bovine Serum (FBS) (\#FBS-GI-12A) from Biocom-Biotech. Tissue culture falcon tubes, 15 ml (\#50015) ml and 50 ml (\#50050), pipette tips and Eppendorf tubes were purchased from B\&M Scientific. Glass bottom Gridded dishes (\#P35G-1.5-14-CGRD) were obtained from MatTek Corporation, USA.

\subsection{Experimental and treatment reagents}
Spermidine (\#S0266), Paraquat dichloride x-hydrate pestnal (\#36541), 2-Cysteamine (\#M9768), WST-1 Cell Proliferative Reagent (\#11644807001) and butyric acid (\#B103500) were purchased from Sigma Aldrich. Rilmenidine hemifumarate (\#0790) was obtained from Whitehead Scientific and Bafilomycin A1 (\#B0025) from Lkt${\textsuperscript{\tiny{\textregistered}}}$ Labs. 

\subsection{Antibodies, plasmids constructs, transfection kits and fluorescence probes}
\subsubsection{Antibodies and fluorescence probes}
Primary antibodies and secondary antibodies [Horse Radish Peroxidase (HRP) and Fluorophore conjugated secondary antibodies] used for western blotting (WB) and immunofluorescence (IF), the company of purchase and the dilution used are listed in \Cref{tab:20_prim} and \Cref{tab:20_sec} respectively. Fluorescence probes such as LysoTracker${\textsuperscript{\tiny{\texttrademark}}}$ Red DND-99 (\#L7528) and MitoSOX TM Red (\#M36008) were purchased from Thermo Fischer Scientific, 2', 7'-Dichlorofluorescin diacetate (DCF) (\#D6883) and Hoechst 33342 (\#H6024) were obtained from Sigma Aldrich.

\subsubsection{DNA plasmids and Transfection reagents}
The DNA plasmids mRFP-GFP-LC3 and GFP-LC3-RFP-LC3$\Delta$G (84572) were purchased from AddGene, Lipofectamine${\textsuperscript{\tiny{\textregistered}}}$ 3000 Transfection kit (\#L3000008) and Neon${\textsuperscript{\tiny{\textregistered}}}$ 10 $\mu$L Transfection Kit (\#MPK1025) were from Thermo Fischer Scientific.

\section{Mammalian cell culture protocol}
Murine hypothalamus-derived GT1-7 neuronal cell and the murine neuroblastoma (Neuro 2a) cell line were used. GT1-7 neuronal cells were received as a gift from Prof Pamela Mellon (University of California, San Diego, USA) \citep{Mellon1990}. Cells were maintained in DMEM supplemented with 10 \% FBS and 1 \% PenStrep. Stably transfected mouse neuroblastoma cell lines, the N2a Swedish mutant form (N2aSwe) of the APP695 plasmid were kindly provided by Professor Sangram Sisodia (University of Chicago, USA) \citep{Sisodia1990}. The generation of the N2aSwe cell line has been extensively described elsewhere \citep{Lo1994}. These cells were cultured in a 1:1 mixture of DMEM and Opti-MEM Reduced Serum supplemented with 5 \% FBS and 1 \% PenStrep. Both cell types were maintained in a humidified incubator (C01901,Snijders Scientific) at 37 degrees Celsius and 5 \% CO\textsubscript{2} atmosphere.

\begin{landscape}
\begin{table}[p]
\centering
\caption[Primary antibodies]{Primary antibodies}
\label{tab:20_prim}
\begin{tabular}{lllccc}
\toprule
Antibody & Company & Source & Expected size (kDa) & Dilution WB & Dilution IF \\
\midrule
Anti-LC3B & Cell Signalling (\#2775) & Rabbit Polyclonal & 16 \& 18 & 1:1000 & 1:200 \\
Anti-SQSTM1/p62 & Abcam (\#ab109012) & Rabbit Ployclonal & 62 & 1:50000 & 1:200 \\
Anti-SQSTM1/p62 & Abcam (\#ab56416) & Mouse Ployclonal & 62 & 1:5000 & 1:200 \\
Anti-Acetylated-$\alpha$-tubulin (6-11B) & Santa Cruz (\#23950) & Mouse Monoclonal & 55 & 1:1000 & 1:200 \\
Anti-Alpha/Beta tubulin & Cell Signalling (\#2148) & Rabbit Polyclonal & - & N/A & 1:100 \\
Anti-APP(NAB228) & Cell Signalling (2450) & Mouse Monoclonal & 140 & 1:1000 & 1:200 \\
Anti-APP & Cell Signalling (2452) & Rabbit Polyclonal & 140 & 1:1000 & 1:200 \\
Anti-LAMP2A & Abcam (ab18528) & Rabbit Polyclonal & 120 & 1:1000 & N/A \\
Anti-4-Hydroxy-2-nonenal & Abcam (ab46545) & Rabbit Polyclonal & 35-76 & 1:1000 & 1:200 \\
\bottomrule
\end{tabular}
\end{table}
\begin{table}[p]
\centering
\caption[Secondary antibodies and probes]{Secondary antibodies and probes}
\label{tab:20_sec}
\begin{tabular}{llcc}
\toprule
Antibody/Probe & Company & Dilution WB & Dilution IF \\
\midrule
Anti-Rabbit IgG,HRP-linked antibody & Cell Signalling (\#CST7074S) & 1:5000 & N/A \\
Anti-Mouse IgG,HRP-linked antibody & Cell Signalling (\#CST7076S) & 1:5000 & N/A \\
Alexa Fluor 488 donkey anti-mouse & ThermoFischerScientific (\#1226927) & N/A & 1:200 \\
Alexa Fluor 488 donkey anti-rabbit & ThermoFischer Scientific (\#1754421) & N/A & 1:200 \\
Alexa Fluor 568 donkey anti-mouse & ThermoFischer Scientific (\#1696197) & N/A & 1:200 \\
Alexa Fluor 568 donkey anti-rabbit & ThermoFischer Scientific (\#1235798) & N/A & 1:200 \\
Alexa Fluor 647 Phalloidin & ThermoFischer Scientific (\#1466629) & N/A & 1:200 \\
\bottomrule
\end{tabular}
\end{table}
\end{landscape}

\subsection{Thawing and culturing of GT1-7 and N2a-Swe cells}
Cryovials containing approximately 1x10\textsuperscript{6} cells in 1 mL FBS, with 10 \% DMSO, were thawed at 37 degrees Celsius in warm water. The suspension was added in a T25 culture flask (25 cm\textsuperscript{2}) containing 4 ml of pre-warmed growth media and incubated at 37 degrees Celsius for 1 - 2 h to allow cells to attach. After cell attachment, media with DMSO was removed and 5 mL of fresh media was added. The media was changed every 2 - 3 days until the cells reached 70 - 80 \% confluency. At this point, the cells were sub-cultured and seeded for experimental conditions as indicated below.

\subsection{Passaging and sub-culturing of GT1-7 and N2aSwe cells for experiments}\label{sec:Passaging_sub-cuof low and high concentrations of spermidinelturing_GT1-7_N2aSwe_experiments}
For passaging and sub-culturing of cells, growth medium was discarded and the cell monolayer was rinsed briefly with 1 mL of pre-warmed 1x phosphate buffer saline (PBS) (\textbf{Addendum A}). 2 mL of fresh trypsin was added to the cell monolayer and incubated until the cells detached (3 - 5 min). Growth medium (twice the volume of trypsin) was added to the cell suspension to neutralize trypsin and the cell suspension was transferred into a 15 mL falcon tube and centrifuged for 3 min at 1500 RPM. The medium was removed, the pellet was re-suspended in fresh medium and cells were counted in a Neubauer Improved haemocytometer (Marenfield) and seeded at desired density into flasks and/or experimental plates for various experiments.

\section{Treatment protocol}
Prior to treatment, growth medium was removed and the cell monolayer was washed twice with warm PBS. GT1-7 cells were left untreated or treated with varying concentrations, 0.1 $\mu$M, 1 $\mu$M and 10 $\mu$M of spermidine (Spd) and rilmenidine (Ril) for 8 h. After 8 h, cells were either incubated with WST-1 reagent for viability analysis or treated for another 4 h with saturating concentrations (400 nM) of the lysosomal inhibitor bafilomycin A1 (BafA1) \citep{DuToit2018b,loos2014}.

\subsection{Effect of low and high concentrationss of spermidine and rilmenidine on autophagic flux}\label{sec:Effect_low_high_Spd_Ril_autophagic flux}
In brief, cells were treated with 1 $\mu$M and 10 $\mu$M Spd and Ril for 8 h. After 8 h, the media was aspirated and cells were treated for another 4 h with BafA1 (400 nM).

\subsection{Paraquat (PQ) concentration test}
Prior to treatment, growth medium was removed and the cell monolayer was washed twice with warm PBS. GT1-7 cells were treated with the following concentrations of PQ: 500 $\mu$M, 1, 2, 3, 5 and 10 mM. 

\subsection{Effect of low and high concentrations spermidine and rilmenidine on PQ induced toxicity}\label{sec:Effect_low_high_Spd_Ril_PQ_toxicity}
After removal of growth medium and rinsing the cell monolayer with warm PBS, cells were treated with 1 $\mu$M and 10 $\mu$M Spd and Ril for 8 h. After 8 h, the treatment was aspirated and cells were treated for another 6 h with 3 mM PQ. 

\subsection{Effect of low and high concentrations spermidine and rilmenidine on APP over-expression}
Prior to treatment, growth medium was aspirated and the cell monolayer was washed twice with warm PBS. N2aSwe cells were treated with 5 mM butyric acid (BA) for 24 and 48 h to induce butyrate-inducible CMV promoter-driven APP transgene expression. 8 h prior to the end of the treatment, cells were incubated for further 8 h in combination medium containing 5 mM BA and 1 $\mu$M Spd and Ril or 10 $\mu$M Spd and Ril.

\section{Reductive capacity assay}
The colorimetric assay WST-1 was used to measure cell viability. This assay is frequently used to measure cell viability or cell proliferation based on the reduction of the tetrazolium salts into formazan crystals by mitochondrial dehydrogenase enzyme. Therefore, media with solubilized formazan crystals can be measured spectrophotometrically as an indicator of cellular viability. An increase in the absorbance signal is directly proportional to an increase in metabolically healthy cells, while its decrease indicates a decrease in viable cells due to cytotoxic effects of treatment compounds. 

Briefly, GT 1-7 cells were seeded at a density of 75,000 cells per well, while N2aSwe were seeded at 20,000 cells per well in 48 well dishes and cultured overnight in 200 $\mu$L media. The following day, media was aspirated and cells were treated as desired over the required period of time. Thereafter, 10 $\mu$L of WST-1 reagent was added to each well and the culture plates were incubated for 2 h protected from light. Subsequently, culture plates were placed in shaking incubator (37 degrees Celsius, 200 RPM) and gently shaken for 2 min to dissolve the formazan crystals. Plates were analysed spectrophotometrically at a wavelength of 595 nm with the KC Junior software using a universal micro plate reader (EL800, Bio-Tek Instruments Inc.). Absorbance values of the treated and untreated control cells were subtracted from the background control (medium containing WST-1). All results were expressed as a percentage of the untreated control.

\section{Western blot analysis}
For the western blot analysis, GT1-7 were seeded at 3,000,000 and N2aSwe at 100,000 into T75 flasks with 8 ml of growth media per flask. GT 1-7 cells were allowed to attach and proliferate for 48 h while N2aSwe were incubated for 24 h, both at 37 degrees Celsius, 5 \% CO2 environment before treatment.

\subsection{Protein extraction}
After appropriate treatments were completed, growth medium was aspirated and cells were immediately placed on ice and washed twice with ice cold 1x PBS. In order to extract total protein, cells were incubated for 5 min with 100 uL of radioprecipitation (RIPA) lysis buffer (\textbf{Addendum A}), which was supplemented with 1x complete Protease Inhibitor Cocktail (as per manufacturers protocol) and phosphatase inhibitors such as 1 mM phenylmethylsulfonyl fluoride (PMSF) (\#93482, Sigma Alrich), 1 mM Sodium Fluoride (NaF) (\#193270, Merck), and 1 mM Sodium Orthovanadate (Na3VO4) (\#S6508, Sigma Alrich). Cells were detached by a scrapping method. Cell lysates were sonicated on ice at 4 Hz for 8 sec and were left for 30 min on ice until the foam subsided and then centrifuged at 8000 RPM for 10 min. Subsequently, supernatant was added into pre-chilled Eppendorf tubes and stored at -80 degrees Celsius until further analysis.

\subsection{Protein determination and sample preparation}
Protein content of cell lysates was determined using a Direct Detect${\textsuperscript{\tiny{\textregistered}}}$ infrared spectrometer (DDHW00010-WW, Merck). Aliquots of 50$\mu$g/ $\mu$L protein diluted in 2:1 ratio in Laemli’s sample buffer (\textbf{Addendum A}), were prepared and stored at -80 degrees Celsius. Prior to loading onto the gels, samples were thawed on ice and boiled at 95 degrees Celsius for 5 min to denature proteins.

\subsection{Sodium-dodecyl-sulfate-polycrylamide gel electrophoresis (SDS-PAGE) and western blot analysis}\label{sec:SDS_PAGE}
Samples (50 $\mu$g/ $\mu$L protein) were loaded into the 4 - 20 \% polyacrylamide precast gels (CriterionTM TGX TM Midi protein gel, Biorad, 5671094). 5 $\mu$l of a pre-stained protein ladder (PM007-0500, Biocom Biotech) was loaded into the first well of each gel in order to orientate the gel and determine molecular weights of separated proteins. Proteins were separated at 100 V and 400 mA using Power Pac 300 (BioRad). Once completed, proteins were transferred to a 0.2 $\mu$m polyvinylidine fluoride (PVDF) membrane using Midi Trans-Blot Turbo Transfer Packs (\#170-4157, Biorad) and the BioRad Trans-Blot Turbo electrotransfer system (170-4155, BioRad,), which was programmed at 120 V and 400 mA for 7 min. To confirm that proteins were transferred correctly, membranes were visualized using the stain free blot protocol provided on the Chemi-DocTM MP system from BioRad. Subsequently, membranes were incubated in 5 \% fat-free milk made up in Tris-buffered saline/0.1 \% Tween 20 (TBS-T) (\textbf{Addendum A}), for 1 h using a bellydancer. Following blocking, membranes were washed 3x for 5 min with TBS-T and incubated with specific primary antibodies diluted in TBS-T (1:1000 or 1:5000) overnight at 4 degrees Celsius. The following day, membranes were washed 3x for 5 min with TBS-T and incubated in corresponding secondary antibodies, Goat anti-rabbit IgG or anti-mouse horseradish peroxidase conjugated secondary antibody (1:5000), gently shaking for 1 h at room temperature (RT). Following incubation, membranes were washed 3x for 5 min with TBS-T and exposed to Clarity${\textsuperscript{\tiny{\textregistered}}}$ ECL (Enhanced chemiluminescence) peroxide reagent kit (\#1705061, Biorad) for 2 min and protein bands were detected and captured using a Chemi-DocTM MP imaging system (BioRad). Detected band intensities were quantified and normalized against total protein using Bio-Rad Image Lab software.

\subsection{Membrane stripping and re-probing}
Membranes were washed 2x 10 min with TBS-T and incubated at RT in the stripping buffer (\textbf{Addendum A}) for 2x 15 min. Subsequently, membranes were rinsed twice in TBS-T, washed for 15 min in TBS-T and again washed for 3x for 5 min in 1x TBS-T. Once completed, the membrane was blocked and re-probed.

\section{Immunofluorescence}
\subsection{Sample preparation for confocal microscopy and SR-SIM}
GT1-7 cells were seeded at 150,000 cells per well on coverslips in 6 well dishes and cultured overnight in 2 ml of growth medium at 37 degrees Celsius, in a 5 \% CO\textsubscript{2} environment. The following day, cells were subjected to the desired treatment protocol. When required, LysoTracker${\textsuperscript{\tiny{\texttrademark}}}$ Red DND-99 at 75 nM was added to the treatment medium 2 h prior to the completion of the period. Next, growth medium was aspirated and cells were washed 3x 5 min with pre-warmed in 1x PBS-A (\textbf{Addendum A}), fixed in pre-warmed double strength fixative at a 1:1 ratio of 8 \% v/v formaldehyde (FA): growth medium for 15 min at 37 degrees Celsius, in a 5 \% CO\textsubscript{2} environment. Following fixation, cells were washed 3x for 5 min and permeabilized with 0.2 \%Triton X in 1x PBS-A for 2 min. Subsequently, cells were washed 3x for 5 min and blocked for 30 min in 5 \% v/v donkey serum in 1x PBS-A to prevent non-specific binding of antibodies. Blocking buffer was removed and cells were probed with specific primary antibodies (\Cref{tab:20_prim}); anti-p62 mouse, anti-LC3 rabbit, anti-LAMP2A mouse, anti-alpha ($\alpha$)/beta ($\beta$)-tubulin rabbit, anti-acetylated-$\alpha$ tubulin mouse, diluted in 3 \% BSA (\#10735078001, Sigma Aldrich) and incubated overnight at 4 degrees Celsius in a humidified chamber. Next, cells were washed 3x for 5 min and incubated with corresponding secondary antibodies diluted in 3 \% BSA for 50 min in a humidified chamber at RT. In this case, Alexa Fluor 488 donkey anti-mouse, Alexa Fluor 488 donkey anti-rabbit, Alexa Fluor 568 donkey anti-mouse, Alexa Fluor 568 donkey anti-rabbit, and Alexa Fluor 647 phalloidin secondary antibodies were used. Cells were counterstained with Hoechst 33342 for further 10 min and then washed 3x for 5 min. Coverslips were inverted and mounted onto glass slide using Dako${\textsuperscript{\tiny{\textregistered}}}$ fluorescent mounting media (\#S302380, Diagnostech), sealed with clear nail polish and stored at 4 degrees Celsius in a dark slide container until image acquisition.

\subsection{dSTORM sample preparation}
In brief, GT1-7 and N2aSwe cells were seeded at a density of 150,000 cells per well using matTek culture dishes with glass bottom and cultured overnight. Thereafter, cells were treated as desired over the required period of time. Following completion of treatment, growth medium was aspirated and cells were washed, fixed and prepared for immunofluorescence. Specifically, anti-acetylated-$\alpha$ tubulin mouse, anti-APP rabbit, Alexa Fluor 568 donkey anti-mouse and Alexa Fluor 568 donkey anti-rabbit were used. After secondary antibody incubation was complete cells were fixed in 4 \% v/v FA for 10 min in order to stabilize fluorophore labelling. Subsequently, fixative was aspirated and cells were incubated in PBS-G (\textbf{Addendum A}) for 2 min to terminate fixation, washed 3x for 5 min and stored in PBS-A until image acquisition. 

\section{Transfection}\label{sec:transfection}
The electroporation (Neon, Thermofischer) or chemical transfection [Lipofectamine 3000 transfection reagent (Invitrogen, \#L3000008)] was used to transfect cells using GFP-LC3-RFP-LC3$\Delta$G or mRFP-GFP-LC3 DNA plasmid into GT 1-7 cell. Manufactures instructions for electroporation and Lipofectamine 3000 transfection reagent were used as a point of departure for transfection optimisation. Different conditions such as cell density, adherent or cell suspension conditions, temperature of transfection buffers and amount of plasmid DNA in using a 6 well plate and T25 culture flask were tested. The Olympus${\textsuperscript{\tiny{\textregistered}}}$ IX81 inverted widefield fluorescence microscope (Olympus${\textsuperscript{\tiny{\textregistered}}}$ Biosystems, GMBH Japan) at 10x objective was used to determine the transfection efficiency.

Once optimised, the following protocols were implemented for the T25 ml flask, 1,000,000 cells exposed to a total of 10 $\mu$g of plasmid DNA were used while for a 6 well plate, 200,000 cells with 2 $\mu$g was were employed. Briefly, when following the electroporation protocol, appropriate number of cells were aliquoted out, washed in PBS and spun down. The supernatant was discarded, and the pallet of cells was re-suspended in 8.34 or 41.7 $\mu$L of sterile Neon${\textsuperscript{\tiny{\textregistered}}}$ resuspension buffer containing 1.66 or 8.3 $\mu$L of GFP-LC3-RFP-LC3$\Delta$G or mRFP-GFP-LC3 DNA plasmid, a total of 2.0 or 10 $\mu$g DNA plasmid. Subsequently, the cell suspension was pipetted into a gold plated Neon${\textsuperscript{\tiny{\textregistered}}}$ Tip using a Neon${\textsuperscript{\tiny{\textregistered}}}$ Pipette, with the tip inserted into Neon${\textsuperscript{\tiny{\textregistered}}}$ Electrolytic buffer inside the Neon${\textsuperscript{\tiny{\textregistered}}}$ Pipette Station. Following this process, cells were electroporated at 1350 V for 1 pulse lasting for 30 ms and plated into T25 flasks or 6 well dish containing fresh media (1:1 DMEM/OptiMEM with 10\% FBS) without antibiotics and incubated 48 h to allow transfection to take place.

Following the chemical transfection protocol and after removing the supernatant, the pallet of cells was re-suspended and cells seeded at a density of 200 000 per well in 6 well dishes in media (1:1 DMEM/OptiMEM) containing 10 \% FBS without antibiotics and transfected immediately in suspension. Transfections were carried out in OptiMEM using a total of 2$\mu$g plasmid DNA per well. Briefly, two mixtures were prepared, one containing optiMEM with 2 $\mu$g of DNA and 4 $\mu$l P3000 reagent to a total of 125 $\mu$l and and another mixture containing 121.5 $\mu$l optiMEM with 3.75 $\mu$l of Lipofectamine 3000 reagent.  These solutions were then mixed together and incubated 15 min at RT prior to adding to wells in a drop-wise manner. Cells were incubated for 48 h. 

\section{Fluorescence Microscopy}
All images were acquired at the Central Analytical Facility (Stellenbosch University) using a Carl Zeiss laser scanning confocal microscope (LSM) 780 equipped with the ELYRA PS1 super-resolution and PALM/STORM platform (Carl Zeiss, Germany) in order to harness the distinct resolving power of each component, i.e. achieving 80 nm (SR-SIM) or 20 nm (PALM/STORM) for a given application.

\subsection{Confocal Laser Scanning Microscopy (LSM)}
Fluorescently labelled and transfected cells were acquired and processed using Zen 2011 imaging software (Carl Zeiss, Germany). Micrographs were collected in a z-stacking mode, acquiring 8 - 10 image frames with increments of 0.60 $\mu$m step width were routinely acquired. Depending on the type of analysis, a LCI Plan-Apochromat 63x/1.4 Oil DIC M27 or Alpha Plan-Apochromat 100x/1.46 Oil DIC M27 objective lens was used. Hoechst, Alexa 488, Alexa 568 and Alexa 647 and LysoTracker were excited with a suitable range of lasers, (405 nm, 488 nm, 561 nm and 633 nm) and signal detected with a GaAsP detector (32$\pm$2 PMT). Beam splitters (MBS 458, MBS 488, MBS 458/514, MBS 488/561, MBS 488/561/633) were selected accordingly depending on the laser line used. Z-stacks were projected as maximum intensity using the Zeiss Zen Black Software (2012) and used for subsequent analysis using FIJI.

\subsection{Super Resolution-Structured Illumination Microscopy (SR-SIM)}
Cells labelled for acetylated-$\alpha$ tubulin and -$\alpha$/$\gamma$ tubulin were imaged using SR-SIM on the Elyra PS1 platform (Carl Zeiss, Germany). Beads were acquired in a z-stacking mode at 5 phases and 3 rotations  of the illumination grid using Alpha Plan-Apochromat 100x/1.46 Oil DIC M27 objective lenses and 405, 488 and 561 laser lines in order to correct for chromatic aberration using the channel alignment algorithm. Subsequently, images were processed and aligned using the ZEN Black Elyra edition software (Carl Zeiss Microscopy). Samples were acquired using the above-mentioned settings and processed for SIM, corrected for channel alignment and projected to maximum intensity. 

\subsection{d-STORM}
Single molecule microscopy of cells labelled for APP and acetylated-$\alpha$ tubulin was performed. Molecules were activated using super resolution Abbelight buffer (a kind gift from Pierre Bauër, Abbelight) or 1 M MEA buffer (M9768, Sigma Aldrich) (\textbf{Addendum A}). Blinking events were acquired using an Alpha Plan-Apochromat 100x/1.46 Oil DIC M27 objective lens using a 561 nm laser and fluorescence was detected with the EMCCD camera. Drifting was minimised using nitrogen gas and definite focus. Imaging was performed in TIRF-uHP mode, 100 \% laser power, camera integration of 33ms and EM gain of 150 initially in order to bleach the sample and force fluorephores to enter the dark state. A recording of 50 000 events were collected per cell. After fluorescence was bleached, the laser power was reduced to 2 \% while recording the events. Raw data images were processed using a ZEN Black software where data was corrected for outliers, drifting, grouping and rendering. For APP analysis, images were further processed using FIJI analysis software for cluster analysis. 

\section{Transmission Electron Microscopy (TEM)}
\subsection{Sample preparation}
Cells were seeded at a density of 1 000 000 cells using a T25 flasks with 4 ml of growth media per flask and allowed to attach and proliferate for 48 h at 37 degrees  celsius, in a 5 \% CO\textsubscript{2} environment. Next, cells were submitted to the treatment regimen described (\Cref{sec:Effect_low_high_Spd_Ril_autophagic flux}). Thereafter, growth medium was aspirated and the monolayer of cells was rinsed 2x with pre-armed PBS and cells were trypsinized. Following trypsinization procedure, pellets were washed 2x in pre-warmed PBS and resuspended in 2.5 \% v/v glutaraldehyde (GA), pelleted at 1500 RPM for 3 min and stored at 4 degrees Celsius overnight. The following day, supernatant was removed, cells resuspended and washed 3 x for 5 min with 0.1 M phosphate buffer (PB), pH 7.4 (\textbf{Addendum A}) and pelleted in between each wash. Pellets were embedded in 2 \% low melting agarose (Merck, SA) which was allowed to solidify on ice and then cut into small pieces (0.2 x 0.2 mm). Subsequently, the sample was fixed in 1 \% osmium tetroxide (SPI) for 1 h to fix lipids contained in cells. Following fixation, cells were washed 1x 5 min in PB and 2x distilled water. After washes were complete, samples were placed in cassettes and underwent through dehydration, substitution and embedding using an automated tissue processor (Leica Biosystems), pre-programmed as follows; 30 min in 2 \% uranyl acetate in 70 \% ethanol, 5 min x 2 in 70 \% ethanol, 5 min in 90 \% ethanol, 10 min in 2 \% uranyl nitrate in 96 \% ethanol, 10 min in 100 \% ethanol x 3, 90 min in 1:1 Spurs resin: 100 \% ethanol, 2 x 1 hour pure resin. After the cycle was complete, the sample was embedded in resin using capsules and polymerized at 60 degrees Celsius in the oven for 48 h.

\subsection{Sectioning and EM image acquisition}
Resin blocks were trimmed and thin sections containing sample were cut at 70 nm using a Leica EM UC7 ultramicrotome (Leica Microsystems,Germany). Thin sections were picked up on 200 nm mesh copper grids (Advanced laboratory solutions) and image acquisition was performed using the JEOL JEM 1011 TEM (JEOL, Inc., Peabody, MA) at 100 kV. A total of 10 random regions were acquired per treatment group, per experiment.  

\subsection{Quantitative micrograph analysis using FIJI}
Morphometric analysis of autophagic vacuoles (AVs) \citep{Kawaoka2017,Swanlund2010} was performed using FIJI, an open source image analysis program as illustrated. Briefly, TEM TIFF images with scale bars inserted were opened in FIJI. For morphometric analysis, images were converted to 8 bit as follows: Image > Type > 8 bit. Thereafter, an outline was drawn around the AVs using a drawing pen and processed as follows. Firstly, the image was calibrated by demarcating the scale bar using a line drawing tool as follows: Analyze > Set scale and the pop up box for setting the scale appears where the known distance and unit of length was inserted.Once the image was calibrated, the entire cell without the scale bar was selected using the rectangle box function and duplicated as follows: Image > duplicate. Thereafter, a threshold was set on the duplicated image using default settings with the black and white background as follows: Image > Adjust> Threshold. The default settings, black and white (B \& W) and dark background were used and the grey values were set at 255 so that pixels with grey levels under a specified threshold are displayed as black pixels, and those above as white pixels, thus demarcating the structures of interest from the background. The number and area ($\mu$m\textsuperscript{2}) of AVs were analysed using the Particle analysis plug-in as follows: Analyse > Analyse particles. The data obtained was analysed using Graphpad prism. 

\section{Correlative Light and Electron Microscopy (CLEM)} 
CLEM combines the capabilities of two different techniques, light microscopy using fluorescence to indicate specificity of e.g. a labelled protein of interest and electron microscopy for the visualization of subcellular ultrastructural detail in a single platform, thus allowing imaging of dynamic biological events and to perform structural analysis at high resolution \citep{Russell2017}. Therefore two different approaches of sample preparation are required and are described below (\Cref{fig:20_CLEM_workflow}). 

\subsection{Sample preparation for fluorescence microscopy}
GT 1-7 cells were transfected with a GFP-LC3-RFP-LC3$\Delta$G DNA plasmid as previously described (\Cref{sec:transfection}), trypsinized and seeded in a grid patterned matTek culture dish with glass bottom, cultured overnight. After overnight culture, cells were treated as described in \Cref{sec:Effect_low_high_Spd_Ril_autophagic flux}. Treatment groups included a (1) Control, (2) BafA1 (4 h), (3) 1 $\mu$M Spd (8 h), (4) 1 $\mu$M Spd (8 h) + BafA1 (4 h), (5) 10 $\mu$M Spd (8 h) \& (6) 10 $\mu$M Spd (8 h) + BafA1 (4 h). After treatment was complete, cells were washed 3x 3 min in 0.1 M PBS-A, fixed in pre-warmed double strength fixative with 1:1 ratio of 8 \% v/v formaldehyde: growth medium for 15 min at 37 degrees Celsius, maintained at 5 \% CO\textsubscript{2}. Next, fixative was removed and cells were washed with 0.1 M PB 5x for 3 min and imaged immediately in PB using a confocal microscopy (\Cref{fig:20_CLEM_workflow}: \textbf{A}). Cells were imaged with a EC “Plan-Neofluar”10x/0.3 M27 using selected filters and a T-PMT to acquire a wide field of view that is inclusive of the grid, using a 4 x 4 or 2 x 2 tile scan. Thereafter, cells of interest in the selected region were imaged at a higher magnification using a LCI "Plan-Apochromat" 63x/1.4 Oil DIC M27 or Alpha Plan-Apochromat 100x/1.46 Oil DIC M27 objective in a z-stacking mode using LSM and SR-SIM respectively. 

\subsection{Sample preparation for Scanning Electron Microscopy (SEM)}\label{sec:sample preparation for SEM}
Following completion of the fluorescence microscopy acquisition, PB was removed and samples were prepared for SEM imaging using a backscattered electron (BSE) detector, allowing TEM-like signal generation. A protocol which involves the use of heavy metal to add contrast into the samples was used \citep{Russell2017}. This process included critical steps such as fixation (to preserve ultrastructure), contrast with heavy metals (to increase conductivity during imaging), dehydration (to remove water in the sample), substitution (to remove ethanol to avoid shrinkage), embedding and polymerization. Briefly, cells were fixed in buffer containing 2.5 \% gluteraldehyde and 4 \% formaldehyde in PB (\textbf{Addendum A}) for 30 min, at RT. Subsequently, cells were washed in PB 5x 3 min on ice and then incubated in 2 \% reduced osmium (2 \% OsO4, 1.5 \% K3Fe(CN) (\textbf{Addendum A}) for 60 min, on ice. Once complete, cells were washed in dH\textsubscript{2}O 5x 3 min, at RT to make sure that all osmium was removed. Thereafter, cells were incubated in 1 \% TCH (\textbf{Addendum A}) for 20 min, at RT, washed in dH\textsubscript{2}O 5x 3 min at RT. Once completed, cells were incubated in 2 \% aqueous OsO4 in dH\textsubscript{2}O (\textbf{Addendum A}) for 30 min, at RT, washed again in dH\textsubscript{2}O for 5x3 min, at RT. Lastly, cells were incubated in 1 \% uranyl acetate (\textbf{Addendum A}) at 4 degrees Celsius, overnight.

Next, cells were washed in dH\textsubscript{2}O, 5x3 min, at RT, and incubated in lead aspartate for 30 min, at 60 degrees Celsius. Thereafter, cells were washed in dH\textsubscript{2}O, 5x3 min, at RT. Subsequently, the coverslips containing cells in the MatTek dishes were detached using a razor blade and placed in aluminium foil dishes containing dH\textsubscript{2}O to avoid cells drying. Thereafter,  cells were dehydrated on ice, using a pre-chilled alcohol series in the following order; 20 \%, 50 \%, 70 \%, 90 \%, 100 \% EtOH, 5 min each, on ice, anhydrous 100 \% EtOH, 5 min, on ice, followed by 10 min, RT, while making sure that the cells are not left without ethanol and the times are not extended to avoid cell shrinkage. Once the dehydration was complete, cells were incubated in 50/50 propylene oxide/Durcupan for 60 min. Thereafter, cells were incubated in pure Durcupan for 90 min , twice. Lastly, cells were embedded in the metal dish by filling up the metal dish with pure Durcupan (\textbf{Addendum A}) for 3D imaging or by inverting a capsule in the position of the ROI for 2D imaging, and dishes were incubated for 48 h at 60 degrees Celsius in the oven to polymerize the resin.

\subsection{Preparation for Image acquisition using electron microscopy (EM)} 
Coverslips were removed from the polymerised resin block with liquid nitrogen resulting in the monolayer of cells positioned at the blockface, over-layered with grid pattern (\Cref{fig:20_CLEM_workflow}: \textbf{B}). 


For 2D acquisition and analysis, the polymerized resin containing cells was trimmed with a razor blade to produce give a square of 2 x 2 mm in a manner that would position the cell of interest in the centre of the blockface (\Cref{fig:20_CLEM_workflow}: \textbf{B}). The sample was sectioned at 70 nm using an ultra-microtome. Thin sections were picked up on conductive wafers which were mounted on the sample holder, inserted into the SEM chamber (Zeiss MERLIN SEM) and pumped to ∼5 Pa. The cell of interest was relocated and imaged at 5 kV using the BSE detector. 

For 3D analysis, the ROI was trimmed in a manner that would position the cell of interest in the centre of the blockface (\Cref{fig:20_CLEM_workflow}: \textbf{B}) and cut from the block to produce a square pyramid of 2 mm height with a face of ∼500 × 500 $\mu$m (\Cref{fig:20_CLEM_workflow}: \textbf{C}). The squared pyramid was mounted onto an aluminium pin using conductive epoxy glue which assists in charge dissipation during imaging (\Cref{fig:20_CLEM_workflow}: \textbf{D}) with FIB-SEM. FIB SEM uses a gallium ion beam to sputter slices of material from the blockface and the revealed surface is imaged using a BSE detector. An average of 800 -1000 image frames per cell were acquired in that manner, allowing the reconstruction of a highly resolved image volume. 

\begin{figure}[!htbp]
  \includegraphics[width=\linewidth]{figures/chapter20/20_CLEM_workflow}
  \caption[CLEM workflow]{\textbf{CLEM workflow.} Shown in the diagram is the sample preparation and imaging with fluorescence microscopy (\textbf{A}), relocating the cell of interest in the polymerized resin block after resin embedding (\textbf{B}), cutting of the region of interest (\textbf{C}), and mounting of the block on an aluminium pin using a conductive epoxy glue and inserting the sample on the sample holder inside the SEM (\textbf{D}) \citep{Russell2017}.}
  \label{fig:20_CLEM_workflow}
\end{figure}


\section{Flow cytometry analysis}
\subsection{Reactive oxygen species (ROS) analysis}
For ROS analysis, GT 1-7 cells were seeded at a density of 1,000,000 in T25 flasks and incubated overnight at 37 degrees Celsius, at 5 \% CO\textsubscript{2}. Following overnight culture, growth medium was removed and the cell monolayer was rinsed twice with warm PBS. Cells were treated as described previously (\Cref{sec:Effect_low_high_Spd_Ril_PQ_toxicity}) and thereafter trypsinized following the protocol described in \Cref{sec:Passaging_sub-culturing_GT1-7_N2aSwe_experiments}. Following trypsinization, the pellet was washed gently in warm PBS and re-suspended in PBS containing 50 $\mu$M DCF or 5 $\mu$M MitoSOX Red, indicators for cytosolic hydrogen peroxide (H\textsubscript{2}O\textsubscript{2}) and mitochondrial superoxide (O\textsubscript{2}\textsuperscript{-}), respectively. Cells were incubated 37 degrees Celsius, protected from light for 10 min. Carbonyl cyanide m-chlorophenyl hydrazone (CCCP) at 5 $\mu$M was used as a positive control. Cells were acquired on the BD FACSAria IIu flow cytometer. A minimum of 20,000 events were collected using a 488 nm laser and 530/30 bandpass filter for DCF, while 610/20 bandpass filter was used for MitoSox. The mean fluorescence intensity was assessed from three independent experiments. 


\subsection{PI exclusion}
Following treatments, trypsinization and washing of GT 1-7 cells as described above, cells were incubated with 1 $\mu$g/ml PI solution for 10 min at 37 degrees Celsius covered in foil and cells were analysed on the flow cytometer. A minimum of 20,000 events were collected using a 488 nm laser and 610/20 bandpass filter. The mean fluorescence intensity was assessed from three independent experiments. 

\section{\textit{In vivo} study: GFP-LC3 transgenic mouse model}
\subsection{Ethical approval, animal care and grouping}
Eight week old male C57BL/6 mice expressing GFP-LC3 were used in this study. GFP-LC3 transgenic mice breeding pair was purchased from Riken, Bio-Resource Centre in Japan (http://www.brc.riken.jp/lab/animal/en/dist.shtml) (\#BRC00806). Mice were bred at the breeding facility at Stellenbosch University, Tygerberg campus according to the previously described protocol \citep{Mizushima2009,Mizushima2004a}. GFP-LC3 protein is a widely used as marker for autophagosome formation, thus its expression in the whole animal is used to used to monitor autophagy in different tissues \citep{Mizushima2009}. One week prior to the study commencement, animals were transported to the animal facility at Stellenbosch University, main campus where upon arrival; they were randomly allocated into six groups of 12 mice per group. Mice were kept on a 12 h day/night cycle at a constant temperature of 22 degrees Celsius and 40 \% humidity and were allowed to acclimatize to this environment for one week before study begin. Mice were monitored and weighed daily for the duration of the study. All animals had free access to standard chow and tap water \textit{ad libitum}. The use of animals for this study was approved by the animal research ethics committee (REC), at Stellenbosch University (SU-ACUD16-00175). This study conformed to the guidelines for the care and use of laboratory animals implemented at Stellenbosch University.

\subsection{Experimental treatments and design}
A total of 72 GFP-LC3 mice were allocated randomly to the following treatment groups: (1) control, (2) PQ, (3) 0.3 mM Spd, (4) 0.3 mM Spd + PQ, (5) 3 mM Spd and (6) 3 mM Spd + PQ as illustrated in \Cref{fig:20_animal_treatment} , with each group consisting of 12 animals. Animals in the control, PQ and the combination groups (0.3 mM Spd + PQ and 3 mM Spd + PQ) were subjected to a total of 6 injections of saline solution (\textbf{Addendum A}) or PQ solution at 10 mg/kg (\textbf{Addendum A}) which was carried out intraperitoneally over a period of 3 weeks \citep{Chen2012}. Animals in the spermidine group and the combination groups were given spermidine at 0.3 mM and 3 mM in drinking water every day for a total duration of 21 days. Every 3 days, spermidine was prepared fresh in tap water. All animals were sacrificed 3 days after the last injection.

\begin{figure}[!htbp]
  \includegraphics[width=\linewidth]{figures/chapter20/20_animal_treatment}
  \caption[GFP-LC3 mice experimental treatment scheme]{\textbf{GFP-LC3 mice experimental treatment scheme.} Mice were divided into groups and treated with (1) saline solution and (2) PQ solution (top), (3) 0.3 mM Spd and (4) 3 mM Spd (middle), (5) 0.3 mM Spd plus PQ and (6) 3 mM Spd plus PQ (bottom.}
  \label{fig:20_animal_treatment}
\end{figure}

\section{Animal sacrifice and brain tissue extraction}
After the treatments were completed, mice were sacrificed using a decapitation method and whole brain tissue was extracted immediately and microdissected for further experiments. Brains were sectioned in the mid-sagittal plane to produce two hemispheres and further dissected into hippocampus and cerebral cortex, rinsed in ice cold PBS and snap frozen in liquid nitrogen and stored at -80 degrees Celsius for western blot analysis. Moreover, 7 hemispheres per group were fixed in 2.5 \% v/v gluteraldehyde and stored at 4 degrees Celsius for TEM analysis while 4 hemispheres per group were dissected into regions of interest (hippocampus and cerebral cortex), rinsed in ice cold PBS and snap frozen in cold iso-pentane placed in liquid nitrogen and stored at -80 degrees Celsius for fluorescence and light microscopy analysis.

\section{Western blot analysis}
\subsection{Tissue protein extraction and protein determination}
Prior to protein extraction, brain tissue from the hippocampus and cerebral cortex were thawed on ice. Next, 200 $\mu$L of total of RIPA lysis buffer supplemented with 1x complete Protease Inhibitor Cocktail (as per manufacturers protocol), 1 mM PMSF, 1 mM NaF, and 1 mM Na3VO4 was added. Total protein was extracted by homogenizing each sample on ice at 2100 RPM using the blade of the homogenizer, which was rinsed with water between samples to prevent cross contamination. Tissues were homogenised for 10 sec 3x with 10 sec pause intervals. Samples were left on ice for 1 h and then centrifuged 20 000 RPM for 30 min x 2 and again for 1 h at 4 degrees Celsius. The supernatant was carefully removed and transferred into a new Eppendorf on ice and samples were analysed for protein content using a Direct Detect${\textsuperscript{\tiny{\texttrademark}}}$ infrared spectrometer (DDHW00010-WW, Merck). Sample aliquots of 50 $\mu$g/ $\mu$L protein diluted in 2:1 ratio in Laemli's sample buffer were prepared and stored at -80 degrees Celsius. Before loading onto the gels, samples were thawed on ice and boiled at 95 degrees Celsius for 5 min to denature proteins. SDS-PAGE and western blot analysis was performed for all samples as previously described (\Cref{sec:SDS_PAGE}). 

\section{Sectioning of brain tissue}
Prior to sectioning, snap frozen tissue samples were orientated and mounted on a small piece of flat cork board using tissue freezing medium (\#14020108926, SMM instruments) and were kept at -20 degrees Celsius in the cryostat (Leica Biosystems) for 1 h before sectioning. Once the samples had equilibrated to -20 degrees Celsius, thin sections were cut at 10 $\mu$m and picked up on poly-lysine coated glass slides for immunofluorescence and light microscopy.

\section{Fluorescence microscopy preparation and imaging}
Sections on the slides were fixed in 4 \% v/v PFA for 10 min at RT. After fixation, the slides were washed in 10 mM 1x PBS (pH 7.4) 3x for 5min each and permeabilized for 15 min with 0.2 \% Triton X in PBS. Subsequently, tissue sections were blocked for non-specific binding of antibodies for 30 min using 5 \% v/v donkey serum made up in PBS. Blocking buffer was removed and tissue sections were probed for p62, 4 Hydroxynonenal (4HNE) and GFAP using dilution specified in \Cref{tab:20_prim}. Nuclei was counterstained with 10 $\mu$g/ml Hoechst stain for 10 min. Thereafter, sections were washed 3x 5 min and blotted with tissue paper to remove excess PBS. Coverslips were mounted onto the glass slides using Dako${\textsuperscript{\tiny{\texttrademark}}}$ fluorescent mounting media and sealed with clear nail polish. Slides were stored at -20 degrees Celsius in a dark slide container until image acquisition was performed, using the Carl Zeiss LSM 780 confocal microscope (Carl Zeiss, Germany). Lasers and filters were set accordingly and images were captured using EC "Plan-Neofluar"10x/0.3 M27 and LD "Plan-Nuofluar" 40x/0.6 Corr M27 objectives.

\section{Light microscopy preparation and imaging}
Following sectioning, tissue samples were stained for structural damage using haematoxylin eosin (H\&E) staining. Samples were subjected to the following staining procedure: xylol, 95 \% and 70 \% absolute ethanol, distilled water, Harris haematoxylin, distilled water, acid alcohol, distilled water, Scott's tap water, distilled water, eosin, distilled water, 70 \%, 95 \% and absolute ethanol followed by xylol. Slides were mounted using mounting media and micrographs were acquired using the Nikon Eclipse E400 phase contrast microscope (Nikon, USA) equipped with a RT Colour Spot Camera (Diagnostics Instruments Inc., USA). 

\section{Statistical Analysis}
The results are expressed as mean values $\pm$ SEM and were analysed by one-way Analysis of Variance (ANOVA) with the Fischer LSD post hoc correction. Graph Pad Prism (v8) was employed to perform statistical tests. Data were considered to be statistically significant with a \textit{p} value $<$ 0.05.