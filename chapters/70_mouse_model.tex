\chapter{The ffect of spermidine on PQ induced brain injury model of Alzheimer’s disease}
\section{Introduction}
In order to assess the role of autophagy enhancement in an \textit{in vivo} scenario, a PQ induced neuronal injury model was employed \citep{Chen2012}, using a GFP-LC3 transgenic mice \citep{Mizushima2004a}. Specifically, since the concentration dependent effect of autophagy induction was of major interest, this part of the work was focussing on spermidine at two distinct concentrations. In this context, we focused primarily on the brain regions that are highly susceptibility to neuronal injury as we previously reviewed in detail \citep{lumkwana2017}. These regions include the hippocampus and the cortex which are impacted in early and late stages of AD respectively and have been shown to have manifest with major pathological changes, including cell loss. So far, the role of spermidine in a mouse model of PQ-induced injury has not been assessed. Therefore, the aims of this chapter were to assess the extent of PQ induced toxicity in terms lipid peroxidation and to assess the role of spermidine in protein clearance, tubulin acetylation and subsequent potential protection. A total of 72 mice expressing GFP-LC3 were randomly selected into 6 groups of 12 consisting of (1) control (saline solution), (2) PQ, (3) 0.3 mM Spd, (4) 0.3 mM Spd + PQ, (5) 3 mM Spd and (6) 3 mM Spd + PQ. The study was conducted for 21 days in which saline and PQ was administered every 3 days for 21 days \citep{Chen2012}, with a total of 6 injections, while other animals had ad libitum access to spermidine in the drinking water for the duration of the treatment. After 21 days, animals were sacrificed and brain was extracted, weighted and dissected into different regions, namely hippocampus and cortex and prepared accordingly for H\&E, western blotting, and immunofluorescence. Markers of oxidative stress (4HNE), autophagy (LC3 II, p62, LAMP2a), brain injury [glial fibrillary acidic protein (GFAP) and amyloid precursor protein (APP)] and microtubule stability (acetylated $\alpha$-tubulin) were assessed. 

\section{Effect of treatment intervention on brain weight}
In order to assess the effect of the treatment intervention on brain weight, cerebral tissue was weighed immediately after animal sacrifice and brain extraction. Results show a significant increase in the brain weight in the PQ treated group (1.08 $\pm$ 0.02, \textit{p} < 0.05), 0.3 mM Spd (1.21 $\pm$ 0.04, \textit{p} < 0.05), 0.3 mM Spd + PQ (1.20 $\pm$ 0.01, p < 0.05), 3 mM Spd (1.16 $\pm$ 0.01, \textit{p} < 0.05) and 3 mM Spd + PQ (1.19 $\pm$ 0.04, \textit{p} < 0.05) compared to the control group (1.00 $\pm$ 0.01) (Fig.7.1). In addition, a significant increase in brain weight was also observed in the 0.3mM Spd, 0.3mM Spd + PQ, 3mM Spd and 3mM Spd + PQ compared to the PQ treated group. 

%figure
%figure references

\section{Effect of treatment intervention on hippocampus and cortex}
The hippocampus and cortex are regions susceptible to pathological changes particularly in PD (Braak et al., 2004, 2003, 1998), but also in AD (Braak et al., 2000, 1998; Braak and Braak, 1991; Braak and Del Tredici, 2012). Therefore, the effect of the treatment intervention in a PQ-induced brain injury model was assessed using Heamatoxylin and Eosin (H\&E), western blotting and immunofluorescence. Protein expression levels indicative of oxidative stress (4HNE), brain injury (GFAP and APP), autophagy activity (LC3 II, p62, and LAMP2A), and microtubule stability (acetylated $\alpha$-tubulin) were assessed in both brain regions and complemented with immunofluorescence.

%add bibliographical references (Braak)

\section{Hippocampus region}
\subsection{Heamatoxylin and Eosin (H\&E) staining}
In order to assess overall structural changes of the hippocampus following treatment intervention, H\&E staining was carried out. No major abnormalities or neuronal cell loss were observed between the treatment groups (Fig.7.2). No signs for architectural damage or protein inclusion were observed.

%figure
%figure references

\subsection{Lipid peroxidation (4HNE) assessment using western blotting}
The lipid peroxidation end product 4-hydroxy-2-nonenal (4HNE), a marker for oxidative stress was assessed in response to treatment intervention. Three distinct 4HNE protein adducts bands (35, 45, 52 kDa) were identified. At 35 kDa, 4HNE signal was significantly increased in the PQ treated group (1.84 $\pm$ 0.18, \textit{p} < 0.05), while significantly decreased in the 0.3 mM Spd (0.29 $\pm$ 0.04, \textit{p} < 0.05), 0.3 mM Spd + PQ (0.39 $\pm$ 0.08, \textit{p} < 0.05), 3 mM Spd (0.36 $\pm$ 0.09, \textit{p} < 0.05) and 3 mM Spd + PQ (0.57 $\pm$ 0.09, \textit{p} < 0.05) compared to the control group (0.88 $\pm$ 0.10) (Fig.7.3.A.i). Moreover, a significant decrease in 4HNE signal was also observed with 0.3 mM Spd, 0.3 mM Spd + PQ, 3 mM Spd and 3mM Spd + PQ (\textit{p} < 0.05) compared to the PQ treated group. 

Similarly at 45 kDa, 4HNE signal was significantly increased in the PQ treated group (1.41 $\pm$ 0.22, \textit{p} < 0.05), while significantly decreased in the 0.3 mM Spd + PQ (0.13 $\pm$ 0.03, \textit{p} < 0.05), 3 mM Spd (0.32 $\pm$ 0.05) and 3 mM Spd + PQ (0.23 $\pm$ 0.05) compared to the control (0.95 $\pm$ 0.13) (Fig.7.3.A.ii). Additionally, in comparison to the PQ treated group, a significant decrease in 4HNE expression was also observed in the 0.3 mM Spd, 0.3 mM Spd + PQ, 3 mM Spd and 3 mM Spd + PQ. Furthermore, a significant decrease was observed in the 0.3 mM Spd + PQ and 3 mM Spd (\textit{p} < 0.05) compared to 0.3 mM Spd.

Lastly at 53 kDa , no significant differences were observed in 4HNE signal in the PQ treated group (1.42 $\pm$ 0.34), 0.3 mM Spd (1.48 $\pm$ 0.22) and in 3 mM Spd (0.76 $\pm$ 0.22) compared to the control (1.00 $\pm$ 0.12), however, a significant decrease in 4HNE signal was detected in the 0.3 mM Spd + PQ (0.46 $\pm$ 0.15) and 3 mM Spd + PQ (0.45 $\pm$ 0.07) group (Fig.7.3.A.iii). In comparison to the PQ treated group, a significant decrease in 4HNE signal was revealed in the 0.3 mM Spd + PQ, 3 mM Spd and 3 mM Spd + PQ. Furthermore, a significant decrease was observed in the 0.3 mM Spd + PQ (\textit{p} < 0.05) compared to 0.3 mM Spd and in 3 mM Spd (\textit{p} < 0.05) compared to 0.3 mM Spd. 

%figure
%figure references

\subsection{4HNE assessment using fluorescence microscopy}
4HNE fluorescence intensity was assessed following treatment intervention using a confocal microscopy. Results show a strong fluorescence intensity signal in the PQ treated group to all other groups; control, 0.3mM Spd, 0.3mM Spd + PQ, 3mM Spd and 3mM Spd + PQ (Fig.7.4).

%figure
%figure references

\subsection{GFAP assessment using fluorescence microscopy} 
Glial fibrillary protein (GFAP) and its upregulation has been utilized as a marker for neuronal damage (Wu et al., 2015). Here, we assessed the role of spermidine in impacting GFAP upon PQ-induced damage. A strong fluorescence intensity signal for GFAP was observed in the PQ treated group with distinct glial derived signal compared to all other groups (Fig.7.5)

%add bibliographical references (Wu2015)
%figure
%figure references

\subsection{APP assessment using western blotting}
Next, APP expression which has been used as an indicator of neuronal damage was assessed following treatments using western blot analysis. Our results reveal that APP expression was significantly increased following treatment with PQ (2.60 $\pm$ 0.17) and 0.3 mM Spd (2.21 $\pm$ 0.13) compared to the control group (1.00 $\pm$ 0.57) (Fig.7.6). Of particular importance is a significant decrease in APP expression with 0.3 mM Spd + PQ (\textit{p} < 0.05), 3 mM Spd (\textit{p} < 0.05) and 3 mM Spd + PQ (\textit{p} < 0.05) compared to the PQ treated group. Furthermore, a significant decrease in APP expression was observed in the 0.3 mM Spd + PQ (\textit{p} < 0.05) and 3 mM Spd (\textit{p} < 0.05) compared to 0.3 mM Spd. 

%figure
%figure references

\subsection{Acetylated $\alpha$-tubulin assessment using western blotting}
In order to assess whether autophagy upregulation through spermidine may impact microtubule stability, acetylated $\alpha$-tubulin was assessed in response to treatment intervention. In comparison to the control group (1.00 $\pm$ 0.08), a significant decrease in tubulin acetylation was observed in the PQ treated group (0.68 $\pm$ 0.05, \textit{p} < 0.05) (Fig.7.7). Moreover, a significant increase in acetylated $\alpha$-tubulin was observed in the 0.3 mM Spd + PQ (1.11 $\pm$ 0.04, \textit{p} < 0.05), and 3 mM Spd + PQ (0.99 $\pm$ 0.13, \textit{p} < 0.05) compared to the PQ treated group (0.68 $\pm$ 0.05). Furthermore, a significant increase was observed in 0.3 mM Spd + PQ (1.11 $\pm$ 0.04, \textit{p} < 0.05) compared to 0.3 mM Spd (0.82 $\pm$ 0.05).

%figure
%figure references

\subsection{Autophagy activity assessment using western blotting}
In order to assess autophagic activity in response to treatment intervention in the hippocampus, LC3 II, p62 and LAMP2A protein levels were assessed using western blot and fluorescence microscopy. Results indicate that LC3 II protein expression was significantly increased in the 0.3 mM Spd + PQ (1.38 $\pm$ 0.11, \textit{p} < 0.05) and the 3 mM Spd + PQ (1.67 $\pm$ 0.12, \textit{p} < 0.05) compared to the control group (1.00 $\pm$ 0.06) and PQ treated group (0.91 $\pm$ 0.06) (Fig.7.8.A.i). Moreover, LC3 II protein levels were significantly increased in the 3 mM Spd + PQ (\textit{p} < 0.05) compared to the 0.3 mM Spd + PQ, with no significant difference observed between 0.3 mM Spd (1.17 $\pm$ 0.05) and 3 mM Spd (1.13 $\pm$ 0.08). Lastly, LC3 II expression was significantly increased in the 3 mM Spd + PQ compared to the 3 mM Spd, \textit{p} < 0.05.

In comparison to the control group (1.00 $\pm$ 0.07) and PQ treated group (0.71 $\pm$ 0.07), a significant increase in the p62 protein expression was observed in the 3 mM Spd (2.60 $\pm$ 0.72, \textit{p} < 0.05) and in 3 mM Spd + PQ (2.59 $\pm$ 0.45, \textit{p} < 0.05) (Fig.7.8.A.ii). In addition, 3 mM Spd exposure significantly increased p62 protein expression compared to 0.3 mM Spd (2.60 $\pm$ 0.72 vs 0.76 $\pm$ 0.29).

LAMP2A protein expression was significantly decreased in the PQ treated group (0.09 $\pm$ 0.02, \textit{p} < 0.05), while significantly increased in 0.3 mM Spd (2.43 $\pm$ 0.27, \textit{p} < 0.05), 0.3 mM Spd + PQ (1.92 $\pm$ 0.22, \textit{p} < 0.05) and 3 mM Spd + PQ (1.90 $\pm$ 0.35, \textit{p} < 0.05) compared to the control group (1.00 $\pm$ 0.33) (Fig.7.8.A.iii). Moreover, LAMP2A protein expression was significantly increased in 0.3 mM Spd (2.43 $\pm$ 0.27), 0.3 mM Spd + PQ (1.92 $\pm$ 0.22), 3 mM Spd (1.80 $\pm$ 0.32) and 3 mM Spd + PQ (1.90 $\pm$ 0.35) compared to the PQ treated group (0.09 $\pm$ 0.02).

%figure
%figure references

\subsection{GFP-LC3 assessment using fluorescence microscopy}
Results show that GFP-LC3 fluorescence intensity signal was strongly expressed in the groups where spermidine was present i.e. in the 0.3 mM Spd, 0.3 mM Spd + PQ, 3 mM Spd, and 3 mM Spd + PQ as well as in the control group compared to the PQ treated group (Fig 7.9).

%figure
%figure references

\subsection{p62 assessment using fluorescence microscopy}
A strong fluorescence intensity signal for p62 was observed in the 3 mM Spd and 3 mM Spd + PQ compared to PQ treated group and control group as shown in Fig.7.10.

%figure
%figure references

\section{Cortex region}
\subsection{Heamatoxylin and Eosin (H\&E) staining}
Qualitative analysis using H\&E staining showed no major abnormal changes in the structural architecture of the cortex upon treatment intervention (Fig.7.11).

%figure
%figure references

\subsection{Lipid peroxidation (4HNE) assessment using western blotting}
At 35 kDa, 4HNE protein adducts signal was significantly decreased in the 0.3 mM Spd (0.59 $\pm$ 0.09, \textit{p} < 0.05), 0.3 mM Spd + PQ group (0.47 $\pm$ 0.08, \textit{p} < 0.05) compared to the control group (1.00 $\pm$ 0.14) with no significant increase observed in the PQ treated group (1.08 $\pm$ 0.17) (Fig.7.12.A.i). Moreover, 4HNE signal was significantly decreased in the 0.3 mM Spd (\textit{p} < 0.05) and 0.3 mM Spd + PQ group (\textit{p} < 0.05) compared to the PQ treated group. 

At 45 kDa, 4HNE signal was significantly decreased in the 0.3 mM Spd (0.45 $\pm$ 0.06, \textit{p} < 0.05), 0.3 mM Spd + PQ (0.14 $\pm$ 0.05, \textit{p} < 0.05), 3 mM Spd (0.26 $\pm$ 0.04) and 3 mM Spd + PQ (0.18 $\pm$ 0.04, \textit{p} < 0.05) compared to the control (1.00 $\pm$ 0.08) and PQ treated group (1.05 $\pm$ 0.130), with no significant differences seen between the control and the PQ treated group (Fig.7.12.A.ii). Moreover, a significant decrease was observed in 0.3 mM Spd + PQ compared to 0.3 mM Spd (\textit{p} < 0.05) and in 3 mM Spd (\textit{p} < 0.05) compared to 0.3mM Spd, 
Lastly, at 52 kDa, a significant decrease in 4HNE signal was observed in the 0.3 mM Spd + PQ (0.14 $\pm$ 0.05, \textit{p} < 0.05), 3 mM Spd (0.15 $\pm$ 0.07) and 3 mM Spd + PQ (0.11 $\pm$ 0.05, \textit{p} < 0.05) group compared to the control (1.00 $\pm$ 0.22) and PQ treated group (0.99 $\pm$ 0.17) (Fig.7.12.A.iii). Moreover, 4HNE expression of was significantly decreased in the 0.3mM Spd + PQ (0.14 $\pm$ 0.05) and in the 3 mM Spd (0.15 $\pm$ 0.07) compared to 0.3mM Spd (0.62 $\pm$ 0.13).

%figure
%figure references

\subsection{4HNE signal assessment using fluorescence microscopy}
A moderately enhanced fluorescence signal for 4HNE was observed in the control and PQ treated group compared to 0.3mM Spd, 0.3mM Spd + PQ, 3mM Spd and 3mM Spd + PQ (Fig.7.13).

%figure
%figure references

\subsection{GFAP assessment using fluorescence microscopy}
Results indicate that GFAP fluorescence intensity signal was strongly enhanced in the PQ treated group compared to all other groups (Fig.7.14).

%figure
%figure references

\subsection{APP assessment using western blotting}
Although no significance differences was reached, a trend towards an increase in APP was observed following PQ treatment (1.57 $\pm$ 0.36), while a trend towards decreased APP signal was observed following treatment with 0.3 mM Spd (1.10 $\pm$ 0.22), 0.3 mM Spd + PQ (0.65 $\pm$ 0.19), 3 mM Spd (1.17 $\pm$ 0.25) and 3 mM Spd + PQ (0.70 $\pm$ 0.23) compared to the control group (0.58 $\pm$ 0.16) (Fig.7.15). Of particular importance is a significant decrease in APP expression with 0.3 mM Spd + PQ (\textit{p} < 0.05) and 3 mM Spd + PQ (\textit{p} < 0.05) compared to the PQ treated group. 

%figure
%figure references

\subsection{Acetylated $\alpha$-tubulin assessment using western blotting}
In comparison to the control (1.00 $\pm$ 0.03) and PQ treated group (0.87 $\pm$ 0.13), no significant differences were observed in the expression of acetylated $\alpha$-tubulin in the 0.3 mM Spd (0.95 $\pm$ 0.17), 0.3 mM Spd + PQ (0.80 $\pm$ 0.08) and 3 mM Spd (1.02 $\pm$ 0.07), however, a significant decrease was observed at 3 mM Spd + PQ (0.65 $\pm$ 0.12, \textit{p} < 0.05) compared to the control and PQ group (Fig.7.16). 

%figure
%figure references

\subsection{Autophagy activity assessment using western blotting}
No significant differences in LC3 II protein levels were observed in the PQ treated group (0.65 $\pm$ 0.03) compared to the control group (1.00 $\pm$ 0.11) (Fig.7.17.A.i). A significant increase in LC3 II expression was however observed in the 0.3 mM Spd + PQ (1.30 $\pm$ 0.27, \textit{p} < 0.05) and 3 mM Spd (1.23 $\pm$ 0.25, \textit{p} < 0.05) compared to the PQ treated group (0.65 $\pm$ 0.03). 

No significant differences were observed in the p62 expression following treatment with PQ (1.15 $\pm$ 0.11), 3 mM Spd (1.66 $\pm$ 0.21) and 3 mM Spd + PQ (1.56 $\pm$ 0.34) compared to the control  group (1.00 $\pm$ 0.06), however, a significant increase in p62 expression was observed in the 0.3 mM Spd (2.53 $\pm$ 0.34) and 0.3 mM Spd + PQ (2.43 $\pm$ 0.36, \textit{p} < 0.05) (Fig.7.17.A.ii). Moreover, a significant increase was also detected in the 0.3 mM Spd and 0.3 mM Spd + PQ group compared to PQ treated group. More importantly, a significant decrease in p62 expression was observed in the combination treatment (3mM Spd + PQ) compared to 0.3mM Spd + PQ as well as in the 3 mM Spd treated group compared to 0.3mM Spd. 

No significant differences were observed in LAMP2A protein expression in the PQ treated group (0.40 $\pm$ 0.13) compared to the control group (1.00 $\pm$ 0.18), however a significant increase in LAMP2A expression was observed with 0.3 mM Spd + PQ (2.02 $\pm$ 0.39, \textit{p} < 0.05) and 3 mM Spd (2.16 $\pm$ 0.53, \textit{p} < 0.05) compared to the control group (Fig.7.17.A.iii). Moreover, LAMP2A expression was significantly increased in 0.3mM Spd (1.85 $\pm$ 0.24), 0.3 mM Spd + PQ (2.02 $\pm$ 0.39), 3 mM Spd (2.16 $\pm$ 0.53) and 3mM Spd + PQ (1.47 $\pm$ 0.14) compared to the PQ treated group (0.40 $\pm$ 0.13).

%figure
%figure references

\subsection{GFP-LC3 assessment using fluorescence microscopy}
Fluorescence microscopy of the cortex showed that GFP-LC3 signal was enhanced in the groups 0.3 mM Spd + PQ and 3mM Spd (Fig.7.18).

%figure
%figure references

\subsection{p62 assessment using fluorescence microscopy}
A strong fluorescence intensity signal for p62 was observed in the PQ treated group compared to all other groups; 0. 3 mM Spd and 0.3 mM Spd + PQ (Fig.7.19).

%figure
%figure references

\section{Discussion: The effect of spermidine on a PQ-induced brain injury model of Alzheimer's Disease}



