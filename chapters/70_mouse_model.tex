\chapter{The effect of spermidine in PQ-induced brain injury}
\label{sec:chapter7}
\section{Introduction}
In order to assess the role of autophagy enhancement in an \textit{in vivo} scenario, a PQ-induced neuronal injury model was employed \citep{Chen2012}, using a GFP-LC3 transgenic mice \citep{Mizushima2004a}. Specifically, since the dose-dependent effect of autophagy induction was of major interest, this part of the work was focussing on spermidine at two distinct dosages. In this context, we focused primarily on the brain regions that are highly susceptibility to neuronal injury as we previously reviewed in detail \citep{lumkwana2017}. These regions include the hippocampus and the cortex which are impacted in early and late stages of AD respectively and have been shown to have manifest with major pathological changes, including cell loss. So far, the role of spermidine in a mouse model of PQ-induced injury has not been assessed. Therefore, the aims of this chapter were to assess the extent of PQ-induced toxicity in terms lipid peroxidation and to assess the role of spermidine in protein clearance, tubulin acetylation and subsequent potential protection. A total of 72 mice expressing GFP-LC3 were randomly selected into 6 groups of 12 consisting of (1) control (saline solution), (2) PQ, (3) 0.3 mM Spd, (4) 0.3 mM Spd + PQ, (5) 3 mM Spd and (6) 3 mM Spd + PQ. The study was conducted for 21 days in which saline and PQ solution was administered at 10 mg/kg every 3 days for 21 days \citep{Chen2012}, with a total of 6 injections, while other animals had \textit{ad libitum} access to spermidine in the drinking water for the duration of the treatment. After 21 days, animals were sacrificed and brain was extracted, weighted and dissected into different regions, namely hippocampus and cortex and prepared accordingly for H\&E, western blotting, and immunofluorescence. Markers of oxidative stress (4HNE), autophagy (LC3-II, p62, LAMP2A), brain injury [glial fibrillary acidic protein (GFAP) and amyloid precursor protein (APP)] and microtubule stability (acetylated $\alpha$-tubulin) were assessed. 

\section{The effect of spermidine treatment intervention on brain weight in PQ-induced brain injury}
In order to assess the effect of the treatment intervention on brain weight, cerebral tissue was weighed immediately after animal sacrifice and brain extraction. Results show a significant increase in the brain weight in the PQ treated group (1.08 $\pm$ 0.02, \textit{p} < 0.05), 0.3 mM Spd (1.21 $\pm$ 0.04, \textit{p} < 0.05), 0.3 mM Spd + PQ (1.20 $\pm$ 0.01, p < 0.05), 3 mM Spd (1.16 $\pm$ 0.01, \textit{p} < 0.05) and 3 mM Spd + PQ (1.19 $\pm$ 0.04, \textit{p} < 0.05) compared to the control group (1.00 $\pm$ 0.01) (\Cref{fig:70_pq_brain_weight_wholebrain}). In addition, a significant increase in brain weight was also observed in the 0.3 mM Spd, 0.3 mM Spd + PQ, 3 mM Spd and 3 mM Spd + PQ compared to the PQ treated group. 

\begin{figure}[!htbp]
\center
  \includegraphics[width=0.495\linewidth]{figures/chapter70/70_pq_brain_weight_wholebrain}
  \caption[The effect of spermidine treatment intervention on brain weight]{\textbf{The effect of spermidine treatment intervention on brain weight.} C57BL/6 mice expressing GFP-LC3 were randomly selected in six groups of 12 animals, control, PQ, 0.3 mM Spd, 0.3 mM Spd + PQ, 3 mM Spd and 3 mM Spd + PQ. Control animals were injected with saline solution. All results are presented as mean $\pm$ SEM, \textit{n} = 12.* = \textit{p} < 0.05 vs control, \# = \textit{p} < 0.05 vs PQ.}
  \label{fig:70_pq_brain_weight_wholebrain}
\end{figure} 

\section{The effect of spermidine treatment intervention on hippocampus and cortex in PQ-induced brain injury}
The hippocampus and cortex are regions susceptible to pathological changes particularly in PD \citep{Braak2004, Braak1998}, but also in AD \citep{Braak2004,Braak1998,Braak1991,Braak2012}. Therefore, the effect of the treatment intervention in a PQ-induced brain injury model was assessed using Heamatoxylin and Eosin (H\&E), western blotting and immunofluorescence. Protein expression levels indicative of oxidative stress (4HNE), brain injury (GFAP and APP), autophagy activity (LC3-II, p62, and LAMP2A), and microtubule stability (acetylated $\alpha$-tubulin) were assessed in both brain regions and complemented with immunofluorescence.

\section{Hippocampus region}
\subsection{The effect of spermidine treatment intervention using Heamatoxylin and Eosin (H\&E) in PQ-induced brain injury}
In order to assess overall structural changes of the hippocampus following treatment intervention, H\&E staining was carried out. No major abnormalities or neuronal cell loss were observed between the treatment groups (\Cref{fig:70_pq_hippo_he_lm}). No signs for architectural damage or protein inclusion were observed.

\begin{figure}[!htbp]
\center
  \includegraphics[width=\linewidth]{figures/chapter70/70_pq_hippo_he_lm}
  \caption[The effect of spermidine treatment intervention using H\&E staining in the hippocampus]{\textbf{The effect of spermidine treatment intervention using H\&E staining in the hippocampus.} Representative micrographs indicate control, PQ, 0.3 mM Spd, 0.3 mM Spd + PQ, 3 mM Spd, and 3 mM Spd + PQ. Images acquired using a 10x and 40x objective lens, \textit{n} = 6. Scale bar: 200 $\mu$m for lower magnification and 50 $\mu$m for higher magnification.}
  \label{fig:70_pq_hippo_he_lm}
\end{figure} 

\subsection{The effect of spermidine treatment intervention on lipid peroxidation (4HNE) in PQ-induced brain injury using WB}
The lipid peroxidation end product 4-hydroxy-2-nonenal (4HNE), a marker for oxidative stress was assessed in response to treatment intervention. Three distinct 4HNE protein adducts bands (35, 45, 52 kDa) were identified. At 35 kDa, 4HNE signal was significantly increased in the PQ treated group (1.84 $\pm$ 0.18, \textit{p} < 0.05), while significantly decreased in the 0.3 mM Spd (0.29 $\pm$ 0.04, \textit{p} < 0.05), 0.3 mM Spd + PQ (0.39 $\pm$ 0.08, \textit{p} < 0.05), 3 mM Spd (0.36 $\pm$ 0.09, \textit{p} < 0.05) and 3 mM Spd + PQ (0.57 $\pm$ 0.09, \textit{p} < 0.05) compared to the control group (0.88 $\pm$ 0.10) (\Cref{fig:70_pq_hippo_hne_wb}: \textbf{A}\textit{i} \& \textbf{B}). Moreover, a significant decrease in 4HNE signal was also observed with 0.3 mM Spd, 0.3 mM Spd + PQ, 3 mM Spd and 3 mM Spd + PQ (\textit{p} < 0.05) compared to the PQ treated group. 

Similarly at 45 kDa, 4HNE signal was significantly increased in the PQ treated group (1.41 $\pm$ 0.22, \textit{p} < 0.05), while significantly decreased in the 0.3 mM Spd + PQ (0.13 $\pm$ 0.03, \textit{p} < 0.05), 3 mM Spd (0.32 $\pm$ 0.05) and 3 mM Spd + PQ (0.23 $\pm$ 0.05) compared to the control (0.95 $\pm$ 0.13), with no significant differences observed in the 0.3 mM Spd (0.74 $\pm$ 0.12) (\Cref{fig:70_pq_hippo_hne_wb}: \textbf{A}\textit{ii} \& \textbf{B}). Additionally, in comparison to the PQ treated group, a significant decrease in 4HNE expression was also observed in the 0.3 mM Spd, 0.3 mM Spd + PQ, 3 mM Spd and 3 mM Spd + PQ. Furthermore, a significant decrease was observed in the 0.3 mM Spd + PQ and 3 mM Spd (\textit{p} < 0.05) compared to 0.3 mM Spd.

Lastly at 53 kDa , no significant differences were observed in 4HNE signal in the PQ treated group (1.42 $\pm$ 0.34), 0.3 mM Spd (1.48 $\pm$ 0.22) and in 3 mM Spd (0.76 $\pm$ 0.22) compared to the control (1.00 $\pm$ 0.12), however, a significant decrease in 4HNE signal was detected in the 0.3 mM Spd + PQ (0.46 $\pm$ 0.15) and 3 mM Spd + PQ (0.45 $\pm$ 0.07) group (\Cref{fig:70_pq_hippo_hne_wb}: \textbf{A}\textit{iii} \& \textbf{B}). In comparison to the PQ treated group, a significant decrease in 4HNE signal was revealed in the 0.3 mM Spd + PQ, 3 mM Spd and 3 mM Spd + PQ. Furthermore, a significant decrease was observed in the 0.3 mM Spd + PQ (\textit{p} < 0.05) compared to 0.3 mM Spd and in 3 mM Spd (\textit{p} < 0.05) compared to 0.3 mM Spd. 

\begin{landscape}
\begin{figure}[!htbp]
\center
  \includegraphics[width=\linewidth]{figures/chapter70/70_pq_hippo_hne_wb}
  \caption[The effect of spermidine treatment intervention on 4HNE adducts in the hippocampus]{\textbf{The effect of spermidine treatment intervention on 4HNE adducts in the hippocampus.} Densitometric analysis (A) and a representative western blot (B) for 4HNE adducts is shown. Data are presented  mean $\pm$ SEM, \textit{n} = 6 - 10.* = \textit{p} < 0.05 vs control, \# = \textit{p} < 0.05 vs PQ and @ = \textit{p} < 0.05 vs 0.3 mM Spd.}
  \label{fig:70_pq_hippo_hne_wb}
\end{figure} 
\end{landscape}

\subsection{The effect of spermidine treatment intervention on lipid peroxidation (4HNE) in PQ-induced brain injury using FM}
4HNE fluorescence intensity was assessed following treatment intervention using a confocal microscopy. Results show a strong fluorescence intensity signal in the PQ treated group to all other groups; control, 0.3 mM Spd, 0.3 mM Spd + PQ, 3 mM Spd and 3 mM Spd + PQ (\Cref{fig:70_pq_hippo_hne_fm}).

\subsection{The effect of spermidine treatment intervention on GFAP in PQ-induced brain injury using FM} 
Glial fibrillary protein (GFAP) and its upregulation has been utilized as a marker for neuronal damage \citep{Wu2015}. Here, we assessed the role of spermidine in impacting GFAP upon PQ-induced damage. A strong fluorescence intensity signal for GFAP was observed in the PQ treated group with distinct glial derived signal compared to all other groups (\Cref{fig:70_pq_hippo_gfap_fm}).

\begin{landscape}
\begin{figure}[!htbp]
\center
  \includegraphics[width=0.75\linewidth]{figures/chapter70/70_pq_hippo_hne_fm}
  \caption[The effect of spermidine treatment intervention on 4HNE signal in the hippocampus]{\textbf{The effect of spermidine treatment  intervention on 4HNE signal in the hippocampus.} Representative fluorescence micrographs indicate control, PQ, 0.3 mM Spd, 0.3 mM Spd + PQ, 3 mM Spd, and 3 mM Spd + PQ. Nuclei counterstained with Hoechst 33342 (blue). Images were acquired using a 10x and 40x objective lens, \textit{n} = 6. Scale bar: 100 $\mu$m for lower magnification and 20 $\mu$m for higher magnification. Arrowheads indicate regions of strong 4HNE signal.}
  \label{fig:70_pq_hippo_hne_fm}
\end{figure} 
\end{landscape}

\begin{landscape}
\begin{figure}[!htbp]
\center
  \includegraphics[width=0.75\linewidth]{figures/chapter70/70_pq_hippo_gfap_fm}
  \caption[The effect of spermidine treatment intervention on GFAP signal in the hippocampus]{\textbf{The effect of spermidine treatment intervention on GFAP signal in the hippocampus.} Representative fluorescence micrographs indicate control, PQ, 0.3 mM Spd, 0.3 mM Spd + PQ, 3 mM Spd, and 3 mM Spd + PQ. Nuclei counterstained with Hoechst 33342 (blue). Images were acquired using a 10x and 40x objective lens, \textit{n} = 6. Scale bar: 100 $\mu$m for lower magnification and 20 $\mu$m for higher magnification. Arrowheads indicate regions of strong GFAP signal.}
  \label{fig:70_pq_hippo_gfap_fm}
\end{figure} 
\end{landscape}

\subsection{The effect of spermidine treatment intervention on APP in PQ-induced brain injury using WB}
Next, APP expression which has been used as an indicator of neuronal damage was assessed following treatments using western blot analysis. Our results reveal that APP expression was significantly increased following treatment with PQ (2.60 $\pm$ 0.17) and 0.3 mM Spd (2.21 $\pm$ 0.13) compared to the control group (1.00 $\pm$ 0.57), with no significant difference observed in the 0.3 mM Spd + PQ (1.26 $\pm$ 0.14), 3 mM Spd (1.09 $\pm$ 0.32) and 3 mM Spd + PQ (1.54 $\pm$ 0.10) (\Cref{fig:70_pq_hippo_app_wb}: \textbf{A} \& \textbf{B}). Of particular importance is a significant decrease in APP expression with 0.3 mM Spd + PQ (\textit{p} < 0.05), 3 mM Spd (\textit{p} < 0.05) and 3 mM Spd + PQ (\textit{p} < 0.05) compared to the PQ treated group. Furthermore, a significant decrease in APP expression was observed in the 0.3 mM Spd + PQ (\textit{p} < 0.05) and 3 mM Spd (\textit{p} < 0.05) compared to 0.3 mM Spd. 

\begin{figure}[!htbp]
\center
  \includegraphics[width=\linewidth]{figures/chapter70/70_pq_hippo_app_wb}
  \caption[The effect of spermidine treatment intervention on APP expression in the hippocampus]{\textbf{The effect of spermidine treatment intervention on APP expression in the hippocampus.} Densitometric analysis (A) and a representative western blot (B) for APP expression is shown. Data are presented  mean $\pm$ SEM, \textit{n} = 6 - 10.* = \textit{p} < 0.05 vs control, \# = \textit{p} < 0.05 vs PQ and @ = \textit{p} < 0.05 vs 0.3 mM Spd.}
  \label{fig:70_pq_hippo_app_wb}
\end{figure} 

\subsection{The effect of spermidine treatment intervention on acetylated $\alpha$-tubulin in PQ-induced brain injury using WB}
In order to assess whether autophagy upregulation through spermidine may impact microtubule stability, acetylated $\alpha$-tubulin was assessed in response to treatment intervention. In comparison to the control group (1.00 $\pm$ 0.08), a significant decrease in tubulin acetylation was observed in the PQ treated group (0.68 $\pm$ 0.05, \textit{p} < 0.05),with no significant differences seen in all other groups [0.3 mM Spd (0.82 $\pm$ 0.05), 0.3 mM Spd + PQ (1.11 $\pm$ 0.04), 3 mM Spd (0.81 $\pm$ 0.06), and 3 mM Spd + PQ (0.99 $\pm$ 0.13)](\Cref{fig:70_pq_hippo_tubulin_wb}: \textbf{A} \& \textbf{B}). Moreover, a significant increase in acetylated $\alpha$-tubulin was observed in the 0.3 mM Spd + PQ (\textit{p} < 0.05), and 3 mM Spd + PQ (\textit{p} < 0.05) compared to the PQ treated group. Furthermore, a significant increase was observed in 0.3 mM Spd + PQ (\textit{p} < 0.05) compared to 0.3 mM Spd.

\begin{figure}[!htbp]
\center
  \includegraphics[width=\linewidth]{figures/chapter70/70_pq_hippo_tubulin_wb}
  \caption[The effect of spermidine treatment intervention on acetylated $\alpha$-tubulin expression in the hippocampus]{\textbf{The effect of spermidine treatment intervention on acetylated $\alpha$-tubulin in the hippocampus.} Densitometric analysis (A) and a representative western blot (B) for acetylated $\alpha$-tubulin is shown. Data are presented  mean $\pm$ SEM, \textit{n} = 6 - 10.* = \textit{p} < 0.05 vs control, \# = \textit{p} < 0.05 vs PQ and @ = \textit{p} < 0.05 vs 0.3 mM Spd.}
  \label{fig:70_pq_hippo_tubulin_wb}
\end{figure} 

\subsection{The effect of spermidine treatment intervention on autophagy in PQ-induced brain injury using WB}
In order to assess autophagic activity in response to treatment intervention in the hippocampus, LC3-II, p62 and LAMP2A protein levels were assessed using western blot and fluorescence microscopy. Results indicate that LC3-II protein expression was significantly increased in the 0.3 mM Spd + PQ (1.38 $\pm$ 0.11, \textit{p} < 0.05) and the 3 mM Spd + PQ (1.67 $\pm$ 0.12, \textit{p} < 0.05) compared to the control group (1.00 $\pm$ 0.06) and PQ treated group (0.91 $\pm$ 0.06), with no significant difference in LC3-II expression observed in the 0.3 mM Spd (1.17 $\pm$ 0.05) and the 3 mM Spd (1.13 $\pm$ 0.08), and between control group and PQ treated group (\Cref{fig:70_pq_hippo_autophagy_wb}: \textbf{A}\textit{i} \& \textbf{B}). Moreover, LC3-II protein levels were significantly increased in the 3 mM Spd + PQ (\textit{p} < 0.05) compared to the 0.3 mM Spd + PQ. Lastly, LC3-II expression was significantly increased in the 3 mM Spd + PQ (\textit{p} < 0.05) compared to the 3 mM Spd.

In comparison to the control group (1.00 $\pm$ 0.07) and PQ treated group (0.71 $\pm$ 0.07), a significant increase in the p62 protein expression was observed in the 3 mM Spd (2.60 $\pm$ 0.72, \textit{p} < 0.05) and in 3 mM Spd + PQ (2.59 $\pm$ 0.45, \textit{p} < 0.05), with no significant difference seen with 0.3 mM Spd (0.76 $\pm$ 0.29) and in the 0.3 mM Spd + PQ (1.70 $\pm$ 0.62) (\Cref{fig:70_pq_hippo_autophagy_wb}: \textbf{A}\textit{ii} \& \textbf{B}). In addition, 3 mM Spd exposure significantly increased p62 protein expression compared to 0.3 mM Spd.

LAMP2A protein expression was significantly decreased in the PQ treated group (0.09 $\pm$ 0.02, \textit{p} < 0.05), while significantly increased in 0.3 mM Spd (2.43 $\pm$ 0.27, \textit{p} < 0.05), 0.3 mM Spd + PQ (1.92 $\pm$ 0.22, \textit{p} < 0.05) and 3 mM Spd + PQ (1.90 $\pm$ 0.35, \textit{p} < 0.05) compared to the control group (1.00 $\pm$ 0.33), with no significant differences observed in the 3 mM Spd (1.80 $\pm$ 0.32) (\Cref{fig:70_pq_hippo_autophagy_wb}: \textbf{A}\textit{iii} \& \textbf{B}). Moreover, LAMP2A protein expression was significantly increased in 0.3 mM Spd, 0.3 mM Spd + PQ, 3 mM Spd and 3 mM Spd + PQ (\textit{p} < 0.05) compared to the PQ treated group.

\subsection{The effect of spermidine treatment intervention on GFP-LC3  in PQ-induced brain injury using FM}
Results show that GFP-LC3 fluorescence intensity signal was strongly expressed in the groups where spermidine was present i.e. in the 0.3 mM Spd, 0.3 mM Spd + PQ, 3 mM Spd, and 3 mM Spd + PQ as well as in the control group compared to the PQ treated group (\Cref{fig:70_pq_hippo_lc3_fm}).

\subsection{The effect of spermidine treatment intervention on p62 in PQ-induced brain injury using FM}
A strong fluorescence intensity signal for p62 was observed in the 3 mM Spd and 3 mM Spd + PQ compared to PQ treated group and control group (\Cref{fig:70_pq_hippo_p62_fm}).

\begin{landscape}
\begin{figure}[!htbp]
\center
  \includegraphics[width=\linewidth]{figures/chapter70/70_pq_hippo_autophagy_wb}
  \caption[The effect of spermidine treatment intervention on autophagic activity in the hippocampus]{\textbf{The effect of spermidine treatment intervention on autophagic activity in the hippocampus.} Densitometric analysis (A) and a representative western blot (B) for LC3-II, p62 and LAMP2A expression is shown. Data are presented  mean $\pm$ SEM, \textit{n} = 6 - 10.* = \textit{p} < 0.05 vs control, \# = \textit{p} < 0.05 vs PQ , @ = \textit{p} < 0.05 vs 0.3 mM Spd and \$ = \textit{p} < 0.05 vs 0.3 mM Spd + PQ.}
  \label{fig:70_pq_hippo_autophagy_wb}
\end{figure} 
\end{landscape}

\begin{landscape}
\begin{figure}[!htbp]
\center
  \includegraphics[width=0.75\linewidth]{figures/chapter70/70_pq_hippo_lc3_fm}
  \caption[The effect of spermidine treatment intervention on GFP-LC3 signal in the hippocampus]{\textbf{The effect of spermidine treatment  intervention on GFP-LC3 signal in the hippocampus.} Representative fluorescence micrographs indicate control, PQ, 0.3 mM Spd, 0.3 mM Spd + PQ, 3 mM Spd, and 3 mM Spd + PQ. Nuclei counterstained with Hoechst 33342 (blue). Images were acquired using a 10x and 40x objective lens, \textit{n} = 6. Scale bar: 100 $\mu$m for lower magnification and 20 $\mu$m for higher magnification. Arrowheads indicate regions of strong GFP-LC3 signal.}
  \label{fig:70_pq_hippo_lc3_fm}
\end{figure} 
\end{landscape}

\begin{landscape}
\begin{figure}[!htbp]
\center
  \includegraphics[width=0.75\linewidth]{figures/chapter70/70_pq_hippo_p62_fm}
  \caption[The effect of spermidine treatment intervention on p62 signal in the hippocampus]{\textbf{The effect of spermidine treatment intervention on p62 signal in the hippocampus.} Representative fluorescence micrographs indicate control, PQ, 0.3 mM Spd, 0.3 mM Spd + PQ, 3 mM Spd, and 3 mM Spd + PQ. Nuclei counterstained with Hoechst 33342 (blue). Images were acquired using a 10x and 40x objective lens, \textit{n} = 6. Scale bar: 100 $\mu$m for lower magnification and 20 $\mu$m for higher magnification. Arrowheads indicate regions of strong p62 signal.}
  \label{fig:70_pq_hippo_p62_fm}
\end{figure} 
\end{landscape}

\section{Cortex region}
\subsection{The effect of spermidine treatment intervention using H\&E in PQ-induced brain injury}
Qualitative analysis using H\&E staining showed no major abnormal changes in the structural architecture of the cortex upon treatment intervention (\Cref{fig:70_pq_cortex_he_lm}).

\begin{figure}[!htbp]
\center
  \includegraphics[width=\linewidth]{figures/chapter70/70_pq_cortex_he_lm}
  \caption[The effect of spermidine treatment intervention using H\&E staining in the cortex]{\textbf{The effect of spermidine treatment intervention using H\&E staining in the cortex.} Representative micrographs indicate control, PQ, 0.3 mM Spd, 0.3 mM Spd + PQ, 3 mM Spd, and 3 mM Spd + PQ. Images acquired using a 10x and 40x objective lens, \textit{n} = 6. Scale bar: 200 $\mu$m for lower magnification and 50 $\mu$m for higher magnification.}
  \label{fig:70_pq_cortex_he_lm}
\end{figure} 

\subsection{The effect of spermidine treatment intervention on lipid peroxidation in PQ-induced brain injury using WB}
At 35 kDa, 4HNE protein adducts signal was significantly decreased in the 0.3 mM Spd (0.59 $\pm$ 0.09, \textit{p} < 0.05), 0.3 mM Spd + PQ group (0.47 $\pm$ 0.08, \textit{p} < 0.05) compared to the control group (1.00 $\pm$ 0.14), with no significant increase observed in the PQ treated group (1.08 $\pm$ 0.17), 3 mM Spd (0.77 $\pm$ 0.15) and 3 mM Spd + PQ (0.76 $\pm$ 0.15) (\Cref{fig:70_pq_cortex_hne_wb}: \textbf{A}\textit{i} \& \textbf{B}). Moreover, 4HNE signal was significantly decreased in the 0.3 mM Spd (\textit{p} < 0.05) and 0.3 mM Spd + PQ group (\textit{p} < 0.05) compared to the PQ treated group. 

At 45 kDa, 4HNE signal was significantly decreased in the 0.3 mM Spd (0.45 $\pm$ 0.06, \textit{p} < 0.05), 0.3 mM Spd + PQ (0.14 $\pm$ 0.05, \textit{p} < 0.05), 3 mM Spd (0.26 $\pm$ 0.04) and 3 mM Spd + PQ (0.18 $\pm$ 0.04, \textit{p} < 0.05) compared to the control (1.00 $\pm$ 0.08) and PQ treated group (1.05 $\pm$ 0.130), with no significant differences observed between the control and the PQ treated group (\Cref{fig:70_pq_cortex_hne_wb}: \textbf{A}\textit{ii} \& \textbf{B}). Moreover, a significant decrease was observed in 0.3 mM Spd + PQ compared to 0.3 mM Spd (\textit{p} < 0.05) and in 3 mM Spd (\textit{p} < 0.05) compared to 0.3 mM Spd.

Lastly, at 52 kDa, a significant decrease in 4HNE signal was observed in the 0.3 mM Spd + PQ (0.14 $\pm$ 0.05, \textit{p} < 0.05), 3 mM Spd (0.15 $\pm$ 0.07) and 3 mM Spd + PQ (0.11 $\pm$ 0.05, \textit{p} < 0.05) group compared to the control (1.00 $\pm$ 0.22) and PQ treated group (0.99 $\pm$ 0.17), with no significant differences observed with 0.3 mM Spd (0.62 $\pm$ 0.13). In addition, no significant differences were observed between the control and PQ group (\Cref{fig:70_pq_cortex_hne_wb}: \textbf{A}\textit{iii} \& \textbf{B}). Moreover, 4HNE expression of was significantly decreased in the 0.3 mM Spd + PQ (\textit{p} < 0.05) and in the 3 mM Spd (\textit{p} < 0.05) compared to 0.3 mM Spd.

\subsection{The effect of spermidine treatment intervention on lipid peroxidation in PQ-induced brain injury using FM}
A moderately enhanced fluorescence signal for 4HNE was observed in the control and PQ treated group compared to 0.3 mM Spd, 0.3 mM Spd + PQ, 3 mM Spd and 3 mM Spd + PQ (\Cref{fig:70_pq_cortex_hne_fm}).

\subsection{The effect of spermidine treatment intervention on GFAP in PQ-induced brain injury using FM}
Results indicate that GFAP fluorescence intensity signal was strongly enhanced in the PQ treated group compared to all other groups (\Cref{fig:70_pq_cortex_gfap_fm}).

\begin{landscape}
\begin{figure}[!htbp]
\center
  \includegraphics[width=\linewidth]{figures/chapter70/70_pq_cortex_hne_wb}
  \caption[The effect of spermidine treatment intervention on 4HNE adducts in the cortex]{\textbf{The effect of spermidine treatment intervention on 4HNE adducts in the cortex.} Densitometric analysis (A) and a representative western blot (B) for 4HNE adducts is shown. Data are presented  mean $\pm$ SEM, \textit{n} = 6 - 10.* = \textit{p} < 0.05 vs control, \# = \textit{p} < 0.05 vs PQ and @ = \textit{p} < 0.05 vs 0.3 mM Spd.}
  \label{fig:70_pq_cortex_hne_wb}
\end{figure} 
\end{landscape}

\begin{landscape}
\begin{figure}[!htbp]
\center
  \includegraphics[width=0.75\linewidth]{figures/chapter70/70_pq_cortex_hne_fm}
  \caption[Effect of treatment intervention on 4HNE signal in the cortex]{\textbf{Effect of treatment intervention on 4HNE signal in the cortex.} Representative fluorescence micrographs indicate control, PQ, 0.3 mM Spd, 0.3 mM Spd + PQ, 3 mM Spd, and 3 mM Spd + PQ. Nuclei counterstained with Hoechst 33342 (blue). Images were acquired using a 10x and 40x objective lens, \textit{n} = 6. Scale bar: 100 $\mu$m for lower magnification and 20 $\mu$m for higher magnification. Arrowheads indicate regions of strong 4HNE signal.}
  \label{fig:70_pq_cortex_hne_fm}
\end{figure} 
\end{landscape}

\begin{landscape}
\begin{figure}[!htbp]
\center
  \includegraphics[width=0.75\linewidth]{figures/chapter70/70_pq_cortex_gfap_fm}
  \caption[The effect of spermidine treatment intervention on GFAP signal in the cortex]{\textbf{The effect of spermidine treatment intervention on GFAP signal in the cortex.} Representative fluorescence micrographs indicate control, PQ, 0.3 mM Spd, 0.3 mM Spd + PQ, 3 mM Spd, and 3 mM Spd + PQ. Nuclei counterstained with Hoechst 33342 (blue). Images were acquired using a 10x and 40x objective lens, \textit{n} = 6. Scale bar: 100 $\mu$m for lower magnification and 20 $\mu$m for higher magnification. Arrowheads indicate regions of strong GFAP signal.}
  \label{fig:70_pq_cortex_gfap_fm}
\end{figure} 
\end{landscape}

\subsection{The effect of spermidine treatment intervention on APP in PQ-induced brain injury using WB}
Although no significance differences were reached, a trend towards an increase in APP was observed following PQ treatment (1.57 $\pm$ 0.36), while a trend towards decreased APP signal was observed following treatment with 0.3 mM Spd (1.10 $\pm$ 0.22), 0.3 mM Spd + PQ (0.65 $\pm$ 0.19), 3 mM Spd (1.17 $\pm$ 0.25) and 3 mM Spd + PQ (0.70 $\pm$ 0.23) compared to the control group (0.58 $\pm$ 0.16)  (\Cref{fig:70_pq_cortex_app_wb}: \textbf{A} \& \textbf{B}). Of particular importance is a significant decrease in APP expression with 0.3 mM Spd + PQ (\textit{p} < 0.05) and 3 mM Spd + PQ (\textit{p} < 0.05) compared to the PQ treated group. 

\begin{figure}[!htbp]
\center
  \includegraphics[width=\linewidth]{figures/chapter70/70_pq_cortex_app_wb}
  \caption[The effect of spermidine treatment intervention on APP expression in the cortex]{\textbf{The effect of spermidine treatment intervention on APP expression in the cortex].} Densitometric analysis (A) and a representative western blot (B) for APP expression is shown. Data are presented  mean $\pm$ SEM, \textit{n} = 6 - 10. \# = \textit{p} < 0.05 vs PQ.}
  \label{fig:70_pq_cortex_app_wb}
\end{figure} 

\subsection{The effect of spermidine treatment intervention on acetylated $\alpha$-tubulin in PQ-induced brain injury using WB}
In comparison to the control (1.00 $\pm$ 0.03) and PQ treated group (0.87 $\pm$ 0.13), no significant differences were observed in the expression of acetylated $\alpha$-tubulin in the 0.3 mM Spd (0.95 $\pm$ 0.17), 0.3 mM Spd + PQ (0.80 $\pm$ 0.08) and 3 mM Spd (1.02 $\pm$ 0.07), however, a significant decrease was observed at 3 mM Spd + PQ (0.65 $\pm$ 0.12, \textit{p} < 0.05), with no differences observed between the control and the PQ-treated group (\Cref{fig:70_pq_cortex_tubulin_wb}: \textbf{A} \& \textbf{B}). 

\begin{figure}[!htbp]
\center
  \includegraphics[width=\linewidth]{figures/chapter70/70_pq_cortex_tubulin_wb}
  \caption[The effect of spermidine treatment intervention on acetylated $\alpha$-tubulin expression in the cortex]{\textbf{The effect of spermidine treatment intervention on acetylated $\alpha$-tubulin in the cortex.} Densitometric analysis (A) and a representative western blot (B) for acetylated $\alpha$-tubulin is shown. Data are presented  mean $\pm$ SEM, \textit{n} = 6 - 10.* = \textit{p} < 0.05 vs control, \# = \textit{p} < 0.05 vs PQ, \$ = \textit{p} < 0.05 vs 0.3 mM Spd + PQ and \% = \textit{p} < 0.05 vs 3 mM Spd.}
  \label{fig:70_pq_cortex_tubulin_wb}
\end{figure} 

\subsection{The effect of spermidine treatment intervention on autophagy in PQ-induced brain injury using WB}
In comparison to the control group (1.00 $\pm$ 0.11), no significant differences were observed in LC3-II protein levels in the PQ treated group (0.65 $\pm$ 0.03), 0.3 mM Spd (0.88 $\pm$ 0.12), 0.3 mM Spd + PQ (1.30 $\pm$ 0.27), 3 mM Spd (1.23 $\pm$ 0.25) and 3 mM Spd + PQ (0.91 $\pm$ 0.07) (\Cref{fig:70_pq_cortex_autophagy_wb}: \textbf{A}\textit{i} \& \textbf{B}). A significant increase in LC3-II expression was however observed in the 0.3 mM Spd + PQ (\textit{p} < 0.05) and 3 mM Spd (\textit{p} < 0.05) compared to the PQ treated group. 

No significant differences were observed in the p62 expression following treatment with PQ (1.15 $\pm$ 0.11), 3 mM Spd (1.66 $\pm$ 0.21) and 3 mM Spd + PQ (1.56 $\pm$ 0.34) compared to the control  group (1.00 $\pm$ 0.06), however, a significant increase in p62 expression was observed in the 0.3 mM Spd (2.53 $\pm$ 0.34, \textit{p} < 0.05) and 0.3 mM Spd + PQ (2.43 $\pm$ 0.36, \textit{p} < 0.05) (\Cref{fig:70_pq_cortex_autophagy_wb}: \textbf{A}\textit{ii} \& \textbf{B}). Moreover, a significant increase was also detected in the 0.3 mM Spd (\textit{p} < 0.05) and 0.3 mM Spd + PQ group (\textit{p} < 0.05) compared to PQ treated group. More importantly, a significant decrease in p62 expression was observed in the 3 mM Spd + PQ (\textit{p} < 0.05) compared to 0.3 mM Spd + PQ as well as in the 3 mM Spd treated group (\textit{p} < 0.05) compared to 0.3 mM Spd. 

No significant differences were observed in LAMP2A protein expression in the PQ treated group (0.40 $\pm$ 0.13), 0.3 mM Spd (1.85 $\pm$ 0.24) and 3 mM Spd + PQ (1.47 $\pm$ 0.14) compared to the control group (1.00 $\pm$ 0.18), however a significant increase in LAMP2A expression was observed with 0.3 mM Spd + PQ (2.02 $\pm$ 0.39, \textit{p} < 0.05) and 3 mM Spd (2.16 $\pm$ 0.53, \textit{p} < 0.05) (\Cref{fig:70_pq_cortex_autophagy_wb}: \textbf{A}\textit{iii} \& \textbf{B}). Moreover, LAMP2A expression was significantly increased in 0.3 mM Spd (\textit{p} < 0.05), 0.3 mM Spd + PQ (\textit{p} < 0.05), 3 mM Spd (\textit{p} < 0.05)  and 3 mM Spd + PQ (\textit{p} < 0.05) compared to the PQ treated group.

\subsection{The effect of spermidine treatment intervention on GFP-LC3 in PQ-induced brain injury using FM}
Fluorescence microscopy of the cortex showed that GFP-LC3 signal was enhanced in the groups 0.3 mM Spd + PQ and 3 mM Spd compared to all other groups; control, PQ,  0.3 mM Spd and 3 mM Spd + PQ (\Cref{fig:70_pq_cortex_lc3_fm}).

\subsection{The effect of spermidine treatment intervention on p62 in PQ-induced brain injury using FM}
A strong fluorescence intensity signal for p62 was observed in the 0.3 mM Spd and 0.3 mM Spd + PQ compared to all other groups; control, PQ, 3 mM Spd and 3 mM Spd + PQ (\Cref{fig:70_pq_cortex_p62_fm}).

\begin{landscape}
\begin{figure}[!htbp]
\center
  \includegraphics[width=\linewidth]{figures/chapter70/70_pq_cortex_autophagy_wb}
  \caption[The effect of spermidine treatment intervention on autophagic activity in the cortex]{\textbf{The effect of spermidine treatment intervention on autophagic activity in the cortex.} Densitometric analysis (A) and a representative western blot (B) for LC3-II, p62 and LAMP2A expression is shown. Data are presented  mean $\pm$ SEM, \textit{n} = 6 - 10.* = \textit{p} < 0.05 vs control, \# = \textit{p} < 0.05 vs PQ , @ = \textit{p} < 0.05 vs 0.3 mM Spd and \$ = \textit{p} < 0.05 vs 0.3 mM Spd + PQ.}
  \label{fig:70_pq_cortex_autophagy_wb}
\end{figure} 
\end{landscape}

\begin{landscape}
\begin{figure}[!htbp]
\center
  \includegraphics[width=0.75\linewidth]{figures/chapter70/70_pq_cortex_lc3_fm}
  \caption[The effect of spermidine treatment intervention on GFP-LC3 signal in the cortex]{\textbf{The effect of spermidine treatment intervention on GFP-LC3 signal in the cortex.} Representative fluorescence micrographs indicate control, PQ, 0.3 mM Spd, 0.3 mM Spd + PQ, 3 mM Spd, and 3 mM Spd + PQ. Nuclei counterstained with Hoechst 33342 (blue). Images were acquired using a 10x and 40x objective lens, \textit{n} = 6. Scale bar: 100 $\mu$m for lower magnification and 20 $\mu$m for higher magnification. Arrowheads indicate regions of strong GFP-LC3 signal.}
  \label{fig:70_pq_cortex_lc3_fm}
\end{figure} 
\end{landscape}

\begin{landscape}
\begin{figure}[!htbp]
\center
  \includegraphics[width=0.75\linewidth]{figures/chapter70/70_pq_cortex_p62_fm}
  \caption[The effect of spermidine treatment intervention on p62 signal in the cortex]{\textbf{The effect of spermidine treatment intervention on p62 signal in the cortex.} Representative fluorescence micrographs indicate control, PQ, 0.3 mM Spd, 0.3 mM Spd + PQ, 3 mM Spd, and 3 mM Spd + PQ. Nuclei counterstained with Hoechst 33342 (blue). Images were acquired using a 10x and 40x objective lens, \textit{n} = 6. Scale bar: 100 $\mu$m for lower magnification and 20 $\mu$m for higher magnification. Arrowheads indicate regions of strong p62 signal.}
  \label{fig:70_pq_cortex_p62_fm}
\end{figure} 
\end{landscape}

\section{Discussion: The effect of spermidine in PQ-induced brain injury}
Mounting evidence points towards autophagy modulation using pharmacological agents as one of the major therapeutic strategies for neurodegenerative diseases \citep{Hebron2013,Jiang2014a,Perera2018,Ravikumar2004,Rose2010}. In particular, spermidine, a natural polyamine, recently received substantial attention as it has been shown to ameliorate neurodegeneration associated pathologies in mouse models of ischemia \citep{Morrison2002}, normal tension glaucoma \citep{Noro2015}, ALS \citep{Wang2012} and  PD \citep{Buttner2014} and recently in mouse and human models of ageing \citep{Schwarz2018,Wirth2019,Wirth2018}. Moreover, as a caloric restriction mimetic (CRM), spermidine is known to enhance longevity by enhancing autophagy \citep{Marino2014}. Whether the protective effects of spermidine are dose-dependent remains, however unclear. Moreover, spermidine has not been assessed in rodent models of PQ-induced brain injury. Hence, the purpose of this study was to assess the potential protective effects of spermidine at two distinct dosages using transgenic mice expressing GFP-LC3 \citep{Mizushima2004a} that were treated with PQ to induce neuronal toxicity associated with neurodegeneration \citep{Chen2012}. PQ is a pesticide that selectively targets neurons in the substantia nigra (SN) pars compacta leading to PD like neuropathology i.e. degeneration of nigral dopaminergic neurons \citep{McCormack2005, McCormack2002}. Although PQ preferentially affects the SN pars compacta, it also impacts the hippocampus and cortex region \citep{Landrigan2005}, thus, it could potentially impair learning and memory functions and affect AD pathogenesis. PQ exerts its toxicity by inducing oxidative stress and mitochondrial damage \citep{Baltazar2014,Drechsel2008,Lin2006}, both of which have been implicated in the pathogenesis of AD \citep{Lin2006}. Hence, here we focussed on the hippocampus and the cortex, two brain regions most susceptible to neuronal injury and impacted in the early and late stages of AD respectively.

\subsection{The effect of spermidine treatment intervention on brain weight in PQ-induced brain injury}
We first assessed the effects of spermidine on total brain weight upon PQ-induced brain injury. Our results revealed a significant increase in brain weight in the PQ treated group, spermidine treatment alone at both dosages (0.3 mM \& 3 mM Spd) and in the combination treatment with PQ at both dosages (0.3 mM Spd + PQ \& 3 mM Spd + PQ) compared to the control group (\Cref{fig:70_pq_brain_weight_wholebrain}). Moreover, a significant increase in brain weight was observed following spermidine treatment, alone and in combination group compared to the PQ treated group. The increase in brain weight following PQ treatment might be due to osmotic swelling and neuronal injury. PQ has been proposed to induce mitochondrial toxicity by inducing osmotic swelling \citep{Cappelletti1996,Mohammadi-Bardbori2008}. However, since spermidine alone and in combination with PQ increased brain weight above the weight observed in the PQ group, these results suggest that the increased brain weight might be brought about by enhanced synaptic plasticity, which plays a role in promoting memory formation. Indeed, previous work showed that spermidine protects against age induced memory impairment in fruitflies \citep{Gupta2016,Gupta2013,Sigrist2014}, and more recent work demonstrated that spermidine works directly at the synapses where it regulates presynaptic specialization in an autophagy-dependent mechanism \citep{Bhukel2017}. Since neuronal tissue is characterized by proteotoxicity and since stress upregulates autophagy \citep{Petrovski2011,Yorimitsu2006}, the increase in brain weight in the PQ group might have been due to enhanced synaptic plasticity caused by the upregulation of autophagy. This warrants further investigation. Here, key molecular proteins associated with oxidative stress and autophagy will be assessed using, using biomedical and techniques.

\subsection{Hippocampus region analysis} 
\subsubsection{The effect of spermidine treatment intervention using Heamatoxylin and Eosin (H\&E) staining in PQ-induced brain injury}
In order to assess overall structural changes of the hippocampus following treatment intervention, heamatoxylin and eosin (H\&E) staining was performed. Our data revealed no major abnormalities i.e. no signs of neuronal cell loss or major architectural damage or protein inclusion between the treatment groups (\Cref{fig:70_pq_hippo_he_lm}). These results suggest that PQ-induced toxicity in the brain was not causing obvious structural damage in this region. Indeed, silver staining of brain sections of PD models revealed degeneration of neurons in the SN pars compacta in mice exposed to PQ once a week for 3 weeks \citep{McCormack2002}. Other studies reported in the same brain region that PQ exposure enhanced $\alpha$-synuclein aggregation \citep{Manning-Bog2002} and induced neuronal injury \citep{Manning-Bog2003}, suggesting that the damage is associated with protein aggregation and proteotoxicity.

\subsubsection{The effect of spermidine treatment intervention on lipid peroxidation in PQ-induced brain injury}
Since PQ induces major oxidative stress, we assessed 4HNE adducts at 35, 45, 52 kDa. Our results revealed a significant increase in 4HNE signal at 35 kDa adducts following PQ treatment and a significant reduction in 4HNE signal following treatment with 0.3 mM Spd, 0.3 mM Spd + PQ, 3 mM Spd, and 3 mM Spd + PQ compared to the control group (\Cref{fig:70_pq_hippo_hne_wb}: \textbf{A}\textit{i} \& \textbf{B}). Moreover, a significant decrease in 4HNE signal was observed following treatment with 0.3 mM Spd, 0.3 mM Spd + PQ, 3 mM Spd, and 3 mM Spd + PQ to the PQ treated group. At 45 kDa, our results showed a significantly enhanced 4HNE signal following PQ treatment and a significant reduction in 4HNE signal following treatment with 0.3 mM Spd + PQ, 3 mM Spd and 3 mM Spd + PQ compared to the control group (\Cref{fig:70_pq_hippo_hne_wb}: \textbf{A}\textit{ii} \& \textbf{B}). Secondly, results show a significant reduction in 4HNE signal following treatment with 0.3 mM Spd, 0.3 mM Spd + PQ, 3 mM Spd, and 3 mM Spd + PQ compared to the PQ treated group. Thirdly, we observed a significant reduction in 4HNE signal in the 0.3 mM Spd + PQ compared to 0.3 mM Spd and in 3 mM Spd compared to 0.3 mM Spd. To our surprise, at 55 kDa, we observed no significant differences in 4HNE signal in the PQ treated group, 0.3 mM Spd and in 3 mM Spd compared to the control, but, a significant reduction in 4HNE signal was detected in the combination groups (0.3 mM Spd + PQ and 3 mM Spd + PQ) (\Cref{fig:70_pq_hippo_hne_wb}: \textbf{A}\textit{iii} \& \textbf{B}). Moreover, we revealed a significant reduction in 4HNE signal following treatment with 0.3 mM Spd + PQ, 3 mM Spd, 3 mM Spd + PQ compared to the PQ treated group. Lastly, we observed a significant reduction in the 0.3 mM Spd + PQ compared to 0.3 mM Spd and in 3 mM Spd compared to 0.3 mM Spd. These results are supported by the immunofluorescence data, where a strong 4HNE signal was observed in the PQ treated group compared to all other groups (\Cref{fig:70_pq_hippo_hne_fm}). Overall, these results suggest that firstly, PQ exposure induces robust oxidative stress or damage, secondly, spermidine supplementation protects against oxidative stress induced by PQ and thirdly, spermidine mediates neuroprotection in a dose-dependent manner. In line with our findings, \citet{Chen2012} revealed that PQ exposure in mice over a period of 21 days resulted in oxidative damage. Furthermore, studies \textit{in vivo} have reported that spermidine reduces oxidative damage in various diseases such as, in a mouse model of retinal degenerative disease \citep{Noro2015}, a \textit{Drosophila melanogaster} model of PQ-induced neurotoxicity \citep{Minois2012}, a mouse model of ageing \citep{Eisenberg2009} and a rat model of HD \citep{Jamwal2016}. In the latter study, systematic exposure to 3 nitropropanoic acid (3NP), a compound that simulates HD like neuropathologies, was found to elevate lipid peroxidation which was reduced by spermidine administration in a dose-dependent manner \citep{Jamwal2016}. Overall, these studies together with our data support the antioxidant properties of spermidine in neurotoxicity.

\subsubsection{The effect of spermidine treatment intervention on GFAP in PQ-induced brain injury}
Next, we assessed glial fibrillary acidic protein (GFAP), a marker neuronal injury. A strong signal for GFAP was detected in the PQ treated group with distinct glial derived signal compared to all other groups (\Cref{fig:70_pq_hippo_gfap_fm}), suggesting that PQ induces astrocyte gliosis, indicative for neuronal injury or neuronal damage in this region. Importantly, spermidine at 0.3 mM Spd + PQ, and 3 mM Spd + PQ reduced neuronal damage compared to the PQ treated group, suggesting that spermidine mediates neuroprotection. In support of our findings, previous reports have shown increased levels of GFAP in the mouse model of AD \citep{Kamphuis2014,Kamphuis2012} and in patients with AD in the cerebrospinal fluid (CSF) \citep{Colangelo2014,Fukuyama2001,Ishiki2016} and in the hippocampus region \citep{Kamphuis2014}. Notably, spermidine has been shown to inhibit gliosis in a mouse model of multiple sclerosis, \citep{Guo2011}, while spermine (produced from spermidine) has been shown to reduce GFAP signal in a rat model of HD \citep{Velloso2009}. Since GFAP is an indicator of an inflammatory response, these studies together with our results suggest that spermidine mediates protection via, at least in part, anti-inflammatory properties. In support of this notion, spermine has been reported to exert anti-inflammatory effects, protecting against lethal sepsis \citep{Zhu2009}. It is however possible that the reduction of astrogliosis by spermidine might also be mediated by mechanisms dependent on autophagy. In support of this notion, others have reported a decrease in astrogliosis after autophagy modulation in models of neurodegeneration \citep{Castillo2013,Rodriguez-Navarro2010,Wang2012}. 

\subsubsection{The effect of spermidine treatment intervention on APP expression in PQ-induced brain injury}
Next, we assessed the expression of APP, another marker for neuronal damage. Consistent with the GFAP data, our western blot results showed a similar pattern in the APP expression, suggesting a neuronal damage or neuronal stress response. APP expression was significantly increased following treatment with PQ compared to the control group (\Cref{fig:70_pq_hippo_app_wb}: \textbf{A} \& \textbf{B}). To our knowledge, our data indicate for the first time an increase in APP following PQ treatment. Previous studies have reported increased APP levels in neurons in the hippocampus following experimentally induced neuronal damage with kainate in a rat model \citep{Siman1989}. In this particular study, high levels of APP were associated with increased astrogliosis as both APP and GFAP co-localized at the sites of damage, thus supporting our observations with GFAP in our study. To our surprise, we found enhanced APP expression with 0.3 mM Spd compared to the control group. Importantly, we observed a significant decrease in APP expression with 0.3 mM Spd + PQ, 3 mM Spd and 3 mM Spd + PQ compared to the PQ treated group, suggesting that spermidine protects against neuronal damage in this context (\Cref{fig:70_pq_hippo_app_wb}: \textbf{A} \& \textbf{B}). Furthermore, a significant decrease in APP expression was observed in the 0.3 mM Spd + PQ compared to 0.3 mM Spd and in 3 mM Spd compared to 0.3 mM Spd, suggesting a dose-dependent protection by spermidine. To our knowledge, this is the first time that a dose-dependent protection by spermidine in the context of APP has been reported. Overall, these results indicate that spermidine intervention ameliorates neuronal damage and confer protection. Protective effects of spermidine against neuronal damage have been reported in various studies. \citet{Yang2017} reported that spermidine protected against I/R induced injury in the hippocampus and cortical areas. Spermidine was found to mitigate $\alpha$-synuclein neurotoxicity in fruit flies and nematodes \citep{Buttner2014}, and to rescue motor dysfunction in mice with frontotemporal lobar degeneration \citep{Wang2012}. Our results are, as far as we know, the first that report protection from ROS-induced damage in this model system.

\subsubsection{The effect of spermidine treatment intervention on acetylated $\alpha$- tubulin in PQ-induced brain injury}
Next, we assessed acetylated $\alpha$- tubulin expression. Tubulin acetylation is a hallmark of microtubule stabilization and unstable microtubules have been implicated in chronic neurodegenerative disease such as AD \citep{Phadwal2018}. Given the neuronal damage observed with PQ, we investigated whether PQ will decrease tubulin acetylation and whether autophagy enhancement through spermidine may possible impact on microtubule stability. Our results revealed that PQ treatment indeed significantly decreased acetylated $\alpha$-tubulin compared to the control group (\Cref{fig:70_pq_hippo_tubulin_wb}: \textbf{A} \& \textbf{B}), suggesting that PQ causes destabilization of the microtubule. In line with our findings, a decrease in acetylated $\alpha$- tubulin was reported in the midbrain of zebrafish treated with PQ \citep{Pinho2019} and in the striata of mice \citep{Wills2012}. Notably, a significant increase in acetylated $\alpha$-tubulin was observed following spermidine treatment with both dosages i.e. 0.3 mM Spd + PQ, and 3 mM Spd + PQ compared to the PQ treated group (\Cref{fig:70_pq_hippo_tubulin_wb}: \textbf{A} \& \textbf{B}). Furthermore, a significant increase was observed in 0.3 mM Spd + PQ compared to 0.3 mM Spd. These results suggest that spermidine rescues from PQ-induced destabilization of the microtubule. Spermidine is known to increase the acetylation of tubulin by supressing EP300, and in turn stimulate autophagic flux in this manner \citep{Madeo2018}. Increased acetylation of microtubules facilitates the retrograde mobilization of autophagosomes from the periphery to the lysosomes localized in the cytoplasm for degradation \citep{Phadwal2018,Xie2010}. Therefore, we hypothesise that spermidine increases the acetylation of tubulin by suppressing EP300, leading to an increased autophagic flux, where toxic protein and organelles induced by PQ treatment are engulfed in the autophagosomes and transported to the lysosomes for subsequent degradation. These findings are the first to report a protective role for spermidine in PQ-induced toxicity context \textit{in vivo}. A recent study reported that spermine increases tubulin acetylation and facilitates the autophagy degradation of aggregate protein in an \textit{in vitro} model of Prion diseases \citep{Phadwal2018}, potentially pointing towards a similar mechanism.

\subsubsection{The effect of spermidine treatment intervention on autophagy in PQ-induced brain injury}
Besides conferring neuroprotective effects, mounting evidence suggests that spermidine plays a role in longevity in various model systems \citep{Eisenberg2009,Garcia-Prat2016,Minois2012,Morselli2011,Pietrocola2015} and that both neuroprotective effects and life span expansion by spermidine are mediated through autophagy \citep{Eisenberg2009,Gupta2016,Gupta2013,Minois2012,Yue2017}. Hence, next, we investigated whether the protective effects of spermidine in the context of PQ-induced toxicity are mediated by autophagy. Expression levels of LC3-II, p62 and LAMP2A were assessed. Our results revealed that LC3-II protein expression was significantly enhanced in spermidine treatment at both dosages i.e. 0.3 mM Spd + PQ and the 3 mM Spd + PQ compared to the PQ treated group and control group, but not between the PQ treated group and the control group (\Cref{fig:70_pq_hippo_autophagy_wb}: \textbf{A}\textit{i} \& \textbf{B}). These results suggest that autophagy activity was enhanced above basal levels in response to spermidine treatment, while PQ exposure alone did not induce autophagy above control levels. Importantly, LC3-II protein levels were significantly increased in the high dose of spermidine i.e. 3 mM Spd + PQ compared to the low dose i.e. 0.3 mM Spd + PQ, with no significant difference observed between 0.3 mM Spd and 3 mM Spd, suggesting a dose-dependent autophagy enhancing effect of spermidine in the presence of PQ. Spermidine alone at both dosages did not induce autophagy robustly above control levels and this was supported by the significant increase in autophagy with 3 mM Spd + PQ compared to the 3 mM Spd. Our data is supported by previous findings that show that spermidine supplementation extended lifespan and prevented liver fibrosis by increasing microtubule-associated protein 1S (MAP1S) levels \citep{Yue2017}. Others have reported that oral supplementation (0.3 or 3 mM) of spermidine in drinking water, as perfomed in our study, enhanced autophagy induction leading to cardiac protection and lifespan extension \citep{Eisenberg2016a} and reversed arterial aging \citep{LaRocca2013}. Spermidine exerts its protective effects by modulating autophagy through suppressing the E1A- binding protein p300 (EP300) \citep{Pietrocola2015}. Our western blot data was supported by the immunofluorescence micrograph which revealed that GFP-LC3 fluorescence signal was strongly expressed in 0.3 mM Spd + PQ, and 3 mM Spd + PQ compared to the PQ treated group (\Cref{fig:70_pq_hippo_lc3_fm}). In contrast to our LC3-II data, we observed enhanced p62 protein levels in 3 mM Spd + PQ and 3 mM Spd compared to the control and PQ treated group, but not in the low dose (0.3 mM Spd + PQ) (\Cref{fig:70_pq_hippo_autophagy_wb}: \textbf{A}\textit{ii} \& \textbf{B}). Importantly, PQ treatment did not enhance p62 expression or aggregation unlike as observed in the APP results. These results are supported by the immunofluorescence analysis which revealed a strong fluorescence intensity signal for p62 in the 3 mM Spd and 3 mM Spd + PQ compared to PQ treated group and control group (\Cref{fig:70_pq_hippo_p62_fm}). p62 levels have been shown to increase or decrease during autophagy modulation, depending on the time of assessment. For example, p62 levels have been shown to decrease after 4 h of starvation, restored to basal levels, but significantly increased after 8 h of starvation, and did not return to basal levels \citep{sahani2014}. In our study, LAMP2A protein expression was significantly decreased in the PQ treated group, while significantly increased in 0.3 mM Spd, 0.3 mM Spd + PQ and 3 mM Spd + PQ compared to the control group (\Cref{fig:70_pq_hippo_autophagy_wb}: \textbf{A}\textit{iii} \& \textbf{B}). Importantly, treatment with spermidine significantly increased LAMP2A expression in 0.3 mM Spd, 0.3 mM Spd + PQ, 3 mM Spd and 3 mM Spd + PQ compared to the PQ treated group (\Cref{fig:70_pq_hippo_autophagy_wb}: \textbf{A}\textit{iii} \& \textbf{B}). These results suggest that PQ causes lysosomal damage, potentially leading to defective proteolysis, and that spermidine rescues the effect, preserving lysosomal abundance. Defective lysosomes are implicated in AD \citep{Nixon2011}, thus, it is possible to assume that PQ might be de-acidifying the lysosomes, leading to lysosomal dysfunction. In support of this, rotenone, another pesticide like PQ has been shown to induce lysosomal dysfunction by increasing lysosomal membrane permeability resulting in impaired autophagic flux, and that trehalose, an autophagy inducer, rescued this effect in an \textit{in vivo} model of PD \citep{Wu2015}. These findings are supporting both the lysosomal damage by PQ and the rescuing effect by spermidine. Taken together, our findings shed light onto the mechanisms of spermidine induced neuronal protection in this model system. Due to the importance of the hippocampus in neurodegeneration and also in memory formation, these results are of significance and point towards spermidine as a therapeutic agent for neurodegenerative diseases.

\subsection{Cortex region assessment}
\subsubsection{The effect of spermidine treatment intervention using Heamatoxylin and Eosin (H\&E) staining in PQ-induced brain injury}
The cortex region is susceptible to pathological changes associated with AD, particularly at the later stages of the disease \citep{Braak1998,Braak1991,Braak2012}, thus effects of the treatment intervention were assessed in a similar fashion as done for hippocampus described earlier in this chapter. H\&E staining showed no major abnormal changes in the structural architecture of the cortex upon treatment intervention (\Cref{fig:70_pq_cortex_he_lm}), in agreement with the H\&E data observed in the hippocampus, suggesting that that PQ-induced toxicity in the brain did not cause any obvious structural damage. 

\subsubsection{The effect of spermidine treatment intervention on lipid peroxidation in PQ-induced brain injury}
Western blot analysis for 4HNE protein adducts revealed to our surprise no significant differences in the PQ treated group compared to the control group at 35, 45 and 52 kDa (1.08 $\pm$ 0.17) (\Cref{fig:70_pq_cortex_hne_wb}: \textbf{A}\textit{i}, \textit{ii}, \textit{iii} \& \textbf{B}). At 35 kDa, a significant decrease in 4HNE signal was observed in 0.3 mM Spd + PQ group compared PQ treated group (\Cref{fig:70_pq_cortex_hne_wb}: \textbf{A}\textit{i} \& \textbf{B}). At 45 \& 55 kDa, 4HNE signal was significantly decreased in both spermidine dosages i.e. 0.3 mM Spd + PQ and 3 mM Spd + PQ compared to the PQ treated group. Additionally, a significant decrease was observed in 0.3 mM Spd + PQ compared to 0.3 mM Spd and in 3 mM Spd compared to 0.3 mM Spd at 45 \& 52 kDa (\Cref{fig:70_pq_cortex_hne_wb}: \textbf{A}\textit{ii}, \textit{iii} \& \textbf{B}). These results are supported by the immunofluorescence micrographs, where a moderately enhanced fluorescence signal for 4HNE was observed in the PQ treated group compared to 0.3 mM Spd + PQ and 3 mM Spd + PQ (\Cref{fig:70_pq_cortex_hne_fm}). Overall, these results suggest that PQ does not induce robust oxidative stress in the cortex, in contrast to the results observed in the hippocampus, which may suggest that the cortex is less susceptible to oxidative damage, as it is only affected later in the AD disease progression \citep{lumkwana2017}. Notably, spermidine i.e. 0.3 mM Spd + PQ and 3 mM Spd + PQ decreased lipid peroxidation below PQ and control levels, suggesting that spermidine mediated favourable effects and did so in a dose-dependent manner as evident by the significant decrease in 4HNE signal with 3 mM Spd compared to 0.3 mM Spd, supporting the notion that spermidine acts as an antioxidant.

\subsubsection{The effect of spermidine treatment intervention on GFAP in PQ-induced brain injury}
Consistent with the observations made in the hippocampus, an enhanced GFAP signal was detected in the PQ treated group compared to all other groups (\Cref{fig:70_pq_cortex_gfap_fm}), further suggesting that PQ induces astrocyte gliosis as likely indication for neuronal injury in this region and that spermidine protects against neuronal damage. In line with these findings, enhanced levels of GFAP were demonstrated in the cortex in 3xTgAD mice \citep{Kamphuis2014,Oddo2003}. Since astrogliosis is a hallmark of AD neuropathology \citep{Gotz2001,Grundke-Iqbal1989}, these findings suggest that PQ treatment in 3 weeks as implemented in this study, induced AD like pathology and that spermidine might have therapeutic potential in AD. In fact, currently, spermidine is part of an ongoing clinical trial (https://clinicaltrials.gov/ct2/show/NCT02755246) where adults (ages of 60 - 80) at risk of developing AD are receiving spermidine. So far, spermidine given long-term as dietary supplement has been found to be safe and-well tolerated in mice and older adults \citep{Schwarz2018}. Moreover, spermidine has been found to improve memory performance in aged adults compared to the placebo group \citep{Wirth2018}, suggesting that spermidine may potentially delay memory loss with age.

\subsubsection{The effect of spermidine treatment intervention on APP expression in PQ-induced brain injury}
Western blot analysis of APP protein expression revealed a trend towards an increase in APP signal in the PQ treated group compared to the control group, but no significance was reached  (\Cref{fig:70_pq_cortex_app_wb}: \textbf{A} \& \textbf{B}). In support of our findings, \citet{Chen2012} found no differences in the APP expression in the PQ treated mice compared to the control mice in the cortex after 3 weeks of treatment, as done in our study. Notably, spermidine at both dosages i.e. 0.3 mM Spd + PQ and 3 mM Spd + PQ significantly decreased APP expression compared to the PQ treated group, suggesting that spermidine mediated protection, decreasing inherent APP syntheisis. 

\subsubsection{The effect of spermidine treatment intervention on acetylated $\alpha$- tubulin in PQ-induced brain injury}
In contrast to the observations made in the hippocampus, no significant differences were observed in the levels of acetylated $\alpha$-tubulin in the 0.3 mM Spd, 0.3 mM Spd + PQ and 3 mM Spd compared to the control and PQ treated group, however, an unexpected decline in acetylated $\alpha$-tubulin below the levels of the control and PQ group was observed at 3 mM Spd + PQ (\Cref{fig:70_pq_cortex_tubulin_wb}: \textbf{A} \& \textbf{B}). These results suggest that PQ did not affect acetylated $\alpha$-tubulin in this region and that the low dose of spermidine in combination with PQ also did not affect microtubules, but the combination of PQ with the high dose of spermidine was destabilizing microtubules by mechanisms that are unknown, potentially related to autophagy. 

\subsubsection{The effect of spermidine treatment intervention on autophagy in PQ-induced brain injury}
Our results reveal that LC3-II expression was enhanced in spermidine treated mice, with low dose i.e. 0.3 mM Spd + PQ compared to the PQ treated group, but not in the PQ treated group itself compared to the control group (\Cref{fig:70_pq_cortex_autophagy_wb}: \textbf{A}\textit{i} \& \textbf{B}). These results suggest that autophagy activity was enhanced above basal levels in response to spermidine treatment with the low dose, while PQ exposure alone did not induce autophagy above control levels. The findings are supported by immunofluorescence micrographs which revealed that GFP-LC3 signal was enhanced only in the groups 0.3 mM Spd + PQ and 3 mM Spd in this region (\Cref{fig:70_pq_cortex_lc3_fm}). Consistent with these observations, a significant increase in p62 protein levels was observed in the 0.3 mM Spd + PQ compared to PQ treated group (\Cref{fig:70_pq_cortex_autophagy_wb}: \textbf{A}\textit{ii} \& \textbf{B}). 

In addition, a significant decrease in p62 was observed in the 3 mM Spd compared to 0.3 mM Spd  and in 3 mM Spd + PQ compared to 0.3 mM Spd + PQ, suggesting a dose-dependent decrease in p62 expression. The findings are supported by immunofluorescence micrographs which revealed that p62 signal was enhanced in 0.3 mM Spd and 0.3 mM Spd + PQ compared to 3 mM Spd and 3 mM Spd + PQ, respectively.
(\Cref{fig:70_pq_cortex_p62_fm}). Lastly, PQ treatment reduced LAMP2A expression below that of the control, but not significantly, yet spermidine at both dosages increased LAMP2A protein levels above that of the PQ treated group (\Cref{fig:70_pq_cortex_autophagy_wb}: \textbf{A}\textit{iii} \& \textbf{B}). These results further suggest that the cortex region is not as highly sensitive to PQ-induced lysosomal damage as observed in the hippocampus and that spermidine protects against lysosomal damage. Taken together, our findings show that even though the cortex is less impacted by PQ toxicity, spermidine still confers protective effects against neuronal damage.  

In conclusion, in this study, we highlight that spermidine protect against PQ-induced neuronal toxicity by oxidative damage, neuronal damage (APP and GFAP), increased the stability of microtubules by enhancing autophagic activity in the hippocampus and cortex. Therefore, administration of spermidine may represent a favourable therapeutic strategies for the treatment of AD. Further studies, using an \textit{in vivo} model over-expressing APP are warranted to further verify the protective effects of spermidine.