\chapter{Method development for Correlative Light and Electron Microscopy (CLEM)}
\section{Introduction}
CLEM is a unique technique that combines two technologies, light/fluorescence microscopy and electron microscopy (TEM/SEM) in order to overcome the limitations presented by each. Using fluorescence microscopy, one is able to visualize labelled proteins and organelles of interest, even using live cell imaging, however, this technique is limited by the spatial resolution and the lack of ultra-structural information \citep{Russell2017}. Using electron microscopy on the other end, one is able to visualize morphology and ultrastructure of the specimen in great detail at a very high resolution, however, this is limited to a fixed sample, hence a single time point and a restricted imaging view {Liss2016,Russell2017}. Thus, a combination of both techniques, CLEM makes it possible to monitor rare dynamic biological processes in live or fixed cells and thereafter obtain ultra-structural information, which is not possible with either of these techniques alone. Although it is a powerful technique, it is technically challenging and requires skills and experience in both fields of fluorescence and electron microscopy, which have additionally often been separate skills. Furthermore, imaging in 3 dimensions is even more challenging. Over the years, CLEM is become more popular among biologists especially in the field of autophagy \citep{Duke2014,Hosseini2014, Russell2017,Gudmundsson2019}. In this chapter, we aim to firstly optimize a protocol for 2 \& 3 dimensional CLEM that can be used in various cell types, and secondly, to use the optimised protocol to localize the autophagic pathway intermediates, i.e. autophagosomes. Finally, we aim to assess the effect of a low and high concentration of spermidine on autophagic flux. For the optimization of the CLEM protocol, the first in a South African setting, various cell types were used, either live or fixed, transfected with plasmid or exposed to antibody staining. The cells were imaged using first confocal microscopy and then prepared for ultra-structural analysis using various SEM based imaging techniques for 2D imaging such as scanning transmission electron microscopy (STEM), scanning electron microscopy (SEM), and 3D imaging such as serial block-face scanning electron microscopy (SBF-SEM) and focused ion beam scanning electron microscopy (FIB-SEM). Next, to assess the localization of autophagosomes and to monitor autophagic flux, GT 1-7 cells were transfected using GFP-LC3-RFP-LC3$\Delta$G plasmid, a probe which allows to measure autophagic flux accurately. Subsequently, cells were imaged in 3D using super resolution structured illumination microscopy (SR-SIM) and then prepared for serial section imaging using FIB-SEM. 

\section{2D \& 3D-CLEM optimization of \textit{in vitro} workflow}
In order to optimise a protocol for 2D \& 3D-CLEM for an \textit{in vitro} model, a published mega-metal protocol was used \citep{Russell2017} and adapted accordingly. Briefly, MEF cells transiently expressing GFP-LC3 (a gift from Prof Noboru Mizushima, Tokyo University) were cultured overnight and imaged with a confocal microscope and then processed for viewing with STEM for ultrastructural assessment. Our results indicated well preserved LC3 puncta fluorescence signal as well as a good ultrastructural perseveration, contrast, and no precipitation using STEM (\Cref{fig:60_spd_clem_mef_a}). 

\begin{figure}[!htbp]
\center
  \includegraphics[width=\linewidth]{figures/chapter60/60_spd_clem_mef_a}
  \caption[CLEM optimization protocol on transfected cells]{\textbf{CLEM optimization protocol on transfected cells.} Shown are fluorescence micrographs indicating the preserved localization of GFP-LC3 signal (left) and STEM micrographs indicating intracellular structural detail such as mitochondrial and various vacuolar structures (middle and right). Scale bar: 10 $\mu$m.}
  \label{fig:60_spd_clem_mef_a}
\end{figure} 

\section{The effect of a low and high concentration of spermidine on the localization and ultrastructure of acetylated $\alpha$-tubulin upon PQ-induced toxicity using CLEM}
\label{sec:Effect_low_high_spermidine_localization_ultrastructure of_tubulin_CLEM}
After successful optimisation of the megametal protocol in transfected MEF cells (\Cref{fig:60_spd_clem_mef_a}), we employed the optimised protocol to assess the localization and ultra-structure of acetylated $\alpha$-tubulin in cells upon treatment. GT1-7 cells were treated with PQ, 1 $\mu$M Spd,  1 $\mu$M Spd + PQ, 10 $\mu$M Spd and 10 $\mu$M Spd + PQ. Subsequently, cells were prepared for immunofluorescence and imaging with STEM. Our fluorescence data showed an increase in acetylated $\alpha$-tubulin signal and density following treatment with PQ, 1 $\mu$M Spd, 1 $\mu$M Spd + PQ, 10 $\mu$M Spd and 10 $\mu$M Spd + PQ compared to the control group (\Cref{fig:60_spd_clem_tubulin_a}).  Moreover, acetylated  $\alpha$-tubulin was  organised around the nuclear regions. More importantly, acetylated $\alpha$-tubulin signal and density was increased in the combinations groups compared to PQ alone. However, STEM data revealed poor contrast and ultra-structure preservation (\Cref{fig:60_spd_clem_tubulin_a}) in comparison to the good ultrastructural perseveration obtained with MEF cells (\Cref{fig:60_spd_clem_mef_a}). 

\begin{landscape}
\begin{figure}[!htbp]
\centering
  \includegraphics[width=\linewidth]{figures/chapter60/60_spd_clem_tubulin_a}
  \caption[Effect of spermidine on the localization and ultrastructure of acetylated $\alpha$-tubulin upon PQ-induced toxicity using CLEM]{\textbf{Effect of spermidine on the localization and ultrastructure of acetylated $\alpha$-tubulin upon PQ-induced toxicity using CLEM.} Shown are FM micrographs indicating the localization of acetylated $\alpha$-tubulin of tubulin network and STEM micrograph showing the poor ultrastructural visualization of the cell (right micrograph). Arrows indicate fragmented mitochondrial and tubulin network. Scale bar: 10 $\mu$m.}
  \label{fig:60_spd_clem_tubulin_a}
\end{figure} 
\end{landscape}

\section{Permeabilization buffer optimization for enhanced contrast}
Next, we optimised the permeabilization buffer since permeabilization with 0.2 \% triton X resulted in a poor ultra-stucture contrast (i.e. poor binding of heavy metals and osmium tetroxide) during STEM imaging. Here, a gentle permeabilization buffer, 0.1 \% saponin was tested in order to improve the abundance of lipid available for binding of osmium tetroxide and heavy metals. Briefly, GT 1-7 cells were cultured overnight in matek dishes. Next, one dish (here referred to as Con group) was fixed, and antibody stained, while the second dish (referred to as  treated group) was fixed, permeabilized with 0.1 \% saponin for 10 min and antibody stained. Both dishes were imaged with FM and prepared using mega-metal protocol for ultrastructural analysis with SBF-SEM. Our results showed that the control group had a very dim fluorescence signal as expected due to poor penetration of antibodies across the membrane, due to the omitted permeabilization step (\Cref{fig:60_spd_clem_saponin_a}: \textit{i}). In contrast, the treated group (0.1 \% saponin) had bright fluorescent signal due to antibody binding (\Cref{fig:60_spd_clem_saponin_a}: \textit{iv}), however, the structural resolution of acetylated $\alpha$-tubulin was compromised compared to the fine structural detail observed in cells that were permeabilized with 0.2 \% triton X (\Cref{fig:60_spd_clem_tubulin_a}). Ultra-structural analysis using FIB-SEM revealed a good preservation and contrast indicated by the clear structures of the mitochondrial cristae, as well as single and double membrane vesicle, potentially AVs (\Cref{fig:60_spd_clem_saponin_a}: \textit{ii}, \textit{iii}, textit{v}, \textit{vi}) \& Fig.6.4).These results together with the results obtained in (\cref{sec:Effect_low_high_spermidine_localization_ultrastructure of_tubulin_CLEM} suggest that the preservation of the ultra-structure as well as good contrast is obtained at a cost of FM resolution or good FM resolution is obtained at a cost ultra-structural preservation and loss of contrast in immuno-stained cells.

\begin{landscape}
\begin{figure}[!htbp]
\center
  \includegraphics[width=\linewidth]{figures/chapter60/60_spd_clem_saponin_a}
  \caption[Optimization of permeabilization buffer for enhanced contrast]{\textbf{Optimization of permeabilization buffer for enhanced contrast.} Shown are FM \& FIB-SEM micrographs indicating the localization acetylated $\alpha$-tubulin and ultra-structure in the control (i, ii \& iii) and treated group (iv, v, \& vi). Arrows indicate mitochondrial (white) and AVs (yellow) in treated and untreated cells. Scale bar: 10 $\mu$m.}
  \label{fig:60_spd_clem_saponin_a}
\end{figure} 
\end{landscape}

\section{Optimization of CLEM workflow for GT 1-7 transfected cells}
Next, we optimised a CLEM workflow for transected GT 1-7 cells in order to assess the localisation of autophagosomes and autolysosomes. Hence, GT 1-7 were transfected with GFP-LC3-RFP-LC3$\Delta$G plasmid, a recently developed probe that allows to measure autophagy. When expressed in cells, it is cleaved by ATG4 into equimolar amounts of GFP-LC3 and RFP-LC3$\Delta$G, where GFP-LC3 is degraded during autophagy, while RFP-LC3$\Delta$G remains in the cytosol serving as an internal control \citep{Kaizuka2016}. Thus, the ratio of GFP:RFP is used to estimate autophagy flux or activity, where a decrease in GFP:RFP indicates higher flux. Transfected cells were untreated or treated with BafA1 for 4 h, imaged with confocal microscopy and prepared using mega-metal protocol for 3D ultrastructural analysis using FIB-SEM

Our results showed an increase in GFP-LC3 (green puncta) following treatment with BafA1 compared to the control group, suggesting that fusion of autophagomes with lysosomes was indeed blocked (\Cref{fig:60_spd_clem_baf_a}). Using FIB-SEM, we observed a good ultra-structural contrast as well as a poor ultra-structural preservation (\Cref{fig:60_spd_clem_baf_a}), suggesting that firstly, cells were not fixed properly and secondly, cell may have dried during dehydration with ethanol. 

\begin{landscape}
\begin{figure}[!htbp]
\center
  \includegraphics[width=\linewidth]{figures/chapter60/60_spd_clem_baf_a}
  \caption[Optimization of CLEM workflow for GT 1-7 transfected cells]{\textbf{Optimization of CLEM workflow for GT 1-7 transfected cells.} CLEM on transfected cells. Left panel: FM micrographs of control and BafA1 treated cells (left panel) showing the localization of autophagosomes. Right panel: CLEM micrographs of BafA1 treated cells with FIB-SEM micrograph showing poor ultra-structure preservation. Arrows indicate mitochondria (white) and AVs (yellow). Scale bar: 5 $\mu$m.}
  \label{fig:60_spd_clem_baf_a}
\end{figure} 
\end{landscape}

\section{Localization of autophagosome in response to spermidine treatment using 2D-CLEM}
In order to further improve ultra-structure for better visualization of the intracellular structures in the cells, additional minor adjustments in the megametal protocol were implemented. During sample preparation, care was taken not to extend the time of the dehydration with ethanol and not to leave the cells without liquid as this affects the ultra-structure preservation. For this part of the protocol, two samples were worked on at a time to ensure highest control over the protocol implementation. Our results showed improvement in the preservation of the ultra-structure, however the precipitation issue remained which hindered a clear visualization the autophagosomes within an ultra-structure context (\Cref{fig:60_spd_2d_clem_a} \& \Cref{fig:60_spd_2d_clem_b}). A decrease in GFP-LC3 (green puncta) and an increase in the RFP-LC3$\Delta$G (red puncta) was observed in the cells treated with 1 $\mu$M Spd and 10 $\mu$M Spd compared to the control group, suggesting a higher autophagic flux. In addition, inhibition of degradation resulted in the increase in the accumulation of GFP-LC3 puncta  and GFP-LC3/RFP-LC3$\Delta$G (yellow puncta) in BafA1 treated cells (BafA1, 1 $\mu$M Spd + BafA1 and 10 $\mu$M Spd + BafA1) (\Cref{fig:60_spd_2d_clem_b}) compared to BafA1 untreated cells (Con, 1 $\mu$M Spd and 10 $\mu$M Spd) (\Cref{fig:60_spd_2d_clem_a}). Furthermore, both green and yellow puncta appear to be localized in vesicles which are low in electron density and appear to be smaller and more non spherical compared to the control. Moreover, unknown large structures of red puncta that appeared to be of electron dense material were observed in 1 $\mu$M Spd. 

%Add figure 6.7 and cross ref it and section 2.12.1
\begin{landscape}
\begin{figure}[!htbp]
\centering
  \includegraphics[width=\linewidth]{figures/chapter60/60_spd_2d_clem_a}
  \caption[Localization of autophagosome in response to spermidine treatment using 2D-CLEM]{\textbf{Localization of autophagosome in response to spermidine treatment using 2D-CLEM.} Representative overlay micrographs indicate control, 1 $\mu$M Spd \& 10 $\mu$M Spd. Cells were acquired with confocal microscopy and SEM. Arrows indicate areas punctate structures are localised.Scale bar: 10 $\mu$m.}
  \label{fig:60_spd_2d_clem_a}
\end{figure} 
\end{landscape}
\begin{landscape}
\begin{figure}[!htbp]
\centering
  \includegraphics[width=\linewidth]{figures/chapter60/60_spd_2d_clem_b}
  \caption[Localization of autophagosome in response to spermidine treatment using 2D-CLEM]{\textbf{Localization of autophagosome in response to spermidine treatment using 2D-CLEM.} Representative overlay micrographs indicate BafA1, 1 $\mu$M Spd + BafA1 \& 10 $\mu$M Spd + BafA1. Cells were acquired with confocal microscopy and SEM. Arrows indicate areas punctate structures are localised.Scale bar: 10 $\mu$m.}
  \label{fig:60_spd_2d_clem_b}
\end{figure} 
\end{landscape}


\section{Final analysis of ultra-structure visualization for localization of autophagosomes in 3D}
In order to clearly localize the autophagosomes, GT1-7 cells expressing GFP-LC3-RFP-LC3$\Delta$G plasmid were treated as described above (section 6.2.2.5). Next, cells were prepared for FIB-EM imaging using the megametal protocol. In addition to the precaution mentioned above, cells were fixed in PFA that was purchased already in a liquid form containing no methanol to preserve the ultra-structure. Secondly, to avoid reversal of fixation in the cells with PFA, cells were imaged immediately and then fixed in the mixture of GA:PFA to cross link structures. Thirdly, absolute care was taken when making lead aspartate (pH and temperature), as lead aspartate can cause precipitation in the sample if the conditions are not optimal, thus leading to poor contrast.

3D-CLEM with SR-SIM and FIB-SEM allowed high resolution imaging where autophagosomes could be identified with high precision. Contents of the cells i.e. mitochondria, nucleus and AVs (electron dense/lucent) were clearly observed FIB-SEM imaging (Fig.6.8). A decrease in GFP-LC3 puncta was observed following treatment with 1 \& 10 $\mu$M Spd and compared to the control group. Moreover, an increase in GFP-LC3 puncta was observed in the BafA1 treated group, 1 $\mu$M Spd + BafA1 and 10 $\mu$M Spd + BafA1 compared to the control group. Importantly, GFP-LC3 puncta accumulated in the combination groups, 1 $\mu$M Spd + BafA1 and 10 $\mu$M Spd + BafA1 compared to BafA1 only. In addition, GFP-LC3 puncta was higher BafA1 treated groups i.e. BafA1 group, 1 $\mu$M Spd + BafA1 and 10 $\mu$M Spd + BafA1 group compared to BafA1 untreated groups (Con, 1 $\mu$M Spd and 10 $\mu$M Spd). These results suggest the presence of basal as well as induced autophagic flux. GFP-LC3 puncta only and a combination of GFP-LC3 puncta and RFP-LC3$\Delta$G (yellow puncta) was localized in AVs of less electron density (Fig.6.9). Furthermore, a large AV was observed with 1 $\mu$M Spd, where the green and yellow puncta appeared to be around the AV, forming a round membrane which contained red puncta inside that was electron dense, suggesting that this structure was an autophagosome that was filled with material (Fig.6.9). With 10 $\mu$M Spd + BafA1, a big vacuole was also observed, which was RFP-LC3$\Delta$G positive and appeared electron dense, suggesting that the red puncta are more electron dense while the autophagosomes (green puncta) and the yellow puncta are less dense, similar to the cytoplasm. 

%cross reference fig 6.8 and 6.9

\subsection{Area and volume analysis of autophagosomes in 3D}

\section{Discussion: Method development for CLEM}

Electron microscopy and fluorescence microscopy have been extensively used to assess autophagy pathway intermediates i.e. autophagosomes and autolysosomes, however, in the electron micrograph it remains challenging  to identify and localize autophagosomes and autolysosomes with high precision. Although FM allows precise identification and localization of autophagosomes and autolysosomes, as cells can be tagged with specific protein markers, it lacks the ability to reveal the underlying ultra-structural content. On the other hand, EM provides the ultra-structural information, but makes it very difficult to distinguish between autophagy intermediates, even for an experienced scientists, hence the autophagy pathway intermediates are ... collectively termed autophagic vacuoles (AVs) \citep{Eskelinen2008,klionsky2016}. However, when both these two microscopy techniques are combined in CLEM, it is possible to reveal information that cannot be obtained using either of these techniques alone. Thus, this chapter aimed at optimizing a protocol for CLEM that can be subsequently used widely to identify and assess autophagosomes with high precision. Secondly, our aim was to use the optimised protocol to assess the accumulation of autophagosomes following treatments with spermidine at various concentrations in the absence and presence of BafA1 and to confirm that the fluorescent structures are indeed autophagosomes, employing 2D as well as 3D CLEM. One caveat of using 2D CLEM is that, although the imaging acquisition is performed using multiple serial optical sections with confocal microscopy, the EM micrographs are acquired as a single plan within the cell, making it challenging to identify autophagosomes due to the difference in the z space. This problem however can be solved by using automated EM based technologies such electron tomography, FIB-SEM and SBF-SEM \citep{Burel2018,Duke2014,Russell2017,Yla-Anttila2009} which allow the entire cell to be acquired and analysed in 3 dimensions, allowing for volume analysis to be perfomed. Here we show that, after a number of optimizations, we have established a working protocol for CLEM, the first to be established in the South Africa. This established protocol can be successfully used in the 2D and 3D context using transfected cells or cells labelled with nanobodies or fluorescence probes. The established protocol can also be used in cells labelled with antibodies, as we have shown using saponin, but at an expense of loosing FM resolution as staining with antibodies require permeabilization of the membrane which affects the preservation of lipids, thus ....impacting the contrast and hence ultra-structural visualization \citep{Eskelinen2008}.

\section{Localization of autophagosome in response to spermidine treatment using 2D-CLEM}
In order to assess the effect of spermidine in the accumulation and localization of  autophagosomes, GT 1-7 cells expressing GFP-LC3-RFP-LC3$\Delta$G were treated with a low and high concentration of spermidine in the absence and presence of BafA1 and assessed using 2D-CLEM. Our results showed distinct differences in the treatment groups in the sense of the distribution of autophagy pathway intermediate, i.e. GFP-LC3 (green puncta), RFP-LC3$\Delta$G (red puncta), and GFP-LC3/ RFP-LC3$\Delta$G (yellow puncta), suggesting that the autophagy pathway is responding to the treatment (\Cref{fig:60_spd_2d_clem_a} \& \Cref{fig:60_spd_2d_clem_b}). Secondly, our results showed that the autophagosomes are characterized by structures that are low in electron density, similar to that of the cytoplasm. In support of our findings, using conventional TEM, autophagosomes have been classified as round vacuoles where the material inside the vacuole resembles that of the cell’s cytoplasm \citep{Eskelinen2008}. Thirdly, we observed an accumulation of green and yellow puncta in the BafA1 treated groups (\Cref{fig:60_spd_2d_clem_b}) compared to the BafA1 untreated groups (\Cref{fig:60_spd_2d_clem_a}) and these appeared to be of similar electron density, suggesting that with EM only, one cannot clearly distinguish these two intermediates. Hence the fluorescence signal allows to localize and discern which structures are GFP-LC3 positive and which are both GFP-LC3 and RFP-LC3$\Delta$G positive, since they both have the same electron density form and shape. Fourthly, we observed that GFP-LC3 is primarily localized to the autophagosome membrane as expected, while the RFP-LC3$\Delta$G was not diffusely distributed in the cytoplasm as expected with this probe \citep{Kaizuka2016}, but it was rather membrane bound together with GFP-LC3 and was also engulfed inside vesicles that were GFP-LC3 positive, suggesting it as part of the cargo. To our knowledge, this is the first time that RFP-LC3$\Delta$G in context of the GFP-LC3-RFP-LC3$\Delta$G probe, is reported to be localized in the membrane as well as part of the cytoplasmic cargo. It was previously reported that the expression of the GFP-LC3-RFP-LC3$\Delta$G probe significantly differs among cell or tissues \citep{MoulisandVindis2017}. Fifthly, treatment with spermidine resulted in the accumulation of red puncta, leading to a decrease in the ratio of GFP-LC3 and RFP-LC3$\Delta$G suggesting an increased autophagic activity. In line with our findings, spermidine at 100 $\mu$M was found to decrease the ratio of GFP-LC3 and RFP-LC3$\Delta$G, thus increasing autophagic activity when measured using a microplate reader in Hela cells expressing GFP-LC3-RFP-LC3$\Delta$G \citep{Kaizuka2016}. An unexpected observation was the presence of a large and complex red punctate structures in cells treated with 1 $\mu$M Spd which was localized in much larger vesicles that were electron dense and GFP-LC3 negative, making their identity unclear (\Cref{fig:60_spd_2d_clem_a}). Using 2D-CLEM, others have monitored and identified autophagosomes in starved Hela cells expressing mRFP-GFP-LC3 \citep{Gudmundsson2019} and in HEK293A-GFP-LC3 \citep{Razi2009}. Our findings are to our knowledge, novel and the first study using the GFP-LC3-RFP-LC3$\Delta$G probe in CLEM to assess autophagosome changes following autophagy induction using a low and high concentration  of an autophagy inducer in the presence and absence of BafA1.

\section{Final analysis of ultra-structure visualization for localization of autophagosomes in 3D}
Due to the poor resolution observed in our 2D-CLEM data, we utilised a  3D-CLEM approach to clarify and assess autophagosomes, combined with SR-SIM for the FM mode. CLEM in 3 dimensions allowed for the visualisation of the entire cell in both FM and more especially in FIB-SEM due to much better resolution power in the z dimension, allowing for the identification of autophagosome with high precision, that would have otherwise been missed in 2 dimensions (Fig.6.9). Autophagy induction with 1 \& 10 $\mu$M Spd resulted in a decrease in GFP-LC3 puncta following autophagy induction compared to the control group, while inhibition of autophagosomal lysosome fusion with BafA1 resulted in the accumulation of GFP-LC3 puncta, which was further enhanced in the combination groups i.e 1 $\mu$M Spd + BafA1 and 10 $\mu$M Spd + BafA1 compared to BafA1 alone (Fig.6.10). Furthermore, GFP-LC3 puncta were higher BafA1 treated groups i.e. BafA1 , 1 $\mu$M Spd + BafA1 and 10 $\mu$M Spd + BafA1 compared to the BafA1 untreated groups (Con, 1 $\mu$M Spd and 10 $\mu$M Spd). These results suggest the presence of basal as well as induced autophagic flux, and overall suggest that the autophagy pathway intermediates are responding distinctly to the treatment. To our knowledge, our study is the first to employing GFP-LC3-RFP-LC3$\Delta$G probe to image cells at high resolution and to reveal autophagosome changes with treatment. Indeed, GFP-LC3-RFP-LC3$\Delta$G has only been used in limited studies to monitor autophagy in \textit{in vitro} and \textit{in vivo}, where autophagy in the \textit{in vitro} case was monitored using a ratio-metric plate reader approach \citep{Kaizuka2016}, and therefore, with no high resolution imaging. Consistent with the 2D-CLEM results, our 3D data reveal that both GFP-LC3 puncta and the yellow puncta were localized in AVs of a low electron density, similar to that of cytoplasm (Fig.6.8 and 9 \& 6.10 and 11). (ADDD In addition, we noticed that both GFP-LC3 puncta and the yellow puncta were sometimes localised in electron dense structures). Using 3D cryo-fluorescence and cryo-soft X-ray microscopy, autophagosomes were identified in starved HEK293A cells expressing GFP-LC3, however, cryo-soft X-ray microscopy resulted in poor resolution \citep{Duke2014}. In this study, fusion of autophagosomes with lysosomes was not inhibited, thus the identified autophagosomes could also have been autolysosomes occurring prior to the quenching of GFP-LC3 in the acidic lysosomal environment. Others have used 3D tomography to study the localization of phagophores and shown elegantly that the structures are connected with rough ER, particularly ER cisternae located inside a nascent autophagosome \citep{Yla-Anttila2009}. Interestingly, we observed a large AV in the 1 $\mu$M Spd group, where the green, red and yellow puncta appeared to be decorating the entire AV, forming a membrane structure, which, inside revealed the presence of only red puncta which appeared highly electron dense, suggesting that this structure is likely an autophagosome that has engulfed RFP-LC3$\Delta$G components (Fig.6.10). Furthermore, in the 10 $\mu$M Spd +Baf A1 group, a large vacuole was also observed, which was RFP-LC3$\Delta$ positive and appeared highly electron dense, suggesting that the red puncta are indeed more electron dense while the autophagosomes (green puncta) and the yellow puncta are less dense. To our knowledge, this study is the first one to report the localization of RFP-LC3$\Delta$G puncta in membrane structures and as part of the cargo. Using this probe, we found that the RFP-LC3$\Delta$G puncta are not only cytoplasmic, but are also engulfed and part of the cargo, suggesting that the mechanisms of action are likely more complex and potentially the RFP-LC3$\Delta$G signal observed is more likely reflective of a later time point, engulfed by merely formed autophagosomes. It is likely that a time dependent effect is being observed, where RFP-LC3$\Delta$G, cleaved  through ATG4 remains at first cytoplasmic, as suggested \citep{Kaizuka2016}, but is subsequently engulfed by new formed autophagosomes. These findings are of importance, especially for quantitative image analysis approach, and deserves further study.

Next, autophagosome were segmented, taking advantage of CLEM and the high resolution achieved in FIB-SEM. Our results reveal





























