\chapter{Method development for Correlative Light and Electron Microscopy (CLEM) to assess autophagosomes}
\label{sec:chapter6}
\section{Introduction}
CLEM is a unique technique that combines two technologies, light/fluorescence microscopy and electron microscopy (TEM/SEM) in order to overcome the limitations presented by either technique. Using fluorescence microscopy, one is able to visualize labelled proteins and organelles of interest, even using live cell imaging, however, this technique is limited by the spatial resolution and the lack of ultrastructural information \citep{Russell2017}. Using electron microscopy on the other hand, one is able to visualize morphology and ultrastructure of the specimen in great detail at a very high resolution, however, this is limited to a fixed sample, hence a single time point and a restricted imaging view \citep{Liss2016,Russell2017}. Thus, a combination of both techniques, CLEM makes it possible to monitor rare dynamic biological processes in live or fixed cells and thereafter obtain ultrastructural information, which is not possible with either of these techniques alone. Although it is a powerful technique, it is technically challenging and requires skills and experience in both fields of fluorescence and electron microscopy, which have traditionally often been separate skills. Furthermore, imaging in 3 dimensions further complicates the process, making it even more challenging. Over the years, CLEM has become more popular among biologists, including those in the field of autophagy \citep{Duke2014,Hosseini2014, Russell2017,Gudmundsson2019}. In this chapter, we aim to firstly optimize a protocol for 2 \& 3 dimensional CLEM that can be used in various cell types, and secondly, to use the optimised protocol to localize the autophagic pathway intermediates, i.e. autophagosomes. Finally, we aim to assess the effect of a low and a high concentration of spermidine on autophagic flux, using 3D-based segmentation and morphometric analysis. For the optimization of the CLEM protocol, the first in a South African setting, various cell types were used, either live or fixed, transfected with plasmid or exposed to antibody staining. The cells were imaged using first confocal microscopy and then prepared for ultrastructural analysis using various SEM based imaging techniques for 2D imaging such as scanning transmission electron microscopy (STEM), scanning electron microscopy (SEM), and 3D imaging such as serial block-face scanning electron microscopy (SBF-SEM) and focused ion beam scanning electron microscopy (FIB-SEM). Next, to assess the localization of autophagosomes and to monitor autophagic flux, GT1-7 cells were transfected using GFP-LC3-RFP-LC3$\Delta$G plasmid, a probe which allows to measure autophagic flux accurately. Subsequently, cells were imaged in 3D using super resolution structured illumination microscopy (SR-SIM) and then prepared for serial section imaging using FIB-SEM. 

\section{2D \& 3D CLEM optimization of \textit{in vitro} workflow}
In order to optimise a protocol for 2D \& 3D CLEM for an \textit{in vitro} model, a published mega-metal protocol was used \citep{Russell2017} and adapted accordingly. Briefly, MEF cells stably expressing GFP-LC3 (a gift from Prof Noboru Mizushima, Tokyo University) were cultured overnight and imaged with a confocal microscope and then processed for viewing with STEM for ultrastructural assessment. Our results indicated well preserved LC3 puncta fluorescence signal as well as  good ultrastructural preservation, contrast, and no precipitation using STEM (\Cref{fig:60_spd_clem_mef_a}). 

\begin{figure}[!htbp]
\centering
 \includegraphics[width=\linewidth]{figures/chapter60/60_spd_clem_mef_a}
 \caption[CLEM optimization protocol using transfected cells]{\textbf{CLEM optimization protocol using transfected cells.} Shown are fluorescence micrographs indicating the preserved localization of GFP-LC3 signal (left) and STEM micrographs indicating intracellular structural detail such as mitochondrial and various vacuolar structures (right). Scale bar: 10 $\mu$m.}
 \label{fig:60_spd_clem_mef_a}
\end{figure} 

\section{The effect of a low and a high concentration of spermidine ultrastructure of acetylated $\alpha$-tubulin upon PQ-induced toxicity using CLEM}
\label{sec:Effect_low_high_spermidine_localization_ultrastructure of_tubulin_CLEM}
After successful optimisation of the mega-metal protocol in transfected MEF cells (\Cref{fig:60_spd_clem_mef_a}), we employed the optimised protocol to assess the ultrastructure of acetylated $\alpha$-tubulin in cells upon treatment. GT1-7 cells were treated with PQ, 1 $\mu$M Spd, 1 $\mu$M Spd + PQ, 10 $\mu$M Spd and 10 $\mu$M Spd + PQ. Subsequently, cells were prepared for immunofluorescence and imaging with STEM. Our fluorescence data showed an increase in acetylated $\alpha$-tubulin signal and density following treatment with PQ, 1 $\mu$M Spd, 1 $\mu$M Spd + PQ, 10 $\mu$M Spd and 10 $\mu$M Spd + PQ compared to the control group (\Cref{fig:60_spd_clem_tubulin_a}). Moreover, acetylated $\alpha$-tubulin was organised around the nuclear regions. More importantly, acetylated $\alpha$-tubulin signal and density was increased in the combinations groups compared to PQ alone. However, STEM data revealed poor contrast and poor ultrastructure preservation (\Cref{fig:60_spd_clem_tubulin_a}) in comparison to the good ultrastructural preservation obtained with the transfected MEF cells (\Cref{fig:60_spd_clem_mef_a}), suggesting that the membrane permeabilization performed during immunostaning may have resulted in the loss of lipids and other components of the cell that are needed for sufficient binding of osmium tetroxide and other heavy metals used in the mega-metal protocol process in order to improve ultrastructural contrast. 

\begin{landscape}
\begin{figure}[!htbp]
\vspace{2cm}
\center
 \includegraphics[width=\linewidth]{figures/chapter60/60_spd_clem_tubulin_a}
 \caption[The effect of a low and a high concentration of spermidine ultrastructure of acetylated $\alpha$-tubulin upon PQ-induced toxicity using CLEM]{\textbf{The effect of a low and a high concentration of spermidine ultrastructure of acetylated $\alpha$-tubulin upon PQ-induced toxicity using CLEM.} Shown are FM micrographs indicating the localization of acetylated $\alpha$-tubulin and STEM micrograph showing poor ultrastructural visualization of the cell (right micrograph). Scale bar: 10 $\mu$m.}
 \label{fig:60_spd_clem_tubulin_a}
\end{figure} 
\end{landscape}

\section{Permeabilization buffer optimization for enhanced contrast}
Next, we optimised the permeabilization buffer since permeabilization with 0.2 \% triton X resulted in a poor ultrastructure contrast (i.e. poor binding of heavy metals and osmium tetroxide) during STEM imaging. Here, a gentle permeabilization buffer, 0.1 \% saponin was tested in order to improve the abundance of lipid available for binding of osmium tetroxide and heavy metals. Briefly, GT1-7 cells were cultured overnight in matek dishes. Next, one dish (here referred to as control group) was fixed, and antibody stained, while the second dish (referred to as treated group) was fixed, permeabilized with 0.1 \% saponin for 10 min and subsequently stained with antibodies. Both dishes were imaged with FM and prepared using the mega-metal protocol for ultrastructural analysis with SBF-SEM. Our results showed that the control group had a very dim fluorescence signal as expected due to poor penetration of antibodies across the membrane, because to the omitted permeabilization step (\Cref{fig:60_spd_clem_saponin_a}: \textit{i}). In contrast, the treated group (0.1 \% saponin) had bright fluorescent signal due to antibody binding (\Cref{fig:60_spd_clem_saponin_a}: \textit{iv}), however, the structural resolution of acetylated $\alpha$-tubulin was compromised compared to the fine structural detail observed in cells that were permeabilized with 0.2 \% triton X (\Cref{fig:60_spd_clem_tubulin_a}). However, ultrastructural analysis using FIB-SEM revealed a good preservation and contrast indicated by the clear structures of the mitochondrial cristae, as well as single and double membraned vesicles, potentially AVs in the control and treated group (\Cref{fig:60_spd_clem_saponin_a}: \textit{ii}, \textit{iii}, \textit{v}, \textit{vi}) \& (\Cref{fig:60_spd_clem_saponin_b}). These results together with the results obtained in \cref{sec:Effect_low_high_spermidine_localization_ultrastructure of_tubulin_CLEM} suggest that the preservation of the ultrastructure as well as good contrast is obtained at a cost of FM resolution or that good FM resolution is obtained at a cost ultrastructural preservation and loss of contrast in immunostained cells.

\begin{landscape}
\begin{figure}[!htbp]
\center
 \includegraphics[width=0.9\linewidth]{figures/chapter60/60_spd_clem_saponin_a}
 \caption[Optimization of the permeabilization buffer for enhanced contrast]{\textbf{Optimization of the permeabilization buffer for enhanced contrast.} Shown are FM \& FIB-SEM micrographs indicating the localization of acetylated $\alpha$-tubulin and ultrastructure in the control (i, ii \& iii) and treated group (iv, v, \& vi). Arrows indicate mitochondria (white) and AVs (yellow) in treated and untreated cells. Scale bar: 10 $\mu$m.}
 \label{fig:60_spd_clem_saponin_a}
\end{figure} 
\end{landscape}

\begin{landscape}
\begin{figure}[!htbp]
\vspace{1.7cm}
\center
 \includegraphics[width=\linewidth]{figures/chapter60/60_spd_clem_saponin_b}
 \caption[FIB-SEM micrographs of the control cell at different z-levels]{\textbf{FIB-SEM micrographs of the control cell at different z-levels.}}
 \label{fig:60_spd_clem_saponin_b}
\end{figure} 
\end{landscape}

\section{Optimization of CLEM workflow for GT1-7 transfected cells}
Next, we optimised a CLEM workflow for transected GT1-7 cells in order to assess the localisation of autophagosomes. Hence, GT1-7 cells were transfected with GFP-LC3-RFP-LC3$\Delta$G plasmid, a recently developed probe that allows to measure autophagy. When expressed in cells, it is cleaved by ATG4 into equimolar amounts of GFP-LC3 and RFP-LC3$\Delta$G, where GFP-LC3 is degraded during autophagy, while RFP-LC3$\Delta$G remains in the cytosol serving as an internal control \citep{Kaizuka2016}. Thus, the ratio of GFP:RFP is used to estimate autophagy flux or activity, where a decrease in GFP:RFP indicates higher flux. Transfected cells were untreated or treated with BafA1 for 4 h, imaged with confocal microscopy and prepared using the mega-metal protocol for 3D ultrastructural analysis using FIB-SEM

Our results showed an increase in GFP-LC3 (green puncta) following treatment with BafA1 compared to the control group, suggesting that fusion of autophagomes with lysosomes was indeed blocked (\Cref{fig:60_spd_clem_baf_a}). Using FIB-SEM, we observed good ultrastructural contrast, but poor ultrastructural preservation (\Cref{fig:60_spd_clem_baf_a}), suggesting that firstly, cells were not fixed favourable and secondly, cells may have dried during the dehydration step with ethanol. Nevertheless, the workflow was improving as cells were identified more swiftly.

\begin{landscape}
\begin{figure}[!htbp]
\center
 \includegraphics[width=\linewidth]{figures/chapter60/60_spd_clem_baf_a}
 \caption[Optimization of CLEM workflow for GT1-7 transfected cells]{\textbf{Optimization of CLEM workflow for GT1-7 transfected cells.} Shown are FM micrographs of the control and BafA1 treated cells (left) indicating the localization of autophagosomes and CLEM micrographs of BafA1 treated cell (right) with FIB-SEM micrograph showing poor ultrastructural preservation. Arrows indicate mitochondria (white) and AVs (yellow). Scale bar: 5 $\mu$m.}
 \label{fig:60_spd_clem_baf_a}
\end{figure} 
\end{landscape}

\section{Localization of autophagosomes in response to a low and a high concentration of spermidine in 2D CLEM}\label{sec:localization_autophagosomes_low_high_spermidine_2DCLEM}
In order to assess the effect of a low and a high concentration of spermidine on autophagosomes, GT1-7 cells expressing GFP-LC3-RFP-LC3$\Delta$G were treated with 1 $\mu$M Spd and 10 $\mu$M Spd for 8 h in the presence and absence of 400 nM BafA1 (4h) and imaged with confocal microscopy and SEM. In order to further improve ultrastructure for better visualization of the intracellular structures, additional minor adjustments in the mega-metal protocol were implemented (\cref{sec:sample preparation for SEM}). During sample preparation, care was taken not to extend the time of the dehydration with ethanol and not to leave the cells at any point without liquid as this affects the ultrastructure preservation. For this part of the protocol, only two samples were worked on at a time to ensure highest control over the protocol implementation. Our results showed improvement in the preservation of the ultrastructure, however the precipitation issue remained which hindered a clear visualization of the autophagosomes within an ultrastructure context (\Cref{fig:60_spd_2d_clem_a} \& \Cref{fig:60_spd_2d_clem_b}). A decrease in GFP-LC3 (green puncta) and an increase in the RFP-LC3$\Delta$G (red puncta) was observed in the cells treated with 1 $\mu$M Spd and 10 $\mu$M Spd compared to the control group, suggesting a higher autophagic flux. In addition, inhibition of degradation resulted in the increase in the accumulation of GFP-LC3 puncta and GFP-LC3/RFP-LC3$\Delta$G (yellow puncta) in BafA1 treated cells (BafA1, 1 $\mu$M Spd + BafA1 and 10 $\mu$M Spd + BafA1) (\Cref{fig:60_spd_2d_clem_b}) compared to BafA1 untreated cells (control, 1 $\mu$M Spd and 10 $\mu$M Spd) (\Cref{fig:60_spd_2d_clem_a}). Furthermore, both green and yellow puncta localized in vesicles which are low in electron density and appeared to be smaller and more non-spherical compared to the control. Moreover, unknown large structures of red puncta that appeared to be of electron dense material were observed in 1 $\mu$M Spd. 

\begin{landscape}
\begin{figure}[!htbp]
\center
 \includegraphics[width=\linewidth]{figures/chapter60/60_spd_2d_clem_a}
 \caption[Localization of autophagosomes in response to a low and a high concentration of spermidine in 2D CLEM]{\textbf{Localization of autophagosomes in response to a low and a high concentration of spermidine in 2D CLEM.} Representative overlay micrographs indicate control, 1 $\mu$M Spd \& 10 $\mu$M Spd. Cells were acquired using confocal microscopy and SEM. Arrowheads indicate areas where punctate structures are localised. Scale bar: 10 $\mu$m.}
 \label{fig:60_spd_2d_clem_a}
\end{figure} 
\end{landscape}

\begin{landscape}
\begin{figure}[!htbp]
\centering
 \includegraphics[width=\linewidth]{figures/chapter60/60_spd_2d_clem_b}
 \caption[Localization of autophagosomes in response to treatment with a low and a high concentration of spermidine in 2D CLEM]{\textbf{Localization of autophagosomes in response to treatment with a low and a high concentration of spermidine in 2D CLEM.} Representative overlay micrographs indicate BafA1, 1 $\mu$M Spd + BafA1 \& 10 $\mu$M Spd + BafA1. Cells were acquired using confocal microscopy and SEM. Arrowheads indicate areas where punctate structures are localised. Scale bar: 10 $\mu$m.}
 \label{fig:60_spd_2d_clem_b}
\end{figure} 
\end{landscape}

\section{3D ultrastructural visualization and localization of autophagosomes in response to a low and a high concentration of spermidine}
The 2D based assessment of autophagosomes did not achieve the desired clarity and resolution, hence, in order to clearly visualise and localize autophagosomes, GT1-7 cells expressing GFP-LC3-RFP-LC3$\Delta$G plasmid were treated with a low and a high concentration of spermidine in the presence and absence of BafA1 and imaged with SR-SIM. Next, cells were prepared for FIB-SEM imaging using the mega-metal protocol. In addition to the precaution taken for the 2D analysis (\cref{sec:localization_autophagosomes_low_high_spermidine_2DCLEM}), cells were fixed in PFA that was purchased already in a liquid form, which contained no methanol to better preserve the ultrastructure. Secondly, to avoid reversal of fixation in the cells with PFA, cells were imaged immediately and then fixed in the mixture of GA:PFA to cross link structures. Thirdly, absolute care was taken when making lead aspartate (pH and temperature), as lead aspartate can cause precipitation in the sample if the conditions are not optimal, thus leading to a poor ultrastructural contrast.

3D CLEM (SR-SIM and FIB-SEM) allowed high-resolution imaging where autophagosomes could be identified with high precision. Organelles i.e. mitochondria, nucleus and AVs (electron dense/lucent) were clearly observed using FIB-SEM imaging (\Cref{fig:60_spd_3d_clem_overlay_a} \& \Cref{fig:60_spd_3d_clem_overlay_b}) . Our results revealed a decrease in GFP-LC3 puncta following treatment with 1 $\mu$M Spd \& 10 $\mu$M Spd compared to the control group. Moreover, an increase in GFP-LC3 puncta was observed in the BafA1 treated group, 1 $\mu$M Spd + BafA1 and 10 $\mu$M Spd + BafA1 compared to the control group. Importantly, GFP-LC3 puncta accumulated in the combination groups, 1 $\mu$M Spd + BafA1 and 10 $\mu$M Spd + BafA1 compared to BafA1 only. Furthermore, GFP-LC3 puncta was higher in BafA1 treated groups i.e. BafA1, 1 $\mu$M Spd + BafA1 and 10 $\mu$M Spd + BafA1 compared to BafA1 untreated groups (control, 1 $\mu$M Spd and 10 $\mu$M Spd). These results suggest the presence of basal as well as induced autophagic flux. GFP-LC3 puncta and a combination of GFP-LC3/RFP-LC3$\Delta$G puncta (yellow puncta) was localized in AVs of less electron density (\Cref{fig:60_spd_3d_clem_model_10spdbaf_a}). Furthermore, a large AV was observed with 1 $\mu$M Spd, where the green and yellow puncta appeared to be decorating the AV, forming a round membrane which contained red puncta that appeared electron dense (\Cref{fig:60_spd_3d_clem_overlay_a}), suggesting that this structure was an autophagosome that was filled with material. With 10 $\mu$M Spd + BafA1 (\Cref{fig:60_spd_3d_clem_overlay_b}), a large vacuole was also observed, which was RFP-LC3$\Delta$G positive and appeared electron dense, suggesting that the red puncta are more electron dense while the autophagosomes (green puncta) and the yellow puncta are less dense, similar to the cytoplasm. 

\begin{landscape}
\begin{figure}[!htbp]
\center
 \includegraphics[width=\linewidth]{figures/chapter60/60_spd_3d_clem_overlay_a}
 \caption[3D ultrastructural visualization and localization of autophagosomes in response to a low and a high concentration of spermidine]{\textbf{3D ultrastructural visualization and localization of autophagosomes in response to a low and a high concentration of spermidine.} Shown are micrographs of FM, TEM \& overlay (TEM \& FM) indicating the ultrastructure and localization of autophagosomes (green and yellow puncta in the control, 1 $\mu$M Spd \& 10 $\mu$M groups. Scale bar: 2 $\mu$m.}
 \label{fig:60_spd_3d_clem_overlay_a}
\end{figure} 
\end{landscape}

\begin{landscape}
\begin{figure}[!htbp]
\vspace{1.2cm}
\center
 \includegraphics[width=\linewidth]{figures/chapter60/60_spd_3d_clem_overlay_b}
 \caption[3D ultrastructural visualization and localization of autophagosomes in response to a low and a high concentration of spermidine]{\textbf{3D ultrastructural visualization and localization of autophagosomes in response to a low and a high concentration of spermidine.} Shown are micrographs of FM, TEM \& overlay (TEM \& FM) indicating the ultrastructure and localization of autophagosomes (green and yellow puncta) in the BafA1, 1 $\mu$M Spd + BafA1 \& 10 $\mu$M + BafA1 groups. Scale bar: 2 $\mu$m.}
 \label{fig:60_spd_3d_clem_overlay_b}
\end{figure} 
\end{landscape}

\subsection{3D models of autophagosomes}
Taking advantage of 3-dimensional data set of both FM and EM, we assessed whether a low and a high concentration of spermidine had an effect on the surface area and volume of autophagosomes, providing indications of size regulation but also capacity of proteinaceous cargo clearance. Hence, autophagosomes were identified and segmented manually from the data obtained from SFB-SEM using the 3dmod program of IMOD \citep{Kremer1996} in order to generate 3D reconstructions of autopgaosomes. Subsequently, the surface area and volume data was generated from rendered models. The resolution of the models allowed us to localize autophagosomes in the cytoplasm and to detect changes in the membrane surfaces following treatment. Cells without BafA1 treatment; i.e. control (\Cref{fig:60_spd_3d_clem_model_con_a}), 1 $\mu$M Spd (\Cref{fig:60_spd_3d_clem_model_1spd_a}) and 10 $\mu$M Spd (\Cref{fig:60_spd_3d_clem_model_10spd_a}) showed a smoother membrane surface, while a rough and irregular membrane surface was observed in cells treated with BafA1 (\Cref{fig:60_spd_3d_clem_model_baf_a}), 1 $\mu$M Spd + BafA1 (\Cref{fig:60_spd_3d_clem_model_1spdbaf_a}) and 10 $\mu$M Spd + BafA1 (\Cref{fig:60_spd_3d_clem_model_10spdbaf_a}). In addition, the volume and surface area of these autophagosomes was upon visual inspection smaller in the BafA1 treated cells compared to the BafA1 untreated cells (\Cref{fig:60_spd_3d_clem_model_all_a}).

\subsection{Morphometric analysis of autophagosomes in response to a low and a high concentration of spermidine in 3D}
Our morphometric analysis revealed a significant increase in the volume of autophagosomes in the 1 $\mu$M Spd group (169942.0 $\pm$ 22511.0 $\mu$m\textsuperscript{3}, \textit{p} < 0.05) and a significant decrease in the 10 $\mu$M Spd + BafA1 (22091.0 $\pm$ 4171.0 $\mu$m\textsuperscript{3}, \textit{p} < 0.05) compared to the control group (114435.0 $\pm$ 19731.0 $\mu$m\textsuperscript{3}), with no significant differences observed in the BafA1 (86991.0 $\pm$ 12455.0 $\mu$m\textsuperscript{3}), 1 $\mu$M Spd + BafA1 (95919.0 $\pm$ 10844.0 $\mu$m\textsuperscript{3}) and 10 $\mu$M Spd (131689.0 $\pm$ 25086.0 $\mu$m\textsuperscript{3}) (\Cref{fig:60_spd_se_volume_a}: Left). In comparison to the BafA1 treated group, a significant increase in the volume of autophagosomes was observed in the 1 $\mu$M Spd group (\textit{p} < 0.05), while a significant decrease was observed in the 10 $\mu$M Spd + BafA1 (\textit{p} < 0.05). In addition, the volume of autophagosomes was significantly decreased in the 1 $\mu$M Spd + BafA1 compared to 1 $\mu$M Spd, and in 10 $\mu$M Spd + BafA1 compared to 10 $\mu$M Spd. More importantly, a significant decrease in the volume of autophagosomes was observed in the 10 $\mu$M Spd + BafA1 compared to 1 $\mu$M Spd + BafA1 (\Cref{fig:60_spd_se_volume_a}: Left). 
$\mu$m\textsuperscript{3}
In terms of surface area, a significant decrease in the surface area of autophagosomes was observed in the BafA1 treated group (883.5 $\pm$ 155.9 $\mu$m\textsuperscript{3}, \textit{p} < 0.05), 1 $\mu$M Spd + BafA1 (1194.0 $\pm$ 104.1 $\mu$m\textsuperscript{3}, \textit{p} < 0.05) and 10 $\mu$M Spd + BafA1 (6415.0 $\pm$ 3542.0 $\mu$m\textsuperscript{3}, \textit{p} < 0.05) compared to the control group (49130.0 $\pm$ 5546.0 $\mu$m\textsuperscript{3}), with no significant differences observed in the 1 $\mu$M Spd (54002.0 $\pm$ 16062.1 $\mu$m\textsuperscript{3}) and 10 $\mu$M Spd (57300.0 $\pm$ 8893.0 $\mu$m\textsuperscript{3}) (\Cref{fig:60_spd_se_volume_a}: Right). A significant increase in the surface area of autophagosomes was observed in the 1 $\mu$M Spd group (\textit{p} < 0.05) and 10 $\mu$M Spd group (\textit{p} < 0.05) compared to the BafA1 treated group. Although no significant differences were observed, an increase in the area of autophagosomes was detected in the 10 $\mu$M Spd + BafA1 compared to 1 $\mu$M Spd + BafA1 and BafA1 treated group. In addition, the surface area of autophagosomes was significantly decreased in the 1 $\mu$M Spd + BafA1 compared to 1 $\mu$M Spd and in 10 $\mu$M Spd + BafA1 compared to 10 $\mu$M Spd (\Cref{fig:60_spd_se_volume_a}: Right). 

\begin{figure}[!htbp]
\center
 \includegraphics[width=\linewidth]{figures/chapter60/60_spd_3d_clem_model_con_a}
 \caption[Render of autophagosomes in the control cell]{\textbf{Render of autophagosomes in the control cell.} Representative micrographs indicating the localization of an autophagosome (i), in different z dimension (ii). Rendered autophagosomes are displayed in 3D view (iii) and in the XY, YZ, XZ orientation (iv \& v), thus revealing the shape and ultrastructural content.}
 \label{fig:60_spd_3d_clem_model_con_a}
\end{figure} 

\begin{figure}[!htbp]
\center
 \includegraphics[width=\linewidth]{figures/chapter60/60_spd_3d_clem_model_1spd_a}
 \caption[Render of autophagosomes in the 1 $\mu$M Spd treated cell]{\textbf{Render of autophagosomes in the 1 $\mu$M Spd treated cell.} Representative micrographs indicating the localization of an autophagosome (i), in different z dimension (ii). Rendered autophagosomes are displayed in 3D view (iii) and in the XY, YZ, XZ orientation (iv \& v), thus revealing the shape and ultrastructural content.}
 \label{fig:60_spd_3d_clem_model_1spd_a}
\end{figure} 

\begin{figure}[!htbp]
\center
 \includegraphics[width=\linewidth]{figures/chapter60/60_spd_3d_clem_model_10spd_a}
 \caption[Render of autophagosomes in the 10 $\mu$M Spd treated cell]{\textbf{Render of autophagosomes in the 10 $\mu$M Spd treated cell.} Representative micrographs indicating the localization of an autophagosome (i), in different z dimension (ii). Rendered autophagosomes are displayed in 3D view (iii) and in the XY, YZ, XZ orientation (iv \& v), thus revealing the shape and ultrastructural content.}
 \label{fig:60_spd_3d_clem_model_10spd_a}
\end{figure} 

\begin{figure}[!htbp]
\center
 \includegraphics[width=\linewidth]{figures/chapter60/60_spd_3d_clem_model_baf_a}
 \caption[Render of autophagosomes in the BafA1 treated cell]{\textbf{Render of autophagosomes in the BafA1 treated cell.} Representative micrographs indicating the localization of autophagosomes (i), in different z dimension (ii). Rendered autophagosomes are displayed in 3D view (iii) and in the XY, YZ, XZ orientation (iv \& v), thus revealing the shape and ultrastructural content.}
 \label{fig:60_spd_3d_clem_model_baf_a}
\end{figure} 

\begin{figure}[!htbp]
\center
 \includegraphics[width=\linewidth]{figures/chapter60/60_spd_3d_clem_model_1spdbaf_a}
 \caption[Render of autophagosomes in the 1 $\mu$M Spd + BafA1 treated cell]{\textbf{Render of autophagosomes in the 1 $\mu$M Spd + BafA1 treated cell.} Representative micrographs indicating the localization of an autophagosome (i), in different z dimension (ii). Rendered autophagosomes are displayed in 3D view (iii) and in the XY, YZ, XZ orientation (iv \& v), thus revealing the shape and ultrastructural content.}
 \label{fig:60_spd_3d_clem_model_1spdbaf_a}
\end{figure} 

\begin{figure}[!htbp]
\center
 \includegraphics[width=\linewidth]{figures/chapter60/60_spd_3d_clem_model_10spdbaf_a}
 \caption[Render of autophagosomes in the 10 $\mu$M Spd + BafA1 treated cell]{\textbf{Render of autophagosomes in the 10 $\mu$M Spd + BafA1 treated cell.} Representative micrographs indicating the localization of autophagosomes (i), in different z dimension (ii). Rendered autophagosomes are displayed in 3D view (iii) and in the XY, YZ, XZ orientation (iv \& v), thus revealing the shape and ultrastructural content.}
 \label{fig:60_spd_3d_clem_model_10spdbaf_a}
\end{figure} 

\begin{landscape}
\begin{figure}[!htbp]
\center
 \includegraphics[width=\linewidth]{figures/chapter60/60_spd_3d_clem_model_all_a}
 \caption[Overview of rendered autophagosomes within the cells]{\textbf{Overview of rendered autophagosomes within the cells.} Shown are models of autophagosomes in the TEM z-stack of the control, BafA1, 1 $\mu$M Spd, 1 $\mu$M Spd + BafA, 10 $\mu$M Spd \& 10 $\mu$M + BafA1.}
 \label{fig:60_spd_3d_clem_model_all_a}
\end{figure} 
\end{landscape}

\begin{figure}[!htbp]
 \begin{subfigure}[b]{0.495\linewidth}
  \includegraphics[width=\linewidth]{figures/chapter60/60_spd_se_volume_a}
 \end{subfigure}
 \begin{subfigure}[b]{0.495\linewidth}
  \includegraphics[width=\linewidth]{figures/chapter60/60_spd_se_area_b}
 \end{subfigure}
  \caption[The effect of a low and a high concentration of spermidine on volume and surface area of autophagosomes in 3D]{\textbf{The effect of a low and a high concentration of spermidine on volume and surface area of autophagosomes in 3D.} Data are presented as mean $\pm$ SEM, with a total of 15 - 25 autophagosomes analysed per treatment group. * = \textit{p} < 0.05 vs control, \# = \textit{p} < 0.05 vs BafA1, @ = \textit{p} < 0.05 vs 1 $\mu$M Spd, \% = \textit{p} < 0.05 vs 10 $\mu$M Spd and \$ = \textit{p} < 0.05 vs 1 $\mu$M Spd + BafA1.}
 \label{fig:60_spd_se_volume_a}
\end{figure}

\section{Discussion: Method development for CLEM to assess autophagosomes}
Electron microscopy and fluorescence microscopy have been extensively used to assess autophagy pathway intermediates i.e. autophagosomes and autolysosomes, however, in the electron micrograph it remains challenging to identify and localize autophagosomes and autolysosomes with high precision. Although FM allows precise identification and localization of autophagosomes and autolysosomes, as cells can be tagged with specific protein markers, it lacks the ability to reveal the underlying ultrastructural content. On the other hand, EM provides the ultrastructural information, but makes it very difficult to distinguish between autophagy intermediates, even for experienced scientists, hence the autophagy pathway intermediates are usually collectively termed autophagic vacuoles (AVs) \citep{Eskelinen2008,klionsky2016}. However, when both these two microscopy techniques are combined in CLEM, it is possible to reveal information that cannot be obtained using either of these techniques alone. Thus, this chapter aimed to optimize a protocol for CLEM that can be subsequently used widely to identify and assess autophagosomes with high precision. Secondly, our aim was to use the optimised protocol to assess the accumulation of autophagosomes following treatments with spermidine at various concentrations in the absence and presence of BafA1 and to confirm that the fluorescent structures are indeed autophagosomes, employing 2D as well as 3D CLEM. One caveat of using 2D CLEM is that, although the imaging acquisition is performed using multiple optical sections with confocal microscopy, the EM micrographs are acquired as a single plan within the cell, making it challenging to identify autophagosomes due to the major difference in the z space. This problem however can be solved by using automated EM based technologies such electron tomography, FIB-SEM and SBF-SEM \citep{Burel2018,Duke2014,Russell2017,Yla-Anttila2009} which allow the entire cell to be acquired and analysed in 3 dimensions, allowing for volume analysis to be performed. Here we show that, after a number of optimizations, we have established a working protocol for CLEM, the first to be established in South Africa. This established protocol can be successfully used in the 2D and 3D context using transfected cells or cells labelled with nanobodies or fluorescence probes. The established protocol can also be used in cells labelled with antibodies, as we have shown using saponin, but at an expense of losing FM resolution as staining with antibodies requires permeabilization of the membrane which affects the preservation of lipids, thus negatively impacting the contrast and hence ultrastructural visualization \citep{Eskelinen2008}.

\subsection{Localization of autophagosomes in response to a low and a high concentration of spermidine in 2D CLEM}
In order to assess the effect of spermidine on the accumulation and abundance of autophagosomes, GT1-7 cells expressing GFP-LC3-RFP-LC3$\Delta$G were treated with a low and a high concentration of spermidine in the absence and presence of BafA1 and assessed using 2D CLEM. Our results showed distinct differences in the treatment groups in the sense of the distribution of autophagy pathway intermediates, i.e. GFP-LC3 (green puncta), RFP-LC3$\Delta$G (red puncta), and GFP-LC3/RFP-LC3$\Delta$G (yellow puncta), suggesting that the autophagy pathway was responding to the treatment (\Cref{fig:60_spd_2d_clem_a} \& \Cref{fig:60_spd_2d_clem_b}). Secondly, our results showed that the autophagosomes are characterized by structures that are low in electron density, similar to that of the cytoplasm. In support of our findings, using conventional TEM, autophagosomes have been classified as round vacuoles where the material inside the vacuole resembles that of the cell’s cytoplasm \citep{Eskelinen2008}. Thirdly, we observed an accumulation of green and yellow puncta in the BafA1 treated groups (\Cref{fig:60_spd_2d_clem_b}) compared to the BafA1 untreated groups (\Cref{fig:60_spd_2d_clem_a}) and these appeared to be of similar electron density, suggesting that with EM only, one cannot clearly distinguish these two intermediates. Hence the fluorescence signal allows to localize and discern which structures are GFP-LC3 positive and which are both GFP-LC3 and RFP-LC3$\Delta$G positive, since they both have the same electron density, form and shape. Fourthly, we observed that GFP-LC3 is primarily localized to the autophagosome membrane as expected, while the RFP-LC3$\Delta$G was not diffusely distributed in the cytoplasm as anticipated with this probe \citep{Kaizuka2016}. Instead it was rather membrane bound together with GFP-LC3 and was also engulfed inside vesicles that were GFP-LC3 positive, suggesting it as part of the cargo. To our knowledge, this is the first time that RFP-LC3$\Delta$G in context of the GFP-LC3-RFP-LC3$\Delta$G probe, is reported to be localized in vesicle membrane structures as well as part of the cytoplasmic cargo. It was previously reported that the expression of the GFP-LC3-RFP-LC3$\Delta$G probe significantly differs among cells or tissues \citep{MoulisandVindis2017}. Fifthly, treatment with spermidine resulted in the accumulation of red puncta, leading to a decrease in the ratio of GFP-LC3/RFP-LC3$\Delta$G suggesting an increased autophagic activity. In line with our findings, spermidine at 100 $\mu$M was found to decrease the ratio of GFP-LC3/RFP-LC3$\Delta$G, thus increasing autophagic activity when measured using a microplate reader in Hela cells expressing GFP-LC3-RFP-LC3$\Delta$G \citep{Kaizuka2016}. An unexpected observation was the presence of large and complex red punctate structures in cells treated with 1 $\mu$M Spd which was localized in much larger vesicles that were electron dense and GFP-LC3 negative, making their identity unclear (\Cref{fig:60_spd_2d_clem_a}). Using 2D CLEM, others have monitored and identified autophagosomes in starved Hela cells expressing mRFP-GFP-LC3 \citep{Gudmundsson2019} and in HEK293A-GFP-LC3 \citep{Razi2009}. Our findings are, to our knowledge, novel and the first study using the GFP-LC3-RFP-LC3$\Delta$G probe in CLEM to assess autophagosome changes following autophagy induction using a low and a high concentration of an autophagy inducer in the presence and absence of BafA1.

\subsection{3D ultrastructural visualization and localization of autophagosomes in response to a low and a high concentration of spermidine}
Due to the poor resolution observed in our 2D CLEM data, we utilised a 3D CLEM approach to clarify and assess autophagosomes, combined with SR-SIM for the FM mode. CLEM in 3 dimensions allowed for the visualisation of the entire cell in both FM and more especially in FIB-SEM due to much better resolution power in the z dimension, allowing for the identification of autophagosomes with high precision, that would have otherwise been missed in 2 dimensions (\Cref{fig:60_spd_3d_clem_overlay_a} \& \Cref{fig:60_spd_3d_clem_overlay_b}). Autophagy induction with 1 \& 10 $\mu$M Spd resulted in a decrease in GFP-LC3 puncta compared to the control group (\Cref{fig:60_spd_3d_clem_overlay_a}), while inhibition of autophagosomal lysosome fusion with BafA1 resulted in the accumulation of GFP-LC3 puncta, which was further enhanced in the combination groups i.e 1 $\mu$M Spd + BafA1 and 10 $\mu$M Spd + BafA1 compared to BafA1 alone (\Cref{fig:60_spd_3d_clem_overlay_b}). Furthermore, GFP-LC3 puncta were higher BafA1 treated groups i.e. BafA1 , 1 $\mu$M Spd + BafA1 and 10 $\mu$M Spd + BafA1 compared to the BafA1 untreated groups (control, 1 $\mu$M Spd and 10 $\mu$M Spd). These results suggest the presence of basal as well as induced autophagic flux, and overall suggest that the autophagy pathway intermediates are responding distinctly to the treatment. To our knowledge, our study is the first employing a GFP-LC3-RFP-LC3$\Delta$G probe to image cells at high resolution and to reveal autophagosome changes with treatment. Indeed, GFP-LC3-RFP-LC3$\Delta$G has only been used in limited studies to monitor autophagy \textit{in vitro} and \textit{in vivo}, where autophagy in the \textit{in vitro} case was monitored using a ratio-metric plate reader approach \citep{Kaizuka2016}. Consistent with the 2D CLEM results, our 3D data reveal that both GFP-LC3 puncta and the yellow puncta were localized in AVs of a low electron density, similar to that of cytoplasm (\Cref{fig:60_spd_3d_clem_model_con_a}, \Cref{fig:60_spd_3d_clem_model_1spd_a}, \Cref{fig:60_spd_3d_clem_model_10spd_a}, \Cref{fig:60_spd_3d_clem_model_baf_a}, \Cref{fig:60_spd_3d_clem_model_1spdbaf_a} \& \Cref{fig:60_spd_3d_clem_model_10spdbaf_a}). Using 3D cryo-fluorescence and cryo-soft X-ray microscopy, autophagosomes were identified in starved HEK293A cells expressing GFP-LC3, however, cryo-soft X-ray microscopy resulted in poor resolution \citep{Duke2014}. In this study, fusion of autophagosomes with lysosomes was not inhibited, thus the identified autophagosomes could also have been autolysosomes occurring prior to the quenching of GFP-LC3 in the acidic lysosomal environment. Others have used 3D tomography to study the localization of phagophores and shown elegantly that the structures are connected with rough ER, particularly ER cisternae located inside a nascent autophagosome \citep{Yla-Anttila2009}. Interestingly, we observed a large AV in the 1 $\mu$M Spd group, where the green, red and yellow puncta appeared to be decorating the entire AV, forming a membrane structure, which, inside revealed the presence of only red puncta which appeared highly electron dense, suggesting that this structure is likely an autophagosome that has engulfed RFP-LC3$\Delta$G components (\Cref{fig:60_spd_3d_clem_overlay_a}). Furthermore, in the 10 $\mu$M Spd +Baf A1 group, a large vacuole was observed, which was RFP-LC3$\Delta$ positive and appeared highly electron dense, suggesting that the red puncta are indeed more electron dense while the autophagosomes (green puncta) and the yellow puncta are less dense. To our knowledge, this study is the first one to report the localization of RFP-LC3$\Delta$G puncta in vesicle membrane structures and as part of the vesicle cargo. Using this probe, we found that the RFP-LC3$\Delta$G puncta are not only cytoplasmic, but are also engulfed and part of the cargo, suggesting that the mechanisms of action are likely more complex and potentially the RFP-LC3$\Delta$G signal observed is more likely reflective of a later time point, engulfed by newly formed autophagosomes. It is likely that a time-dependent effect is being observed, where RFP-LC3$\Delta$G, cleaved through ATG4 remains at first cytoplasmic, as suggested \citep{Kaizuka2016}, but is subsequently engulfed by newly formed autophagosomes. These findings are of importance, especially for quantitative image analysis approaches, and deserve further study.

\subsubsection{Morphometric analysis of autophagosomes in response to a low and a high concentration of spermidine in 3D}
Next, autophagosomes were segmented, taking advantage of CLEM and the high resolution achieved in FIB-SEM. Our render of autophagosomes revealed firstly that the autophagosome were only localised in the cytoplasm regardless treatment, but BafA1 treated cells presented with a rough and irregular autophagosome membrane surface while autophagosomes of BafA1 untreated cells displayed a smoother membrane surface (\Cref{fig:60_spd_3d_clem_model_all_a}). Secondly, rendered autophagosomes were smaller in volume and surface area in the cells treated with BafA1 compared to BafA1 untreated cells. These results were consistent with the morphometric analysis. Our volume analysis revealed, firstly, a significant increase in the volume of the autophagosomes following treatment with 1 $\mu$M Spd and a significant decrease following treatment with 10 $\mu$M Spd + BafA1 compared to the control (\Cref{fig:60_spd_se_volume_a}: Left). Secondly, our results revealed a significant decrease on the volume of autophagosomes in the 10 $\mu$M Spd + BafA1 compared to BafA1 treated group and more importantly to 1 $\mu$M Spd + BafA1. Lastly, the volume of autophagosomes was significantly decreased in the 1 $\mu$M Spd + BafA1 compared to 1 $\mu$M Spd, in 10 $\mu$M Spd compared to 10 $\mu$M Spd + BafA1. Overall, these results suggest firstly, that the autophagosomes were responding distinctively to the treatment and secondly, these results suggest a concentration-dependent impact on the volume of autophagosomes and thirdly, that the presence of BafA1 impacts on the volume of autophagosome (\Cref{fig:60_spd_se_volume_a}: Left). With regards to the surface area of autophagosomes, we observed a significant decrease following treatment with BafA1, 1 $\mu$M Spd + BafA1 and 10 $\mu$M Spd + BafA1 compared to the control group (\Cref{fig:60_spd_se_volume_a}). In addition, a significant increase in the surface area of autophagosomes was observed following treatment with spermidine i.e 1 $\mu$M Spd and 10 $\mu$M Spd compared to the BafA1 treated group, suggesting that BafA1 reduces the surface area. Furthermore, an increase in the area of autophagosome was observed in the 10 $\mu$M Spd + BafA1 compared to 1 $\mu$M Spd + BafA1 and BafA1 treated group, although no significant differences were reached, suggesting a concentration-dependent increase in the surface area. Moreover, surface area was significantly decreased in the 1 $\mu$M Spd + BafA1 compared to 1 $\mu$M Spd and in 10 $\mu$M Spd + BafA1 compared to 10 $\mu$M Spd, suggesting that the presence of BafA1 does indeed reduces the surface area (\Cref{fig:60_spd_se_volume_a}). 

Overall, these results suggest that low spermidine concentrations induce autophagosome formation capable of larger volume clearance, while high spermidine concentrations were not able to elicit such robust effects. Futhermore BafA1 impacts the volume and surface area of autophagosomes and that spermidine in combination with BafA1 reduces the volume but increases the surface area compared to BafA1 alone and compared to 1 $\mu$M Spd + BafA1. Our results are the first to reveal a concentration-dependent effect of spermidine on the volume and surface area of autophagosomes and are the first to reveal that BafA1 impacts on autophagosome volume and area in 3D. This  deserves further investigation.

In conclusion, in this study, we have successfully implemented a 3D CLEM protocol, and identified autophagosomes with high precision in 3D. Furthermore, we have modelled autophagosomes in 3D and revealed with morphometric analysis that spermidine, in combination with BafA1 decreases autophagosome volume while increasing their surface area in a concentration-dependent manner. 



























