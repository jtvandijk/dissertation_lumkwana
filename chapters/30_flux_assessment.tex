\chapter{Autophagic flux assessment}
\section{Introduction}

Autophagy is a physiological cellular process that is characterized by a fine balance between the rate of AV formation and the rate of their clearance, with autophagic flux referring  to the rate of flow of material through the autophagic pathway \citep{loos2014,klionsky2016}. Many tools and techniques are available to assess autophagy, including western blotting (WB), fluorescence microscopy (FM), Transmission Electron Microscopy (TEM), flow cytometry \citep{klionsky2016} and Correlative Light and Electron Microscopy (CLEM) with each technique having particular advantages, but also inherent disadvantages.

Autophagy plays a significant role in health where primarily decreased activation of autophagy has been shown to have detrimental effects. Autophagy can be induced via mTOR dependent and mTOR independent pathways \citep{sarkar2013}. Although we have progressed substantially in the understanding of the autophagy machinery and its regulation, its dysfunction in pathology as well as its dynamic changes however in the disease progression remains often largely unclear. Furthermore, although a substantial progress has been made in unravelling the role and regulation of autophagy and its modulation in neurodegenerative disease using pharmacological agents, major differences exist between experimental design, drug concentrations used, duration of treatment intervention as well as the model system used to assess autophagic flux \citep{lumkwana2017}. This contributes to inability to modulate autophagy with high precision. Specifically, spermidine and rilmenidine have been shown to induce autophagy, but the impact of concentration ranges on autophagic flux and on subsequent protein clearance and neuronal toxicity is unclear. Importantly, whether a concentration dependent effect on autophagy activity exists achieving defined, but distinguishable heightened autophagy flux remains unclear. Particularly in terms of fine-tuning autophagy activity with a given flux deviation in neurodegeneration, this aspect is of critical importance. In this chapter we firstly aimed to establish a suitable concentration of spermidine (Spd) and rilmenidine (Ril) using GT 1-7 neuronal cells that induces autophagic flux without causing cellular toxicity, and secondly, to characterize the autophagic flux profile of a low and high concentration of both drugs established in terms of (1) the autophagic machinery and pathway intermediates (autophagosomes, autolysosome),(2) the autophagosome flux, (3) p62 puncta, and (4) autophagic vacuoles (AVs). For this purpose, GT1-7 neuronal cells were treated with three concentrations of Spd and Ril in the presence and absence of BafA1. Subsequently, cellular viability and WB for LC3 II and p62 was performed. Next, to assess the effect of a low and high concentration of Spd and Ril, GT 1-7 cells were treated with 1 \& {10.0 $\mu$M respectively in the presence and absence of BafA1 and thereafter extensively analysed using FM and TEM linked to quantitative morphometric analysis. In addition, GT1-7 cells were transfected with mRFP-GFP-LC3 \citep{yoshii2017} and subsequently treated with 1 \& 10 $\mu$M of Spd and Ril in the presence and absence of BafA1 and analysed with FM.
 
\section{Effect of spermidine and rilmenidine on cell viability}
In order to choose a suitable, yet non-toxic concentration of autophagy modulators, WST-1 was performed. For the concentrations assessed, no decrease in viability was observed. In fact, a minor metabolic response associated with enhanced reductive capacity was observed. Reductive capacity was significantly increased using 1 $\mu$M Spd 104.70\% $\pm$ 1.64\%, \textit{p} < 0.05 and 10 $\mu$M Spd (109.20 $\pm$ 1.38\%, \textit{p} < 0.05) compared to the control (100.00\% $\pm$ 1.50\%), with no significant differences observed using 0.1 $\mu$M Spd (104.20\% $\pm$ 1.46\%) (\Cref{fig:30_spd_cell_viability}: \textbf{A}). More importantly, a significant increase in reductive capacity was observed with 10 $\mu$M Spd (\textit{p} < 0.01) compared to 0.1 $\mu$M Spd and 1 $\mu$M Spd.

Following rilmenidine treatment, a significant increase in reductive capacity was observed in the 10 $\mu$M Ril treated group (105.70 $\pm$ 2.26\%, \textit{p} < 0.05) compared to the control (100.40 $\pm$ 1.16\%), with no significant differences observed in the 0.1 $\mu$M Ril (98.76\% $\pm$ 1.49\%) and 1 $\mu$M Ril (100.50 $\pm$ 2.12\%) (\Cref{fig:30_spd_cell_viability}: \textbf{B}). Importantly, a significant increase in cellular viability was observed in the 10 $\mu$M Ril group compared to 0.1 $\mu$M and 1 $\mu$M Ril. This data suggests no toxicity and allowed to choose a low and high concentration of spermidine and rilmenidine for subsequent experiments.

\begin{figure}[!htbp]
  \centering
  \begin{subfigure}[b]{0.495\linewidth}
    \includegraphics[width=\linewidth]{figures/chapter30/30_spd_ril_cell_viability_a}
    \caption{Spermidine}
  \end{subfigure}
  \begin{subfigure}[b]{0.495\linewidth}
    \includegraphics[width=\linewidth]{figures/chapter30/30_spd_ril_cell_viability_b}
    \caption{Rilmenidine.}
  \end{subfigure}
  \caption[Effect of spermidine and rilmenidine on cell viability]{Effect of spermidine and rilmenidine on cell viability. GT1-7 cells were treated with 0.1, 1 \& 10 $\mu$M of Spd (\textbf{A}) and Ril (\textbf{B}) for 8 hr. Subsequently, reductive capacity as a measure of cell viability was assessed using WST1 assay. All results are presented as a percentage of the control (mean $\pm$ SEM), \textit{n}=3, with 6 replicates per group. * = \textit{p} < 0.05 vs control, \# = \textit{p} < 0.05 vs 0.1 $\mu$M Spd and \$ = \textit{p} < 0.05 vs 1 $\mu$M Spd (graph \textbf{A}), * = \textit{p} < 0.05 vs control, \# = \textit{p} < 0.05 vs 0.1 $\mu$M Ril and \$ = \textit{p} < 0.05 vs 1 $\mu$M Ril (graph \textbf{B})}
  \label{fig:30_spd_cell_viability}
\end{figure}

\section{Effect of spermidine and rilmenidine on autophagic flux using western blotting}
To further identify the concentration that would have maximal autophagy inducing effects, protein extraction was performed following 8 hr treatment with a range of Spd and Ril concentrations. Western blot analysis was performed to assess the protein expression of LC3-II and p62. LC3 II is formed through conjugation of LC3-I to PE and is localized on the inner and outer surface of the autophagosome membrane, where it is either degraded upon fusion of autophagosomes with lysosomes or removed through deconjugation and recycled \citep{kabeya2000}. Importantly, the abundance of LC3 II correlates with the number of autophagosomes and hence is used as a key marker to indicate autophagy or the size of the autophagosome pool \citep{loos2014} p62 serves as a scaffolding for ubiquitinated proteins \citep{sahani2014}. It binds directly to LC3 and is in turn selectively degraded through autophagy; thus it is widely used as an additional indicator for autophagy activity \citep{pankiv2007}.

\subsection{Abundance of LC3 II and p62}
A significant increase in the abundance of LC3 II was observed in the 10 $\mu$M Spd + BafA1 (1.67 $\pm$ 0.28) compared to the control group (1.00 $\pm$ 0.00) (\Cref{fig:30_spd_ril_abun}: \textbf{A}\textit{i}, \textbf{A}\textit{iii}). Moreover, a significant increase in the abundance of LC3 II was observed in the 10 $\mu$M Spd + BafA1 (\textit{p} < 0.05) compared to 0.1 $\mu$M Spd + BafA1. No significant differences were observed between 0.1 $\mu$M Spd, 1 $\mu$M Spd and 10 $\mu$M Spd. A significant increase was observed in the 10 $\mu$M Spd + BafA1 compared to10 $\mu$M Spd, with no significant differences seen between 0.1 $\mu$M Spd and 0.1 $\mu$M Spd + BafA1 and between 1 $\mu$M Spd versus 1 $\mu$M Spd + BafA1, however it is important to mention that in both the 0.1 $\mu$M Spd  and 1 $\mu$M Spd, inhibition of degradation with BafA1 resulted in an increased LC3 II accumulation compared to the Baf untreated groups (\Cref{fig:30_spd_ril_abun}: \textbf{A}\textit{i}, \textbf{A}\textit{iii}), suggesting the presence of a basal as well as increased autophagic activity.

The abundance of p62 was significantly increased in the BafA1 treated group (1.72 $\pm$ 0.26, \textit{p} < 0.05) and  10 $\mu$M Spd + BafA1 (1.72 $\pm$ 0.17, \textit{p} < 0.05) compared to Con (1.00 $\pm$ 0.00) (\Cref{fig:30_spd_ril_abun}: \textbf{A}\textit{ii}, \textbf{A}\textit{iii}). Moreover, in comparison to the Bafa1 treated group, a significant decrease in p62 abundance was observed in the 1 $\mu$M Spd, with no significant differences seen in all other groups. A higher abundance was indicated in the presence of BafA1 in all treatment groups compared to the respective BafA1 untreated groups (\Cref{fig:30_spd_ril_abun}: \textbf{A}\textit{i}, \textbf{A}\textit{iii}).

With regards to rilmenidine treatment (\Cref{fig:30_spd_ril_abun}: \textbf{B}\textit{i}, \textbf{B}\textit{iii}), no significant differences in the abundance of LC3 II were observed, however an increasing trend in the amount of LC3 II in the BafA1 treated groups compared to the BafA1 untreated groups was noted [Con (1.00 $\pm$ 0.00), BafA1 (1.45 $\pm$ 0.49), 0.1 $\mu$M (1.16 $\pm$ 0.09), 0.1 $\mu$M Ril + BafA1 (1.46 $\pm$ 0.36), 1 $\mu$M Ril (0.94 $\pm$ 0.14), 1 $\mu$M Ril + BafA1 (1.22 $\pm$ 0.42), 10 $\mu$M Ril (1.02 $\pm$ 0.27) and 10 $\mu$M Ril + BafA1 (1.17 $\pm$ 0.63)].

A significant increase in p62 protein expression was observed in the 1 $\mu$M Ril (1.68 $\pm$ 0.19, \textit{p} < 0.05), 1 $\mu$M Ril + BafA1 (2.21 $\pm$ 0.31, \textit{p} < 0.05) and 10 $\mu$M Ril + BafA1 (1.81 $\pm$ 0.08, \textit{p} < 0.05) compared to the Con group (1.00 $\pm$ 0.00) (\Cref{fig:30_spd_ril_abun}: \textbf{B}\textit{ii}, \textbf{B}\textit{iii}). Moreover, a significant increase in p62 protein levels was revealed in the 1 $\mu$M Ril + BafA1 compared to BafA1 treated group. Furthermore, a significant increase in p62 abundance was observed in the 1 $\mu$M Ril + BafA1 (\textit{p} < 0.05) compared to 0.1 $\mu$M Ril + BafA1, with no significant differences observed when compared to 10 $\mu$M Ril + BafA1. Finally, overall p62 protein abundance accumulated in the BafA1 treated groups compared to the respective BafA1 untreated groups, however not reaching significance (\Cref{fig:30_spd_ril_abun}: \textbf{B}\textit{ii}, \textbf{B}\textit{iii}). No response was observed in 0.1 $\mu$M Ril + BafA1 compared to its BafA1 untreated group. This data suggests an autophagy enhancing effect, which is best revealed through the impacted p62 clearance.

\subsection{LC3 II and p62 protein turnover}
In order to better estimate autophagic flux, turnover of LC3 II was assessed. To estimate autophagic flux, it is necessary to determine the extent of change in the LC3 II protein abundance at a given time point in the presence of lysosomal inhibitors (i.e. BafA1) \citep{Martinez-Lopez2013}. Thus, mean values of the BafA1 treated groups (BafA1, 0.1 $\mu$M Spd + BafA1, 1 $\mu$M Spd + Baf A1and 10 $\mu$M Spd + BafA1) were subtracted from the BafA1 untreated groups (control, 0.1 $\mu$M Spd, 1 $\mu$M Spd and 10 $\mu$M Spd), respectively.

\begin{landscape}
\begin{figure}[!htbp]
\center
  \includegraphics[width=0.75\linewidth]{figures/chapter30/30_spd_ril_abun}
  \caption[Effect of spermidine and rilmenidine concentrations on LC3-II and p62 abundance]{Effect of spermidine and rilmenidine concentrations on LC3-II and p62 abundance. Quantification and representative blots for LC3-II and p62 in in GT 1-7s following treatment with concentrations of spermidine (\textbf{A}) and rilmenidine (\textbf{B}). Data are presented as mean $\pm$ SEM, \textit{n} = 3 - 4. * = \textit{p} < 0.05 vs control, \# = \textit{p} < 0.05 vs Baf, \$ = \textit{p} < 0.05 vs 0.1$\mu$M Spd + Baf and \S = \textit{p} < 0.05 vs 10 $\mu$M Spd (graph \textbf{A}), * = \textit{p} < 0.05 vs control, \# = \textit{p} < 0.05 vs Baf and \$ = \textit{p} < 0.05 vs 0.1$\mu$M Ril + Baf (graph \textbf{B}).}
  \label{fig:30_spd_ril_abun}
\end{figure}
\end{landscape}

A significant increase in LC3-II protein turnover was observed using 10 $\mu$M Spd (0.91 $\pm$ 0.15, \textit{p} < 0.05) compared to the control group (0.24 $\pm$ 0.11) as well as the 0.1 $\mu$M Spd (0.12 $\pm$ 0.09) and 1 $\mu$M Spd (0.26 $\pm$ 0.07), with no significant differences observed using 0.1 $\mu$M Spd  and 1 $\mu$M Spd (\Cref{fig:30_spd_ril_lc3ii}: \textbf{A}). With regards to rilmenidine treatment, no significant differences in the LC3-II turnover were observed in all groups [control (0.45 $\pm$ 0.49), 0.1 $\mu$M Ril (0.30 $\pm$ 0.29), 1 $\mu$M Ril (0.27 $\pm$ 0.32) and 10 $\mu$M Ril (0.15 $\pm$ 0.36)], however, there is a progressive trend in decline with an increase in Ril concentration (\Cref{fig:30_spd_ril_lc3ii}: \textbf{B}). 

\begin{figure}[!htbp]
  \centering
  \begin{subfigure}[b]{0.495\linewidth}
    \includegraphics[width=\linewidth]{figures/chapter30/30_spd_ril_lc3II_a}
    \caption{Spermidine}
  \end{subfigure}
  \begin{subfigure}[b]{0.495\linewidth}
    \includegraphics[width=\linewidth]{figures/chapter30/30_spd_ril_lc3II_b}
    \caption{Rilmenidine.}
  \end{subfigure}
  \caption[Effect of spermidine and rilmenidine concentrations on LC3-II turnover]{Effect of spermidine and rilmenidine concentrations on LC3-II turnover. Following quantification of LC3-II protein levels, mean values of the Baf treated group were subtracted from the corresponding Baf untreated groups. Data are presented as mean $\pm$ SEM, \textit{n} = 3 - 4. * = \textit{p} < 0.05 vs control, \# = p < 0.05 vs 0.1 $\mu$M Spd and \$ = \textit{p} < 0.05 vs 1 $\mu$M Spd.}
  \label{fig:30_spd_ril_lc3ii}
\end{figure}

\section{Effect of low and high concentration spermidine and rilmenidine on autophagy pathway intermediates and autophagosome flux using quantitative FM}
In order to assess the effect of spermidine and rilmenidine on the autophagy pathway intermediates i.e. autophagosomes, autolysosomes as well as autophagosome flux, GT 1-7 cells were transfected with a mRFP-GFP-LC3 plasmid \citep{yoshii2017} and imaged using a fluorescence microscopy. The mRFP-GFP-LC3 plasmid allows to detect pH dependent changes as it fluoresces red and green in the autophagosomes resulting in a yellow signal, while only fluorescing red in the autolysosome due to the quenching of GFP signal as a result of the auto-lysosomal acidity \citep{yoshii2017}. Transfected cells were treated with a low (1 $\mu$M) and a high (10 $\mu$M) concentration of spermidine and rilmenidine for 8 h. Subsequently, cells were exposed to BafA1 (400 nM) for further 4 h to completely block the fusion of autophagosomes with lysosomes \citep{DuToit2018a} and imaged with fluorescence microscopy using z-stack acquisition. The number of autophagosome puncta (nA, yellow puncta) and autolysomes (nAL, red puncta) before and after autophagy induction with spermidine and rilmenidine as well as before and after complete inhibition of autophagosome fusion with lysosomes was assessed. In addition, autophagic flux (J), which is the initial rate of increase in nA and nAL per cell after treatment with Baf A1 and transition time required for a cell to clean its pool were also measured \citep{DuToit2018a,DuToit2018b,loos2014}.

\subsection{Assessment of autophagosome pool size}
The number of autophagosome significantly increased in the BafA1 treated group (11.93  $\pm$ 1.69), 1 $\mu$M Spd + BafA1 (14.48 $\pm$ 2.69, \textit{p} < 0.05) and 10 $\mu$M Spd + BafA1 (17.33 $\pm$ 2.82, \textit{p} < 0.05) compared to the control (3.90 $\pm$ 0.75) (\Cref{fig:30_spd_ril_poolsize1}: \textbf{A}). A significant increase was revealed in the 10 $\mu$M Spd + Baf compared to the BafA1 group. Moreover, a significant increase autophagosome number was observed in 1 $\mu$M Spd + Baf versus 1 $\mu$M Spd (2.00 $\pm$ 0.62) and in 10 $\mu$M Spd + Baf versus 10 $\mu$M Spd (4.03 $\pm$ 0.82) (\Cref{fig:30_spd_ril_poolsize1}: \textbf{A}).

Rilmenidine treatment resulted in a significant increase in the number of autophagosomes in the BafA1 treated group (11.93 $\pm$ 1.69, \textit{p} < 0.05), and a significant decrease in the 1 $\mu$M Ril (0.03 $\pm$ 0.03, \textit{p} < 0.05) and 10 $\mu$M Ril (0.03 $\pm$ 0.03, \textit{p} < 0.05) compared to the control group (3.90 $\pm$ 0.75) (\Cref{fig:30_spd_ril_poolsize1}: \textbf{B}). A significant decrease was revealed in the 1 $\mu$M Ril, 1 $\mu$M Ril + BafA1 (2.45 $\pm$ 0.60, \textit{p} < 0.05), 10 $\mu$M Ril and 10$\mu$M Ril + BafA1 (3.00 $\pm$ 0.72, \textit{p} < 0.05) compared to the BafA1 treated group. Furthermore, a significant increase autophagosome number was observed in 1 $\mu$M Ril + Baf versus 1 $\mu$M Ril and in 10 $\mu$M Ril + Baf versus 10$\mu$M Ril.

\begin{figure}[!htbp]
  \begin{subfigure}[b]{0.495\linewidth}
    \includegraphics[width=\linewidth]{figures/chapter30/30_spd_ril_poolsize_a}
    \caption{Spermidine}
  \end{subfigure}
  \begin{subfigure}[b]{0.495\linewidth}
    \includegraphics[width=\linewidth]{figures/chapter30/30_spd_ril_poolsize_b}
    \caption{Rilmenidine.}
  \end{subfigure}
  \caption[Effect of low and high concentration of spermidine and rilmenidine on autolysosome pool size]{Effect of low and high concentration of spermidine and rilmenidine on autolysosome pool size. Quantitative analysis for autolysosome puncta in GT 1-7s treated with spermidine (\textbf{A}) and rilmenidine (\textbf{B}). Data are presented as mean $\pm$ SEM, \textit{n} = 3, with a total of 30 cells analysed. * = \textit{p} < 0.05 vs control (graph \textbf{A}), * = \textit{p} < 0.05 vs control, \# = \textit{p} < 0.05 vs BafA1, \& = \textit{p} < 0.05 vs 1 $\mu$M Ril (graph \textbf{B}).}
  \label{fig:30_spd_ril_poolsize1}
\end{figure}

\subsection{Assessment of autolysosome pool size}
Next, the total number of autolysosomes was quantitatively assessed.  Autolysosomes number (AL) significantly increased in 1 $\mu$M Spd (36.79 $\pm$ 4.31, \textit{p} < 0.05) and 10 $\mu$M Spd (36.63 $\pm$ 2.04, \textit{p} < 0.05) compared to the control groups (24.60 $\pm$ 3.04) (\Cref{fig:30_spd_ril_poolsize2}: \textbf{A}). No significant differences were observed in the BafA1 treated group compared to all other groups.

With regards to the rilmenidine treatment, a robust and significant increase in the number of autolysosomes was observed in 1 $\mu$M Ril (138.90 $\pm$ 6.61), 1 $mu$M Ril + BafA1 (107.60 $\pm$ 7.45), 10 $\mu$M Ril (104.70 $\pm$ 5.75) and 10 $\mu$M Ril + BafA1 (102.00 $\pm$ 7.13) compared to the control (24.60 $\pm$ 3.04) and BafA1 treated group (28.13 $\pm$ 3.0), with no significant differences observed between control and BafA1 group (\Cref{fig:30_spd_ril_poolsize2}: \textbf{B}). In addition, a significant decrease in the number of AL was observed in 1 $\mu$M Ril + BafA1 compared to 1 $\mu$M Ril and in 10 $\mu$M Ril compared 1 $\mu$M Ril (\Cref{fig:30_spd_ril_poolsize2}: \textbf{B}).

\begin{figure}[!htbp]
  \begin{subfigure}[b]{0.495\linewidth}
    \includegraphics[width=\linewidth]{figures/chapter30/30_spd_ril_poolsize_c}
    \caption{Spermidine}
  \end{subfigure}
  \begin{subfigure}[b]{0.495\linewidth}
    \includegraphics[width=\linewidth]{figures/chapter30/30_spd_ril_poolsize_d}
    \caption{Rilmenidine.}
  \end{subfigure}
  \caption[Effect of low and high concentration of spermidine and rilmenidine on autolysosome pool size]{Effect of low and high concentration of spermidine and rilmenidine on autolysosome pool size. Quantitative analysis for autolysosome puncta in GT 1-7s treated with spermidine (\textbf{A}) and rilmenidine (\textbf{B}). Data are presented as mean $\pm$ SEM, \textit{n} = 3, with a total of 30 cells analysed. * = \textit{p} < 0.05 vs control (graph \textbf{A}), * = \textit{p} < 0.05 vs control, \# = \textit{p} < 0.05 vs BafA1, \& = \textit{p} < 0.05 vs 1 $\mu$M Ril (graph \textbf{B}).}
  \label{fig:30_spd_ril_poolsize2}
\end{figure}

\subsection{Assessment of autophagosomes flux and transition time}
Next, autophagy flux i.e. the rate of protein degradation through autophagy \citep{klionsky2016,loos2014} and the transition time ($\tau$) i.e. the time required for a cell to clear its autophagosome pool size \citep{DuToit2018b,loos2014} were assessed. A concentration dependent increase in rate of autophagosome number accumulation following complete inhibition with BafA1 was observed following 8 h treatment with spermidine at 1 $\mu$M (3.12 nA/h/cell) and 10 $\mu$M Spd (3.32 nA/h/cell) compared to the basal flux in control cells (2.01 nA/h/cell) (\Cref{fig:30_spd_ril_flux}: Left). In contrast, treatment with rilmenidine resulted in a decrease in autophagosome flux at 1 $\mu$M (0.60 nA/h/cell) and 10 $\mu$M (0.74 nA/h/cell) compared to the basal flux in control cells (2.01 nA/h) as shown in \Cref{fig:30_spd_ril_flux} (Right). Of note, a major increase in the autolysosomes pool size was observed upon rilmenidine treatment. Further analysis of the transition time showed that 1 $\mu$M Spd required 0.64 h to turn over the autophagosome pool, while 10 $\mu$M Spd required 1.21 h compared to 2 h needed by the control cells to turn over its pool. On the other hand, rilmenidine required 0.05 h and 0.040 h with a low and high dose respectively to clear its entire pool compared to the control (2 h) (\Cref{tab:30_flux}).

\begin{figure}[!htbp]
\center
  \includegraphics[width=\linewidth]{figures/chapter30/30_spd_ril_flux}
  \caption[Effect of low and high concentration spermidine and rilmenidine on autophagosome flux]{Effect of low and high concentration spermidine and rilmenidine on autophagosome flux. Following quantification of autophagosome and autolysosome number in spermidine and rilmenidine. Data are presented as mean $\pm$ SEM, \textit{n} = 3, with a total of 30 cells analysed 6 per group. BafA1 was added after 8 h.}
  \label{fig:30_spd_ril_flux}
\end{figure} 

\begin{table}[!htbp]
\centering
\caption[Autophagosome flux and transition time under basal and induced autophagy in GT 1-7 cells]{Autophagosome flux and transition time under basal and induced autophagy in GT 1-7 cells}
\label{tab:30_flux}
  \begin{tabular}{lcc}
\toprule
Treatment & Autophagy flux J (nA/h/cell) & Transition time $\tau$\\
\midrule
Con & 2.01 & 2.00 h \\
1 $\mu$M Spd & 3.12 & 0.64 h \\
10 $\mu$M Spd & 3.32 & 1.21 h \\
1 $\mu$M Ril & 0.60 & 0.05 h \\
10 $\mu$M Ril & 0.74 & 0.04 h \\
\end{tabular}
\end{table}

\section{Effect of low and high concentration of spermidine and rilmenidine on p62 puncta using FM}
\subsection{Assessment of p62 puncta}

In order to assess the effect of spermidine and rilmenidine on autophagy substrate p62, GT 1-7 cells were treated as required and immuno-stained for p62. Cells were imaged using fluorescence microscopy and the total number of p62 puncta was counted using image J. 

A significant increase in the number of p62 puncta was observed following treatment with Baf (105.10 $\pm$ 16.56, \textit{p} < 0.05), 1 $\mu$M Spd + Baf (130.40 $\pm$ 22.67, \textit{p} < 0.05) and 10 $\mu$M Spd + Baf (134.50 $\pm$ 35.71, \textit{p} < 0.05) compared to the control (37.13 $\pm$ 8.02) (\Cref{fig:30_fluorescent_graph}: \textbf{A} and \Cref{fig:30_fluorescent}: Left). Similarly, a significant increase in p62 puncta was observed in 1 $\mu$M Spd + Baf versus 1 $\mu$M Spd and in 10 $\mu$M Spd + Baf versus 10 $\mu$M Spd, suggesting that the accumulation of p62 cargo takes place at basal and autophagy induced conditions. 

Similarly, upon rilmenidine treatment, a significant increase in the number of p62 puncta was observed in the Baf treated group (197.60 $\pm$ 18.87, \textit{p} < 0.05), 1 $\mu$M Ril + Baf (164.40 $\pm$ 28.48, \textit{p} < 0.05) and 10 $\mu$M Ril + Baf (146.80 $\pm$ 16.34, \textit{p} < 0.05) compared to the control (92.39 $\pm$ 8.71) (\Cref{fig:30_fluorescent_graph}: \textbf{B} and \Cref{fig:30_fluorescent}: Right). In addition, a significant increase was observed in 1 $\mu$M Ril + Baf versus 1 $\mu$M Ril and in 10 $\mu$M Ril + Baf versus 10 $\mu$M Ril.

\begin{figure}[!htbp]
  \begin{subfigure}[b]{0.495\linewidth}
    \includegraphics[width=\linewidth]{figures/chapter30/30_fluorescent_a}
    \caption{Spermidine}
  \end{subfigure}
  \begin{subfigure}[b]{0.495\linewidth}
    \includegraphics[width=\linewidth]{figures/chapter30/30_fluorescent_b}
    \caption{Rilmenidine.}
  \end{subfigure}
    \caption[Effect of low and high concentration of spermidine and rilmenidine on p62 puncta]{Effect of low and high concentration of spermidine and rilmenidine on p62 puncta. Data are presented as mean $\pm$ SEM, \textit{n} = 3 with a total of 40-60 cells analysed per treatment group. * = \textit{p} < 0.05 vs control, \& = \textit{p} < 0.05 vs 1 $\mu$M Spd, \% = \textit{p} < 0.05 vs 10 $\mu$M Spd (graph \textbf{A}), * = \textit{p} < 0.05 vs control, \& = \textit{p} < 0.05 vs 1 $\mu$M Ril, \% = \textit{p} < 0.05 vs 10 $\mu$M Ril (graph \textbf{B}).}
  \label{fig:30_fluorescent_graph}
\end{figure}

\begin{figure}[!htbp]
\center
  \includegraphics[width=\linewidth]{figures/chapter30/30_fluorescent}
  \caption[Effect of low and high concentration of spermidine and rilmenidine on p62 puncta - fluorescence micrographs]{Effect of low and high concentration of spermidine and rilmenidine on p62 puncta. Representative fluorescence micrographs showing quantitative analysis of p62 puncta in GT 1-7s treated with spermidine (Left) and rilmenidine (Right).}
  \label{fig:30_fluorescent}
\end{figure} 

\section{Effect of low and high concentration spermidine and rilmenidine on autophagic vacuoles using TEM}
\subsection{Assessment of autophagic vacuoles (AVs)}
 
