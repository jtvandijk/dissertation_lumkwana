\chapter{Autophagic flux assessment}
\label{sec:chapter3}
\section{Introduction}
Autophagy is a physiological cellular process that is characterized by a fine balance between the rate of AV formation and the rate of their clearance, with autophagic flux referring  to the rate of flow of material through the autophagic pathway \citep{loos2014,klionsky2016}. Many tools and techniques are available to assess autophagy, including western blotting (WB), fluorescence microscopy (FM), Transmission Electron Microscopy (TEM), flow cytometry \citep{klionsky2016} and Correlative Light and Electron Microscopy (CLEM) with each technique having particular advantages, but also inherent disadvantages.

Autophagy plays a significant role in health where primarily decreased activation of autophagy has been shown to have detrimental effects. Autophagy can be induced via mTOR dependent and mTOR independent pathways \citep{sarkar2013}. Although we have progressed substantially in the understanding of the autophagy machinery and its regulation, its dysfunction in pathology as well as its dynamic changes however in the disease progression remains often largely unclear. Furthermore, although a substantial progress has been made in unravelling the role and regulation of autophagy and its modulation in neurodegenerative disease using pharmacological agents, major differences exist between experimental design, drug concentrations used, duration of treatment intervention as well as the model system used to assess autophagic flux \citep{lumkwana2017}. This contributes to inability to modulate autophagy with high precision. Specifically, spermidine and rilmenidine have been shown to induce autophagy, but the impact of concentration ranges on autophagic flux and on subsequent protein clearance and neuronal toxicity is unclear. Importantly, whether a concentration-dependent effect on autophagy activity exists achieving defined, but distinguishable heightened autophagy flux remains unclear. Particularly in terms of fine-tuning autophagy activity with a given flux deviation in neurodegeneration, this aspect is of critical importance. In this chapter, we first aimed to establish a suitable concentration of spermidine (Spd) and rilmenidine (Ril) using GT 1-7 neuronal cells that induces autophagic flux without causing cellular toxicity, and secondly, to characterize the autophagic flux profile of a low and high concentration of both drugs established in terms of (1) the autophagic machinery and pathway intermediates (autophagosomes, autolysosome),(2) the autophagosome flux, (3) transition time, (4) p62 puncta, and (5) autophagic vacuoles (AVs). For this purpose, GT1-7 neuronal cells were treated with three concentrations of Spd and Ril in the presence and absence of BafA1. Subsequently, cellular viability and WB for LC3-II and p62 was performed. Next, to assess the effect of a low and high concentration of Spd and Ril, GT 1-7 cells were treated with 1 \& 10 $\mu$M respectively in the presence and absence of BafA1 and thereafter extensively analysed using FM and TEM linked to quantitative morphometric analysis. In addition, GT1-7 cells were transfected with mRFP-GFP-LC3 \citep{yoshii2017} and subsequently treated with 1 \& 10 $\mu$M of Spd and Ril in the presence and absence of BafA1 and analysed with FM.
 
\section{Effect of spermidine and rilmenidine on cell viability}
In order to choose a suitable, yet non-toxic concentration of autophagy modulators, WST-1 was performed. For the concentrations assessed, no decrease in viability was observed. In fact, a minor metabolic response associated with enhanced reductive capacity was observed. Reductive capacity was significantly increased using 1 $\mu$M Spd 104.70 \% $\pm$ 1.64 \%, \textit{p} < 0.05 and 10 $\mu$M Spd (109.20 $\pm$ 1.38 \%, \textit{p} < 0.05) compared to the control (100.00 \% $\pm$ 1.50 \%), with no significant differences observed using 0.1 $\mu$M Spd (104.20 \% $\pm$ 1.46 \%) (\Cref{fig:30_spd_ril_cell_viability_a}: \textbf{A}). More importantly, a significant increase in reductive capacity was observed with 10 $\mu$M Spd (\textit{p} < 0.01) compared to 0.1 $\mu$M Spd and 1 $\mu$M Spd.

Following rilmenidine treatment, a significant increase in reductive capacity was observed in the 10 $\mu$M Ril treated group (105.70 $\pm$ 2.26 \%, \textit{p} < 0.05) compared to the control (100.40 $\pm$ 1.16 \%), with no significant differences observed in the 0.1 $\mu$M Ril (98.76 \% $\pm$ 1.49 \%) and 1 $\mu$M Ril (100.50 \% $\pm$ 2.12 \%) (\Cref{fig:30_spd_ril_cell_viability_a}: \textbf{B}). Importantly, a significant increase in cellular viability was observed in the 10 $\mu$M Ril group compared to 0.1 $\mu$M and 1 $\mu$M Ril. This data suggests no toxicity and allowed to choose a low and high concentration of spermidine and rilmenidine for subsequent experiments.

\begin{figure}[!htbp]
  \centering
  \begin{subfigure}[b]{0.495\linewidth}
    \includegraphics[width=\linewidth]{figures/chapter30/30_spd_ril_cell_viability_a}
    \caption{Spermidine}
  \end{subfigure}
  \begin{subfigure}[b]{0.495\linewidth}
    \includegraphics[width=\linewidth]{figures/chapter30/30_spd_ril_cell_viability_b}
    \caption{Rilmenidine}
  \end{subfigure}
  \caption[Effect of spermidine and rilmenidine on cell viability]{\textbf{Effect of spermidine and rilmenidine on cell viability.} GT1-7 cells were treated with 0.1, 1 \& 10 $\mu$M of Spd (\textbf{A}) and Ril (\textbf{B}) for 8 hr. Subsequently, reductive capacity as a measure of cell viability was assessed using WST1 assay. All results are presented as a percentage of the control (mean $\pm$ SEM), \textit{n}=3, with 6 replicates per group. * = \textit{p} < 0.05 vs control, \# = \textit{p} < 0.05 vs 0.1 $\mu$M Spd or 0.1 $\mu$M Ril and \$ = \textit{p} < 0.05 vs 1 $\mu$M Spd or 1 $\mu$M Ril.}
  \label{fig:30_spd_ril_cell_viability_a}
\end{figure}

\section{Effect of spermidine and rilmenidine on autophagic flux using western blotting}
To further identify the concentration that would have maximal autophagy inducing effects, protein extraction was performed following 8 hr treatment with a range of Spd and Ril concentrations. Western blot analysis was performed to assess the protein expression of LC3-II and p62. LC3-II is formed through conjugation of LC3-I to PE and is localized on the inner and outer surface of the autophagosome membrane, where it is either degraded upon fusion of autophagosomes with lysosomes or removed through deconjugation and recycled \citep{kabeya2000}. Importantly, the abundance of LC3 II correlates with the number of autophagosomes and hence is used as a key marker to indicate autophagy or the size of the autophagosome pool \citep{loos2014} p62 serves as a scaffolding for ubiquitinated proteins \citep{sahani2014}. It binds directly to LC3 and is in turn selectively degraded through autophagy; thus it is widely used as an additional indicator for autophagy activity \citep{pankiv2007}.

\subsection{Abundance of LC3-II and p62}
A significant increase in the abundance of LC3-II was observed in the 10 $\mu$M Spd + BafA1 (1.67 $\pm$ 0.28) compared to the control group (1.00 $\pm$ 0.00) (\Cref{fig:30_spd_ril_abun_a}: \textbf{A}\textit{i}, \textbf{A}\textit{iii}). Moreover, a significant increase in the abundance of LC3-II was observed in the 10 $\mu$M Spd + BafA1 (\textit{p} < 0.05) compared to 0.1 $\mu$M Spd + BafA1. No significant differences were observed between 0.1 $\mu$M Spd, 1 $\mu$M Spd and 10 $\mu$M Spd. A significant increase was observed in the 10 $\mu$M Spd + BafA1 compared to10 $\mu$M Spd, with no significant differences seen between 0.1 $\mu$M Spd and 0.1 $\mu$M Spd + BafA1 and between 1 $\mu$M Spd versus 1 $\mu$M Spd + BafA1, however it is important to mention that in both the 0.1 $\mu$M Spd  and 1 $\mu$M Spd, inhibition of degradation with BafA1 resulted in an increased LC3 II accumulation compared to the BafA1 untreated groups (\Cref{fig:30_spd_ril_abun_a}: \textbf{A}\textit{i}, \textbf{A}\textit{iii}), suggesting the presence of a basal as well as increased autophagic activity.

The abundance of p62 was significantly increased in the BafA1 treated group (1.72 $\pm$ 0.26, \textit{p} < 0.05) and  10 $\mu$M Spd + BafA1 (1.72 $\pm$ 0.17, \textit{p} < 0.05) compared to Con (1.00 $\pm$ 0.00) (\Cref{fig:30_spd_ril_abun_a}: \textbf{A}\textit{ii}, \textbf{A}\textit{iii}). Moreover, in comparison to the BafA1 treated group, a significant decrease in p62 abundance was observed in the 1 $\mu$M Spd, with no significant differences seen in all other groups. A higher abundance was indicated in the presence of BafA1 in all treatment groups compared to the respective BafA1 untreated groups (\Cref{fig:30_spd_ril_abun_a}: \textbf{A}\textit{i}, \textbf{A}\textit{iii}).

With regards to rilmenidine treatment (\Cref{fig:30_spd_ril_abun_a}: \textbf{B}\textit{i}, \textbf{B}\textit{iii}), no significant differences in the abundance of LC3-II were observed, however an increasing trend in the amount of LC3-II in the BafA1 treated groups compared to the BafA1 untreated groups was noted [Con (1.00 $\pm$ 0.00), BafA1 (1.45 $\pm$ 0.49), 0.1 $\mu$M (1.16 $\pm$ 0.09), 0.1 $\mu$M Ril + BafA1 (1.46 $\pm$ 0.36), 1 $\mu$M Ril (0.94 $\pm$ 0.14), 1 $\mu$M Ril + BafA1 (1.22 $\pm$ 0.42), 10 $\mu$M Ril (1.02 $\pm$ 0.27) and 10 $\mu$M Ril + BafA1 (1.17 $\pm$ 0.63)].

A significant increase in p62 protein expression was observed in the 1 $\mu$M Ril (1.68 $\pm$ 0.19, \textit{p} < 0.05), 1 $\mu$M Ril + BafA1 (2.21 $\pm$ 0.31, \textit{p} < 0.05) and 10 $\mu$M Ril + BafA1 (1.81 $\pm$ 0.08, \textit{p} < 0.05) compared to the Con group (1.00 $\pm$ 0.00) (\Cref{fig:30_spd_ril_abun_a}: \textbf{B}\textit{ii}, \textbf{B}\textit{iii}). Moreover, a significant increase in p62 protein levels was revealed in the 1 $\mu$M Ril + BafA1 compared to BafA1 treated group. Furthermore, a significant increase in p62 abundance was observed in the 1 $\mu$M Ril + BafA1 (\textit{p} < 0.05) compared to 0.1 $\mu$M Ril + BafA1, with no significant differences observed when compared to 10 $\mu$M Ril + BafA1. Finally, overall p62 protein abundance accumulated in the BafA1 treated groups compared to the respective BafA1 untreated groups, however not reaching significance (\Cref{fig:30_spd_ril_abun_a}: \textbf{B}\textit{ii}, \textbf{B}\textit{iii}). No response was observed in 0.1 $\mu$M Ril + BafA1 compared to its BafA1 untreated group. This data suggests an autophagy enhancing effect, which is best revealed through the impacted p62 clearance.

\begin{landscape}
\begin{figure}[!htbp]
  \centering
  \begin{subfigure}[b]{0.35\linewidth}
    \includegraphics[width=\linewidth]{figures/chapter30/30_spd_ril_abun_a}
    \caption{Spermidine}
  \end{subfigure}
  \begin{subfigure}[b]{0.35\linewidth}
    \includegraphics[width=\linewidth]{figures/chapter30/30_spd_ril_abun_b}
    \caption{Rilmenidine}
  \end{subfigure}
  \caption[Effect of spermidine and rilmenidine concentrations on LC3-II and p62 abundance]{\textbf{Effect of spermidine and rilmenidine concentrations on LC3-II and p62 abundance.} Quantification and representative blots for LC3-II and p62 in in GT 1-7s following treatment with concentrations of spermidine (\textbf{A}) and rilmenidine (\textbf{B}). Data are presented as mean $\pm$ SEM, \textit{n} = 3 - 4. * = \textit{p} < 0.05 vs control, \# = \textit{p} < 0.05 vs BafA1, \$ = \textit{p} < 0.05 vs 0.1$\mu$M Spd + BafA1 or 0.1$\mu$M Ril + Baf and \% = \textit{p} < 0.05 vs 10 $\mu$M Spd.}
  \label{fig:30_spd_ril_abun_a}
\end{figure}
\end{landscape}

\subsection{LC3-II protein turnover}
In order to better estimate autophagic flux, turnover of LC3-II was assessed. To estimate autophagic flux, it is necessary to determine the extent of change in the LC3 II protein abundance at a given time point in the presence of lysosomal inhibitors (i.e. BafA1) \citep{Martinez-Lopez2013}. Thus, mean values of the BafA1 treated groups (BafA1, 0.1 $\mu$M Spd + BafA1, 1 $\mu$M Spd + Baf A1and 10 $\mu$M Spd + BafA1) were subtracted from the BafA1 untreated groups (control, 0.1 $\mu$M Spd, 1 $\mu$M Spd and 10 $\mu$M Spd), respectively.

A significant increase in LC3-II protein turnover was observed using 10 $\mu$M Spd (0.91 $\pm$ 0.15, \textit{p} < 0.05) compared to the control group (0.24 $\pm$ 0.11) as well as the 0.1 $\mu$M Spd (0.12 $\pm$ 0.09) and 1 $\mu$M Spd (0.26 $\pm$ 0.07), with no significant differences observed using 0.1 $\mu$M Spd  and 1 $\mu$M Spd (\Cref{fig:30_spd_ril_lc3ii}: \textbf{A}). 

With regards to rilmenidine treatment, no significant differences in the LC3-II turnover were observed in all groups [control (0.45 $\pm$ 0.49), 0.1 $\mu$M Ril (0.30 $\pm$ 0.29), 1 $\mu$M Ril (0.27 $\pm$ 0.32) and 10 $\mu$M Ril (0.15 $\pm$ 0.36)], however, there is a progressive trend in decline with an increase in Ril concentration (\Cref{fig:30_spd_ril_lc3ii}: \textbf{B}). 

\begin{figure}[!htbp]
  \centering
  \begin{subfigure}[b]{0.495\linewidth}
    \includegraphics[width=\linewidth]{figures/chapter30/30_spd_ril_lc3II_a}
    \caption{Spermidine}
  \end{subfigure}
  \begin{subfigure}[b]{0.495\linewidth}
    \includegraphics[width=\linewidth]{figures/chapter30/30_spd_ril_lc3II_b}
    \caption{Rilmenidine.}
  \end{subfigure}
  \caption[Effect of spermidine and rilmenidine concentrations on LC3-II turnover]{\textbf{Effect of spermidine and rilmenidine concentrations on LC3-II turnover.} Following quantification of LC3-II protein levels, mean values of the Baf treated group were subtracted from the corresponding Baf untreated groups. Data are presented as mean $\pm$ SEM, \textit{n} = 3 - 4. * = \textit{p} < 0.05 vs control, \# = p < 0.05 vs 0.1 $\mu$M Spd and \$ = \textit{p} < 0.05 vs 1 $\mu$M Spd.}
  \label{fig:30_spd_ril_lc3ii}
\end{figure}

\section{Effect of low and high concentrations spermidine and rilmenidine on autophagy pathway intermediates, autophagosome flux and transition time using quantitative FM}
In order to assess the effect of spermidine and rilmenidine on the autophagy pathway intermediates i.e. autophagosomes, autolysosomes as well as autophagosome flux, GT 1-7 cells were transfected with a mRFP-GFP-LC3 plasmid \citep{yoshii2017} and imaged using a fluorescence microscopy. The mRFP-GFP-LC3 plasmid allows to detect pH dependent changes as it fluoresces red and green in the autophagosomes resulting in a yellow signal, while only fluorescing red in the autolysosome due to the quenching of GFP signal as a result of the auto-lysosomal acidity \citep{yoshii2017}. Transfected cells were treated with a low (1 $\mu$M) and a high (10 $\mu$M) concentration of spermidine and rilmenidine for 8 h. Subsequently, cells were exposed to BafA1 (400 nM) for further 4 h to completely block the fusion of autophagosomes with lysosomes \citep{DuToit2018b} and imaged with fluorescence microscopy using z-stack acquisition. The number of autophagosome puncta (nA, yellow puncta) and autolysomes (nAL, red puncta) before and after autophagy induction with spermidine and rilmenidine as well as before and after complete inhibition of autophagosome fusion with lysosomes was assessed. In addition, autophagic flux (J), which is the initial rate of increase in nA and nAL per cell after treatment with Baf A1 and transition time required for a cell to clean its pool were also measured \citep{DuToit2018a,DuToit2018b,loos2014}.

\subsection{Assessment of autophagosome pool size}
The number of autophagosome significantly increased in the BafA1 treated group (11.93  $\pm$ 1.69), 1 $\mu$M Spd + BafA1 (14.48 $\pm$ 2.69, \textit{p} < 0.05) and 10 $\mu$M Spd + BafA1 (17.33 $\pm$ 2.82, \textit{p} < 0.05) compared to the control (3.90 $\pm$ 0.75) (\Cref{fig:30_spd_ril_poolsize1}: \textbf{A}). A significant increase was revealed in the 10 $\mu$M Spd + Baf compared to the BafA1 group. Moreover, a significant increase autophagosome number was observed in 1 $\mu$M Spd + Baf versus 1 $\mu$M Spd (2.00 $\pm$ 0.62) and in 10 $\mu$M Spd + Baf versus 10 $\mu$M Spd (4.03 $\pm$ 0.82) (\Cref{fig:30_spd_ril_poolsize1}: \textbf{A}).

Rilmenidine treatment resulted in a significant increase in the number of autophagosomes in the BafA1 treated group (11.93 $\pm$ 1.69, \textit{p} < 0.05), and a significant decrease in the 1 $\mu$M Ril (0.03 $\pm$ 0.03, \textit{p} < 0.05) and 10 $\mu$M Ril (0.03 $\pm$ 0.03, \textit{p} < 0.05) compared to the control group (3.90 $\pm$ 0.75) (\Cref{fig:30_spd_ril_poolsize1}: \textbf{B}). A significant decrease was revealed in the 1 $\mu$M Ril, 1 $\mu$M Ril + BafA1 (2.45 $\pm$ 0.60, \textit{p} < 0.05), 10 $\mu$M Ril and 10$\mu$M Ril + BafA1 (3.00 $\pm$ 0.72, \textit{p} < 0.05) compared to the BafA1 treated group. Furthermore, a significant increase autophagosome number was observed in 1 $\mu$M Ril + Baf versus 1 $\mu$M Ril and in 10 $\mu$M Ril + Baf versus 10$\mu$M Ril.

\begin{figure}[!htbp]
  \begin{subfigure}[b]{0.495\linewidth}
    \includegraphics[width=\linewidth]{figures/chapter30/30_spd_ril_poolsize_a}
    \caption{Spermidine}
  \end{subfigure}
  \begin{subfigure}[b]{0.495\linewidth}
    \includegraphics[width=\linewidth]{figures/chapter30/30_spd_ril_poolsize_b}
    \caption{Rilmenidine.}
  \end{subfigure}
  \caption[Effect of low and high concentrations of spermidine and rilmenidine on autolysosome pool size]{\textbf{Effect of low and high concentrations of spermidine and rilmenidine on autolysosome pool size.} Quantitative analysis for autolysosome puncta in GT 1-7s treated with spermidine (\textbf{A}) and rilmenidine (\textbf{B}). Data are presented as mean $\pm$ SEM, \textit{n} = 3, with a total of 30 cells analysed. * = \textit{p} < 0.05 vs control, \# = \textit{p} < 0.05 vs BafA1, \@ = \textit{p} < 0.05 vs 1 $\mu$M Spd or 1 $\mu$M Ril and \% = \textit{p} < 0.05 vs 10 $\mu$M Spd or 10 $\mu$M Ril .}
  \label{fig:30_spd_ril_poolsize1}
\end{figure}

\subsection{Assessment of autolysosome pool size}
Next, the total number of autolysosomes was quantitatively assessed.  Autolysosomes number (AL) significantly increased in 1 $\mu$M Spd (36.79 $\pm$ 4.31, \textit{p} < 0.05) and 10 $\mu$M Spd (36.63 $\pm$ 2.04, \textit{p} < 0.05) compared to the control groups (24.60 $\pm$ 3.04) (\Cref{fig:30_spd_ril_poolsize2}: \textbf{A}). No significant differences were observed in the BafA1 treated group compared to all other groups.

With regards to the rilmenidine treatment, a robust and significant increase in the number of autolysosomes was observed in 1 $\mu$M Ril (138.90 $\pm$ 6.61), 1 $mu$M Ril + BafA1 (107.60 $\pm$ 7.45), 10 $\mu$M Ril (104.70 $\pm$ 5.75) and 10 $\mu$M Ril + BafA1 (102.00 $\pm$ 7.13) compared to the control (24.60 $\pm$ 3.04) and BafA1 treated group (28.13 $\pm$ 3.0), with no significant differences observed between control and BafA1 group (\Cref{fig:30_spd_ril_poolsize2}: \textbf{B}). In addition, a significant decrease in the number of AL was observed in 1 $\mu$M Ril + BafA1 compared to 1 $\mu$M Ril and in 10 $\mu$M Ril compared 1 $\mu$M Ril (\Cref{fig:30_spd_ril_poolsize2}: \textbf{B}).

\begin{figure}[!htbp]
  \begin{subfigure}[b]{0.495\linewidth}
    \includegraphics[width=\linewidth]{figures/chapter30/30_spd_ril_poolsize_c}
    \caption{Spermidine}
  \end{subfigure}
  \begin{subfigure}[b]{0.495\linewidth}
    \includegraphics[width=\linewidth]{figures/chapter30/30_spd_ril_poolsize_d}
    \caption{Rilmenidine.}
  \end{subfigure}
  \caption[Effect of low and high concentrations of spermidine and rilmenidine on autolysosome pool size]{\textbf{Effect of low and high concentrations of spermidine and rilmenidine on autolysosome pool size.} Quantitative analysis for autolysosome puncta in GT 1-7s treated with spermidine (\textbf{A}) and rilmenidine (\textbf{B}). Data are presented as mean $\pm$ SEM, \textit{n} = 3, with a total of 30 cells analysed. * = \textit{p} < 0.05 vs control, \# = \textit{p} < 0.05 vs BafA1, \@ = \textit{p} < 0.05 vs 1 $\mu$M Ril.}
  \label{fig:30_spd_ril_poolsize2}
\end{figure}

\subsection{Assessment of autophagosomes flux and transition time}
Next, autophagy flux i.e. the rate of protein degradation through autophagy \citep{klionsky2016,loos2014} and the transition time ($\tau$) i.e. the time required for a cell to clear its autophagosome pool size \citep{DuToit2018b,loos2014} were assessed. A concentration-dependent increase in rate of autophagosome number accumulation following complete inhibition with BafA1 was observed following 8 h treatment with spermidine at 1 $\mu$M (3.12 nA/h/cell) and 10 $\mu$M Spd (3.32 nA/h/cell) compared to the basal flux in control cells (2.01 nA/h/cell) (\Cref{fig:30_spd_ril_flux}: Left). In contrast, treatment with rilmenidine resulted in a decrease in autophagosome flux at 1 $\mu$M (0.60 nA/h/cell) and 10 $\mu$M (0.74 nA/h/cell) compared to the basal flux in control cells (2.01 nA/h) as shown in \Cref{fig:30_spd_ril_flux} (Right). Of note, a major increase in the autolysosomes pool size was observed upon rilmenidine treatment. Further analysis of the transition time showed that 1 $\mu$M Spd required 0.64 h to turn over the autophagosome pool, while 10 $\mu$M Spd required 1.21 h compared to 2 h needed by the control cells to turn over its pool. On the other hand, rilmenidine required 0.05 h and 0.040 h with a low and high dose respectively to clear its entire pool compared to the control (2 h) (\Cref{tab:30_flux}).

\begin{figure}[!htbp]
\center
  \includegraphics[width=\linewidth]{figures/chapter30/30_spd_ril_flux}
  \caption[Effect of low and high concentrations spermidine and rilmenidine on autophagosome flux]{\textbf{Effect of low and high concentrations spermidine and rilmenidine on autophagosome flux.} Following quantification of autophagosome and autolysosome number in spermidine and rilmenidine. Data are presented as mean $\pm$ SEM, \textit{n} = 3, with a total of 30 cells analysed 6 per group. BafA1 was added after 8 h.}
  \label{fig:30_spd_ril_flux}
\end{figure} 

\begin{table}[!htbp]
\centering
\caption[Autophagosome flux and transition time under basal and induced autophagy in GT 1-7 cells]{Autophagosome flux and transition time under basal and induced autophagy in GT 1-7 cells}
\label{tab:30_flux}
  \begin{tabular}{lcc}
\toprule
Treatment & Autophagy flux J (nA/h/cell) & Transition time $\tau$\\
\midrule
Con & 2.01 & 2.00 h \\
1 $\mu$M Spd & 3.12 & 0.64 h \\
10 $\mu$M Spd & 3.32 & 1.21 h \\
1 $\mu$M Ril & 0.60 & 0.05 h \\
10 $\mu$M Ril & 0.74 & 0.04 h \\
\end{tabular}
\end{table}

\section{Effect of low and high concentrations of spermidine and rilmenidine on p62 puncta using FM}
\subsection{Assessment of p62 puncta}
In order to assess the effect of spermidine and rilmenidine on autophagy substrate p62, GT 1-7 cells were treated as required and immuno-stained for p62. Cells were imaged using fluorescence microscopy and the total number of p62 puncta was counted using image J. 

A significant increase in the number of p62 puncta was observed following treatment with Baf (105.10 $\pm$ 16.56, \textit{p} < 0.05), 1 $\mu$M Spd + Baf (130.40 $\pm$ 22.67, \textit{p} < 0.05) and 10 $\mu$M Spd + Baf (134.50 $\pm$ 35.71, \textit{p} < 0.05) compared to the control (37.13 $\pm$ 8.02) (\Cref{fig:30_fluorescent_graph}: \textbf{A} and \Cref{fig:30_fluorescent}: Left). Similarly, a significant increase in p62 puncta was observed in 1 $\mu$M Spd + Baf versus 1 $\mu$M Spd and in 10 $\mu$M Spd + Baf versus 10 $\mu$M Spd, suggesting that the accumulation of p62 cargo takes place at basal and autophagy induced conditions. 

Similarly, upon rilmenidine treatment, a significant increase in the number of p62 puncta was observed in the Baf treated group (197.60 $\pm$ 18.87, \textit{p} < 0.05), 1 $\mu$M Ril + Baf (164.40 $\pm$ 28.48, \textit{p} < 0.05) and 10 $\mu$M Ril + Baf (146.80 $\pm$ 16.34, \textit{p} < 0.05) compared to the control (92.39 $\pm$ 8.71) (\Cref{fig:30_fluorescent_graph}: \textbf{B} and \Cref{fig:30_fluorescent}: Right). In addition, a significant increase was observed in 1 $\mu$M Ril + Baf versus 1 $\mu$M Ril and in 10 $\mu$M Ril + Baf versus 10 $\mu$M Ril.

\begin{figure}[!htbp]
  \begin{subfigure}[b]{0.495\linewidth}
    \includegraphics[width=\linewidth]{figures/chapter30/30_fluorescent_a}
    \caption{Spermidine}
  \end{subfigure}
  \begin{subfigure}[b]{0.495\linewidth}
    \includegraphics[width=\linewidth]{figures/chapter30/30_fluorescent_b}
    \caption{Rilmenidine}
  \end{subfigure}
    \caption[Effect of low and high concentrations of spermidine and rilmenidine on p62 puncta]{\textbf{Effect of low and high concentrations of spermidine and rilmenidine on p62 puncta.} Data are presented as mean $\pm$ SEM, \textit{n} = 3 with a total of 40 - 60 cells analysed per treatment group. * = \textit{p} < 0.05 vs control, \# = \textit{p} < 0.05 vs BafA1, \@ = \textit{p} < 0.05 vs 1 $\mu$M Spd or 1 $\mu$M Ril and \% = \textit{p} < 0.05 vs 10 $\mu$M Spd or 10 $\mu$M Ril.}
  \label{fig:30_fluorescent_graph}
\end{figure}

\begin{figure}[!htbp]
\center
  \includegraphics[width=\linewidth]{figures/chapter30/30_fluorescent}
  \caption[Effect of low and high concentrations of spermidine and rilmenidine on p62 puncta - fluorescence micrographs]{\textbf{Effect of low and high concentrations of spermidine and rilmenidine on p62 puncta.} Representative fluorescence micrographs showing quantitative analysis of p62 puncta in GT 1-7s treated with spermidine (Left) and rilmenidine (Right).}
  \label{fig:30_fluorescent}
\end{figure} 

\section{Effect of low and high concentrations spermidine and rilmenidine on autophagic vacuoles using TEM}
\subsection{Assessment of AVs}
TEM remains one of the critical tools to assess the autophagic pathway \citep{klionsky2016}, particularly, the abundance and size of the AVs.  In order to assess the effect of spermidine and rilmenidine on AVs, GT 1-7 cells were treated as required and prepared for TEM analysis. After imaging acquisition, the surface area and the number of AVs were analysed using image J. The results indicate a significant increase in the surface area of AVs in the BafA1 group (14.55 $\pm$ 0.33, \textit{p} < 0.05) compared to the control cells (12.54 $\pm$ 0.61), while a significant decrease was observed with 1 $\mu$M Spd + BafA1 (9.65 $\pm$ 0.24, \textit{p} < 0.05) (\Cref{fig:30_spd_ril_avs_a}: \textbf{A} \& \textbf{B}\textit{i}).  A significant reduction in size of AVs was observed in 1 $\mu$M Spd + BafA1 and 10 $\mu$M Spd + BafA1 compared to the BafA1 treated group. In addition, the surface area of AVs was significantly increased in 10 $\mu$M Spd + BafA1 compared to 1 $\mu$M Spd + BafA1, while it was significantly decreased in 1 $\mu$M Spd + BafA1 compared to 1 $\mu$M Spd and in 10 $\mu$M Spd + BafA1 compared to10 $\mu$M Spd (\Cref{fig:30_spd_ril_avs_a}: \textbf{A} \& \textbf{B}\textit{i}). 
 
In terms of autophagic vacuole abundance, a significant increase in the number of AVs per cell was observed in the BafA1 treated group (18.58 $\pm$ 2.27, \textit{p} < 0.05), in 1 $\mu$M Spd + BafA1 (16.56 $\pm$ 2.64, \textit{p} < 0.05) and in 10 $\mu$M Spd + BafA1 (17.42 $\pm$ 1.87, \textit{p} < 0.05) compared to the control (5.05 $\pm$ 0.72) (Fig. 3.8 A \& B.ii). 1 $\mu$M Spd + BafA1 and 10 $\mu$M Spd + BafA1 resulted in a significant increase in the number of AVs when compared to 1 $\mu$M Spd versus 10 $\mu$M Spd, respectively (\Cref{fig:30_spd_ril_avs_a}: \textbf{A} \& \textbf{B}\textit{ii}).

\begin{landscape}
\begin{figure}[!htbp]
\center
  \includegraphics[width=0.75\linewidth]{figures/chapter30/30_spd_ril_avs_a}
  \caption[Effect of low and high concentrations of spermidine on AVs]{\textbf{Effect of low and high concentrations of spermidine on AVs.} Representative TEM images (\textbf{A}) and morphometric analysis (\textbf{B}) of AVs in GT 1-7 cells following treatments with low and high concentrations of spermidine. Data are presented as mean $\pm$ SEM, \textit{n} = 2 with a total of 30 cells analysed per treatment.* = \textit{p} < 0.05 vs control, \# = \textit{p} < 0.05 vs BafA1, \$ = \textit{p} < 0.05 vs 1$\mu$M Spd + BafA1, \@ = \textit{p} < 0.05 vs 1 $\mu$M Spd and  \% = \textit{p} < 0.05 vs 10 $\mu$M Spd.}
  \label{fig:30_spd_ril_avs_a}
\end{figure} 
\end{landscape}

Rilmenidine treatment (\Cref{fig:30_spd_ril_avs_b}: \textbf{A} \& \textbf{B}\textit{i}) resulted in a significant increase in the surface area of AVs following treatment with 1 $\mu$M Ril (37.59 $\pm$ 1.04, p \textit{p} < 0.05), 1 $\mu$M Ril + BafA1 (46.06 $\pm$ 1.25, \textit{p} < 0.05), 10 $\mu$M Ril (49.71 $\pm$ 2.02), and 10 $\mu$M Ril + BafA1 (36.55 $\pm$ 0.89, \textit{p} < 0.05) compared to the control (25.39 $\pm$ 1.91) and BafA1 group (25.64 $\pm$ 0.94), with, however, no significant differences between the control and BafA1 group. In addition, a significant increase in the surface area of the AVs was detected in 1 $\mu$M Ril + BafA1 versus 1 $\mu$M Ril, and in 10 $\mu$M Ril treatment versus 1 $\mu$M Ril, while a significant decrease was observed in 10 $\mu$M Ril + BafA1 versus 10 $\mu$M Ril  treated cells (Fig. 3.9 A \& B.i).

A significant increase in the number of AVs was observed in the BafA1 treated group (18.65 $\pm$ 1.91, \textit{p} < 0.05), 1 $\mu$M Ril + BafA1 (21.67 $\pm$ 2.74, \textit{p} < 0.05), and 10$\mu$M Ril + BafA1 (24.35 $\pm$ 2.23, \textit{p} < 0.05) compared to the control (5.75 $\pm$ 0.71) (\Cref{fig:30_spd_ril_avs_b}: \textbf{A} \& \textbf{B}\textit{ii}). In addition, a significant increase was observed in 1 $\mu$M Ril + BafA1 versus 1 $\mu$M Ril and in 10 $\mu$M Ril + Baf versus 10 $\mu$M Ril, with no significant differences observed the between low and high concentrations of rilmenidine (1 $\mu$M versus 10 $\mu$M).

\begin{landscape}
\begin{figure}[!htbp]
\center
  \includegraphics[width=0.8\linewidth]{figures/chapter30/30_spd_ril_avs_b}
  \caption[Effect of low and high concentrations of rilmenidine on AVs]{\textbf{Effect of low and high concentrations of rilmenidine on AVs.} Representative TEM images (\textbf{A}) and morphometric analysis (\textbf{B}) of AVs in GT 1-7 cells following treatments with low and high concentrations of rilmenidine. Data are presented as mean $\pm$ SEM, \textit{n} = 2 with a total of 30 cells analysed per treatment.* = \textit{p} < 0.05 vs control, \# = \textit{p} < 0.05 vs BafA1, \@ = \textit{p} < 0.05 vs 1 $\mu$M Ril and  \% = \textit{p} < 0.05 vs 10 $\mu$M Ril.}
  \label{fig:30_spd_ril_avs_b}
\end{figure} 
\end{landscape}

\section{Discussion: Autophagic flux assessment}
Although autophagy modulation using pharmacological agents has received considerable attention over the past years and although many autophagy modulating drugs are now known, it remains unclear to what extent different concentrations would change autophagic activity or autophagic flux, especially given the dynamic and cell specific nature of protein degradation and proteolysis. One of the challenges in this context is to accurately assess and dissect the pathway intermediates i.e. autophagosomal and autolysosomal pool sizes and their turnover in a highly sensitive manner, in order to better understand the underlying mechanisms that govern a particular autophagic activity. Although rilmenidine and spermidine are known to induce autophagy, the extent to which different concentrations of rilmenidine and spermidine  change  autophagic flux, is unclear. To this end, the autophagic flux profile of spermidine and rilmenidine was assessed by blocking autophagosome/lysosome fusion while monitoring the accumulation of pathway intermediates ( i.e. autophagosomes, autolysosomes, p62 protein levels and AVs), quantifying autophagic flux and transition time by using a combination of western blotting, z-stack based fluorescence imaging and TEM. Here we report that autophagy flux through standard analysis using either western blotting, TEM or FM analysis could result in the misinterpretation of data because not all methods are equally sensitive. We show that, in order to measure and discern autophagic flux accurately, one requires western blotting to assess the changes in LC3-II and p62 protein expression, TEM, to assess AVs and FM to assess p62 puncta, LC3 turnover and most importantly, to distinguish between autophagosome and autolysosome pool sizes. Secondly, we report that the size in vacuolar structures in response to the treatment assessment plays a major role in autophagic flux and deserves further investigation.

\subsection{The effect of spermidine and rilmenidine on cell viability}
Although various concentrations of rilmenidine and spermidine have been used to modulate autophagic flux in different cell types \citep{Perera2018,Wang2018}, a suitable, yet non-toxic concentration of rilmenidine and spermidine that induces autophagic flux in GT 1-7 cells had to be established. Therefore, a concentration-dependent response study of rilmenidine and spermidine modulation on cellular viability was performed using WST-1. The results showed no decrease in viability using spermidine (\Cref{fig:30_spd_ril_cell_viability_a}: \textbf{A}) and rilmenidine (\Cref{fig:30_spd_ril_cell_viability_a}: \textbf{B}), suggesting that both drugs exerted no toxicity at the concentrations tested. In agreement with our results assessing rilmenidine,  a recent \textit{in vitro} study by \citet{Perera2018} demonstrated that rilmenidine  at the  0.1, 1, \& 10 $\mu$M does not impair cell viability in mouse neuroblastoma x spinal cord (NSC-34) cells after 24 h treatment intervention. To our surprise, we found a small increase in reductive capacity which reached significance with 1 and 10 $\mu$M Spd and 10 $\mu$M Ril compared to the control group (\Cref{fig:30_spd_ril_cell_viability_a}: \textbf{A})  \& (\Cref{fig:30_spd_ril_cell_viability_a}: \textbf{B}). In addition, 10 $\mu$M Spd and 10 $\mu$M Ril resulted in the highest reductive capacity compared to 0.1 and 1 $\mu$M Spd and with 0.1 and 1 $\mu$M Ril, respectively, suggesting a potential mitochondrial or metabolic response. Although previous studies have assessed cellular viability in the context of rilmenidine and spermidine, none have reported an enhancement, suggesting a potential metabolic effect. In fact, previous studies  have reported cytotoxic effects of spermidine in different cells lines \citep{He1993,Poulin1995,Poulin1993}, supporting our findings of 100 $\mu$M Spd resulting in a significant reduction in cell viability, as shown in the supplementary material (\textbf{S1}) (\Cref{fig:30_S1_cell_viability}). A recent study reported a dose-dependent cytotoxicity effect after 48 h treatment with spermidine and spermine (0, 5,10, 20, 40, 60 , 80 and 100 $\mu$M) in drug-resistant cell lines, but not in their wild type cell lines \citep{Wang2018}. It has been suggested that the cytotoxic effects observed with spermidine or spermine might not be caused by these drugs directly but may be related to the metabolites of polyamines, since other polyamine such as putrescine and triethylenetetramine (TETA) did not show any significant cytotoxicity effect on all the tested cell lines tested \citep{Wang2018}. Taken together, our results allowed us to choose a low and high concentration of spermidine and rilmenidine for subsequent experiments.

\subsection{The effect of spermidine and rilmenidine on autophagic flux using WB}
To further identify a concentration that would have maximal autophagy inducing effects, western blot analysis was performed, to assess the abundance and turnover of LC3-II and p62 protein. Results show the highest abundance of LC3-II only with 10 $\mu$M Spd + BafA1 compared to the control group and 0.1 $\mu$M Spd + BafA1 (\Cref{fig:30_spd_ril_abun_a}: \textbf{A}\textit{i}, \textbf{A}\textit{iii}). However, trends towards an increase in the accumulation of LC3-II were observed when degradation was inhibited with BafA1 in all treatment groups compared to the respective BafA1 untreated groups, with an enhanced accumulation indicated at the highest concentration of spermidine. These results suggest the presence of a basal as well as increased autophagic activity. In addition, these results suggest a concentration-dependent effect. Basal autophagy flux is better revealed in the p62 western blot data, where the highest abundance of p62 was indicated in the BafA1 treated group compared to control group (\Cref{fig:30_spd_ril_abun_a}: \textbf{A}\textit{ii}, \textbf{A}\textit{iii}). Moreover, significant abundance of p62 was indicated in 10 $\mu$M Spd + BafA1 compared to the control group, although trends were indicated in the presence of BafA1 in all treatment groups compared to the respective BafA1 untreated groups (\Cref{fig:30_spd_ril_abun_a}: \textbf{A}\textit{ii}, \textbf{A}\textit{iii}). These results suggest the presence of autophagic flux and show that indeed both LC3-II and p62 protein levels need to be assessed in order to monitor changes in autophagic flux in a sensitive manner. Notably, only the high spermidine concentration induced heavily autophagic flux which can be clearly seen in the turnover of LC3-II (\Cref{fig:30_spd_ril_lc3ii}: \textbf{A}). In line with our findings, 10 $\mu$M Spd was found to increase LC3 lipidation and GPF-LC3 puncta above  basal levels yet not significantly in Human colorectal carcinoma (HCT) 116 cells when treated for 2 h \citep{Morselli2011}. Although no significance was observed, treatment with spermidine in this particular study induced autophagic flux 2 times above basal levels \citep{Morselli2011}. The none significant increase observed in this study might be due to the short duration of incubation with spermidine in these cells. In support of this, a recent study by \citet{Yue2017} reported a concentration-dependent increase in LC3-II after 4 h of incubation with spermidine at  0, 12.5, 25, 50 and 100 $\mu$M in MEF and Hela cells.  In another study, 20 nM of Spd which is 5 times lower than the lowest concentration used in our study (0.1 $\mu$M), was found to increase LC3-II and p62 levels above control levels in MEF cells after exposure of 8 h \citep{DuToit2018a}. Others reported a basal as well as induced autophagic flux with 100 $\mu$M Spd incubated for 4 h in U2OS cells  \citep{Pietrocola2015}, in Hela and MEF \citep{Yue2017} and in HCT cells \citep{Morselli2011}. We found that 100 $\mu$M Spd reduces cell viability in GT 1-7 cells \textbf{S1}, hence we did not use this concentration. Taken together these results suggest that autophagy induction by spermidine is dependent on the concentration, time of incubation and cell type used, which is critical for future clinical translation.

In contrast, treatment with rilmenidine had no effect on the abundance of LC3-II protein levels (\Cref{fig:30_spd_ril_abun_a}: \textbf{B}\textit{i}, \textbf{B}\textit{iii}) and on LC3-II turnover (\Cref{fig:30_spd_ril_lc3ii}: \textbf{B}), however, trends were observed in the presence of BafA1 in all treatment groups compared to the respective BafA1 untreated groups (\Cref{fig:30_spd_ril_abun_a}: \textbf{B}\textit{i}, \textbf{B}\textit{iii}), suggesting the presence of a basal flux.  In contrast to our findings, various studies reported a significant increase in LC3-II following treatment with multiple concentrations of rilmenidine. \citet{Rose2010} reported a significant increase in LC3-II using 1 $\mu$M Ril in primary cortical neurons after 8 h of incubation in the presence and absence of BafA1, while \citet{Perera2018} reported the same effect with 10 $\mu$M Ril for 18 and 24 h  in the absence of BafA1 in NSC-34 and H9 human embryonic stem cell-derived spinal motor neurons, respectively, both expressing human wild-type (WT) SOD1. In our study, response with rilmenidine was better revealed when assessing p62 protein levels where a significant accumulation of p62 was detected in 1 $\mu$M Ril, 1 $\mu$M Ril + BafA1, and 10 $\mu$M Ril + BafA1 compared to control cells, and in 1 $\mu$M Ril + BafA1 versus BafA1 and 1 $\mu$M Ril + BafA1 versus 0.1 $\mu$M Ril + BafA1 (\Cref{fig:30_spd_ril_abun_a}: \textbf{B}\textit{ii}, \textbf{B}\textit{iii}). This data suggests likely autophagy enhancing effect, which is best revealed through the impacted p62 clearance. In support of our findings, rilmenidine at 10 $\mu$M for 18 h exposure was found to significantly increase p62 accumulation in NSC-34 expressing WT SOD1 \citep{Perera2018}. In the same study, rilmenidine at similar concentration but different incubation period was found to significantly decrease p62 accumulation in H9 human embryonic stem cell-derived spinal motor neurons expressing WT SOD1  \citep{Perera2018}, suggesting possibly a cell type dependent effect. Since p62 and LC3-II protein levels can sometimes result in a distinct response, both their expression needs to be assessed.

\subsection{The effect of low and high concentrations of spermidine and rilmenidine on autophagy pathway intermediates, autophagosome flux and transition time using quantitative FM}
Next, in order to assess the effect of low and high concentrations of spermidine and rilmenidine on autophagic flux, the total number of autophagosomes and autolysosome, the autophagosome flux and transition time \citep{loos2014} were assessed using single cell analysis-based fluorescence microscopy. Our results when using spermidine revealed a significant increase in the accumulation of autophagosomes following treatment with BafA1, 1 $\mu$M Spd + BafA1 and 10 $\mu$M Spd + BafA1 compared to the control (\Cref{fig:30_spd_ril_poolsize1}: \textbf{A}). Secondly, our results revealed a significant increase in the accumulation of autophagosomes using 10 $\mu$M Spd + BafA1 compared to the BafA1 group, suggesting a concentration-dependent induced autophagic flux, in agreement with our western blotting data (\Cref{fig:30_spd_ril_abun_a}: \textbf{A}\textit{i}, \textbf{A}\textit{iii}).  Thirdly, we reveal a significant accumulation of autophagosomes following treatment with BafA1, 1 $\mu$M Spd + BafA1 and 10 $\mu$M Spd + BafA1 compared to the control, 1 $\mu$M Spd and in 10 $\mu$M Spd respectively, suggesting the presence of basal and enhanced autophagic flux in all three groups. These results are in contrast with the data obtained using western blot analysis where a significant change in autophagic flux was only measurable with the highest concentration of spermidine, suggesting that autophagosome pool size assessment using FM is indeed more sensitive. In line with our findings, other studies have shown an increase in autophagosome accumulation following treatment with 100 $\mu$M spermidine after 4 h exposure in the presence and absence of BafA1 in U2OS cells expressing GFP-LC3  \citep{Pietrocola2015} and MEF cells expressing RFP-LC3 \citep{Yue2017}. Similarly, treatment with rilmenidine revealed a significant increase in the accumulation of autophagosomes following treatment with BafA1, 1 $\mu$M Ril + BafA1 and 10 $\mu$M Ril + BafA1 compared to the control, 1 $\mu$M Ril and in 10 $\mu$M Ril respectively (\Cref{fig:30_spd_ril_poolsize1}: \textbf{B}), suggesting the presence of basal autophagic flux in all three groups. These results are in contrast with the data obtained using western blot analysis (\Cref{fig:30_spd_ril_abun_a}: \textbf{B}\textit{i}, \textbf{B}\textit{iii}) where only trends were observed but no significance was revealed, suggesting that autophagosome pool size assessment with FM is likely more sensitive.  Importantly, we also observed a significant decrease in the autophagosome pool size when using the lowest and high concentration of rilmenidine in the presence and absence of BafA1, compared to cells treated with BafA1, in line with our data obtained using western blot analysis (\Cref{fig:30_spd_ril_lc3ii}: \textbf{B}), suggesting that rilmenidine stimulates autophagy resulting in autophagosomes being shunted into the lysosomal pool, forming autolysosomes (\Cref{fig:30_spd_ril_poolsize2}: \textbf{B}). 

Indeed, we observed a robust and significant accumulation in the number of autolysosomes when using low and high concentrations of rilmenidine in the absence and presence of BafA1 compared to the control and BafA1 treated group (\Cref{fig:30_spd_ril_poolsize2}: \textbf{B}), suggesting that autophagosomes were likely shunted towards autolysosomes. In support of these findings,  rilmenidine (10 $\mu$M Ril) was shown to significantly increase the accumulation of mature autolysosomes compared to untreated cells \citep{Perera2018}. It is reasonable to propose that the enhanced autophagic flux upon rilmenidine treatment  also enhances the conversion of autophagosomes to autolysosomes. Autolysosomes accumulated primarily when using a lower concentration of rilmenidine compared with the lower concentration of rilmenidine plus BafA1, indicating a possibly recycling of autolysosomes back to lysosomes. To our surprise, the lower concentration of rilmenidine induced the accumulation of autolysomes compared to the higher concentration, suggesting that low rilmendine concentration might enhance autophagy better than the higher concentration, leading to the rapid conversion of autophagosomes to autolysosomes (\Cref{fig:30_spd_ril_poolsize2}: \textbf{B}). In contrast, following spermidine treatment, autolysosomes significantly accumulated with both low and high concentrations compared to the control group (\Cref{fig:30_spd_ril_poolsize2}: \textbf{A}), with no significant differences observed in the BafA1 treated group compared to all other groups. Other studies have found an increase in the autolysosomal pool following autophagy induction with pharmacological agents including spermidine \citep{DuToit2018a}. All together, these results suggest that both autophagosome and autolysosome pool size should be assessed to reveal the true autophagic flux and associated vessicle dynamics, otherwise it may have been missed that both low and high concentrations of rilmenidine induces flux by assessing autolysosomes, as this was not revealed with the assessment of autophagosomes only. 

Lastly, our results indicate a concentration-dependent enhanced autophagosome flux following spermidine treatment compared to the basal flux present in control cells (\Cref{fig:30_spd_ril_flux}: Left). Treatment with rilmenidine resulted in a decline in autophagosome flux below basal flux which remained decreased but at the same time there was a major shift of the autophagosome pool to the autolysosome pool \Cref{fig:30_spd_ril_flux} (Right). Further analysis revealed that cells treated with low concentration of spermidine required 0.64 h, while while cells exposed to a higher concentration of spermidine required 1.21 h to clear their autophagosome pool compared to 2 h needed by the control cells to turn over their pool (\Cref{tab:30_flux}). On the other end, rilmenidine treated cells required 0.05 h and 0.040 h when exposed to a low and high concentration respectively to clear their entire autophagosome pool compared to the control (2 h). Overall, our data reveal novel and critical insights into the autophagic flux profile of spermidine and rilmenidine under physiological conditions. Here, we were able to define the basal flux for GT 1-7 cells and were able to define the transition time to turn over the basal pool as 2 h. Secondly, we were able to reveal differences in spermidine concentrations doses as it increased flux in a concentration-dependent manner, unlike rilmenidine, which reduced flux.

\subsection{The effect of low and high concentrations spermidine and rilmenidine on p62 puncta using FM}
Next, to assess whether a low and high concentration of spermidine and rilmenidine has an effect on the autophagy substrate p62, a single cell analysis-based fluorescence microscopy approach was performed using GT 1-7 cells, immuno-stained for p62 in the absence and presence of saturating concentration of BafA1. The results revealed a significant increase in p62 puncta following treatment with BafA1, 1 $\mu$M Spd + BafA1 and 10 $\mu$M Spd + BafA1 compared to the control cells (\Cref{fig:30_fluorescent_graph}: \textbf{A} and \Cref{fig:30_fluorescent}: Left) and following treatment  with BafA1, 1 $\mu$M Ril + BafA1  and 10 $\mu$M Ril + BafA1 compared to the control cells (\Cref{fig:30_fluorescent_graph}: \textbf{B} and \Cref{fig:30_fluorescent}: Right). Moreover, a significant increase in p62 puncta was revealed in 1 $\mu$M Spd + BafA1 versus 1 $\mu$M Spd, 10 $\mu$M Spd + BafA1 versus 10 $\mu$M Spd, 1 $\mu$M Ril + BafA1 versus 1 $\mu$M Ril and in 10 $\mu$M Ril + BafA1 versus 10 $\mu$M Ril. These results suggest that clearance of p62 cargo takes place at basal and autophagy induced conditions. Overall, these results imply that p62 puncta counts are more sensitive compared to assessment of p62 protein expression measured using western blotting as a flux can be revealed in each group. Interestingly, we observed a cargo increase with p62 puncta in the rilmenidine treatment but not when assessing the LC3 response (i.e. autophagome pool). These results suggest that the p62 puncta observed may in fact be cargo that is accumulating outside of LC3 positive structures, hence not associated with LC3 and thus unable to be degraded. This might also be explained by the trends observed in the LC3 western blots that did not reach significance  (\Cref{fig:30_spd_ril_abun_a}: \textbf{B}\textit{i}, \textbf{B}\textit{iii}) and the major changes observed in the autolysosome pool (\Cref{fig:30_spd_ril_poolsize2}: \textbf{B}), potentially responsible for this particular phenotype. Hence, the association of cargo (p62) and machinery (LC3) maybe a critical aspect that requires distinction.
  
\subsection{The effect of low and high concentrations spermidine and rilmenidine on autophagic vacuoles using TEM}
TEM was the first technique used to describe and detect autophagy in the 1950s \citep{Deter1967}. It remains a most powerful technique for assessing the autophagic pathway intermediates due to its ability to reveal ultrastructural information of autophagosomes, lysosomes and autolysosomes at highest resolution (0.1 - 10 nm), allowing for quantitative analysis of these structures to be performed \citep{klionsky2016}. Despite the high resolution achieved, it remains often challenging to discern between autophagosomes and autolysosomes based on morphology only, thus these structures are collectively termed AVs \citep{Eskelinen2008,klionsky2016} and are usually assessed in the presence and absence of lysosomal inhibitors such as BafA1 in order to distinguish changes in autophagic flux in response to treatment \citep{Eskelinen2011,klionsky2016}.

Thus, in order to assess whether a low and high concentration of spermidine and rilmenidine has an effect on AVs, surface area and the number of AVs was analysed in the presence and absence of BafA1 following image acquisition. Our results indicate a decrease in the size of AVs after autophagy induction using low and high concentrations of spermidine in the presence of BafA1 and in the high spermidine group compared to BafA1 treated cells (\Cref{fig:30_spd_ril_avs_a}: \textbf{A} \& \textbf{B}\textit{i}). To our knowledge, this is the first study to report a decrease in AV size upon spermidine treatment in the presence of BafA1. Moreover, our results reveal a rapid accumulation of AVs following inhibition of degradation in the BafA1 group, 1 $\mu$M Spd + BafA1 and 10 $\mu$M Spd + BafA1 compared to the control group, 1 $\mu$M Spd and 10 $\mu$M Spd, respectively  (\Cref{fig:30_spd_ril_avs_a}: \textbf{A} \& \textbf{B}\textit{ii}), suggesting indeed the presence of basal autophagic flux.  Treatment with rilmenidine revealed, to our surprise, a concentration-dependent increase in the size of AVs following autophagy induction (\Cref{fig:30_spd_ril_avs_b}: \textbf{A} \& \textbf{B}\textit{i}). We further detected an increase in the size of AVs above basal levels following autophagy induction with rilmenidine in the presence of BafA1 (\Cref{fig:30_spd_ril_avs_b}: \textbf{A} \& \textbf{B}\textit{i}). These results suggest that the cargo degradation is subsequently likely impacted, and deserves further investigation. In addition, our results reveal a rapid accumulation of AVs following inhibition of degradation in the BafA1 group, 1 $\mu$M Ril + BafA1 and 10 $\mu$M Ril + BafA1 compared to the control group, 1 $\mu$M Ril and 10 $\mu$M Ril, respectively (\Cref{fig:30_spd_ril_avs_b}: \textbf{A} \& \textbf{B}\textit{ii}), supporting the presence of a basal flux. Lastly, a significant accumulation of AVs was detected with a high rilmenidine concentration plus BafA1 compared to BafA1 alone, suggesting indeed a concentration-dependent effect.  To our knowledge, our results provide for the first time evidence for a concentration-dependent effect in the size and number of AVs following autophagy induction.  In line with our findings, others have reported an increase in the size and number of AVs following autophagy induction using EM, albeit no concentration range was tested. \citet{Mizushima2004a} reported an increase in the number and size of AVs following autophagy induction in cells and in tissues derived from mice. In another \textit{in vitro} study, an increase in the number of AVs was reported  following autophagy induction \citep{Lum2005}. Similar findings have been reported by others \textit{in vivo} \citep{Alirezaei2010,Ericsson1969,Mizushima2004a}. Since we observed no changes with BafA1 in terms of size but a major increase in AV accumulation which remained increased upon BafA1 treatment, above basal control levels, these results suggest that the accumulated AVs might be primarily autolysosomes, as we observed substantial accumulation of autolysosomes (\Cref{fig:30_spd_ril_poolsize2}: \textbf{B}) but not autophagosomes (\Cref{fig:30_spd_ril_poolsize1}: \textbf{B}) with rilmenidine. This requires further investigation. One way to discern between autolysosomes and autophagosomesdo is through the use of CLEM which will be discussed in chapter 6.

In summary, our results provide evidence for a concentration-dependent effect of spermidine and to a certain extent with rilmenidine on autophagic flux under physiological conditions. Spermidine is a suitable autophagy inducer in GT 1-7 cells. It is possible that the effect of rilmenidine observed by others is due to the moving of cargo to autolysosomes, but the cargo is not being degraded. We show that the extent of change in flux detected is method dependent, with each having its own sensitivity, and that autophagy flux through standard analysis using either of these techniques in isolation could result in the misinterpretation of data. We show that,  in order to measure autophagic flux accurately, it is required to use a combination of western blot analysis, TEM, FM and most importantly to distinguish between the autophagosome and autolysosome pool using FM. We report that the size in vacuolar structures play a major role in the treatment response, which deserves further investigation. In conclusion, western blotting appears rather insensitive and only provides rough indication while vesicle analysis allows to better describe discrepancies in pool sizes. The turnover (net flux) analysis allows to identify differences better, at least in the here investigated conditions. The analysis of AVs using EM allows to observe different sizes in the structures and autolysosomes, all of which allows to gain insights into cargo clearance and pool size dynamics. Therefore, it is advisable to choose a number of sensitive techniques if one wishes to fully understand the effect of drugs and their concentrations on autophagy under normal and disease conditions.

\section*{Supplementary material}\label{sec:supp_mat}
\begin{figure}[!htbp]
  \center
    \includegraphics[width=0.495\linewidth]{figures/chapter30/30_S1_cell_viability}
  	\caption[Effect of spermidine on cell viability]{\textbf{Effect of spermidine on cell viability.} GT1-7 cells were treated with 0.1, 1, 10 \& 100 $\mu$M of Spd for 8 hr. Subsequently, reductive capacity as a measure of cell viability was assessed using WST1 assay. All results are presented as a percentage of the control (mean $\pm$ SEM), \textit{n}=3, with 6 replicates per group. * = \textit{p} < 0.05 vs control, \# = \textit{p} < 0.05 vs 0.1 $\mu$M Spd, \$ = \textit{p} < 0.05 vs 1 $\mu$M Spd and \& = \textit{p} < 0.05 vs 10 $\mu$M Spd.}
  \label{fig:30_S1_cell_viability}
\end{figure}