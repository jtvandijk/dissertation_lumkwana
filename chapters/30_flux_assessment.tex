\chapter{Autophagic flux assessment}
\section{Introduction}

Autophagy is a physiological cellular process that is characterized by a fine balance between the rate of AV formation and the rate of their clearance, with autophagic flux referring  to the rate of flow of material through the autophagic pathway \citep{loos2014,klionsky2016}. Many tools and techniques are available to assess autophagy, including western blotting (WB), fluorescence microscopy (FM), Transmission Electron Microscopy (TEM), flow cytometry \citep{klionsky2016} and Correlative Light and Electron Microscopy (CLEM) with each technique having particular advantages, but also inherent disadvantages.

Autophagy plays a significant role in health where primarily decreased activation of autophagy has been shown to have detrimental effects. Autophagy can be induced via mTOR dependent and mTOR independent pathways \citep{sarkar2013}. Although we have progressed substantially in the understanding of the autophagy machinery and its regulation, its dysfunction in pathology as well as its dynamic changes however in the disease progression remains often largely unclear. Furthermore, although a substantial progress has been made in unravelling the role and regulation of autophagy and its modulation in neurodegenerative disease using pharmacological agents, major differences exist between experimental design, drug concentrations used, duration of treatment intervention as well as the model system used to assess autophagic flux \citep{lumkwana2017}. This contributes to inability to modulate autophagy with high precision. Specifically, spermidine and rilmenidine have been shown to induce autophagy, but the impact of concentration ranges on autophagic flux and on subsequent protein clearance and neuronal toxicity is unclear. Importantly, whether a concentration dependent effect on autophagy activity exists achieving defined, but distinguishable heightened autophagy flux remains unclear. Particularly in terms of fine-tuning autophagy activity with a given flux deviation in neurodegeneration, this aspect is of critical importance. In this chapter we firstly aimed to establish a suitable concentration of spermidine (Spd) and rilmenidine (Ril) using GT 1-7 neuronal cells that induces autophagic flux without causing cellular toxicity, and secondly, to characterize the autophagic flux profile of a low and high concentration of both drugs established in terms of (1) the autophagic machinery and pathway intermediates (autophagosomes, autolysosome),(2) the autophagosome flux, (3) p62 puncta, and (4) autophagic vacuoles (AVs). For this purpose, GT1-7 neuronal cells were treated with three concentrations of Spd and Ril in the presence and absence of BafA1. Subsequently, cellular viability and WB for LC3 II and p62 was performed. Next, to assess the effect of a low and high concentration of Spd and Ril, GT 1-7 cells were treated with 1 \& {10.0 $\mu$M respectively in the presence and absence of BafA1 and thereafter extensively analysed using FM and TEM linked to quantitative morphometric analysis. In addition, GT1-7 cells were transfected with mRFP-GFP-LC3 \citep{yoshii2017} and subsequently treated with 1 \& 10 $\mu$M of Spd and Ril in the presence and absence of BafA1 and analysed with FM.
 
\section{Effect of spermidine and rilmenidine on cell viability}
In order to choose a suitable, yet non-toxic concentration of autophagy modulators, WST-1 was performed. For the concentrations assessed, no decrease in viability was observed. In fact, a minor metabolic response associated with enhanced reductive capacity was observed. Reductive capacity was significantly increased using 1 $\mu$M Spd $104.70\% \pm 1.64\%, p < 0.05$ and 10 $\mu$M Spd ($109.20 \pm 1.38\%, p < 0.05$) compared to the control ($100.00\% \pm 1.50\%$), with no significant differences observed using 0.1 $\mu$M Spd ($104.20\% \pm 1.46\%$). More importantly, a significant increase in reductive capacity was observed with 10 $\mu$M Spd ($p < 0.01$) compared to 0.1 $\mu$M Spd and 1 $\mu$M Spd.

Following rilmenidine treatment, a significant increase in reductive capacity was observed in the 10 $\mu$M Ril treated group ($105.70 \pm 2.26\%, p < 0.05$) compared to the control ($100.40 \pm 1.16\%$), with no significant differences observed in the 0.1 $\mu$M Ril ($98.76\% \pm 1.49\%$) and 1 $\mu$M Ril ($100.50 \pm 2.12\%$). Importantly, a significant increase in cellular viability was observed in the 10 $\mu$M Ril group compared to 0.1 $\mu$M and 1 $\mu$M Ril. This data suggests no toxicity and allowed to choose a low and high concentration of spermidine and rilmenidine for subsequent experiments.

\section{Effect of spermidine and rilmenidine on autophagic flux using western blotting}
To further identify the concentration that would have maximal autophagy inducing effects, protein extraction was performed following 8 hr treatment with a range of Spd and Ril concentrations. Western blot analysis was performed to assess the protein expression of LC3-II and p62. LC3 II is formed through conjugation of LC3-I to PE and is localized on the inner and outer surface of the autophagosome membrane, where it is either degraded upon fusion of autophagosomes with lysosomes or removed through deconjugation and recycled \citep{kabeya2000}. Importantly, the abundance of LC3 II correlates with the number of autophagosomes and hence is used as a key marker to indicate autophagy or the size of the autophagosome pool \citep{loos2014} p62 serves as a scaffolding for ubiquitinated proteins \citep{sahani2014}. It binds directly to LC3 and is in turn selectively degraded through autophagy; thus it is widely used as an additional indicator for autophagy activity \citep{pankiv2007}.

\subsection{Abundance of LC3 II and p62}
A significant increase in the abundance of LC3 II was observed in the 10 $\mu$M Spd + BafA1 ($1.67 \pm 0.28$) compared to the control group ($1.00 \pm 0.00$). Moreover, a significant increase in the abundance of LC3 II was observed in the 10 $\mu$M Spd + BafA1 ($p < 0.05$) compared to 0.1 $\mu$M Spd + BafA1. No significant differences were observed between 0.1 $\mu$M Spd, 1 $\mu$M Spd and 10 $\mu$M Spd. A significant increase was observed in the 10 $\mu$M Spd + BafA1 compared to10 $\mu$M Spd, with no significant differences seen between 0.1 $\mu$M Spd and 0.1 $\mu$M Spd + BafA1 and between 1 $\mu$M Spd versus 1 $\mu$M Spd + BafA1, however it is important to mention that in both the 0.1 $\mu$M Spd  and 1 $\mu$M Spd, inhibition of degradation with BafA1 resulted in an increased LC3 II accumulation compared to the Baf untreated groups, suggesting the presence of a basal as well as increased autophagic activity.

The abundance of p62 was significantly increased in the BafA1 treated group ($1.72 \pm 0.26, p < 0.05$) and  10 $\mu$M Spd + BafA1 ($1.72 \pm 0.17, p< 0.05$) compared to Con ($1.00 \pm 0.00$). Moreover, in comparison to the Bafa1 treated group, a significant decrease in p62 abundance was observed in the 1 $\mu$M Spd, with no significant differences seen in all other groups. A higher abundance was indicated in the presence of BafA1 in all treatment groups compared to the respective BafA1 untreated groups.

With regards to rilmenidine treatment, no significant differences in the abundance of LC3 II were observed, however an increasing trend in the amount of LC3 II in the BafA1 treated groups compared to the BafA1 untreated groups was noted [Con ($1.00 \pm 0.00$), BafA1 ($1.45 \pm 0.49$), 0.1 $\mu$M ($1.16 \pm 0.09$), 0.1 $\mu$M Ril + BafA1 ($1.46 \pm 0.36$), 1 $\mu$M Ril ($0.94 \pm 0.14$), 1 $\mu$M Ril + BafA1 ($1.22 \pm 0.42$), 10 $\mu$M Ril ($1.02 \pm 0.27$) and 10 $\mu$M Ril + BafA1 ($1.17 \pm 0.63$)].

A significant increase in p62 protein expression was observed in the 1 $\mu$M Ril ($1.68 \pm 0.19, p < 0.05$), 1 $\mu$M Ril + BafA1 ($2.21 \pm 0.31, p < 0.05$) and 10 $\mu$M Ril + BafA1 ($1.81 \pm 0.08, p < 0.05$) compared to the Con group ($1.00 \pm 0.00$). Moreover, a significant increase in p62 protein levels was revealed in the 1 $\mu$M Ril + BafA1 compared to BafA1 treated group. Furthermore, a significant increase in p62 abundance was observed in the1 $\mu$M Ril + BafA1 ($p < 0.05$) compared to 0.1 $\mu$M Ril + BafA1, with no significant differences observed when compared to 10 $\mu$M Ril + BafA1. Finally, overall p62 protein abundance accumulated in the BafA1 treated groups compared to the respective BafA1 untreated groups, however not reaching significance. No response was observed in 0.1 $\mu$M Ril + BafA1 compared to its BafA1 untreated group. This data suggests an autophagy enhancing effect, which is best revealed through the impacted p62 clearance.

\subsection{LC3 II and p62 protein turnover}










