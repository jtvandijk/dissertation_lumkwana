\chapter{The role of spermidine and rilmenidine in a paraquat-induced neuronal toxicity model}
\label{sec:chapter4}
\section{Introduction}
Paraquat (PQ) is a pesticide that is used extensively in experimental models of AD and PD. Although the mechanisms of action of PQ in AD are not fully elucidated, it is known that PQ induces oxidative stress and mitochondrial damage \citep{Chen2012} and ultimately neuronal cell death. Both oxidative stress and mitochondrial damage have been implicated in the pathogenesis of AD \citep{Chen2012,Lin2006}. Indeed, high levels of lipid peroxidation, a marker for oxidative stress have been reported in the brains of patients with AD \citep{Wang2014,Zhao2013}. mTOR dependent and mTOR independent autophagy modulators have been shown consistently to counteract age-associated pathology in various organisms \citep{lumkwana2017}. For example, lifestyle interventions such as caloric restriction (CR), defined as a reduction in food intake without malnutrition,  and physical exercise have have been shown to extend life span and overall health through autophagy modulation in primates and no primates \citep{Law2018,Frederiksen2017,Liu2019,Lin2017}. Spermidine, a caloric restriction mimetic (CRM) which mimic the pharmacological beneficial effects of CR but without chronic reduction of calorie intake has been shown to extend life span in an autophagy dependent manner and ameliorate age-associated pathologies in various organisms \citep{Buttner2014,Eisenberg2016a,Gupta2016,Morselli2011,Morselli2009,Sigrist2014,Yue2017,Zhang2017} through reduction of anti-inflammatory processes, increased mitochondrial function as well as increased proteostasis \citep{Madeo2018}. Moreover, rilmenidine has been shown to improve age-associated pathologies by promoting the clearance of aggregate-prone proteins in neuronal cells and mouse models of HD and ALS \citep{Perera2018,Rose2010}. Although, spermidine and rilmenidine both enhance autophagy, the exact relationship between autophagy activity, the extent of protein clearance, oxidative stress, microtubule organisation and cell death onset remains unclear. Moreover, the impact of concentration differences in autophagy modulation and on subsequent protein clearance and neuronal protection in a PQ induced neuronal model is unclear. Therefore, this part of the study aimed to assess the effect of spermidine and rilmenidine in terms of (i) oxidative stress, (ii) cell death, (iii) autophagy, (iv) microtubule stability and (v) the degree of protection, i.e. whether a higher autophagic activity will translate in a higher degree of protection. For this, GT1-7 cells were pre-treated with a low and high concentration (1 \& 10 $\mu$M) spermidine and rilmenidine for 8 h, followed by exposure to PQ for 6 h.

\section{The effect of PQ on cell viability}
In order to choose a suitable concentration of PQ that causes toxicity in GT 1-7, a concentration dependent toxicity study was performed using WST-1. Different concentration of PQ [(500 $\mu$M, 1, 2, 3, 5 and 10 mM) were assessed. Cell viability in response to PQ treatment significantly decreased progressively with an increase in concentration 500 $\mu$M (93.58 \% $\pm$ 1.23 \%, \textit{p} < 0.05), 1 mM (84.51 \% $\pm$ 1.48 \%, \textit{p} < 0.05), 2 mM (62.70 \% $\pm$ 3.27 \%, \textit{p} < 0.05), 3 mM (48.48 \% $\pm$ 3.58 \%, \textit{p} < 0.05), 5 mM (34.74 \% $\pm$ 3.11 \%, \textit{p} < 0.05) and 10 mM (23.41 \% $\pm$ 0.53 \%, \textit{p} < 0.05)] compared to the control group (100 \% $\pm$ 2.57 \%) (\Cref{fig:40_pq_cell_viability_a}). PQ at 3 mM reduced cell viability by approximately 50 \% and was selected for subsequent experiments.

\begin{figure}[!htbp]
\center
  \includegraphics[width=0.495\linewidth]{figures/chapter40/40_pq_cell_viability_a}
  \caption[Effect of PQ on cell viability]{\textbf{Effect of PQ on cell viability.} GT1-7 cells were treated with increasing concentrations of PQ for 6 h. Subsequently, a decrease in cell viability was assessed using WST-1 assay. All results are presented as a percentage of the control (mean $\pm$ SEM), \textit{n} = 3, with 6 replicates per group. * = \textit{p} < 0.05 vs control.}
  \label{fig:40_pq_cell_viability_a}
\end{figure} 

\section{The effect of spermidine and rilmenidine on neuronal toxicity induced by PQ}
In order to assess whether Spd and Ril protect against PQ induced neuronal toxicity, GT1-7 cells were pre-treated with 1 \& 10 $\mu$M of Spd and Ril for 8 h followed by exposure to 3 mM PQ for 6 h. Cells were assessed for cell viability, ROS damage, cell death, autophagy and microtubule stability.

\subsection{The effect of low and high concentrations spermidine and rilmenidine on cellular viability upon PQ-induced neuronal toxicity} 
Cell viability significantly decreased in PQ treated group (72.67 \% $\pm$ 3.42 \%, \textit{p} < 0.05) compared to the control group (100.00 \% $\pm$ 1.13 \%) (\Cref{fig:40_pq_spd_ril_cell_viability_a}: \textbf{A}). In addition, combination of spermidine with PQ at both concentrations; 1 $\mu$M Spd + PQ (89.25 \% ± 4.40 \%, \textit{p} < 0.05) and 10 $\mu$M Spd + PQ (93.04 \% $\pm$ 5.84 \%, \textit{p} < 0.05) significantly improved cell viability compared to the PQ treated group. When rilmenidine was used, a significant decreased in cell viability was observed in the PQ treated group (70.97 \% $\pm$ 4.00 \%, \textit{p} < 0.05), 1 $\mu$M Ril + PQ (74.43 \% $\pm$ 4.08 \%) and 10 $\mu$M Ril + PQ (78.41 \% $\pm$ 3.98 \%) compared to the control (100.00 \% $\pm$ 3.27 \%) (\Cref{fig:40_pq_spd_ril_cell_viability_a}: \textbf{B}). Furthermore, no significant differences were observed in cellular viability in the 1 $\mu$M Ril + PQ and 10 $\mu$M Ril + PQ compared to the PQ treated group.

\begin{figure}[!htbp]
  \center
  \begin{subfigure}[b]{0.495\linewidth}
    \includegraphics[width=\linewidth]{figures/chapter40/40_pq_spd_ril_cell_viability_a}
    \caption{Spermidine}
  \end{subfigure}
  \begin{subfigure}[b]{0.495\linewidth}
    \includegraphics[width=\linewidth]{figures/chapter40/40_pq_spd_ril_cell_viability_b}
    \caption{Rilmenidine}
  \end{subfigure}
  \caption[Effect of spermidine and rilmenidine on cellular viability upon PQ-induced neuronal toxicity]{\textbf{Effect of spermidine and rilmenidine on cellular viability upon PQ-induced neuronal toxicity.} GT1-7 cells were pre-treated with 1 \& 10 $\mu$M of Spd (\textbf{A}) and Ril (\textbf{B}) followed by exposure to 3 mM PQ for 6 h and assessed for cell viability using WST-1 assay. All results are presented as a percentage of the control (mean $\pm$ SEM), textit{n}=3, with 6 replicates per group. * = \textit{p} < 0.05 vs control, \# = \textit{p} < 0.05 vs PQ.}
  \label{fig:40_pq_spd_ril_cell_viability_a}
\end{figure}

\subsection{The effect of low and high concentrations spermidine and rilmenidine on reactive oxygen species (ROS) generation upon PQ-induced neuronal toxicity}
Following treatments, cells were stained for mitochondrial superoxide (O\textsubscript{2}\textsuperscript{-}) using mitoSOX and for cytosolic hydrogen peroxide (H\textsubscript{2}O\textsubscript{2}) using DCF. Mean fluorescence intensities were measured using flow cytometry. Mitochondrial O\textsubscript{2}\textsuperscript{-} was significantly increased in PQ treated group (3.92 $\pm$ 0.40, \textit{p} < 0.05), 1 $\mu$M Spd + PQ (2.11 $\pm$ 0.19, \textit{p} < 0.05) and 10 $\mu$M Spd + PQ (3.80 $\pm$ 0.52, \textit{p} < 0.05) compared to the control (1.00 $\pm$ 0.02), with no significant differences observed in the 1 $\mu$M Spd (1.05 $\pm$ 0.10) and 10 $\mu$M Spd (1.07 $\pm$ 0.03) compared to the control (\Cref{fig:40_pq_spd_ril_ros_a}: \textbf{A}\textit{i}). A significant decrease in mitochondrial O\textsubscript{2}\textsuperscript{-} was observed in the 1$\mu$M Spd, 1 $\mu$M Spd + PQ and 10 $\mu$M Spd compared to the PQ treated group. Similarly, a significant decrease was observed in 1 $\mu$M Spd compared to 1 $\mu$M Spd + PQ, 10 $\mu$M Spd compared to 10 $\mu$M Spd + PQ and more importantly in 1 $\mu$M Spd + PQ compared to 10 $\mu$M Spd + PQ (\Cref{fig:40_pq_spd_ril_ros_a}: \textbf{A}\textit{i}). 

Cytosolic H\textsubscript{2}O\textsubscript{2} was significantly increased in the PQ treated group (1.39 $\pm$ 0.10, \textit{p} < 0.05) and 10 $\mu$M Spd + PQ (1.40 $\pm$ 0.09, p < 0.05) compared to the control (1.00 $\pm$ 0.06), with no significant differences observed in the 1 $\mu$M Spd (1.00 $\pm$ 0.06), 1 $\mu$M Spd + PQ (1.18 $\pm$ 0.06) and 10 $\mu$M Spd (1.04 $\pm$ 0.06) compared to the control group (\Cref{fig:40_pq_spd_ril_ros_a}: \textbf{A}\textit{ii}). A significant decrease in cytosolic H\textsubscript{2}O\textsubscript{2} was observed in the 1 $\mu$M Spd, 1 $\mu$M Spd + PQ and 10 $\mu$M Spd compared to the PQ treated group. Similarly, a significant decrease was observed in 10 $\mu$M Spd compared to 10 $\mu$M Spd + PQ.

With regards to rilmenidine treatment, a significant increase in mitochondrial O\textsubscript{2}\textsuperscript{-} was observed in the PQ treated group (1.41 $\pm$ 0.14, \textit{p} < 0.05), in 1 $\mu$M Ril + PQ (1.29 $\pm$ 0.07, \textit{p} < 0.05) and 10 $\mu$M Ril + PQ (1.47 $\pm$ 0.11, \textit{p} < 0.05) compared to the to the control group (1.00 $\pm$ 0.10), with no significant differences detected with 1 $\mu$M Ril (0.89 $\pm$ 0.08) and 10 $\mu$M Ril (0.98 $\pm$ 0.03) (\Cref{fig:40_pq_spd_ril_ros_a}: \textbf{B}\textit{i}). A significant decrease in mitochondrial O\textsubscript{2}\textsuperscript{-} was observed in the 1 $\mu$M Ril and 10 $\mu$M Ril compared to the PQ treated group. Similarly, a significant decrease was observed in 1 $\mu$M Ril compared to 1 $\mu$M Ril + PQ, and in 10 $\mu$M Ril compared to 10 $\mu$M Ril + PQ (\Cref{fig:40_pq_spd_ril_ros_a}: \textbf{B}\textit{i}). 

A significant increase was observed in cytosolic H\textsubscript{2}O\textsubscript{2} in the PQ treated group (1.16 $\pm$ 0.07) compared to the control group (1.00 $\pm$ 0.03), with no significant differences observed in 1 $\mu$M Ril (1.07 $\pm$ 0.04), 1 $\mu$M Ril + PQ (1.06 $\pm$ 0.04), 10 $\mu$M Ril (1.09 $\pm$ 0.04) and 10 $\mu$M Ril + PQ (1.07 $\pm$ 0.03) (\Cref{fig:40_pq_spd_ril_ros_a}: \textbf{B}\textit{ii}).

\begin{figure}[!htbp]
  \center
  \begin{subfigure}[b]{0.495\linewidth}
    \includegraphics[width=\linewidth]{figures/chapter40/40_pq_spd_ril_ros_a}
    \caption{Spermidine}
  \end{subfigure}
  \begin{subfigure}[b]{0.495\linewidth}
    \includegraphics[width=\linewidth]{figures/chapter40/40_pq_spd_ril_ros_b}
    \caption{Rilmenidine}
  \end{subfigure}
  \caption[Effect of spermidine and rilmenidine on mitochondrial ROS and cytosolic ROS in PQ induced neuronal toxicity]{\textbf{Effect of spermidine and rilmenidine on mitochondrial ROS and cytosolic ROS in PQ induced neuronal toxicity.} Quantitative analysis of MitoSox (\textbf{top}) and DCF (\textbf{bottom}) following treatment with spermidine (\textbf{A}) and rilmenidine (\textbf{B}). Data are presented as mean $\pm$ SEM, \textit{n} = 3, with 2 - 3 technical repeats per group. * = \textit{p} < 0.05 vs control group, \# = \textit{p} < 0.05 vs PQ, \@ = \textit{p} < 0.05 vs 1 $\mu$M Spd/Ril, \% = \textit{p} < 0.05 vs 10 $\mu$M Spd/Ril, and \$ = \textit{p} < 0.05 vs 1 $\mu$M Spd + PQ.}
  \label{fig:40_pq_spd_ril_ros_a}
\end{figure} 

\subsection{The effect of low and high concentrations of spermidine and rilmenidine on cell death onset upon PQ-induced neuronal toxicity} 
Cell death was monitored following treatment using propidium iodide (PI), a membrane-impermeable dye which intercalates with DNA. PI accumulates in cells with compromised membrane integrity, a hallmark for necrotic cell death. Mean fluorescent intensities were measured using flow cytometry. Percentage of PI + cells was significantly increased in the PQ treated group (32.92 \% $\pm$ 2.50 \%, \textit{p} < 0.05), 1 $\mu$M Spd + PQ (24.11 \% $\pm$ 2.74 \%, \textit{p} < 0.05) and 10 $\mu$M Spd + PQ (29.44 \% $\pm$ 3.04 \%, \textit{p} < 0.05) compared to the control (3.91 \% $\pm$ 0.64 \%), with no significant differences observed in the 1 $\mu$M Spd (4.07 \% $\pm$ 0.27 \%) and 10 $\mu$M Spd (5.53 \% $\pm$ 0.64 \%) (\Cref{fig:40_pq_spd_ril_pi_a}: \textbf{A}). Importantly, a significant decrease in the number of PI + cells was observed in the 1 $\mu$M Spd + PQ, 1 $\mu$M Spd and 10 $\mu$M Spd group compared to the PQ treated group. Furthermore, a significant decrease was observed in 1 $\mu$M Spd compared to 1 $\mu$M Spd + PQ and in 10 $\mu$M Spd compared to 10 $\mu$M Spd + PQ (\Cref{fig:40_pq_spd_ril_pi_a}: \textbf{A}).

With regards to rilmenidine treatment, a significant increase in the percentage of PI + cells was observed in the PQ treated group (42.00 \% $\pm$ 1.62 \%, \textit{p} < 0.05), 1 $\mu$M Ril + PQ (47.83 \% $\pm$ 1.06 \%, \textit{p} < 0.05) and 10 $\mu$M Ril + PQ (51.35 \% $\pm$ 4.48 \%, \textit{p} < 0.05) compared to the control (6.87 \% $\pm$ 1.05 \%), with no significant differences observed in the 1 $\mu$M Ril (5.38 \% $\pm$ 0.36 \%) and 10 $\mu$M Ril (7.82 \% $\pm$ 0.72 \%) (\Cref{fig:40_pq_spd_ril_pi_a}: \textbf{B}). A significant decrease in the PI + cells was observed in the low and high concentration of rilmenidine (1 \& 10 $\mu$M Ril), while a significant increase was detected in the 10 $\mu$M Ril + PQ compared to the PQ treated group.  Moreover, a significant decrease was observed 1 $\mu$M Ril compared to 1 $\mu$M Ril + PQ and in 10 $\mu$M Ril compared to 10 $\mu$M Ril + PQ (\Cref{fig:40_pq_spd_ril_pi_a}: \textbf{B}).

\begin{figure}[!htbp]
  \center
  \begin{subfigure}[b]{0.495\linewidth}
    \includegraphics[width=\linewidth]{figures/chapter40/40_pq_spd_ril_pi_a}
    \caption{Spermidine}
  \end{subfigure}
  \begin{subfigure}[b]{0.495\linewidth}
    \includegraphics[width=\linewidth]{figures/chapter40/40_pq_spd_ril_pi_b}
    \caption{Rilmenidine}
  \end{subfigure}
  \caption[Effect of spermidine and rilmenidine on cell death onset upon PQ induced neuronal toxicity]{\textbf{Effect of spermidine and rilmenidine on cell death onset upon PQ induced neuronal toxicity.} GT1-7 cells were pre-treated with 1 \& 10 $\mu$M of Spd (\textbf{A}) and Ril (\textbf{B}) followed by exposure to 3 mM PQ for 6 h and assessed for cell viability using WST-1 assay. All results are presented as a percentage of the control (mean $\pm$ SEM), textit{n}=3, with 6 replicates per group. * = \textit{p} < 0.05 vs control, \# = \textit{p} < 0.05 vs PQ.}
  \label{fig:40_pq_spd_ril_pi_a}
\end{figure}


\subsection{The effect of low and high concentrations of spermidine and rilmenidine on autophagy upon PQ-induced neuronal toxicity} 
\subsubsection{Autophagy assessment using fluorescence microscopy}
Following treatments, immunofluorescence and imaging was performed. LC3, p62 and lysosome puncta were counted using the thresholding method or by eye on image J. Our results indicate a progressive increase in the average number of LC3 puncta per cell in PQ treated group (83.69 $\pm$ 14.35), 1 µM Spd (101.00 $\pm$ 19.75) with significant increase observed in the in 1 $\mu$M Spd + PQ (142.90 $\pm$ 24.44, \textit{p} < 0.05), 10 $\mu$M Spd (196.70 $\pm$ 23.83, \textit{p} < 0.05) and 10 $\mu$M Spd + PQ (244.80 $\pm$ 42.03, \textit{p} < 0.05) compared to the control (75.58 $\pm$ 14.99), (\Cref{fig:40_pq_spd_ril_lc3_fm_a}: \textbf{A}\textit{i}, \textit{ii}). Moreover, a significant increase was observed in the 10 $\mu$M Spd and 10 $\mu$M Spd + PQ (\textit{p} < 0.05) compared to the PQ group. Furthermore, a significant increase was observed in 10 $\mu$M Spd (\textit{p} < 0.05) compared to 1 $\mu$M Spd  and in 10 $\mu$M Spd + PQ (\textit{p} < 0.05) compared to 10 $\mu$M Spd.


p62 puncta was significantly increased in the 10 $\mu$M Spd + PQ group (17.40 $\pm$ 4.42, \textit{p} < 0.05) compared to the control (2.56 $\pm$ 0.94), and PQ (5.39 $\pm$ 1.05), 1 $\mu$M Spd + PQ (2.90 $\pm$ 0.74) and 10 $\mu$M Spd group (8.00 $\pm$ 1.40) (\Cref{fig:40_pq_spd_ril_p62_fm_a}: \textbf{A}\textit{i}, \textit{ii}). Moreover, no significant differences were detected in the control and PQ group compared to 1 $\mu$M Spd (3.00 $\pm$ 0.82), 1 $\mu$M Spd + PQ (2.90 $\pm$ 0.74) and 10 $\mu$M Spd group (8.00 $\pm$ 1.40) and between control and PQ group.

Lysosomal puncta was significantly increased in the PQ treated group (25.38 $\pm$ 5.26, \textit{p} < 0.05), 10 $\mu$M Spd (27.09 $\pm$ 5.29, \textit{p} < 0.05) and 10 $\mu$M Spd + PQ (29.00 $\pm$ 7.34, \textit{p} < 0.05) compared to the control group (11.27 $\pm$ 3.33), with no significant differences observed in the 1 $\mu$M Spd (16.90 $\pm$ 3.47) and 1 $\mu$M Spd + PQ (17.00 $\pm$ 3.77) (\Cref{fig:40_pq_spd_ril_lyso_fm_a}: \textbf{A}\textit{i}, \textit{ii}).

With regards to rilmenidine treatment, no significant differences were observed in the average number of LC3 puncta in all the groups [Con (69.43 $\pm$ 25.51), PQ (80.40 $\pm$ 11.57), 1 $\mu$M Ril (79.00 $\pm$ 21.80), 1 $\mu$M Ril + PQ (81.50 $\pm$ 46.31), 10 $\mu$M Ril (90.60 $\pm$ 34.47) and 10 $\mu$M Ril + PQ (67.29 $\pm$ 17.29)] (\Cref{fig:40_pq_spd_ril_lc3_fm_a}: \textbf{B}\textit{i}, \textit{ii}).

Similarly with p62 puncta, no significant differences were observed in all the groups [Con (3.67 $\pm$ 1.28), PQ (4.20 $\pm$ 0.97), 1 $\mu$M Ril (7.00 $\pm$ 2.02), 1 $\mu$M Ril + PQ (7.17 $\pm$ 3.97), 10 $\mu$M Ril (6.00 $\pm$ 2.07) and 10 $\mu$M Ril + PQ (4.57 $\pm$ 2.06)], (\Cref{fig:40_pq_spd_ril_p62_fm_a}: \textbf{B}\textit{i}, \textit{ii}).

Similarly with lysosomal puncta, no significant differences were observed in all the groups [Con (12.29 $\pm$ 3.93), PQ (10.80 $\pm$ 1.83), 1 $\mu$M Ril (17.00 $\pm$ 2.49), 1 $\mu$M Ril + PQ (27.50 $\pm$ 15.12), 10 $\mu$M Ril (20.00 $\pm$ 6.96) and 10 $\mu$M Ril + PQ (19.29 $\pm$ 3.20)], (\Cref{fig:40_pq_spd_ril_lyso_fm_a}: \textbf{B}\textit{i}, \textit{ii}).

\begin{figure}[!htbp]
  \center
  \begin{subfigure}[b]{0.495\linewidth}
    \includegraphics[width=\linewidth]{figures/chapter40/40_pq_spd_ril_lc3_fm_a}
    \caption{Spermidine}
  \end{subfigure}
  \begin{subfigure}[b]{0.495\linewidth}
    \includegraphics[width=\linewidth]{figures/chapter40/40_pq_spd_ril_lc3_fm_b}
    \caption{Rilmenidine}
  \end{subfigure}
  \caption[Effect of spermidine and rilmenidine on LC3 puncta upon PQ induced neuronal toxicity]{\textbf{Effect of spermidine and rilmenidine on LC3 puncta upon PQ induced neuronal toxicity.} Representative fluorescence micrographs and quantitative analysis of LC3 puncta following treatment with 1 \& 10 $\mu$M of Spd (\textbf{A}) and Ril (\textbf{B}). Data are presented as mean $\pm$ SEM, textit{n}=3, with a total of 30 - 60 cells analysed per group. * = \textit{p} < 0.05 vs control, \# = \textit{p} < 0.05 vs PQ, \$ = \textit{p} < 0.05 vs 1 $\mu$M Spd + PQ, and \@ = \textit{p} < 0.05 vs 1 $\mu$M Spd. Scale bar: 10 $\mu$M.}
  \label{fig:40_pq_spd_ril_lc3_fm_a}
\end{figure}

\begin{figure}[!htbp]
  \center
  \begin{subfigure}[b]{0.495\linewidth}
    \includegraphics[width=\linewidth]{figures/chapter40/40_pq_spd_ril_p62_fm_a}
    \caption{Spermidine}
  \end{subfigure}
  \begin{subfigure}[b]{0.495\linewidth}
    \includegraphics[width=\linewidth]{figures/chapter40/40_pq_spd_ril_p62_fm_b}
    \caption{Rilmenidine}
  \end{subfigure}
  \caption[Effect of spermidine and rilmenidine on p62 puncta upon PQ induced neuronal toxicity]{\textbf{Effect of spermidine and rilmenidine on p62 puncta upon PQ induced neuronal toxicity.} Representative fluorescence micrographs and quantitative analysis of p62 puncta following treatment with 1 \& 10 $\mu$M of Spd (\textbf{A}) and Ril (\textbf{B}). Data are presented as mean $\pm$ SEM, textit{n}=3, with a total of 30 - 60 cells analysed per group. * = \textit{p} < 0.05 vs control, \# = \textit{p} < 0.05 vs PQ, \$ = \textit{p} < 0.05 vs 1 $\mu$M Spd + PQ, and \% = \textit{p} < 0.05 vs 10 $\mu$M Spd. Scale bar: 10 $\mu$M.}
  \label{fig:40_pq_spd_ril_p62_fm_a}
\end{figure}

\begin{figure}[!htbp]
  \center
  \begin{subfigure}[b]{0.495\linewidth}
    \includegraphics[width=\linewidth]{figures/chapter40/40_pq_spd_ril_lyso_fm_a}
    \caption{Spermidine}
  \end{subfigure}
  \begin{subfigure}[b]{0.495\linewidth}
    \includegraphics[width=\linewidth]{figures/chapter40/40_pq_spd_ril_lyso_fm_b}
    \caption{Rilmenidine}
  \end{subfigure}
  \caption[Effect of spermidine and rilmenidine on lysosomal puncta upon PQ induced neuronal toxicity]{\textbf{Effect of spermidine and rilmenidine on lysosomal puncta upon PQ induced neuronal toxicity.} Representative fluorescence micrographs and quantitative analysis of lysosomal puncta following treatment with 1 \& 10 $\mu$M of Spd (\textbf{A}) and Ril (\textbf{B}). Data are presented as mean $\pm$ SEM, textit{n}=3, with a total of 30 - 60 cells analysed per group. * = \textit{p} < 0.05 vs control.}
  \label{fig:40_pq_spd_ril_lyso_fm_a}
\end{figure}

\subsubsection{Autophagy assessment using western blotting}
Following treatments, autophagy was assessed using western blotting. No significant differences were observed in LC3-II protein expression in all groups [Con (1.00 $\pm$ 0.00), PQ (1.61 $\pm$ 0.41),1 $\mu$M Spd (1.32 $\pm$ 0.51), 1 $\mu$M Spd + PQ (2.27 $\pm$ 0.77), 10 $\mu$M Spd (1.50 $\pm$ 0.71), 10 $\mu$M Spd + PQ (2.17 $\pm$ 0.93)] (\Cref{fig:40_pq_spd_ril_lc3_wb_a}: \textbf{A}\textit{i}, \textit{ii}), however an trend towards an increase in LC3 II was observed in the PQ treated group compared to the PQ untreated groups and in the the combination groups compared to PQ alone.

No significant differences were observed in p62 protein expression in all groups [Con (1.00 $\pm$ 0.00), PQ (0.97 $\pm$ 0.18),1 µM Spd (0.85 $\pm$ 0.16), 1 $\mu$M Spd + PQ (0.75 $\pm$ 0.20), 10 $\mu$M Spd (1.04 $\pm$ 0.09), 10 $\mu$M Spd + PQ (0.82 $\pm$ 0.20)] (\Cref{fig:40_pq_spd_ril_p62_wb_a}: \textbf{A}\textit{i}, \textit{ii}).

LAMP2A protein expression progressively decreased following treatment in the Con (1.00 $\pm$ 0.00), PQ (0.84 $\pm$ 0.24),1 $\mu$M Spd (0.72 $\pm$ 0.24), 1 $\mu$M Spd + PQ (0.52 $\pm$ 0.06), 10 $\mu$M Spd (0.48 $\pm$ 0.14), with a  significant decrease observed in the 10 $\mu$M Spd + PQ (0.41 $\pm$ 0.22, \textit{p} < 0.05) (\Cref{fig:40_pq_spd_ril_lamp2a_wb_a}: \textbf{A}\textit{i}, \textit{ii}). 

Regarding rilmenidine treatment, no significant differences were observed in the expression of LC3-II protein in all the groups tested however, the expression was higher in PQ (1.61 $\pm$ 0.41) compared to Con (1.00 $\pm$ 0.00), in 1 $\mu$M Ril + PQ (0.95 $\pm$ 0.34) compared to 1 $\mu$M Ril (1.10 $\pm$ 0.49), and in 10 $\mu$M Ril + PQ (1.18 $\pm$ 0.53) compared to 10 $\mu$M Ril (0.70 $\pm$ 0.12) (\Cref{fig:40_pq_spd_ril_lc3_wb_a}: \textbf{B}\textit{i}, \textit{ii}).

Similarly with p62 (\Cref{fig:40_pq_spd_ril_p62_wb_a}: \textbf{B}\textit{i}, \textit{ii}), no significant differences were seen in all the groups tested [Con (1.00 ± 0.00), PQ (0.97 ± 0.18),1 $\mu$M Ril (1.55 ± 0.50), 1 $\mu$M Ril + PQ (1.39 $\pm$ 0.65), 10 $\mu$M Ril (1.38 $\pm$ 0.45), 10 $\mu$M Ril + PQ (1.37 $\pm$ 0.35)].

Lastly, LAMP2A was similarly expressed in all the groups tested [Con (1.00 $\pm$ 0.00), PQ (1.03 $\pm$ 0.22),1 $\mu$M Ril (0.85 $\pm$ 0.16), 1 $\mu$M Ril + PQ (0.75 $\pm$ 0.20), 10 $\mu$M Ril (1.04 $\pm$ 0.09), 10 $\mu$M Ril + PQ (0.82 $\pm$ 0.20)] (\Cref{fig:40_pq_spd_ril_lamp2a_wb_a}: \textbf{B}\textit{i}, \textit{ii}).

\begin{figure}[!htbp]
  \center
  \begin{subfigure}[b]{0.495\linewidth}
    \includegraphics[width=\linewidth]{figures/chapter40/40_pq_spd_ril_lc3_wb_a}
    \caption{Spermidine}
  \end{subfigure}
  \begin{subfigure}[b]{0.495\linewidth}
    \includegraphics[width=\linewidth]{figures/chapter40/40_pq_spd_ril_lc3_wb_b}
    \caption{Rilmenidine}
  \end{subfigure}
  \caption[Effect of spermidine and rilmenidine on LC-II protein expression upon PQ induced neuronal toxicity]{\textbf{Effect of spermidine and rilmenidine on LC-II protein expression upon PQ induced neuronal toxicity.} Densitometric analysis (i) and representative western blot  (ii) of LC3-II. Data are presented as mean $\pm$ SEM, \textit{n} = 3.}
  \label{fig:40_pq_spd_ril_lc3_wb_a}
\end{figure}
  
\begin{figure}[!htbp]
  \center
  \begin{subfigure}[b]{0.495\linewidth}
    \includegraphics[width=\linewidth]{figures/chapter40/40_pq_spd_ril_p62_wb_a}
    \caption{Spermidine}
  \end{subfigure}
  \begin{subfigure}[b]{0.495\linewidth}
    \includegraphics[width=\linewidth]{figures/chapter40/40_pq_spd_ril_p62_wb_b}
    \caption{Rilmenidine}
  \end{subfigure}
  \caption[Effect of spermidine and rilmenidine on p62 protein expression upon PQ induced neuronal toxicity]{\textbf{Effect of spermidine and rilmenidine on p62 protein expression upon PQ induced neuronal toxicity.} Densitometric analysis (i) and representative western blot  (ii) of p62. Data are presented as mean $\pm$ SEM, \textit{n} = 3.}
  \label{fig:40_pq_spd_ril_p62_wb_a}
\end{figure}
  
\begin{figure}[!htbp]
  \center
  \begin{subfigure}[b]{0.495\linewidth}
    \includegraphics[width=\linewidth]{figures/chapter40/40_pq_spd_ril_lamp2a_wb_a}
    \caption{Spermidine}
  \end{subfigure}
  \begin{subfigure}[b]{0.495\linewidth}
    \includegraphics[width=\linewidth]{figures/chapter40/40_pq_spd_ril_lamp2a_wb_b}
    \caption{Rilmenidine}
  \end{subfigure}
  \caption[Effect of spermidine and rilmenidine on LAMP2A protein expression upon PQ induced neuronal toxicity]{\textbf{Effect of spermidine and rilmenidine on LAMP2A protein expression upon PQ induced neuronal toxicity.} Densitometric analysis (i) and representative western blot  (ii) of LAMP2A. Data are presented as mean $\pm$ SEM, \textit{n} = 3.* = \textit{p} < 0.05 vs control.}
  \label{fig:40_pq_spd_ril_lamp2a_wb_a}
\end{figure}

\subsection{The effect of low and high concentrations of spermidine on microtubule stability and structure upon PQ-induced neuronal toxicity}
Next, acetylated $\alpha$-tubulin was assessed using western blotting and super-resolution techniques. Although, no significant differences were observed in the expression of acetylated $\alpha$-tubulin, a trend towards an increase was observed in the PQ treated group (2.23 ± 0.92), 1 $\mu$M Spd (1.46 $\pm$ 0.24), 1 $\mu$M Spd + PQ (3.32 $\pm$ 1.21), 10 $\mu$M Spd (1.10 $\pm$ 0.35), 10 $\mu$M Spd + PQ (2.64 $\pm$ 1.06) compared to the control group (1.00 $\pm$ 0.00) (\Cref{fig:40_pq_spd_ril_tubulin_wb_a}: \textit{i}, \textit{ii}). More importantly, a trend towards an increase in acetylated $\alpha$-tubulin was observed in the combination groups compared to PQ group alone. 

\begin{figure}[!htbp]
\center
  \includegraphics[width=0.75\linewidth]{figures/chapter40/40_pq_spd_ril_tubulin_wb_a}
  \caption[Effect of spermidine on microtubule stability and structure upon PQ-induced neuronal toxicity]{\textbf{Effect of spermidine on microtubule stability and structure upon PQ-induced neuronal toxicity.} Densitometric analysis (i) and representative western blot  (ii) of acetylated $\alpha$-tubulin. Data are presented as mean $\pm$ SEM, \textit{n} = 6.}
  \label{fig:40_pq_spd_ril_tubulin_wb_a}
\end{figure} 

Using SR-SIM, results show a decrease in the intensity of total $\alpha$/$\beta$ tubulin in the PQ treated groups compared to the PQ untreated group (\Cref{fig:40_pq_spd_ril_tubulin_srsim_a}). In addition, an increase in acetylated $\alpha$-tubulin intensity signal and density was observed following treatment in all the groups (PQ, 1 $\mu$M Spd, 1 $\mu$M Spd + PQ, 10 $\mu$M Spd, 10 $\mu$M Spd + PQ) compared to the control group. In addition, acetylated $\alpha$-tubulin were organised in the perinuclear regions. More importantly, acetylated $\alpha$-tubulin signal and density was increased in the 1 $\mu$M Spd + PQ and 10 $\mu$M Spd + PQ group compared to PQ alone, with the 1 $\mu$M Spd + PQ showing increased signal and density (\Cref{fig:40_pq_spd_ril_tubulin_srsim_a}). 

Similarly with the SR-SIM, d-STORM images showed an increased signal in acetylated $\alpha$-tubulin in the PQ treated groups compared to PQ untreated groups (\Cref{fig:40_pq_spd_ril_tubulin_dstorm_a}). Moreover, fine tubulin network were observed with d-STORM (\Cref{fig:40_pq_spd_ril_tubulin_dstorm_a}) compared to SR-SIM (\Cref{fig:40_pq_spd_ril_tubulin_srsim_a}).

\section{Discussion: The role of spermidine and rilmenidine in a paraquat-induced neuronal toxicity model}
To date, a growing consensus exist suggesting that modulation of autophagy by pharmacological interventions reduces age-related pathologies. Spermidine and rilmenidine are known autophagy modulators that have been shown to clear aggregate prone proteins associated with neurodegeneration \citep{Bhukel2017,Buttner2014,Rose2010,Sigrist2014} and to extend life span in the case of spermidine \citep{Madeo2010,Morselli2009}. Despite the autophagy enhancing effects of rilmenidine and spermidine, the relationship between autophagy activity, the extent of protein clearance, oxidative stress, as well as microtubule organisation and cell death onset remains largely unclear. Moreover, the impact of concentration differences in autophagy modulation and on subsequent protein clearance and neuronal toxicity , such as in PQ-induced neuronal injury is unclear. Therefore, in this study, we used PQ to induce neuronal toxicity in GT 1-7 cells in order to assess the potential protective effects of spermidine and rilmenidine in the context of cellular viability, ROS generation, and cell death onset and to determine firstly whether effects are exerted through autophagy and secondly whether a higher autophagic activity will translate in a higher degree of protection. 

\subsection{ The effect of PQ on cell viability}
In order to determine a suitable concentration of PQ that induces neuronal toxicity, GT 1-7 cells were treated with PQ at various concentrations, so as to select a suitable concentration of PQ for subsequent experiments. Our results reveal that PQ induced neuronal toxicity in a concentration dependent manner (\Cref{fig:40_pq_cell_viability_a}). In line with our findings, others have reported a concentration dependent decrease in cellular viability using PQ \citep{Chen2012b,Jaroonwitchawan2017,Mehdi2013}. 

\begin{landscape}\centering
\begin{figure}[!htbp]
\vspace*{\fill}
\centering
  \includegraphics[width=\linewidth]{figures/chapter40/40_pq_spd_ril_tubulin_srsim_a}
  \caption[Effect of spermidine on PQ induced neuronal toxicity on microtubule stability and structure]{\textbf{Effect of spermidine on PQ induced neuronal toxicity on microtubule stability and structure.} Representative images of SR-SIM for acetylated $\alpha$-tubulin (green) and $\alpha$/$\beta$ tubulin (red) in the Con, PQ, 1 $\mu$M Spd, 1 $\mu$M Spd + PQ, 10 $\mu$M Spd \& 10 $\mu$M Spd + PQ. Scale bar: 5 $\mu$M and 1 $\mu$M.}
  \label{fig:40_pq_spd_ril_tubulin_srsim_a}
\end{figure} 
\vfill
\end{landscape}

\begin{landscape}
\begin{figure}[!htbp]
\center
  \includegraphics[width=\linewidth]{figures/chapter40/40_pq_spd_ril_tubulin_dstorm_a}
  \caption[Effect of spermidine on PQ induced neuronal toxicity on microtubule stability and structure]{\textbf{Effect of spermidine on PQ induced neuronal toxicity on microtubule stability and structure.} Representative images of d-STORM for acetylated $\alpha$-tubulin in the Con, PQ, 1 $\mu$M Spd, 1 $\mu$M Spd + PQ, 10 $\mu$M Spd \& 10 $\mu$M Spd + PQ. Scale bar: 2 $\mu$M and 0.5 $\mu$M.}
  \label{fig:40_pq_spd_ril_tubulin_dstorm_a}
\end{figure} 
\end{landscape}

\subsection{The effect of low and high concentrations spermidine and rilmenidine on cellular viability upon PQ-induced neuronal toxicity} 
Next, we assessed whether spermidine and rilmenidine protect against PQ-induced neuronal toxicity. Our results reveal that PQ significantly reduced cellular viability compared to the control group (\Cref{fig:40_pq_spd_ril_cell_viability_a}). However, combination treatment of PQ and spermidine at both concentrations (1 $\mu$M Spd + PQ and 10 $\mu$M Spd + PQ) significantly improved cellular viability compared to the PQ group (\Cref{fig:40_pq_spd_ril_cell_viability_a}: \textbf{A}), while combination treatment of PQ and rilmenidine (1 $\mu$M Ril + PQ and 10 $\mu$M Ril + PQ) did not protect against loss in cellular viability (\Cref{fig:40_pq_spd_ril_cell_viability_a}: \textbf{B}). These results suggest that only spermidine and not rilmenidine protect against cellular injury induced by PQ, in this model system.

\subsection{The effect of low and high concentration spermidine and rilmenidine on ROS generation upon PQ-induced neuronal toxicity}
Next, we assessed whether spermidine and rilmenidine protect against PQ-induced ROS production. Various studies have demonstrated that PQ accumulates in the mitochondria and induces the production of superoxide (O\textsubscript{2}\textsuperscript{-}), hydrogen peroxide (H\textsubscript{2}O\textsubscript{2}) and other types of reactive oxygen species (ROS) \citep{Cocheme2008,Jones2000,Yumino2002}. Thus, flow cytometer was used to assess ROS generation using DCF and MitoSox probes to measure to measure H\textsubscript{2}O\textsubscript{2} and O\textsubscript{2}\textsuperscript{-}, respectively. Our results revealed an enhanced mitochondrial O\textsubscript{2}\textsuperscript{-}  generation in the PQ group and combination groups i.e. 1 $\mu$M Spd + PQ and 10 $\mu$M Spd + PQ compared to the control (\Cref{fig:40_pq_spd_ril_ros_a}: \textbf{A}\textit{i}). Notably, a significant reduction in mitochondrial O\textsubscript{2}\textsuperscript{-} was detected following treatment with the combination group i.e 1 $\mu$M Spd + PQ compared to the PQ group. Moreover, and importantly, 1 $\mu$M Spd + PQ significantly reduced mitochondrial O2- compared to 10 $\mu$M Spd + PQ (\Cref{fig:40_pq_spd_ril_ros_a}: \textbf{A}\textit{i}). These results suggest that firstly, the increase observed in mitochondrial O\textsubscript{2}\textsuperscript{-} in the PQ group, 1 $\mu$M Spd + PQ and 10 $\mu$M Spd + PQ is indeed likely the result of PQ and not spermidine, since treatment with spermidine only at low and high concentrations resulted in the same amount of mitochondrial O\textsubscript{2}\textsuperscript{-} as observed in the control group. Secondly, the combination group with the lowest concentration of spermidine and not the high concentration decreases ROS levels and confers neuroprotection. Overall, these results suggest that the concentration of the autophagy modulating drug indeed matters and that a higher concentration does not necessarily translate to enhanced protection. Consistent with the mitochondrial O\textsubscript{2}\textsuperscript{-} results, a significant increase cytosolic H\textsubscript{2}O\textsubscript{2} (DCF signal) was observed in the PQ treated group and the 10 $\mu$M Spd + PQ compared to the control (\Cref{fig:40_pq_spd_ril_ros_a}: \textbf{A}\textit{ii}). Moreover and importantly, a significant reduction in cytosolic H\textsubscript{2}O\textsubscript{2} was detected following treatment with 1 $\mu$M Spd + PQ compared to the PQ group alone (\Cref{fig:40_pq_spd_ril_ros_a}: \textbf{A}\textit{ii}). These results suggest that spermidine mitigates PQ induced ROS production. In line with these findings, it was reported in cultured mouse fibroblasts that spermidine as well as spermine (produced from spermidine) mediated protection against oxidative stress caused by H\textsubscript{2}O\textsubscript{2} \citep{Rider2007}. Since spermidine is a known anti-oxidant, we speculate that it protects against ROS damage by increasing the generation of anti-oxidants enzymes such as superoxide dismutase (SOD). In line with these findings, others have reported a decrease in PQ-induced ROS generation in SH-SY-5Y cells following pre-treatment with antioxidants such as \textit{N-acetylcysteine} (NAC) \citep{Zhou2017} and curcumin \citep{Jaroonwitchawan2017}, where curcumin also stimulated an increase in antioxidant enzymes, including superoxide dismutase (SOD) and glutathione peroxidase (GPX). Moreover, spermidine has been shown to serve dual roles, as it can also act as substrate for enzymes responsible for ROS production \citep{Stewart2018}. Thus, this might explain why the high concentration of spermidine in the combination group failed to protect against ROS damage. In contrast to the protective effects of spermidine against ROS generation, our results revealed that rilmenidine failed to protect against mitochondrial O\textsubscript{2}\textsuperscript{-} and cytosolic H\textsubscript{2}O\textsubscript{2} (\Cref{fig:40_pq_spd_ril_ros_a}: \textbf{B}\textit{i}, \textit{ii}). To our knowledge, our study is the first to report on the role of  rilmenidine in the context of PQ induced injury. Previous studies have shown that rilmenidine does not have an effect on cytotoxicity induced by rotenone, a neurotoxin that elicits mechanism similar to PQ \citep{Choi2002}, supporting results observed in this study.

\subsection{The effect of low and high concentration spermidine and rilmenidine on cell death upon PQ-induced neuronal toxicity}
Next, we assessed whether spermidine and rilmenidine protect against PQ-induced cell death onset. Mounting evidence suggests that several types of cells including neurons, are highly sensitive to PQ induced toxicity and that PQ eventually results in cell death onset \citep{Chun2001,Gonzalez-Polo2004,Niso-Santano2006}. Hence, here we assessed cell death, in particular loss of membrane integrity by measuring propidium iodide (PI) uptake using flow cytometry. Consistent with the increase in ROS production, cell death analysis using PI revealed a significant increase in cell death onset in the PQ group, 1 $\mu$M Spd + PQ and 10 $\mu$M Spd + PQ compared to the control group (\Cref{fig:40_pq_spd_ril_pi_a}: \textbf{A}). However, a significant reduction in cell death was observed in the 1 $\mu$M Spd + PQ compared to the PQ group (\Cref{fig:40_pq_spd_ril_pi_a}: \textbf{A}). These results suggest that spermidine protects against PQ induced cell damage by inhibiting necrosis or loss in membrane integrity. In support of our findings, spermidine administration was found to protect against age- induced necrotic cell death in yeast cells by inhibiting histone H3 acetylation \citep{Eisenberg2009}. In addition, spermidine was found to protect against cell death induced by H\textsubscript{2}O\textsubscript{2} in retinal ganglion cells \citep{Guo2011}. Consistent with the ROS data, we observed a significant increase in cell death with 10 $\mu$M Ril + PQ compared to PQ alone (\Cref{fig:40_pq_spd_ril_pi_a}: \textbf{B}), suggesting that rilmenidine does not protect against cell death induced by ROS generation. Indeed, in models of ALS, rilmenidine treatment was found to augment disease progression and neurodegeneration in motor neurons expressing mutant SOD1G93A \citep{Perera2018}. The augment in disease progression in this study was associated with an increase in the accumulation and aggregation of insoluble and damaged SOD1 species, which were thought to occur outside of the autophagy pathway. Moreover, as a result of overactivation of mitophagy, severe mitochondrial depletion was found in motor neurons of mice treated with rilmenidine.

\subsection{The effect of low and high concentration of spermidine and rilmenidine on autophagy upon PQ-induced neuronal toxicity}
Next, we assessed whether spermidine and rilmenidine potentially protect against PQ-induced toxicity by up-regulating autophagy. Literature suggests that spermidine and rilmenidine protect against neurodegeneration by clearing aggregate protein associated with pathology through upregulating autophagy \citep{Buttner2014,Rose2010}. In order to investigate the role of autophagy in the context of PQ-induced neuronal toxicity, autophagy markers were assessed using FM and western blotting. Our results revealed a progressive accumulation of LC3 puncta following treatment with PQ and 1 $\mu$M Spd with a significant increase observed in 1 $\mu$M Spd + PQ, 10 $\mu$M Spd and 10 $\mu$M Spd + PQ compared to the control group (\Cref{fig:40_pq_spd_ril_lc3_fm_a}: \textbf{A}\textit{i}, \textit{ii}). These results suggest that firstly, autophagy was induced in this model above basal levels and was enhanced further in the combination groups at both concentrations (1 $\mu$M Spd + PQ and 10 $\mu$M Spd + PQ), while exposure to PQ alone did not induce autophagy robustly. Secondly, our results revealed a significant accumulation of LC3 puncta in 1 $\mu$M Spd + PQ and 10 $\mu$M Spd + PQ compared to the PQ group and 10 $\mu$M Spd + PQ compared to 1 $\mu$M Spd + PQ, suggesting a concentration dependent autophagy modulation. Thirdly, we observed a significant increase in the accumulation of LC3 puncta in 10 $\mu$M Spd compared to 1 $\mu$M Spd and in 10 $\mu$M Spd + PQ compared to 1 $\mu$M Spd + PQ, as well as an increase in LC3 although not significantly between 1 $\mu$M Spd + PQ compared to 1 $\mu$M Spd  and between 10 $\mu$M Spd + PQ compared to 10 $\mu$M Spd suggesting that, in addition to a concentration dependent effect of autophagy modulation by spermidine, PQ also modulates autophagy to some extent, likely resulting in an additive effect (\Cref{fig:40_pq_spd_ril_lc3_fm_a}: \textbf{A}\textit{i}, \textit{ii}). Overall, these results suggest that spermidine confers cellular protection against PQ induced neurotoxicity by inducing autophagy. In line with our findings, an increase in LC3-II above the levels induced by PQ alone was reported when a combination of rapamycin and PQ was used in \textit{in vitro}) studies \citep{Gonzalez-Polo2007a}. Moreover, beneficial effects of spermidine through autophagy upregulation resulting in a reduction of oxidative damage has been shown in several  \textit{in vivo} models. Spermidine has been shown upregulate autophagy and confer protection against synaptic aging \citep{Bhukel2017}, arterial aging \citep{LaRocca2013}, memory loss associated with aging \citep{Gupta2013,Sigrist2014,Wirth2018} and neurotoxicity \citep{Buttner2014,Yang2017}. Spermidine is known to exert its protective effects through autophagy by inhibiting EP300 \citep{Pietrocola2015}. In support of the FM data, western blot analysis revealed an increase in LC3-II expression in the combination groups (1 $\mu$M Spd + PQ and 10 $\mu$M Spd + PQ) compared to PQ alone, but no significance was reached  (\Cref{fig:40_pq_spd_ril_lc3_wb_a}: \textbf{A}\textit{i}, \textit{ii}). 

 Consistent with the LC3 data, assessment of p62 puncta using FM revealed an enhanced accumulation of p62 in the 10 $\mu$M Spd + PQ group compared to the control group, PQ treated group, 1 $\mu$M Spd + PQ and 10 $\mu$M Spd group (\Cref{fig:40_pq_spd_ril_p62_fm_a}: \textbf{A}\textit{i}, \textit{ii}). The concurrent accumulation of p62 and LC3 puncta may suggest that autophagy is induced but p62 is not yet completely cleared. In support of this notion, using western blot analysis, we revealed a trend towards a decrease in p62 expression in the combination groups compared to PQ alone, suggesting that autophagy was modulated leading to the degradation of p62 (\Cref{fig:40_pq_spd_ril_p62_wb_a}: \textbf{A}\textit{i}, \textit{ii}). In line with our findings, others have shown a decrease in p62 levels following autophagy induction \citep{Bjorkoy2005,Jones2000,Zhou2017}. Importantly, we observed a trend towards an increase in LC3 puncta (\Cref{fig:40_pq_spd_ril_lc3_fm_a}: \textbf{A}\textit{i}, \textit{ii}) and LC3-II protein expression (\Cref{fig:40_pq_spd_ril_lc3_wb_a}: \textbf{A}\textit{i}, \textit{ii}) and a trend towards a decrease in p62 puncta (\Cref{fig:40_pq_spd_ril_p62_fm_a}: \textbf{A}\textit{i}, \textit{ii}) and p62 protein expression (\Cref{fig:40_pq_spd_ril_p62_wb_a}: \textbf{A}\textit{i}, \textit{ii}) in the combination group with the low concentration of spermidine. These results suggest that the low concentration of spermidine stimulates autophagy and confers protection against PQ induced neuronal toxicity, as evident by the increase in cellular viability (\Cref{fig:40_pq_spd_ril_cell_viability_a}) and decrease in ROS generation (\Cref{fig:40_pq_spd_ril_ros_a}: \textbf{A}\textit{i}, \textit{ii}). 

When assessing lysosomal abundance, our results revealed that lysosomal puncta was significantly increased in the PQ treated group, 10 $\mu$M Spd and 10 $\mu$M Spd + PQ compared to the control group (\Cref{fig:40_pq_spd_ril_lyso_fm_a}: \textbf{A}\textit{i}, \textit{ii}). In addition, we observed no significant difference in the number of lysosomes between the PQ group, 1 $\mu$M Spd + PQ and 10 $\mu$M Spd + PQ, suggesting that spermidine did not affect  lysosomal abundance. In contrast to the FM data, western blot analysis of LAMP2A protein levels revealed a progressive decrease with significant differences detected at 10 $\mu$M Spd + PQ compared to the control group \Cref{fig:40_pq_spd_ril_lamp2a_wb_a}: \textbf{A}\textit{i}, \textit{ii}), suggesting a likely decline in autophagy activity, potentially CMA. Whether spermidine or PQ indeed fail to activate CMA activity remains to be elucidated. Moreover, we observed different results with lysosomal markers; a trend towards an increase with lysotracker using FM and a trend towards a decrease with LAMP2A using western blot analysis, suggesting that the increase observed with FM might be due to an enhanced endocytic activity and not necessarily an increase in mature lysosomes. This requires further assessment, for example assessing the levels of endosomes using RAP7 (Rhoptry-associated protein), early lysosomes using LAMP1 and late lysosome using LAMP2A in order to better understand the response. In contrast to data obtained with spermidine, when using rilmenidine, we observed no significant differences in LC3 (\Cref{fig:40_pq_spd_ril_lc3_fm_a}: \textbf{B}\textit{i}, \textit{ii}), p62 (\Cref{fig:40_pq_spd_ril_p62_fm_a}: \textbf{B}\textit{i}, \textit{ii}) and lysosomal puncta (\Cref{fig:40_pq_spd_ril_lyso_fm_a}: \textbf{B}\textit{i}, \textit{ii}) in all treatment groups, as well as in the expression of LC3-II (\Cref{fig:40_pq_spd_ril_lc3_wb_a}: \textbf{B}\textit{i}, \textit{ii}), p62 (\Cref{fig:40_pq_spd_ril_p62_wb_a}: \textbf{B}\textit{i}, \textit{ii}), and LAMP2A  (\Cref{fig:40_pq_spd_ril_lamp2a_wb_a}: \textbf{B}\textit{i}, \textit{ii}) protein, suggesting that rilmenidine did not induce autophagy in this context. In contrast to our findings, others have shown that rilmenidine clears mutant huntingtin, $\alpha$-synuclein and mutant SOD1 in various cell culture models by up-regulating autophagy \citep{Rose2010,Perera2018,Williams2008}. 

\subsection{ The effect of low and high concentrations of spermidine on microtubule stability and structure upon PQ-induced neuronal toxicity}
Given the neuronal damage observed with PQ, we investigated the impact of PQ on acetylated $\alpha$-tubulin and the potential protective effects of spermidine on acetylated $\alpha$-tubulin after PQ treatment. Using western blotting, no significant differences were observed in the levels of acetylated $\alpha$-tubulin, however, a trend towards an increase was observed especially in the presence of PQ compared to the control group (\Cref{fig:40_pq_spd_ril_tubulin_wb_a}: \textit{i}, \textit{ii}). Combination groups presented higher levels of acetylated $\alpha$-tubulin compared to the PQ alone (\Cref{fig:40_pq_spd_ril_tubulin_wb_a}: \textit{i}, \textit{ii}). These results suggest that PQ did not result in destabilization of microtubules but rather increased their stability, which appeared enhanced further in the combination groups. We speculate that the increase in the stability of microtubules observed following PQ treatment might be the result of stress response, likely in part due to autophagy activation as we observed an increase in LC3-II expression following PQ treatment, while increased further in the combination groups (\Cref{fig:40_pq_spd_ril_lc3_wb_a}: \textbf{A}\textit{i}, \textit{ii}). Low concentrations of PQ have been shown to stimulate characteristics of autophagy in \textit{in vitro} models, with neurons ultimately succumbing to cell death \citep{Gonzalez-Polo2009,Gonzalez-Polo2007a,Gonzalez-Polo2007b,Niso-Santano2006}. Due to the increased ROS generation we observed following PQ treatment, this may suggest that PQ act as a stressor that induces autophagy, probably in the initial stages, prior to mitochondrial dysfunction. Indeed, it has been shown that PQ induces an early ER stress response which was concurrent with activation of autophagy as indicated by the increase in the accumulation of autophagic vacuoles and LC3-II levels, beclin-1 activation, decreased p62 levels and dephosphorylation of  mammalian target of rapamycin \citep{Gonzalez-Polo2007a,Gonzalez-Polo2007b,Niso-Santano2011}. Acetylation of microtubules is directly involved in autophagy as microtubules facilitate the transport of autophagosomes to lysosomes for degradation \citep{Phadwal2018,Xie2010}. Thus spermidine increases the acetylation of tubulin and may facilitate autophagy degradation. In line with these findings, \citet{Phadwal2018} reported that spermine, produced from spermidine, increased tubulin acetylation resulting in the degradation of aggregate proteins in an \textit{in vitro} model of Prion diseases. The western blot data is supported by the SR-SIM (\Cref{fig:40_pq_spd_ril_tubulin_srsim_a}) and d-STORM micrographs (\Cref{fig:40_pq_spd_ril_tubulin_dstorm_a}), where an enhanced acetylation signal and density was observed following treatment with PQ,1 $\mu$M Spd + PQ and 10 $\mu$M Spd + PQ compared to the control cells. High-intensity signal of acetylated tubulin was observed in the combination groups compared to the PQ alone, with 1 µM Spd + PQ showing the highest intensity (\Cref{fig:40_pq_spd_ril_tubulin_srsim_a} \& \Cref{fig:40_pq_spd_ril_tubulin_dstorm_a}). Notably, acetylated tubulin was organised in the perinuclear region (\Cref{fig:40_pq_spd_ril_tubulin_srsim_a}). In addition, total $\alpha$/$\beta$ tubulin intensity decreased in the presence of PQ compared to the PQ untreated groups, suggesting a role of PQ in the acetylation of tubulin.

In conclusion, this part of the study highlighted the neuroprotective effects of spermidine. Spermidine but not rilmenidine protected against PQ-induced neuronal injury by suppressing the production of ROS and likely through the activation of autophagy. These results suggest that spermidine may serve as promising therapeutic compound  in the context of neuronal injury and potentially for neurodegeneration. Furthermore, spermidine is naturally produced, safe and tolerable even at higher concentration, making it a promising therapeutic agent.