\chapter{Role of spermidine and rilmenidine in a paraquat-induced neuronal toxicity model}
\section{Introduction}
Paraquat (PQ) is a pesticide that is used extensively in experimental models of AD and PD. Although the mechanisms of action of PQ in AD are not fully elucidated, it is known that PQ induces oxidative stress and mitochondrial damage \citep{Chen2012} and ultimately neuronal cell death. Both oxidative stress and mitochondrial damage have been implicated in the pathogenesis of AD \citep{Chen2012,Lin2006}. 

\section{Effect of PQ on cell viability}
In order to choose a suitable concentration of PQ that causes toxicity in GT 1-7, a concentration dependent toxicity study was performed using WST-1. Different concentration of PQ [(500 $\mu$M, 1, 2, 3, 5 and 10 mM) were assessed. Cell viability in response to PQ treatment significantly decreased progressively with an increase in concentration 500 $\mu$M (93.58 \% $\pm$ 1.23 \%, \textit{p} < 0.05), 1 mM (84.51 \% $\pm$ 1.48 \%, \textit{p} < 0.05), 2 mM (62.70 \% $\pm$ 3.27 \%, \textit{p} < 0.05), 3 mM (48.48 \% $\pm$ 3.58 \%, \textit{p} < 0.05), 5 mM (34.74 \% $\pm$ 3.11 \%, \textit{p} < 0.05) and 10 mM (23.41 \% $\pm$ 0.53 \%, \textit{p} < 0.05)] compared to the control group (100 \% $\pm$ 2.57 \%) (Fig.4.1). PQ at 3 mM reduced cell viability by approximately 50 \% and was selected for subsequent experiments.
%cross ref fig and add fig
\section{Effect Spd and Ril on neuronal toxicity induced by PQ}
In order to assess whether Spd and Ril protect against PQ induced neuronal toxicity, GT1-7 cells were pre-treated with 1 \& 10 $\mu$M of Spd and Ril for 8 h followed by exposure to 3 mM PQ for 6 h. Cells were assessed for cell viability, ROS damage, cell death, autophagy and microtubule stability.

\subsection{Effect of low and high concentration Spd and Ril on cellular viability} 
Cell viability significantly decreased in PQ treated group (72.67 \% $\pm$ 3.42 \%, \textit{p} < 0.05) compared to the control group (100.00 \% $\pm$ 1.13 \%) ( Fig.4.2.A). In addition, combination of spermidine with PQ at both concentrations; 1 $\mu$M Spd + PQ (89.25 \% ± 4.40 \%, \textit{p} < 0.05) and 10 $\mu$M Spd + PQ (93.04 \% $\pm$ 5.84 \%, \textit{p} < 0.05) significantly improved cell viability compared to the PQ treated group. When rilmenidine was used, a significant decreased in cell viability was observed in the PQ treated group (70.97 \% $\pm$ 4.00 \%, \textit{p} < 0.05), 1 $\mu$M Ril + PQ (74.43 \% $\pm$ 4.08 \%) and 10 $\mu$M Ril + PQ (78.41 \% $\pm$ 3.98 \%) compared to the control (100.00 \% $\pm$ 3.27 \%) ( Fig.4.2. B). Furthermore, no significant differences were observed in cellular viability in the 1 $\mu$M Ril + PQ and 10 $\mu$M Ril + PQ compared to the PQ treated group.
% croos ref fig and add fig

\subsection{Effect of low and high concentration of Spd and Ril reactive oxygen species (ROS)}
Following treatments, cells were stained for mitochondrial superoxide (O\textsubscript{2}\textsuperscript{-}) using mitoSOX and for cytosolic hydrogen peroxide (H\textsubscript{2}O\textsubscript{2}) using DCF. Mean fluorescence intensities were measured using flow cytometry. Mitochondrial O\textsubscript{2}\textsuperscript{-} was significantly increased in PQ treated group (3.92 $\pm$ 0.40, \textit{p} < 0.05), 1 $\mu$M Spd + PQ (2.11 $\pm$ 0.19, \textit{p} < 0.05) and 10 $\mu$M Spd + PQ (3.80 $\pm$ 0.52, \textit{p} < 0.05) compared to the control (1.00 $\pm$ 0.02), with no significant differences observed in the 1 $\mu$M Spd (1.05 $\pm$ 0.10) and 10 $\mu$M Spd (1.07 $\pm$ 0.03) compared to the control (Fig 4.3 A). A significant decrease in mitochondrial O\textsubscript{2}\textsuperscript{-} was observed in the 1$\mu$M Spd, 1 $\mu$M Spd + PQ and 10 $\mu$M Spd compared to the PQ treated group. Similarly, a significant decrease was observed in 1 $\mu$M Spd compared to 1 $\mu$M Spd + PQ, 10 $\mu$M Spd compared to 10 $\mu$M Spd + PQ and more importantly in 1 $\mu$M Spd + PQ compared to 10 $\mu$M Spd + PQ. 

Cytosolic H\textsubscript{2}O\textsubscript{2} was significantly increased in the PQ treated group (1.39 $\pm$ 0.10, \textit{p} < 0.05) and 10 $\mu$M Spd + PQ (1.40 $\pm$ 0.09, p < 0.05) compared to the control (1.00 $\pm$ 0.06), with no significant differences observed in the 1 $\mu$M Spd (1.00 $\pm$ 0.06), 1 $\mu$M Spd + PQ (1.18 $\pm$ 0.06) and 10 $\mu$M Spd (1.04 $\pm$ 0.06) compared to the control group (Fig 4.3.B). A significant decrease in cytosolic H\textsubscript{2}O\textsubscript{2} was observed in the 1 $\mu$M Spd, 1 $\mu$M Spd + PQ and 10 $\mu$M Spd compared to the PQ treated group. Similarly, a significant decrease was observed in 10 $\mu$M Spd compared to 10 $\mu$M Spd + PQ.

With regards to rilmenidine treatment, a significant increase in mitochondrial O\textsubscript{2}\textsuperscript{-} was observed in the 
in PQ treated group (1.41 $\pm$ 0.14, \textit{p} < 0.05), in 1 $\mu$M Ril + PQ (1.29 $\pm$ 0.07, \textit{p} < 0.05) and 10 $\mu$M Ril + PQ (1.47 $\pm$ 0.11, \textit{p} < 0.05) compared to the to the control group (1.00 $\pm$ 0.10), with no significant differences detected with 1 $\mu$M Ril (0.89 $\pm$ 0.08) and 10 $\mu$M Ril (0.98 $\pm$ 0.03) (Fig 4.3.C). A significant decrease in mitochondrial O\textsubscript{2}\textsuperscript{-} was observed in the 1 $\mu$M Ril and 10 $\mu$M Ril compared to the PQ treated group. Similarly, a significant decrease was observed in 1 $\mu$M Ril compared to 1 $\mu$M Ril + PQ, and in 10 $\mu$M Ril compared to 10 $\mu$M Ril + PQ. 

No significant differences in cytosolic H\textsubscript{2}O\textsubscript{2} were observed in all the groups tested [Control (1.00 $\pm$ 0.05), PQ (1.00 $\pm$ 0.03), 1 $\mu$M Ril (1.07 $\pm$ 0.04), 1 $\mu$M Ril + PQ (1.06 $\pm$ 0.04), 10 $\mu$M Ril (1.09 $\pm$ 0.04) and 10 $\mu$M Ril + PQ (1.07 $\pm$ 0.03) (Fig 4.3.D)].

\subsection{Effect of low and high concentration of Spd and Ril on reactive oxygen species cell death} 
Cell death was monitored following treatment using propidium iodide (PI), a membrane-impermeable dye which intercalates with DNA. PI accumulates in cells with compromised membrane integrity, a hallmark for necrotic cell death. Mean fluorescent intensities were measured using flow cytometry. Percentage of PI + cells was significantly increased in the PQ treated group (32.92 \% $\pm$ 2.50 \%, \textit{p} < 0.05), 1 $\mu$M Spd + PQ (24.11 \% $\pm$ 2.74 \%, \textit{p} < 0.05) and 10 $\mu$M Spd + PQ (29.44 \% $\pm$ 3.04 \%, \textit{p} < 0.05) compared to the control (3.91 \% $\pm$ 0.64 \%), with no significant differences observed in the 1 $\mu$M Spd (4.07 \% $\pm$ 0.27 \%) and 10 $\mu$M Spd (5.53 \% $\pm$ 0.64 \%) (Fig 4.4.A). Importantly, a significant decrease in the number of PI + cells was observed in the 1 $\mu$M Spd + PQ, 1 $\mu$M Spd and 10 $\mu$M Spd group compared to the PQ treated group. Furthermore, a significant decrease was observed in 1 $\mu$M Spd compared to 1 $\mu$M Spd + PQ and in 10 $\mu$M Spd compared to 10 $\mu$M Spd + PQ.

With regards to rilmenidine treatment, a significant increase in the percentage of PI + cells was observed in the PQ treated group (42.00 \% $\pm$ 1.62 \%, \textit{p} < 0.05), 1 $\mu$M Ril + PQ (47.83 \% $\pm$ 1.06 \%, \textit{p} < 0.05) and 10 $\mu$M Ril + PQ (51.35 \% $\pm$ 4.48 \%, \textit{p} < 0.05) compared to the control (6.87 \% $\pm$ 1.05 \%), with no significant differences observed in the 1 $\mu$M Ril (5.38 \% $\pm$ 0.36 \%) and 10 $\mu$M Ril (7.82 \% $\pm$ 0.72 \%) (Fig 4.4.A). A significant decrease in the PI + cells was observed in the low and high concentration of rilmenidine (1 \& 10 $\mu$M Ril), while a significant increase was detected in the 10 $\mu$M Ril + PQ compared to the PQ treated group.  Moreover, a significant decrease was observed 1 $\mu$M Ril compared to 1 $\mu$M Ril + PQ and in 10 $\mu$M Ril compared to 10 $\mu$M Ril + PQ (Fig 4.4.B).



























