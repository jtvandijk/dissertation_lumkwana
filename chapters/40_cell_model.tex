\chapter{Role of spermidine and rilmenidine in a paraquat-induced neuronal toxicity model}
\section{Introduction}
Paraquat (PQ) is a pesticide that is used extensively in experimental models of AD and PD. Although the mechanisms of action of PQ in AD are not fully elucidated, it is known that PQ induces oxidative stress and mitochondrial damage \citep{Chen2012} and ultimately neuronal cell death. Both oxidative stress and mitochondrial damage have been implicated in the pathogenesis of AD \citep{Chen2012,Lin2006}. 

\section{Effect of PQ on cell viability}
In order to choose a suitable concentration of PQ that causes toxicity in GT 1-7, a concentration dependent toxicity study was performed using WST-1. Different concentration of PQ [(500 $\mu$M, 1, 2, 3, 5 and 10 mM) were assessed. Cell viability in response to PQ treatment significantly decreased progressively with an increase in concentration 500 $\mu$M (93.58 \% $\pm$ 1.23 \%, \textit{p} < 0.05), 1 mM (84.51 \% $\pm$ 1.48 \%, \textit{p} < 0.05), 2 mM (62.70 \% $\pm$ 3.27 \%, \textit{p} < 0.05), 3 mM (48.48 \% $\pm$ 3.58 \%, \textit{p} < 0.05), 5 mM (34.74 \% $\pm$ 3.11 \%, \textit{p} < 0.05) and 10 mM (23.41 \% $\pm$ 0.53 \%, \textit{p} < 0.05)] compared to the control group (100 \% $\pm$ 2.57 \%) (\Cref{fig:40_pq_cell_viability_a}). PQ at 3 mM reduced cell viability by approximately 50 \% and was selected for subsequent experiments.


\begin{figure}[!htbp]
\center
  \includegraphics[width=0.495\linewidth]{figures/chapter40/40_pq_cell_viability_a}
  \caption[Effect of PQ on cell viability]{\textbf{Effect of PQ on cell viability.} GT1-7 cells were treated with increasing concentrations of PQ for 6 h. Subsequently, a decrease in cell viability was assessed using WST-1 assay. All results are presented as a percentage of the control (mean $\pm$ SEM), \textit{n} = 3, with 6 replicates per group. * = \textit{p} < 0.05 vs control.}
  \label{fig:40_pq_cell_viability_a}
\end{figure} 

\section{Effect Spd and Ril on neuronal toxicity induced by PQ}
In order to assess whether Spd and Ril protect against PQ induced neuronal toxicity, GT1-7 cells were pre-treated with 1 \& 10 $\mu$M of Spd and Ril for 8 h followed by exposure to 3 mM PQ for 6 h. Cells were assessed for cell viability, ROS damage, cell death, autophagy and microtubule stability.

\subsection{Effect of low and high concentration Spd and Ril on cellular viability} 
Cell viability significantly decreased in PQ treated group (72.67 \% $\pm$ 3.42 \%, \textit{p} < 0.05) compared to the control group (100.00 \% $\pm$ 1.13 \%) (\Cref{fig:40_pq_spd_ril_cell_viability_a}: \textbf{A}). In addition, combination of spermidine with PQ at both concentrations; 1 $\mu$M Spd + PQ (89.25 \% ± 4.40 \%, \textit{p} < 0.05) and 10 $\mu$M Spd + PQ (93.04 \% $\pm$ 5.84 \%, \textit{p} < 0.05) significantly improved cell viability compared to the PQ treated group. When rilmenidine was used, a significant decreased in cell viability was observed in the PQ treated group (70.97 \% $\pm$ 4.00 \%, \textit{p} < 0.05), 1 $\mu$M Ril + PQ (74.43 \% $\pm$ 4.08 \%) and 10 $\mu$M Ril + PQ (78.41 \% $\pm$ 3.98 \%) compared to the control (100.00 \% $\pm$ 3.27 \%) (\Cref{fig:40_pq_spd_ril_cell_viability_a}: \textbf{B}). Furthermore, no significant differences were observed in cellular viability in the 1 $\mu$M Ril + PQ and 10 $\mu$M Ril + PQ compared to the PQ treated group.

\begin{figure}[!htbp]
  \center
  \begin{subfigure}[b]{0.495\linewidth}
    \includegraphics[width=\linewidth]{figures/chapter40/40_pq_spd_ril_cell_viability_a}
    \caption{Spermidine}
  \end{subfigure}
  \begin{subfigure}[b]{0.495\linewidth}
    \includegraphics[width=\linewidth]{figures/chapter40/40_pq_spd_ril_cell_viability_b}
    \caption{Rilmenidine}
  \end{subfigure}
  \caption[Effect of spermidine and rilmenidine on PQ induced neuronal toxicity]{\textbf{Effect of spermidine and rilmenidine on PQ induced neuronal toxicity.} GT1-7 cells were pre-treated with 1 \& 10 $\mu$M of Spd (\textbf{A}) and Ril (\textbf{B}) followed by exposure to 3 mM PQ for 6 h and assessed for cell viability using WST-1 assay. All results are presented as a percentage of the control (mean $\pm$ SEM), textit{n}=3, with 6 replicates per group. * = \textit{p} < 0.05 vs control, \# = \textit{p} < 0.05 vs PQ.}
  \label{fig:40_pq_spd_ril_cell_viability_a}
\end{figure}

\subsection{Effect of low and high concentration of Spd and Ril on reactive oxygen species (ROS)}
Following treatments, cells were stained for mitochondrial superoxide (O\textsubscript{2}\textsuperscript{-}) using mitoSOX and for cytosolic hydrogen peroxide (H\textsubscript{2}O\textsubscript{2}) using DCF. Mean fluorescence intensities were measured using flow cytometry. Mitochondrial O\textsubscript{2}\textsuperscript{-} was significantly increased in PQ treated group (3.92 $\pm$ 0.40, \textit{p} < 0.05), 1 $\mu$M Spd + PQ (2.11 $\pm$ 0.19, \textit{p} < 0.05) and 10 $\mu$M Spd + PQ (3.80 $\pm$ 0.52, \textit{p} < 0.05) compared to the control (1.00 $\pm$ 0.02), with no significant differences observed in the 1 $\mu$M Spd (1.05 $\pm$ 0.10) and 10 $\mu$M Spd (1.07 $\pm$ 0.03) compared to the control (\Cref{fig:40_pq_spd_ril_ros_a}: \textbf{A}\textit{i}). A significant decrease in mitochondrial O\textsubscript{2}\textsuperscript{-} was observed in the 1$\mu$M Spd, 1 $\mu$M Spd + PQ and 10 $\mu$M Spd compared to the PQ treated group. Similarly, a significant decrease was observed in 1 $\mu$M Spd compared to 1 $\mu$M Spd + PQ, 10 $\mu$M Spd compared to 10 $\mu$M Spd + PQ and more importantly in 1 $\mu$M Spd + PQ compared to 10 $\mu$M Spd + PQ (\Cref{fig:40_pq_spd_ril_ros_a}: \textbf{A}\textit{i}). 

Cytosolic H\textsubscript{2}O\textsubscript{2} was significantly increased in the PQ treated group (1.39 $\pm$ 0.10, \textit{p} < 0.05) and 10 $\mu$M Spd + PQ (1.40 $\pm$ 0.09, p < 0.05) compared to the control (1.00 $\pm$ 0.06), with no significant differences observed in the 1 $\mu$M Spd (1.00 $\pm$ 0.06), 1 $\mu$M Spd + PQ (1.18 $\pm$ 0.06) and 10 $\mu$M Spd (1.04 $\pm$ 0.06) compared to the control group (\Cref{fig:40_pq_spd_ril_ros_a}: \textbf{A}\textit{ii}). A significant decrease in cytosolic H\textsubscript{2}O\textsubscript{2} was observed in the 1 $\mu$M Spd, 1 $\mu$M Spd + PQ and 10 $\mu$M Spd compared to the PQ treated group. Similarly, a significant decrease was observed in 10 $\mu$M Spd compared to 10 $\mu$M Spd + PQ.

With regards to rilmenidine treatment, a significant increase in mitochondrial O\textsubscript{2}\textsuperscript{-} was observed in the 
in PQ treated group (1.41 $\pm$ 0.14, \textit{p} < 0.05), in 1 $\mu$M Ril + PQ (1.29 $\pm$ 0.07, \textit{p} < 0.05) and 10 $\mu$M Ril + PQ (1.47 $\pm$ 0.11, \textit{p} < 0.05) compared to the to the control group (1.00 $\pm$ 0.10), with no significant differences detected with 1 $\mu$M Ril (0.89 $\pm$ 0.08) and 10 $\mu$M Ril (0.98 $\pm$ 0.03) (\Cref{fig:40_pq_spd_ril_ros_a}: \textbf{B}\textit{i}). A significant decrease in mitochondrial O\textsubscript{2}\textsuperscript{-} was observed in the 1 $\mu$M Ril and 10 $\mu$M Ril compared to the PQ treated group. Similarly, a significant decrease was observed in 1 $\mu$M Ril compared to 1 $\mu$M Ril + PQ, and in 10 $\mu$M Ril compared to 10 $\mu$M Ril + PQ (\Cref{fig:40_pq_spd_ril_ros_a}: \textbf{B}\textit{i}). 

No significant differences in cytosolic H\textsubscript{2}O\textsubscript{2} were observed in all the groups tested [Control (1.00 $\pm$ 0.05), PQ (1.00 $\pm$ 0.03), 1 $\mu$M Ril (1.07 $\pm$ 0.04), 1 $\mu$M Ril + PQ (1.06 $\pm$ 0.04), 10 $\mu$M Ril (1.09 $\pm$ 0.04) and 10 $\mu$M Ril + PQ (1.07 $\pm$ 0.03)] (\Cref{fig:40_pq_spd_ril_ros_a}: \textbf{B}\textit{ii}).

\begin{figure}[!htbp]
\center
  \includegraphics[width=\linewidth]{figures/chapter40/40_pq_spd_ril_ros_a}
  \caption[Effect of spermidine and rilmenidine on mitochondrial ROS and cytosolic ROS in PQ induced neuronal toxicity]{\textbf{Effect of spermidine and rilmenidine on mitochondrial ROS and cytosolic ROS in PQ induced neuronal toxicity.} Quantitative analysis of MitoSox (\textbf{top}) and DCF (\textbf{bottom}) following treatment with spermidine (\textbf{A}) and rilmenidine (\textbf{B}). Data are presented as mean $\pm$ SEM, \textit{n} = 3, with 2 - 3 technical repeats per group. * = \textit{p} < 0.05 vs control group, \# = \textit{p} < 0.05 vs PQ, \@ = \textit{p} < 0.05 vs 1 $\mu$M Spd/Ril, \% = \textit{p} < 0.05 vs 10 $\mu$M Spd/Ril, and \$ = \textit{p} < 0.05 vs 1 $\mu$M Spd + PQ.}
  \label{fig:40_pq_spd_ril_ros_a}
\end{figure} 

\subsection{Effect of low and high concentration of Spd and Ril on cell death} 
Cell death was monitored following treatment using propidium iodide (PI), a membrane-impermeable dye which intercalates with DNA. PI accumulates in cells with compromised membrane integrity, a hallmark for necrotic cell death. Mean fluorescent intensities were measured using flow cytometry. Percentage of PI + cells was significantly increased in the PQ treated group (32.92 \% $\pm$ 2.50 \%, \textit{p} < 0.05), 1 $\mu$M Spd + PQ (24.11 \% $\pm$ 2.74 \%, \textit{p} < 0.05) and 10 $\mu$M Spd + PQ (29.44 \% $\pm$ 3.04 \%, \textit{p} < 0.05) compared to the control (3.91 \% $\pm$ 0.64 \%), with no significant differences observed in the 1 $\mu$M Spd (4.07 \% $\pm$ 0.27 \%) and 10 $\mu$M Spd (5.53 \% $\pm$ 0.64 \%) (\Cref{fig:40_pq_spd_ril_pi_a}: \textbf{A}). Importantly, a significant decrease in the number of PI + cells was observed in the 1 $\mu$M Spd + PQ, 1 $\mu$M Spd and 10 $\mu$M Spd group compared to the PQ treated group. Furthermore, a significant decrease was observed in 1 $\mu$M Spd compared to 1 $\mu$M Spd + PQ and in 10 $\mu$M Spd compared to 10 $\mu$M Spd + PQ (\Cref{fig:40_pq_spd_ril_pi_a}: \textbf{A}).

With regards to rilmenidine treatment, a significant increase in the percentage of PI + cells was observed in the PQ treated group (42.00 \% $\pm$ 1.62 \%, \textit{p} < 0.05), 1 $\mu$M Ril + PQ (47.83 \% $\pm$ 1.06 \%, \textit{p} < 0.05) and 10 $\mu$M Ril + PQ (51.35 \% $\pm$ 4.48 \%, \textit{p} < 0.05) compared to the control (6.87 \% $\pm$ 1.05 \%), with no significant differences observed in the 1 $\mu$M Ril (5.38 \% $\pm$ 0.36 \%) and 10 $\mu$M Ril (7.82 \% $\pm$ 0.72 \%) (\Cref{fig:40_pq_spd_ril_pi_a}: \textbf{B}). A significant decrease in the PI + cells was observed in the low and high concentration of rilmenidine (1 \& 10 $\mu$M Ril), while a significant increase was detected in the 10 $\mu$M Ril + PQ compared to the PQ treated group.  Moreover, a significant decrease was observed 1 $\mu$M Ril compared to 1 $\mu$M Ril + PQ and in 10 $\mu$M Ril compared to 10 $\mu$M Ril + PQ (\Cref{fig:40_pq_spd_ril_pi_a}: \textbf{B}).

\begin{figure}[!htbp]
  \center
  \begin{subfigure}[b]{0.495\linewidth}
    \includegraphics[width=\linewidth]{figures/chapter40/40_pq_spd_ril_pi_a}
    \caption{Spermidine}
  \end{subfigure}
  \begin{subfigure}[b]{0.495\linewidth}
    \includegraphics[width=\linewidth]{figures/chapter40/40_pq_spd_ril_pi_b}
    \caption{Rilmenidine}
  \end{subfigure}
  \caption[Effect of spermidine and rilmenidine on PQ induced neuronal toxicity]{\textbf{Effect of spermidine and rilmenidine on PQ induced neuronal toxicity.} GT1-7 cells were pre-treated with 1 \& 10 $\mu$M of Spd (\textbf{A}) and Ril (\textbf{B}) followed by exposure to 3 mM PQ for 6 h and assessed for cell viability using WST-1 assay. All results are presented as a percentage of the control (mean $\pm$ SEM), textit{n}=3, with 6 replicates per group. * = \textit{p} < 0.05 vs control, \# = \textit{p} < 0.05 vs PQ.}
  \label{fig:40_pq_spd_ril_pi_a}
\end{figure}


\subsection{Effect of low and high concentration of Spd and Ril on autophagy} 
\subsubsection{Autophagy assessment using fluorescence microscopy}
Following treatments, immunofluorescence and imaging was performed. LC3, p62 and lysosome puncta were counted using the thresholding method or by eye on image J. Our results indicate a progressive increase in the average number of LC3 puncta per cell in PQ treated group (83.69 $\pm$ 14.35), 1 µM Spd (101.00 $\pm$ 19.75) with significant increase observed in the in 1 $\mu$M Spd + PQ (142.90 $\pm$ 24.44, \textit{p} < 0.05), 10 $\mu$M Spd (196.70 $\pm$ 23.83, \textit{p} < 0.05) and 10 $\mu$M Spd + PQ (244.80 $\pm$ 42.03, \textit{p} < 0.05) compared to the control (75.58 $\pm$ 14.99), (\Cref{fig:40_pq_spd_ril_lc3_fm_a}: \textbf{A}\textit{i}, \& \textit{ii}. Moreover, a significant increase was observed in the 10 $\mu$M Spd and 10 $\mu$M Spd + PQ (\textit{p} < 0.05) compared to the PQ group. Furthermore, a significant increase was observed in 10 $\mu$M Spd (\textit{p} < 0.05) compared to 1 $\mu$M Spd  and in 10 $\mu$M Spd + PQ (\textit{p} < 0.05) compared to 10 $\mu$M Spd.


p62 puncta was significantly increased in the 10 $\mu$M Spd + PQ group (17.40 $\pm$ 4.42, \textit{p} < 0.05) compared to the control (2.56 $\pm$ 0.94), and PQ (5.39 $\pm$ 1.05), 1 $\mu$M Spd + PQ (2.90 $\pm$ 0.74) and 10 $\mu$M Spd group (8.00 $\pm$ 1.40) (\Cref{fig:40_pq_spd_ril_p62_fm_a}: \textbf{A}\textit{i}, \& \textit{ii}. Moreover, no significant differences were detected in the control and PQ group compared to 1 $\mu$M Spd (3.00 $\pm$ 0.82), 1 $\mu$M Spd + PQ (2.90 $\pm$ 0.74) and 10 $\mu$M Spd group (8.00 $\pm$ 1.40) and between control and PQ group.

Lysosomal puncta was significantly increased in the PQ treated group (25.38 $\pm$ 5.26, \textit{p} < 0.05), 10 $\mu$M Spd (27.09 $\pm$ 5.29, \textit{p} < 0.05) and 10 $\mu$M Spd + PQ (29.00 $\pm$ 7.34, \textit{p} < 0.05) compared to the control group (11.27 $\pm$ 3.33), with no significant differences observed in the 1 $\mu$M Spd (16.90 $\pm$ 3.47) and 1 $\mu$M Spd + PQ (17.00 $\pm$ 3.77) (\Cref{fig:40_pq_spd_ril_lyso_fm_a}: \textbf{A}\textit{i}, \& \textit{ii}.

With regards to rilmenidine treatment, no significant differences were observed in the average number of LC3 puncta in all the groups [Con (69.43 $\pm$ 25.51), PQ (80.40 $\pm$ 11.57), 1 $\mu$M Ril (79.00 $\pm$ 21.80), 1 $\mu$M Ril + PQ (81.50 $\pm$ 46.31), 10 $\mu$M Ril (90.60 $\pm$ 34.47) and 10 $\mu$M Ril + PQ (67.29 $\pm$ 17.29)] (\Cref{fig:40_pq_spd_ril_lc3_fm_a}: \textbf{B}\textit{i}, \& \textit{ii}.

Similarly with p62 puncta, no significant differences were observed in all the groups [Con (3.67 $\pm$ 1.28), PQ (4.20 $\pm$ 0.97), 1 $\mu$M Ril (7.00 $\pm$ 2.02), 1 $\mu$M Ril + PQ (7.17 $\pm$ 3.97), 10 $\mu$M Ril (6.00 $\pm$ 2.07) and 10 $\mu$M Ril + PQ (4.57 $\pm$ 2.06)], (\Cref{fig:40_pq_spd_ril_p62_fm_a}: \textbf{B}\textit{i}, \& \textit{ii}.

Similarly with lysosomal puncta, no significant differences were observed in all the groups [Con (12.29 $\pm$ 3.93), PQ (10.80 $\pm$ 1.83), 1 $\mu$M Ril (17.00 $\pm$ 2.49), 1 $\mu$M Ril + PQ (27.50 $\pm$ 15.12), 10 $\mu$M Ril (20.00 $\pm$ 6.96) and 10 $\mu$M Ril + PQ (19.29 $\pm$ 3.20)], (\Cref{fig:40_pq_spd_ril_lyso_fm_a}: \textbf{B}\textit{i}, \& \textit{ii}.

\begin{figure}[!htbp]
  \center
  \begin{subfigure}[b]{0.495\linewidth}
    \includegraphics[width=\linewidth]{figures/chapter40/40_pq_spd_ril_lc3_fm_a}
    \caption{Spermidine}
  \end{subfigure}
  \begin{subfigure}[b]{0.495\linewidth}
    \includegraphics[width=\linewidth]{figures/chapter40/40_pq_spd_ril_lc3_fm_b}
    \caption{Rilmenidine}
  \end{subfigure}
  \caption[Effect of spermidine and rilmenidine on LC3 puncta upon PQ induced neuronal toxicity]{\textbf{Effect of spermidine and rilmenidine on LC3 puncta upon PQ induced neuronal toxicity.} Representative fluorescence micrographs and quantitative analysis of LC3 puncta following treatment with 1 \& 10 $\mu$M of Spd (\textbf{A}) and Ril (\textbf{B}). Data are presented as mean $\pm$ SEM, textit{n}=3, with a total of 30 - 60 cells analysed per group. * = \textit{p} < 0.05 vs control, \# = \textit{p} < 0.05 vs PQ, \$ = \textit{p} < 0.05 vs 1 $\mu$M Spd + PQ, and \@ = \textit{p} < 0.05 vs 1 $\mu$M Spd. Scale bar:10 $\mu$M.}
  \label{fig:40_pq_spd_ril_lc3_fm_a}
\end{figure}

\begin{figure}[!htbp]
  \center
  \begin{subfigure}[b]{0.495\linewidth}
    \includegraphics[width=\linewidth]{figures/chapter40/40_pq_spd_ril_p62_fm_a}
    \caption{Spermidine}
  \end{subfigure}
  \begin{subfigure}[b]{0.495\linewidth}
    \includegraphics[width=\linewidth]{figures/chapter40/40_pq_spd_ril_p62_fm_b}
    \caption{Rilmenidine}
  \end{subfigure}
  \caption[Effect of spermidine and rilmenidine on p62 puncta upon PQ induced neuronal toxicity]{\textbf{Effect of spermidine and rilmenidine on p62 puncta upon PQ induced neuronal toxicity.} Representative fluorescence micrographs and quantitative analysis of p62 puncta following treatment with 1 \& 10 $\mu$M of Spd (\textbf{A}) and Ril (\textbf{B}). Data are presented as mean $\pm$ SEM, textit{n}=3, with a total of 30 - 60 cells analysed per group. * = \textit{p} < 0.05 vs control, \# = \textit{p} < 0.05 vs PQ, \$ = \textit{p} < 0.05 vs 1 $\mu$M Spd + PQ, and \% = \textit{p} < 0.05 vs 10 $\mu$M Spd. Scale bar:10 $\mu$M.}
  \label{fig:40_pq_spd_ril_p62_fm_a}
\end{figure}

\begin{figure}[!htbp]
  \center
  \begin{subfigure}[b]{0.495\linewidth}
    \includegraphics[width=\linewidth]{figures/chapter40/40_pq_spd_ril_lyso_fm_a}
    \caption{Spermidine}
  \end{subfigure}
  \begin{subfigure}[b]{0.495\linewidth}
    \includegraphics[width=\linewidth]{figures/chapter40/40_pq_spd_ril_lyso_fm_b}
    \caption{Rilmenidine}
  \end{subfigure}
  \caption[Effect of spermidine and rilmenidine on lysosomal puncta upon PQ induced neuronal toxicity]{\textbf{Effect of spermidine and rilmenidine on lysosomal puncta upon PQ induced neuronal toxicity.} Representative fluorescence micrographs and quantitative analysis of lysosomal puncta following treatment with 1 \& 10 $\mu$M of Spd (\textbf{A}) and Ril (\textbf{B}). Data are presented as mean $\pm$ SEM, textit{n}=3, with a total of 30 - 60 cells analysed per group. * = \textit{p} < 0.05 vs control.}
  \label{fig:40_pq_spd_ril_lyso_fm_a}
\end{figure}



















