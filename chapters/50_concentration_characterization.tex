\chapter{The characterization of autophagy modulation using Spermidine and Rilmenidine in an APP over-expression model }
\section{Introduction}

The N2a cells expressing Sweddish double mutations (N2aSwe) associated with AD pathology have been extensively used to unravel the mechanisms related to A$\beta$ pathology \citep{Lee2015,Park2011, Schlachetzki2013}. Accumulation of toxic A$\beta$ peptide due to increased APP cleavage or decreased A$\beta$ degradation has been shown to be an early event in AD progression \citep{Selkoe2016}. Moreover, large body of evidence indicates that increased A$\beta$-induced neurotoxicity and impaired protein degradation may be key factors in AD pathogenesis \citep{Selkoe2016} . Modulation of autophagy using pharmacological agents such as rapamycin, temsirolimus and cilostazol to induce A$\beta$ clearance has been reported to be neuroprotective \citep{Lee2015,Park2011,Jiang2014a}. The potential neuroprotective effects of rilmenidine and spermidine using the N2aSwe cell line, however, has not been investigated. In addition, a concentration dependent effect of these drugs on cellular protection remains to be elucidated. Therefore, the aims of this chapter were to assess (i) the effects of a low and high concentration spermidine and rilmenidine on autophagy modulation and subsequent protein clearance as well as neuronal protection following APP over-expression (over-time). For this purpose, N2aSwe cells were treated with 5 mM BA for 24 and 48 h in the presence and absence of rilmenidine and spermidine.

\section{The effect of a low and high concentration spermidine and rilmenidine on cellular viability in the context of APP over-expression}
In order to assess whether a low and high concentration of spermidine and rilmenidine protect against neuronal toxicity induced upon  24 h and 48 h APP over-expression, cellular viability was assessed using a WST-1. Reductive capacity was significantly decreased in the 24 h BA treated group (73.99 \% $\pm$ 2.08 \%, \textit{p} < 0.05), 24 h BA + 1 $\mu$M Spd (83.98 \% $\pm$ 2.67 \%, \textit{p} < 0.05), and 24 h BA + 10 $\mu$M Spd (88.91 \% $\pm$ 2.07 \%, \textit{p} < 0.05), 48 h BA treated group (61.37 \% $\pm$ 2.75 \%, \textit{p} < 0.05), 48 h BA + 1 $\mu$M Spd (83.51 \% $\pm$ 2.67 \%, \textit{p} < 0.05 ) and 48 h BA + 10 $\mu$M Spd (85.91 \% $\pm$ 2.55 \%, \textit{p} < 0.05) compared to the control cells (100.00 \% $\pm$ 5.80 \%) (\Cref{fig:50_spd_ril_cell_viability_a}: \textbf{A}). More importantly, and in comparison to the 24 h BA treated group, reductive capacity was significantly increased in 24 h BA + 1 $\mu$M Spd (\textit{p} < 0.05), and 24 h BA + 10 $\mu$M Spd (\textit{p} < 0.05), while it was significantly reduced further in the 48 h BA (\textit{p} < 0.05). Moreover, reductive capacity was significantly increased in 48 h BA + 1 $\mu$M Spd (\textit{p} < 0.05 ) and 48 h BA + 10 $\mu$M Spd (\textit{p} < 0.05) compared to 48 h BA treated group. 

Similarly, with the use of rilmenidine in (\Cref{fig:50_spd_ril_cell_viability_a}: \textbf{B}), cellular viability was significantly decreased in the 24 h BA treated group (75.14 \% $\pm$ 2.62 \%, \textit{p} < 0.05), 24 h BA + 1 $\mu$M Ril (86.45 \% $\pm$ 2.25 \%, \textit{p} < 0.05), and 24 h BA + 10 $\mu$M Ril (87.72 \% $\pm$ 3.81 \%, \textit{p} < 0.05), 48 h BA treated group (57.01 \% $\pm$ 3.25 \%, \textit{p} < 0.05), 48 h BA + 1 $\mu$M Ril (75.91 \% $\pm$ 1.99 \%, \textit{p} < 0.05) and 48 h BA + 10 $\mu$M Ril (80.22 \% $\pm$ 2.49 \%, \textit{p} < 0.05) compared to the control cells (100.00 \% $\pm$ 5.39 \%). More importantly, cellular viability was significantly increased in 24 h BA + 1 $\mu$M Ril (\textit{p} < 0.05), and 24 h BA + 10 $\mu$M Ril (\textit{p} < 0.05) compared to 24 h BA treated group, while it was significantly reduced further in the 48 h BA. Furthermore, reductive capacity was significantly increased in 48 h BA + 1 $\mu$M Ril (\textit{p} < 0.05) and 48 h BA + 10 $\mu$M Ril (\textit{p} < 0.05) compared to 48 h BA treated group. 

\begin{figure}[!htbp]
  \centering
  \begin{subfigure}[b]{0.495\linewidth}
    \includegraphics[width=\linewidth]{figures/chapter50/50_spd_ril_cell_viability_a}
    \caption{Spermidine}
  \end{subfigure}
  \begin{subfigure}[b]{0.495\linewidth}
    \includegraphics[width=\linewidth]{figures/chapter50/50_spd_ril_cell_viability_b}
    \caption{Rilmenidine}
  \end{subfigure}
  \caption[Effect of spermidine and rilmenidine on cell viability in response to APP-over-expression]{\textbf{Effect of spermidine and rilmenidine on cell viability in response to APP-over-expression}. N2aSwe cells were treated with 5 mM BA for 24 h and 48 h in the presence/absence of 1 \& 10 $\mu$M Spd (\textbf{A}) and Ril (\textbf{B}). Subsequently, reductive capacity as a measure of cell viability was assessed using the WST1 assay. All results are presented as a percentage of the control (mean $\pm$ SEM), \textit{n}=3, with 6 replicates per group. * = \textit{p} < 0.05 vs control, \# = \textit{p} < 0.05 vs 24 h BA and \$ = \textit{p} < 0.05 vs 48 h BA.}
  \label{fig:50_spd_ril_cell_viability_a}
\end{figure}

\section{Effect of a low and high concentration spermidine on the clearance of APP clusters using d-STORM}
Following imaging with d-STORM and post processing (drifting, grouping and rendering), the number of APP clusters and their size was analysed by using the thresholding plugin on Image J \citep{Schindelin2012}. A significant increase in the number of APP clusters was observed following treatment with 48 h BA (339.50 $\pm$ 112.50, \textit{p} < 0.05) compared to the control cells (105.20 $\pm$ 24.49), with no significant differences seen in the 24 h BA group (150.30 $\pm$ 29.29), 24 h BA + 1 $\mu$M Spd (106.00 $\pm$ 23.00), 24 h BA + 10 $\mu$M Spd (120.00 $\pm$ 54.00), 48 h BA + 1 $\mu$M Spd (151.00 $\pm$ 121.00) and 48 h BA + 10 $\mu$M Spd (333.00 $\pm$ 63.00) (\Cref{fig:50_spd_ril_app_clusters_a} \& \Cref{fig:50_spd_ril_app_clusters_b}: \textbf{A} \& \textbf{B}). More importantly, and in comparison to the 24 h BA treated group, APP cluster number was decreased following treatment with 24 h BA + 1 $\mu$M Spd and 24 h BA + 10 $\mu$M Spd, while increased following treatment with 48 h BA, however, no significant differences were reached. Moreover, APP cluster number was decreased in the 48 h BA + 1 $\mu$M Spd compared to 48 h BA treated group, again without reaching significance, while it remained similar in the 48 h BA + 10 $\mu$M Spd. Furthermore, no significant differences were observed in the 24 h BA + 1 $\mu$M Spd versus 24 h BA + 10 $\mu$M Spd, 48 h BA + 1 $\mu$M Spd versus 48 hrs BA + 10 $\mu$M Spd, 24 h BA + 1 $\mu$M Spd versus 48 h BA + 1 $\mu$M Spd and 24 h BA + 10 $\mu$M Spd versus 48 hrs BA + 10 $\mu$M Spd, however combinational treatment with the lowest concentration of spermidine (1 $\mu$M) reduced the number of APP clusters compared to the highest concentration (10 $\mu$M) (\Cref{fig:50_spd_ril_app_clusters_a} \& \Cref{fig:50_spd_ril_app_clusters_b}: \textbf{A} \& \textbf{B}).

Regarding the size distribution profile of APP clusters, results show a distribution of APP clusters at sizes varying from < 5 nm\textsuperscript{2} to 50 nm\textsuperscript{2}, with morphology that was mostly spherical, regardless of the treatment. Treatment with BA for 48 h resulted in an increase in the number of APP clusters compared to the control at all sizes, while treatment with BA for 24 h resulted in increased frequency at sizes < 5 to 10 nm\textsuperscript{2} (\Cref{fig:50_spd_ril_app_clusters_b}: \textbf{C}\textit{i}, \textbf{C}\textit{ii}). In addition, both 24 h BA + 1 $\mu$M Spd and 24 h BA + 10 $\mu$M Spd resulted in the clearance of APP clusters as evident by the reduction in the number of APP clusters at sizes < 5 to 25 nm\textsuperscript{2}. Similarly, 48 h BA + 1 $\mu$M Spd and 48 h BA + 10 $\mu$M Spd cleared APP clusters, leading to a decrease in frequency at sizes < 5 to 25 and 26 to 50 nm\textsuperscript{2} (\Cref{fig:50_spd_ril_app_clusters_b}: \textbf{C}\textit{i}, \textbf{C}\textit{ii}). 

\begin{landscape}
\begin{figure}[!htbp]
\center
  \includegraphics[width=0.80\linewidth]{figures/chapter50/50_spd_ril_app_clusters_a}
  \caption[dSTORM micrographs of APP clusters]{\textbf{dSTORM micrographs of APP clusters}. Shown are representative fluorescence micrographs in gauss modec(magenta) of the control group, 24 h BA, 24 h BA + 1 $\mu$M Spd, 24 h BA + 10 $\mu$M Spd, 48 h BA, 48 h BA + 1 $\mu$M Spd and 48 h BA + 10 $\mu$M Spd. Images were taken using at 100x objective lens. Scale bar = 1 $\mu$m and 0.5 $\mu$m.}
  \label{fig:50_spd_ril_app_clusters_a}
\end{figure} 
\end{landscape}

\begin{landscape}
\begin{figure}[!htbp]
\center
  \includegraphics[width=0.80\linewidth]{figures/chapter50/50_spd_ril_app_clusters_b}
  \caption[Effect of spermidine in response to APP over-expression]{\textbf{Effect of spermidine in response to APP over-expression}. Quantitative analysis of the number of APP clusters (\textbf{A}), representative western blot (\textbf{B}) and size distribution (\textbf{C}) puncta. Data are presented as mean $\pm$ SEM, \textit{n} = 3. * = \textit{p} < 0.05 vs control.}
  \label{fig:50_spd_ril_app_clusters_b}
\end{figure} 
\end{landscape}

\section{Effect of a low and high concentration rilmenidine on the clearance of APP clusters using d-STORM}
The number of APP clusters was significantly increased in the 48 h BA treated group (339.50 $\pm$ 112.50, \textit{p} < 0.05) compared to the control cells (105.20 $\pm$ 24.49), with no significant differences observed in the 24 h BA treated group (150.30 $\pm$ 29.29), 24 h BA + 1 $\mu$M Ril (225.50 $\pm$ 29.50) and 24 h BA + 10 $\mu$M Ril (100.50 $\pm$ 32.50), 48 h BA + 1 $\mu$M Ril (293.50 $\pm$ 40.50) and 48 h BA + 10 $\mu$M Ril (78.50 $\pm$ 16.50) (\Cref{fig:50_spd_ril_app_clusters_c} \& \Cref{fig:50_spd_ril_app_clusters_d}: \textbf{A} \& \textbf{B}). In comparison to the 24 h BA treated group, APP clusters were increased in 24 h BA + 1 $\mu$M Ril, decreased in 24 h BA + 10 $\mu$M Ril, while increased as expected in the 48 h BA group, however, no significant differences were reached. Moreover, APP clusters were decreased in the 48 h + 1 $\mu$M Ril and decreased further in the 48 h BA + 10 $\mu$M Ril compared to 48 h BA treated group, no significant differences were reached. Furthermore, no significant differences were observed in the 24 h BA + 1 $\mu$M Ril versus 24 h BA + 10 $\mu$M Ril, 48 h BA + 1 $\mu$M Ril versus 48 h BA + 10 $\mu$M Ril, 24 h BA + 1 $\mu$M Ril versus 48 h BA + 1 $\mu$M Ril and 24 h BA + 10 $\mu$M Ril versus 48 h BA + 10 $\mu$M Ril, however combinational treatment with the highest concentration of rilmenidine (10 $\mu$M) reduced the number of APP clusters compared to the lowest concentration (1 $\mu$M) (10 $\mu$M) (\Cref{fig:50_spd_ril_app_clusters_c} \& \Cref{fig:50_spd_ril_app_clusters_d}: \textbf{A} \& \textbf{B}).

With regards to the size distribution profile of APP clusters shown in  (\Cref{fig:50_spd_ril_app_clusters_d}: \textbf{C}\textit{i}, \textbf{C}\textit{ii}), 48 h BA resulted in an increase in the number of APP clusters compared to the control at all sizes, while BA 24 h displayed a higher frequency at sizes < 5 to 10 nm\textsuperscript{2}. In addition, both 24 h BA + 1 $\mu$M Ril and 24 h BA + 10 $\mu$M Ril resulted in the clearance of APP clusters as evident by the reduction in the number of APP clusters at sizes < 5 to 10 nm\textsuperscript{2}. Furthermore, 48 h BA + 1 $\mu$M Ril reduced APP clusters at sizes < 5 to 40 and 45 to 50 nm\textsuperscript{2}, while 48 h BA + 10 $\mu$M Ril reduced APP clusters at all sizes < 5 to 50 nm\textsuperscript{2}  (\Cref{fig:50_spd_ril_app_clusters_d}: \textbf{C}\textit{i}, \textbf{C}\textit{ii}), suggesting that 10 $\mu$M Ril clears APP clusters better than 1 $\mu$M Ril, and this is clearly visible after 48 h APP over-expression.

\begin{landscape}
\begin{figure}[!htbp]
\center
  \includegraphics[width=0.80\linewidth]{figures/chapter50/50_spd_ril_app_clusters_c}
  \caption[dSTORM micrographs of APP clusters]{\textbf{dSTORM micrographs of APP clusters}. Shown are representative fluorescence micrographs in gauss modec(magenta) of the control group, 24 h BA, 24 h BA + 1 $\mu$M Ril, 24 h BA + 10 $\mu$M Ril, 48 h BA, 48 h BA + 1 $\mu$M Ril and 48 h BA + 10 $\mu$M Ril. Images were taken using at 100x objective lens. Scale bar = 1 $\mu$m and 0.5 $\mu$m.}
  \label{fig:50_spd_ril_app_clusters_c}
\end{figure} 
\end{landscape}

\begin{landscape}
\begin{figure}[!htbp]
\center
  \includegraphics[width=0.80\linewidth]{figures/chapter50/50_spd_ril_app_clusters_d}
  \caption[Effect of rilmenidine in response to APP over-expression]{\textbf{Effect of rilmenidine in response to APP over-expression}. Quantitative analysis of the number of APP clusters (\textbf{A}), representative western blot (\textbf{B}) and size distribution (\textbf{C}) puncta. Data are presented as mean $\pm$ SEM, \textit{n} = 3. * = \textit{p} < 0.05 vs control.}
  \label{fig:50_spd_ril_app_clusters_d}
\end{figure} 
\end{landscape}

\section{Discussion: The characterization of autophagy modulation using Spermidine and Rilmenidine in an APP over-expression model}
To date, the therapeutic potential of spermidine and rilmenidine in models of AD has not been fully investigated. In particular, concentration dependent effects of these drugs on cellular protection remains to be elucidated. In this chapter, we assessed the effects of low and high concentrations of spermidine and rilmenidine on autophagy modulation and subsequent protein clearance as well as neuronal protection following APP over-expression. N2a cells expressing a Swedish double mutation (APPswe) which is linked to the familial onset of AD was used. In order to assess the potential protective effects of spermidine and rilmenidine, N2aSwe cells were treated with 5 mM BA for 24 and 48 h to induce APP transgene expression followed by exposure to spermidine or rilmenidine for the last 8 h of the treatment intervention. Overproduction of APP results in increased levels of A$\beta$ and APP C-terminal fragments (CTF$\beta$), both of which contribute to the pathology of AD \citep{Walsh2007}. Thus, cells were assessed for cellular viability using WST-1, APP protein expression and clearance using western blotting, while protein clusters were assessed and quantified using single molecule imaging, i.e, d-STORM.

\subsection{The effect of a low and high concentration spermidine and rilmenidine on cellular viability in the context of APP over-expression}
Our results reveal a significant reduction in cellular viability following treatment with BA alone at 24 and 48 h and in the combination groups of spermidine and rilmenidine at both time points compared to the control group (\Cref{fig:50_spd_ril_cell_viability_a}: \textbf{A}). Notably, combination treatment of 24 h BA and spermidine at low and high concentrations (24 h BA + 1 $\mu$M Spd and 24 h BA + 10 $\mu$M Spd) enhanced cellular viability compared to the 24 h BA alone. Furthermore, enhanced cellular viability was also observed in the combination treatment of 48 h BA and spermidine at low and high concentrations (48 h BA + 1 $\mu$M Spd and 48 h BA + 10 $\mu$M Spd) compared to 48 h BA alone. To our surprise, similar results were observed with rilmenidine (\Cref{fig:50_spd_ril_cell_viability_a}: \textbf{B}). Combination treatment at 24 h (24 h BA + 1 $\mu$M Ril and 24 h BA + 10 $\mu$M Ril) and 48 h (48 h BA + 1 $\mu$M Ril and 48 h BA + 10 $\mu$M Ril) enhanced cellular viability compared to 24 h and 48 h BA alone, respectively (\Cref{fig:50_spd_ril_cell_viability_a}: \textbf{B}). Moreover, cellular viability was further decreased in the 48 h BA compared to the 24 h BA. These results suggest that firstly, APP over-expression induces cytotoxicity in N2aSwe cells in a time dependent manner and secondly, that spermidine and rilmenidine at both concentrations protect against cytotoxicity effects induced by APP over-expression. To our knowledge, this study is the first to demonstrate the protective effects of spermidine and rilmenidine against cytotoxicity induced by APP over-expression in N2aSwe cells. In line with these findings, \citet{Lee2015} reported that cilostazol, a phosphodiesterase III inhibitor that induces autophagy, enhanced cellular viability and protected against neurotoxocity in N2a cells treated with exogenous A$\beta$1 - 42 and in N2aSwe cells endogenously overproducing A$\beta$. Others have shown that administration of autophagy modulators protect against APP induced cytotoxicity effects in N2aSwe cells. Indeed, \citet{Jiang2014a} demonstrated that temsirolimus protected against A$\beta$ induced neurotoxicity and apoptosis in HEK293 cells over-expressing APP695 and in mouse models of APP by enhancing the clearance of soluble and insoluble A$\beta$40 and A$\beta$42. \citet{Park2011} reported a reduction in A$\beta$ and phosphorylated tau levels using cilostazol, resulting in neuronal protection in the same cells.

\subsection{The effect of a low and high concentration spermidine and rilmenidine on the clearance of APP clusters using d-STORM}

Next, we assessed the role of low and high concentration spermidine on the clearance of APP clusters using d-STORM. Our results revealed that APP clusters are spherical in shape, regardless of the treatment but vary in size (\Cref{fig:50_spd_ril_app_clusters_a} \& \Cref{fig:50_spd_ril_app_clusters_b}). In addition, results showed an increase in APP clusters after 24 h of APP over-expression with BA with a significant increase observed in the 48 h BA treatment group compared to the control (\Cref{fig:50_spd_ril_app_clusters_a} \& \Cref{fig:50_spd_ril_app_clusters_b}: \textbf{A} \& \textbf{B}). Although no significant difference was observed, a trend towards a decrease in the number of APP clusters was detected in the combination groups (24 h BA + 1 $\mu$M Spd and 24 h BA + 10 $\mu$M Spd ) compared to 24 h BA alone, and only in the 48 h BA + 1 $\mu$M Spd compared to 48 h BA. Similar results were observed with rilmenidine (\Cref{fig:50_spd_ril_app_clusters_c} \& \Cref{fig:50_spd_ril_app_clusters_d}). A trend towards a decrease in the number of APP clusters was detected in the 24 h BA + 10  $\mu$M Ril compared to 24 h BA alone, and in both 48 h BA + 1  $\mu$M Ril and 48 h BA + 10  $\mu$M Ril compared to 48 h BA (\Cref{fig:50_spd_ril_app_clusters_c} \& \Cref{fig:50_spd_ril_app_clusters_d}: \textbf{A} \& \textbf{B}). These results suggest firstly, a time dependent accumulation of APP clusters manifests upon APP over-expression with BA. Secondly, these results suggest that both concentrations of spermidine protect against APP over-expression leading to the clearance of APP clusters after 24 h of APP induction, while only the low concentration of spermidine lead to a clearance of APP after 48 h of APP over-expression, suggesting that autophagy induction in the context of spermidine using a high concentration does not necessarily translate into protection. Thirdly, these results suggest that the high concentration of rilmenidine clears APP clusters leading to protection after 24 h of APP over-expression, while both low and high concentrations of rilmenidine clear APP clusters after 48 h of APP over-expression. Since the high concentration of rilmenidine at 24 h and 48 h resulted in the enhanced clearance of APP clusters, these results suggest that a high concentration of rilmenidine is required for the clearance of APP in this model. 

We questioned, whether the clearance of APP clusters by spermidine and rilmenidine was dependent on the size of the clusters. Hence we assessed the frequency distribution profile of APP clusters. Our results revealed a distribution of APP clusters of sizes varying from < 5 nm\textsuperscript{2} to 50 nm\textsuperscript{2}, with 48 h BA resulting in the accumulation of APP clusters at a range of sizes compared to the control, while 24 hrs BA displayed a higher frequency at to increased frequency at sizes < 5 to 10 nm\textsuperscript{2} (Fig.5.3 & 5.5 C). Moreover, combination treatment of low and high concentration spermidine at 24 hrs reduced APP clusters leading to the clearance of APP clusters that were less than 5 to 25 nm\textsuperscript{2} compared to the 24 BA alone while combination treatment of low and high concentration spermidine at 48 hrs resulted in the clearance APP clusters of sizes from < 5 to 25 and 26 to 50 nm\textsuperscript{2} compared to 48BA alone (Fig.5.3.C.i &ii). Notably, while both concentration of spermidine cleared APP clusters in a similar manner at 24 hrs after APP overexpression, the lower concentration of spermidine cleared APP clusters better after 48 hrs of APP overexpression at all sizes better. These results suggest after 24 hrs of APP induction, spermidine is able to only clear APP clusters that are less than 25 nm\textsuperscript{2}, while after long exposure to BA, spermidine clears all different sizes of APP clusters. In addition, combination treatment of low and high concentration rilmenidine at 24 hrs APP overexpression resulted in the clearance of APP clusters that were less than 5 to 10 nm\textsuperscript{2}, while combination treatment of low and high concentration rilmenidine resulted in the clearance of APP clusters at sizes < 5 to 50 nm\textsuperscript{2} (Fig.5.5.C.i &ii), with highest concentration rilmenidine resulting in more APP clusters being cleared than the low concentration, particularly after 48 hrs APP overexpression. In overall, these results suggest that spermidine and rilmenidine clear APP clusters effectively when cells are exposed to APP overexpression for longer. In addition, after 48 hrs APP overexpression, spermidine at lower concentration offers better protection while rilmenidine at high concentration offer protection suggesting that different drugs act differently on the same cells and that the concentration of the drug used matters. To our knowledge, our study is the first to assess APP clusters and their clearance by spermidine and rilmenidine using d-STORM in models of AD and supports the findings obtained with WST1 analysis. We speculate that rilmenidine and spermidine protect against APP induced toxicity by inducing autophagy. However, this warrants for further investigation were autophagy markers are assessed. This study also contributes to the current body of literature by demonstrating that the concentration of the drug used in models of neurodegeneration matters in autophagy modulation and the subsequent protein clearance and protection. Others however have used d-STORM in vitro models of AD. Esbjörner et al. (2014) demonstrated that intracellular aggregates of A40 and A42 are mostly spherical and increase in size after 48 hrs of overexpression compared to 24 hrs in the case of A40, while A42 were bigger in size at both time, suggesting that A40 and A42 have different aggregation kinetics, while Kaminski Schiele et al. (2011) demonstrated different forms of Aβ 1- 42 within HeLa cells including oligomeric and fibrillar structures. In another study, Apetri et al. (2016) assessed the uptake of α-syn preformed aggregates in SH-SY5Y human neuroblastoma cells using d-STORM and showed that α-syn clusters were decreased in size as they move through the endosomal pathway. 

In conclusion, in this study, we showed that spermidine and rilmenidine protect against APP induced neuronal toxicity by enhancing cellular viability and reducing APP clusters, probably by mechanisms that are dependent on autophagy. Therefore, administration of these drugs may represent therapeutic strategies for AD. 




























