\chapter{Concentration dependent characterization of autophagy modulation using Spermidine and Rilmenidine in an APP over-expression model }
\section{Introduction}

The N2a cells expressing Sweddish double mutations (N2aSwe) associated with AD pathology have been extensively used to unravel the mechanisms related to A$\beta$ pathology \citep{Lee2015,Park2011, Schlachetzki2013}. Accumulation of toxic A$\beta$ peptide due to increased APP cleavage or decreased A$\beta$ degradation has been shown to be an early event in AD progression \citep{Selkoe2016}. Moreover, large body of evidence indicates that increased A$\beta$-induced neurotoxicity and impaired protein degradation may be key factors in AD pathogenesis \citep{Selkoe2016} . Modulation of autophagy using pharmacological agents such as rapamycin, temsirolimus and cilostazol to induce A$\beta$ clearance has been reported to be neuroprotective \citep{Lee2015,Park2011,Jiang2014a}. The potential neuroprotective effects of rilmenidine and spermidine using the N2aSwe cell line, however, has not been investigated. In addition, a concentration dependent effect of these drugs on cellular protection remains to be elucidated. Therefore, the aims of this chapter were to assess (i) the effects of a low and high concentration spermidine and rilmenidine on autophagy modulation and subsequent protein clearance as well as neuronal protection following APP over-expression (over-time). For this purpose, N2aSwe cells were treated with 5 mM BA for 24 and 48 h in the presence and absence of rilmenidine and spermidine.

\section{Effect of a low and high concentration spermidine and rilmenidine on the clearance of APP clusters}
In order to assess whether a low and high concentration of spermidine and rilmenidine protect against neuronal toxicity induced upon  24 h and 48 h APP over-expression, cellular viability was assessed using a WST-1. Reductive capacity was significantly decreased in the 24 h BA treated group (73.99 \% $\pm$ 2.08 \%, \textit{p} < 0.05), 24 h BA + 1 $\mu$M Spd (83.98 \% $\pm$ 2.67 \%, \textit{p} < 0.05), and 24 h BA + 10 $\mu$M Spd (88.91 \% $\pm$ 2.07 \%, \textit{p} < 0.05), 48 h BA treated group (61.37 \% $\pm$ 2.75 \%, \textit{p} < 0.05), 48 h BA + 1 $\mu$M Spd (83.51 \% $\pm$ 2.67 \%, \textit{p} < 0.05 ) and 48 h BA + 10 $\mu$M Spd (85.91 \% $\pm$ 2.55 \%, \textit{p} < 0.05) compared to the control cells (100.00 \% $\pm$ 5.80 \%) (\Cref{fig:50_spd_ril_cell_viability_a}: \textbf{A}). More importantly, and in comparison to the 24 h BA treated group, reductive capacity was significantly increased in 24 h BA + 1 $\mu$M Spd (\textit{p} < 0.05), and 24 h BA + 10 $\mu$M Spd (\textit{p} < 0.05), while it was significantly reduced further in the 48 h BA (\textit{p} < 0.05). Moreover, reductive capacity was significantly increased in 48 h BA + 1 $\mu$M Spd (\textit{p} < 0.05 ) and 48 h BA + 10 $\mu$M Spd (\textit{p} < 0.05) compared to 48 h BA treated group. 

Similarly, with the use of rilmenidine in (\Cref{fig:50_spd_ril_cell_viability_a}: \textbf{B}), cellular viability was significantly decreased in the 24 h BA treated group (75.14 \% $\pm$ 2.62 \%, \textit{p} < 0.05), 24 h BA + 1 $\mu$M Ril (86.45 \% $\pm$ 2.25 \%, \textit{p} < 0.05), and 24 h BA + 10 $\mu$M Ril (87.72 \% $\pm$ 3.81 \%, \textit{p} < 0.05), 48 h BA treated group (57.01 \% $\pm$ 3.25 \%, \textit{p} < 0.05), 48 h BA + 1 $\mu$M Ril (75.91 \% $\pm$ 1.99 \%, \textit{p} < 0.05) and 48 h BA + 10 $\mu$M Ril (80.22 \% $\pm$ 2.49 \%, \textit{p} < 0.05) compared to the control cells (100.00 \% $\pm$ 5.39 \%). More importantly, cellular viability was significantly increased in 24 h BA + 1 $\mu$M Ril (\textit{p} < 0.05), and 24 h BA + 10 $\mu$M Ril (\textit{p} < 0.05) compared to 24 h BA treated group, while it was significantly reduced further in the 48 h BA. Furthermore, reductive capacity was significantly increased in 48 h BA + 1 $\mu$M Ril (\textit{p} < 0.05) and 48 h BA + 10 $\mu$M Ril (\textit{p} < 0.05) compared to 48 h BA treated group. 

\begin{figure}[!htbp]
  \centering
  \begin{subfigure}[b]{0.495\linewidth}
    \includegraphics[width=\linewidth]{figures/chapter50/50_spd_ril_cell_viability_a}
    \caption{Spermidine}
  \end{subfigure}
  \begin{subfigure}[b]{0.495\linewidth}
    \includegraphics[width=\linewidth]{figures/chapter50/50_spd_ril_cell_viability_b}
    \caption{Rilmenidine}
  \end{subfigure}
  \caption[Effect of spermidine and rilmenidine on cell viability in response to APP-over-expression]{\textbf{Effect of spermidine and rilmenidine on cell viability in response to APP-over-expression}. N2aSwe cells were treated with 5 mM BA for 24 h and 48 h in the presence/absence of 1 \& 10 $\mu$M Spd (\textbf{A}) and Ril (\textbf{B}). Subsequently, reductive capacity as a measure of cell viability was assessed using the WST1 assay. All results are presented as a percentage of the control (mean $\pm$ SEM), \textit{n}=3, with 6 replicates per group. * = \textit{p} < 0.05 vs control, \# = \textit{p} < 0.05 vs 24 h BA and \$ = \textit{p} < 0.05 vs 48 h BA.}
  \label{fig:50_spd_ril_cell_viability_a}
\end{figure}

\section{Effect of a low and high concentration spermidine on the clearance of APP clusters using d-STORM}
Following imaging with d-STORM and post processing (drifting, grouping and rendering), the number of APP clusters and their size was analysed by using the thresholding plugin on Image J \citep{Schindelin2012}. A significant increase in the number of APP clusters was observed following treatment with 48 h BA (339.50 $\pm$ 112.50, \textit{p} < 0.05) compared to the control cells (105.20 $\pm$ 24.49), with no significant differences seen in the 24 h BA group (150.30 $\pm$ 29.29), 24 h BA + 1 $\mu$M Spd (106.00 $\pm$ 23.00), 24 h BA + 10 $\mu$M Spd (120.00 $\pm$ 54.00), 48 h BA + 1 $\mu$M Spd (151.00 $\pm$ 121.00) and 48 h BA + 10 $\mu$M Spd (333.00 $\pm$ 63.00) (\Cref{fig:50_spd_ril_app_clusters_a} \& \Cref{fig:50_spd_ril_app_clusters_b}: \textbf{A} \& \textbf{B}). More importantly, and in comparison to the 24 h BA treated group, APP cluster number was decreased following treatment with 24 h BA + 1 $\mu$M Spd and 24 h BA + 10 $\mu$M Spd, while increased following treatment with 48 h BA, however, no significant differences were reached. Moreover, APP cluster number was decreased in the 48 h BA + 1 $\mu$M Spd compared to 48 h BA treated group, again without reaching significance, while it remained similar in the 48 h BA + 10 $\mu$M Spd. Furthermore, no significant differences were observed in the 24 h BA + 1 $\mu$M Spd versus 24 h BA + 10 $\mu$M Spd, 48 h BA + 1 $\mu$M Spd versus 48 hrs BA + 10 $\mu$M Spd, 24 h BA + 1 $\mu$M Spd versus 48 h BA + 1 $\mu$M Spd and 24 h BA + 10 $\mu$M Spd versus 48 hrs BA + 10 $\mu$M Spd, however combinational treatment with the lowest concentration of spermidine (1 $\mu$M) reduced the number of APP clusters compared to the highest concentration (10 $\mu$M) (\Cref{fig:50_spd_ril_app_clusters_a} \& \Cref{fig:50_spd_ril_app_clusters_b}: \textbf{A} \& \textbf{B}).

Regarding the size distribution profile of APP clusters, results show a distribution of APP clusters at sizes varying from < 5 nm\textsuperscript{2} to 50 nm\textsuperscript{2}, with morphology that was mostly spherical, regardless of the treatment. Treatment with BA for 48 h resulted in an increase in the number of APP clusters compared to the control at all sizes, while treatment with BA for 24 h resulted in increased frequency at sizes < 5 to 10 nm\textsuperscript{2} (Fig.5.3.C.i \&ii). In addition, both 24 h BA + 1 $\mu$M Spd and 24 h BA + 10 $\mu$M Spd resulted in the clearance of APP clusters as evident by the reduction in the number of APP clusters at sizes < 5 to 25 nm\textsuperscript{2}. Similarly, 48 h BA + 1 $\mu$M Spd and 48 h BA + 10 $\mu$M Spd cleared APP clusters, leading to a decrease in frequency at sizes < 5 to 25 and 26 to 50 nm\textsuperscript{2}2(Fig.5.3.C.i \&ii). 

\begin{landscape}
\begin{figure}[!htbp]
\center
  \includegraphics[width=0.90\linewidth]{figures/chapter50/50_spd_ril_app_clusters_a}
  \caption[dSTORM micrographs of APP clusters]{\textbf{dSTORM micrographs of APP clusters}. Shown are representative fluorescence micrographs in gauss modec(magenta) of the control group, 24 hh BA, 24 h BA + 1 $\mu$M Spd, 24 h BA + 10 $\mu$M Spd, 48 h BA, 48 h BA + 1 $\mu$M Spd and 48 h BA + 10 $\mu$M Spd. Images were taken using at 100x objective lens. Scale bar = 1 $\mu$m and 0.5 $\mu$m.}
  \label{fig:50_spd_ril_app_clusters_a}
\end{figure} 
\end{landscape}

\begin{landscape}
\begin{figure}[!htbp]
\center
  \includegraphics[width=0.90\linewidth]{figures/chapter50/50_spd_ril_app_clusters_b}
  \caption[Effect of spermidine in response to APP over-expression]{\textbf{Effect of spermidine in response to APP over-expression}. Quantitative analysis of the number of APP clusters (\textbf{A}), representative western blot (\textbf{B}) and size distribution (\textbf{C}) puncta. Data are presented as mean $\pm$ SEM, \textit{n} = 3. * = \textit{p} < 0.05 vs control.}
  \label{fig:50_spd_ril_app_clusters_b}
\end{figure} 
\end{landscape}


